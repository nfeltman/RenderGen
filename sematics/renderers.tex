\documentclass{article}
\usepackage{amsmath}
\usepackage{stmaryrd}
\usepackage{proof}
\usepackage{amssymb,amsthm}
\usepackage[left=3cm,top=3cm,right=4cm,nohead,bottom=3cm]{geometry}
\parindent 0pt
\parskip .75em

%bnf stuff
\newcommand {\myit} [1]{\operatorname{\it{#1}}}
\newcommand {\expr} {\langle\mathit{expr}\rangle}
\newcommand {\decompG} {\langle\mathit{breakMethodG}\rangle}
\newcommand {\decompS} {\langle\mathit{breakMethodS}\rangle}
\newcommand {\decompH} {\langle\mathit{breakMethodH}\rangle}
\newcommand {\sizeRange} {\langle\mathit{sizeRange}\rangle}
\newcommand {\gbar} {~~|~~}


\newcommand {\lcata}{\llparenthesis}
\newcommand {\rcata}{\rrparenthesis}
\newcommand {\lhylo}{\llbracket}
\newcommand {\rhylo}{\rrbracket}

%source nodes 
\newcommand {\chain}{\mathtt{>>}}
\newcommand {\rightFish}{\mathtt{>=>}}
\newcommand {\fix}{\mathtt{fix}}
\newcommand {\call}{\mathtt{call}}
\newcommand {\test}{\mathtt{test}}
\newcommand {\filt}{\mathtt{filt}}
\newcommand {\isect}{\mathtt{hit}}
\newcommand {\shade}{\mathtt{Shade}}
\newcommand {\bmrg}{\mathtt{BreakMapReduceG}}
\newcommand {\bmrs}{\mathtt{BreakMapReduceS}}
\newcommand {\bmrh}{\mathtt{BreakMapReduceH}}
\newcommand {\buildG}{\mathtt{buildG}}
\newcommand {\buildS}{\mathtt{buildS}}
\newcommand {\id}{\mathtt{id}}
\newcommand {\unboundG}{\mathtt{unboundG}}
\newcommand {\unboundS}{\mathtt{unboundS}}
\newcommand {\mmrG}{\mathtt{mmrG}}
\newcommand {\caseG}{\mathtt{caseG}}
\newcommand {\mmrS}{\mathtt{mmrS}}
\newcommand {\oneS}{\mathtt{1S}}
\newcommand {\oneG}{\mathtt{1G}}
\newcommand {\oneH}{\mathtt{1H}}
\newcommand {\flatD}{\mathtt{flat}}
\newcommand {\sampP}{\mathtt{16^2\_SP}}
\newcommand {\twoGP}{\mathtt{2\_GP}}
\newcommand {\geosamp}{\mathtt{GeoSamp}}
\newcommand {\hit}{\mathtt{Hit}}
\newcommand {\frag}{\mathtt{Frag}}
\newcommand {\semimap}{\mathtt{semimap}}
\newcommand {\map}{\mathtt{map}}

%semantics
\newcommand {\calG}{\mathcal{G}}
\newcommand {\calS}{\mathcal{S}}
\newcommand {\calK}{\mathcal{K}}
\newcommand {\bound}{\mathtt{bound}}


% Equation spacing
\newcommand {\inferenceSpacing}{\setlength{\jot}{3ex}}
\newcommand {\normalSpacing}{\setlength{\jot}{1ex}}
\newcommand {\tab}{~~~~~~}

\title{\Large\textbf{Mockingbird Semantics}}
\author{Nicolas Feltman}
\begin{document}
\maketitle

\section{MbFilter}

\subsection{Grammar}
\begin{align*}
\decompG &= \oneG~|~\twoGP \\
\decompS &= \oneS~|~\sampP \\
\decompH &= \oneH \\
\\
\expr &= \expr \chain \expr \\
&\gbar \fix~x. \expr \\
&\gbar x \\
&\gbar \test \{ \expr \} \\
&\gbar \isect \\
&\gbar \shade \\
&\gbar \mathtt{if~}|g| > n \mathtt{~then~} \expr \mathtt{~else~} \expr \\
&\gbar \bmrg (\decompG) \{ \expr \} \\
&\gbar \bmrs (\decompS) \{ \expr \} \\
&\gbar \bmrh (\decompH) \{ \expr \} \\
\\
\sizeRange &= [a,b]~|~[a+] \\
\langle\myit{entity-type}\rangle &= \geosamp~\sizeRange~\sizeRange\\
 &\gbar \hit~\sizeRange \\
 & \gbar \frag~\sizeRange \\
\langle\myit{transition-type}\rangle &= \langle\myit{entity-type}\rangle \to \langle\myit{entity-type}\rangle \\
\langle\myit{breakG-type}\rangle &= (\sizeRange \to \sizeRange)~\sizeRange  \\
\langle\myit{breakS-type}\rangle &= \sizeRange~(\sizeRange \to \sizeRange)  \\
\langle\myit{breakH-type}\rangle &= \sizeRange \to \sizeRange
\end{align*}

\subsection{Typing Rules}
\inferenceSpacing
\begin{gather}
\infer {\Gamma\vdash e_1 \chain e_2 : \tau \to \tau''}{\Gamma \vdash e_1 : \tau\to \tau' & \Gamma \vdash e_2 : \tau' \to \tau''} \\
\infer {\Gamma\vdash \fix~x.e : \phi}{\Gamma, x:\phi \vdash e : \phi} \\
\infer {\Gamma\vdash \call~x : \phi}{\Gamma(x) = \phi} \\
\infer {\Gamma\vdash \test \{ e \} : \geosamp~\sigma_1~\sigma_2 \to \tau}{\Gamma \vdash e : \geosamp~\sigma_1~\sigma_2  \to \tau} \\
\infer {\Gamma \vdash \isect : \geosamp~[1,1]~[1,1] \to \hit} {\cdot} \\
\infer {\Gamma \vdash \shade : \hit~[1,1] \to \frag} {\cdot} \\
\infer {\Gamma \vdash  \mathtt{if~}|g| > n \mathtt{~then~} e_1 \mathtt{~else~} e_2 : \geosamp~[a,b]~\sigma \to \tau} {\Gamma \vdash e_1 : \geosamp~[n+1,b]~\sigma \to \tau & \Gamma \vdash e_2 : \geosamp~[a,n]~\sigma \to \tau} \\
\infer {\Gamma \vdash \bmrg (d) \{ e \} : \geosamp \to \tau} {\Gamma \vdash e : \geosamp \to \tau } \\
\infer {\Gamma \vdash \bmrs (d) \{ e \} : \geosamp \to \tau} {\Gamma \vdash e : \geosamp \to \tau } \\
\infer {\Gamma \vdash \bmrh (d) \{ e \} : \hit \to \tau} {\Gamma \vdash e : \hit \to \tau }
\end{gather}

\subsection{Denotational Semantics}

\begin{align}
|e_1 \chain e_2 | &= |e_2| \circ |e_1| \\
|\fix~x. e| &= |[(\fix~x. e)/(\call~x)]e| \\
|\test\{e\}| &=  |e| \circ id_\between \cup  (\lambda (G,S).\{(\bot,k):(s,k)\in S\}) \circ id_{\not \between} \\
|\isect| &= \{((\{g\},\{(s,k)\}), \{(h,k)\}) : h = isect(g,s)\}\\
|\shade| &= \{(\{(s,k)\}, \{(f,k)\}) : f = shade(h)\} \\
| \mathtt{if~}|g| > n \mathtt{~then~} e_1 \mathtt{~else~} e_2| &=  |e_1| \circ id_{|g|>n} \cup  |e_2| \circ id_{|g|\le n}\\
|\bmrg (d) \{ e \} | &= \begin{array}{l} \{((G,S), \{F_1 \oplus \cdots \oplus F_n\}) \\
	: ((G_i,S), f_i) \in |e|, (G, \{G_1,...,G_n\}) \in |d| \} \end{array}\\
|\bmrs (d) \{ e \} | &= \begin{array}{l} \{((G,S), \{F_1 \cup \cdots \cup F_n\}) \\
	: ((G,S_i), f_i) \in |e|, (S, \{S_1,...,S_n\}) \in |d| \} \end{array} \\
|\bmrh (d) \{ e \} | &= \begin{array}{l} \{(H, \{F_1 \cup \cdots \cup F_n\}) \\
	: (H_i, F_i) \in |e|, (H, \{H_1,...,H_n\}) \in |d| \} \end{array}
\end{align}

\subsection{Theorems}

\begin{enumerate}
\item For any expression $e$, if $((G,S),F) \in |e|$, then $keys(S) = keys(F)$.
\item For any expression $e:\geosamp \to \frag$, if $((G,S),F) \in |e|$, then for all $(s,k) \in S$, $(\bigoplus_{g\in G} shade(isect(g,s)),k) \in F$.
\end{enumerate}



\section{MbOrder}
\normalSpacing
\subsection{Grammar}
\begin{align*}
e &= e~\chain~e \\
&\gbar \fix~x.~e \\
&\gbar x \\
&\gbar \isect \\
&\gbar \test \{ e \} \\
&\gbar \filt \{ e \} \\
&\gbar \mathtt{ifsizeG}(>n)~e \mathtt{~else~} e \\
&\gbar \buildG ( d_G ) \\
&\gbar \buildS ( d_S ) \\
&\gbar \unboundG \\
&\gbar \unboundS \\
&\gbar \mmrG \{ e \} \\
&\gbar \mmrS \{ e \} \\
\\
d_\alpha &= \mathtt{id} \\
&\gbar p_\alpha~\rightFish~d_\alpha \\
&\gbar \fix~x.~d_\alpha \\
&\gbar x \\
&\gbar \bound (d_\alpha) \\
&\gbar \mathtt{ifsize\alpha}(>n)~d_\alpha \mathtt{~else~} d_\alpha \\
\\
p_G &= \oneG~|~\twoGP \\
p_S &= \oneS~|~\sampP \\
\\
\tau &= \sigma \to \sigma \\
\sigma &= \geosamp~\delta~\delta \\
&\gbar \hit~\delta \\
\delta &= * \\
&\gbar [\delta] \\
&\gbar \# \delta \\
&\gbar \delta + \delta \\
&\gbar \mu~\alpha.~\delta \\
&\gbar \alpha \\
\end{align*}

\subsection{Typing Rules}
\inferenceSpacing
\begin{gather}
\infer {\Gamma\vdash e_1 \chain e_2 : \sigma \to \sigma''}{\Gamma \vdash e_1 : \sigma\to \sigma' & \Gamma \vdash e_2 : \sigma' \to \sigma''} \\
\infer {\Gamma\vdash \fix~x.e : \tau}{\Gamma, x:\tau \vdash e : \tau} \\
\infer {\Gamma\vdash x : \tau}{\Gamma(x) = \tau} \\
\infer {\Gamma\vdash \test \{ e \} : \geosamp~\#\delta_1~\#\delta_2 \to \sigma}{\Gamma \vdash e : \geosamp~\#\delta_1~\#\delta_2  \to \sigma} \\
\infer {\Gamma \vdash \isect : \geosamp~\ast~\ast \to \hit~\ast} {\cdot} \\
\infer {\Gamma \vdash \mmrG \{ e \} : \geosamp~[\delta_1]~\delta_2 \to \hit~\ast} {\Gamma \vdash e : \geosamp~\delta_1~\delta_2 \to \hit~\ast }\\
\infer {\mathtt{ifsizeG}(>n)~e_1 \mathtt{~else~} e_2 : \geosamp~\delta_1~\delta_2 \to \sigma} {\Gamma \vdash e_1 : \geosamp~\delta_1~\delta_2 \to \sigma & \Gamma \vdash e_2 : \geosamp~\delta_1~\delta_2 \to \sigma} \\
\infer { \unboundG : \geosamp~\#\delta_1~\delta_2 \to \geosamp~\delta_1~\delta_2} {\cdot}\\
\infer { \unboundS : \geosamp~\delta_1~\#\delta_2 \to \geosamp~\delta_1~\delta_2} {\cdot}\\
\infer { \buildG (d): \geosamp~\ast~\delta_2 \to \geosamp~\delta_1~\delta_2} {d : \delta_1}\\
\infer { \buildS (d): \geosamp~\delta_1~\ast \to \geosamp~\delta_1~\delta_2} {d : \delta_2}
\end{gather}

\subsection {Big Step Semantics}
\begin{align*}
g &= [t, \ldots, t] \\
&\gbar [g, \ldots, g]\\
&\gbar (bound, g)\\
&\gbar inL (g)\\
&\gbar inR (g)\\
s &= [r, \ldots, r] \\
&\gbar [s, \ldots, s]\\
&\gbar (bound, s)\\
&\gbar inL (s)\\
&\gbar inR (s)\\
v &= (g,s) 
\end{align*}

\begin{gather}
\infer {(e_1 \chain e_2)~v \Downarrow v''}{e_1~ v \Downarrow v' & e_2~ v' \Downarrow v''} \\
\infer { \isect ~([g],[s]) \Downarrow [h]}{h = intersect(g,s)} \\
\infer { \buildG (d)~(g,s) \Downarrow (g',s)}{d~g \Downarrow_G g'} \\
\infer { \bound (d)~g \Downarrow_G (box(g'),v)}{d~g \Downarrow_G g'} \\
\infer { (p >=> d)~g \Downarrow_G [g_1', g_2',...,g_n']}{p(g) =[g_1, g_2,...,g_n] & d~g_i \Downarrow_G g_i'} \\
\infer { \test \{ e \}~((b_1,g),(b_2,s)) \Downarrow v}{intersects(b_1,b_2) & e~((b_1,g),(b_2,s)) \Downarrow v} \\
\infer { \mmrG \{ e \}~([g_1,\ldots,g_n],s) \Downarrow v}{e~(g_i,s) \Downarrow v_i & v = v_1 \oplus \cdots \oplus v_n}\\
\infer { \mmrS \{ e \}~(g,[s_1,\ldots,s_n]) \Downarrow v}{e~(g,s_i) \Downarrow v_i & v = v_1 \cup \cdots \cup v_n}\\
\infer { \caseG \{ e_1 | e_2 \}~(inL(g),s) \Downarrow v}{e_1~(g,s) \Downarrow v}\\
\infer { \caseG \{ e_1 | e_2 \}~(inR(g),s) \Downarrow v}{e_2~(g,s) \Downarrow v}
\end{gather}

\section{Examples}
\normalSpacing
We start with the source for the repeat-work raytracer.
\begin{align*}
&\buildS (\oneS >=> \id) >> \\
&\mmrS \{ \\
&\tab\fix~x. \\
&\tab\tab \buildS(\bound (\mathtt{id})) >> \\
&\tab\tab \buildG(\bound (\mathtt{id})) >> \\
&\tab\tab\test \{ \\
&\tab\tab\tab \unboundS >> \\
&\tab\tab\tab \unboundG >> \\
&\tab\tab\tab \mathtt{ifsizeG}(>1)~\buildG (\twoGP >=> \id) >> \mmrG \{x\} \\
&\tab\tab\tab \mathtt{else}~\isect \\
&\tab \tab \} \\
&\}
\end{align*}
We now precompute the 2GP split.
\begin{align*}
&\buildS (\oneS >=> \id) >> \\
&\mmrS \{ \\
&\tab\fix~x. \\
&\tab\tab \buildS(\bound (\mathtt{id})) >> \\
&\tab\tab \buildG(\bound (\mathtt{id})) >> \\
&\tab\tab\test \{ \\
&\tab\tab\tab \unboundS >> \\
&\tab\tab\tab \unboundG >> \\
&\tab\tab\tab \buildG (\mathtt{ifsizeG}(>1)~(\twoGP >=> \id)~\mathtt{else}~\id) >> \\
&\tab\tab\tab \mathtt{caseG}~\mmrG \{x\}~|~\isect\\
&\tab \tab \} \\
&\}
\end{align*}
Reorder the bounding and unbounding to make the next step less daunting.
\begin{align*}
&\buildS (\oneS >=> \id) >> \\
&\mmrS \{ \\
&\tab\fix~x. \\
&\tab\tab \buildS(\bound (\mathtt{id})) >> \\
&\tab\tab \buildG(\bound (\mathtt{id})) >> \\
&\tab\tab\test \{ \\
&\tab\tab\tab \unboundG >> \\
&\tab\tab\tab \buildG (\mathtt{ifsizeG}(>1)~(\twoGP >=> \id)~\mathtt{else}~\id) >> \\
&\tab\tab\tab \unboundS >> \\
&\tab\tab\tab \mathtt{caseG}~\mmrG \{x\}~|~\isect\\
&\tab \tab \} \\
&\}
\end{align*}
Pull through the bound-test-unbound (I have to show this all at once, since the intermediate stages are inexpressible in this language).
\begin{align*}
&\buildS (\oneS >=> \id) >> \\
&\mmrS \{ \\
&\tab\fix~x. \\
&\tab\tab \buildS(\bound (\mathtt{id})) >> \\
&\tab\tab \buildG (\bound(\mathtt{ifsizeG}(>1)~(\twoGP >=> \id)~\mathtt{else}~\id)) >> \\
&\tab\tab\test \{ \\
&\tab\tab\tab \unboundG >> \\
&\tab\tab\tab \unboundS >> \\
&\tab\tab\tab \mathtt{caseG}~\mmrG \{x\}~|~\isect\\
&\tab \tab \} \\
&\}
\end{align*}
Push the unboundS through the case and into the mmrG.
\begin{align*}
&\buildS (\oneS >=> \id) >> \\
&\mmrS \{ \\
&\tab\fix~x. \\
&\tab\tab \buildS(\bound (\mathtt{id})) >> \\
&\tab\tab \buildG (\bound(\mathtt{ifsizeG}(>1)~(\twoGP >=> \id)~\mathtt{else}~\id)) >> \\
&\tab\tab\test \{ \\
&\tab\tab\tab \unboundG >> \\
&\tab\tab\tab \mathtt{caseG}~\mmrG \{\unboundS >> x\}~|~(\unboundS >> \isect)\\
&\tab \tab \} \\
&\}
\end{align*}
Lift out the buildS.
\begin{align*}
&\buildS (\oneS >=> \id) >> \\
&\mmrS \{ \\
&\tab \buildS(\bound (\mathtt{id})) >> \\
&\tab\fix~x. \\
&\tab\tab \buildG (\bound(\mathtt{ifsizeG}(>1)~(\twoGP >=> \id)~\mathtt{else}~\id)) >> \\
&\tab\tab\test \{ \\
&\tab\tab\tab \unboundG >> \\
&\tab\tab\tab \mathtt{caseG}~\mmrG \{x\}~|~(\unboundS >> \isect)\\
&\tab \tab \} \\
&\}
\end{align*}
Lift the buildG through the fix.
\begin{align*}
&\buildS (\oneS >=> \id) >> \\
&\mmrS \{ \\
&\tab \buildS(\bound (\mathtt{id})) >> \\
&\tab \buildG (\fix~z.~\bound(\mathtt{ifsizeG}(>1)~(\twoGP >=> z)~\mathtt{else}~\id)) >> \\
&\tab\fix~x. \\
&\tab\tab\test \{ \\
&\tab\tab\tab \unboundG >> \\
&\tab\tab\tab \mathtt{caseG}~\mmrG \{x\}~|~(\unboundS >> \isect)\\
&\tab \tab \} \\
&\}
\end{align*}
Finally list the bvh build through mmrS and we have the bvh-accelerated raytracer.
\begin{align*}
&\buildS (\oneS >=> \id) >> \\
&\buildG (\fix~z.~\bound(\mathtt{ifsizeG}(>1)~(\twoGP >=> z)~\mathtt{else}~\id)) >> \\
&\mmrS \{ \\
&\tab \buildS(\bound (\mathtt{id})) >> \\
&\tab\fix~x. \\
&\tab\tab\test \{ \\
&\tab\tab\tab \unboundG >> \\
&\tab\tab\tab \mathtt {caseG}~\mmrG\{x\}~|~(\unboundS>>\isect) \\
&\tab \tab \} \\
&\}
\end{align*}

Now here's the dumb raytracer expressed in another language:
\begin{align*}
&\fix~y. \\
&\tab \map^{(G\times)} >> push^{(G\times)}_{S} >> \\
&\tab \fix~x. \\
&\tab\tab \%G(boundBox) >> \\
&\tab\tab \%S(boundRay) >> \\
&\tab\tab \mathtt{if}~\%GS(intersect)~\mathtt{then} \\
&\tab\tab\tab \%G(unbound) >> \\
&\tab\tab\tab \%S(unbound) >> \\
&\tab\tab\tab \mathtt{if}~\%G(sizeGT1)~\mathtt{then}\\
&\tab\tab\tab\tab \%G(2GP) >> \semimap_G^{2}(x) >> \%H(closer) \\
&\tab\tab\tab \mathtt{else}\\
&\tab\tab\tab\tab \%GS(hit)  \\
&\tab\tab \mathtt{else} \\
&\tab\tab\tab \%S(allMiss)
\end{align*}


Now here's the dumb raytracer expressed in yet another language:
\begin{align*}
&\fix~y. \\
&\tab \%G. \\
&\tab \fix~x. \\
&\tab\tab \%G(boundBox) >> \\
&\tab\tab \%S(boundRay) >> \\
&\tab\tab \mathtt{if}~\%GS(intersect)~\mathtt{then} \\
&\tab\tab\tab \%G(unbound) >> \\
&\tab\tab\tab \%S(unbound) >> \\
&\tab\tab\tab \mathtt{if}~\%G(sizeGT1)~\mathtt{then}\\
&\tab\tab\tab\tab \%G(2GP) >> \semimap_G^{2}(x) >> \%H(closer) \\
&\tab\tab\tab \mathtt{else}\\
&\tab\tab\tab\tab \%GS(hit)  \\
&\tab\tab \mathtt{else} \\
&\tab\tab\tab \%S(allMiss)\\
&
\end{align*}
And the bvh-accelerated raytracer:
\begin{align*}
&\%S (\oneS) >> \\
&\%G(\fix~z.~ boundBox >> \map^{\times \overline{B}} ( \\
&\tab\tab\tab \mathtt{if}~(sizeGT1)~\mathtt{then}\\
&\tab\tab\tab\tab 2GP >> \map^{2}(x) >> \iota_1 \\
&\tab\tab\tab \mathtt{else}~\iota_2 ) >> \\
&*mapOverS* \{ \\
&\tab \%S(boundRay) >> \\
&\tab\fix~x. \\
&\tab\tab \mathtt{if}~\%GS(intersect)~\mathtt{then} \\
&\tab\tab\tab \%G(unbound) >> \\
&\tab\tab\tab \mathtt {case} \\
&\tab\tab\tab ~~(\semimap_G^{2}(x) >> \%H(closer)) \\
&\tab\tab\tab |~(\%S(unbound)>>\%GS(\isect)) \\
&\tab\tab \mathtt{else} \\
&\tab\tab\tab \%S(allMiss)\\
&\}
\end{align*}

\section{Splitting}
\begin{gather}
% @ sign
\infer {@e \rightsquigarrow [e, x.x]}{e \text{ has no @}} \\
% variable
\infer {v \rightsquigarrow [(), \_.v]}{\cdot} \\
% constant
\infer {c \rightsquigarrow [(), \_.c]}{c \text{ is a constant}} \\
% tuple
\infer {(e_1,e_2) \rightsquigarrow [(p_1,p_2),x.\mathtt{let}~(x_1,x_2)=x~\mathtt{in}~(r_1,r_2)]}{e_1 \rightsquigarrow [p_1,x_1.r_1] & e_2 \rightsquigarrow [p_2,x_2.r_2]} \\
% case without @
\infer {
\begin{array}{r}
\mathtt{case}~e_1~\mathtt{of}~v_2=>e_2~|~v_3=>e_3 \rightsquigarrow  [(\mathtt{case}~e_1~\mathtt{of}~v_2=>injL~p_2~|~v_3=>injR~p_3), \\ x.(\mathtt{case}~x~\mathtt{of}~x_2=>r_2~|~x_3=>r_3)]\end{array}
}{e_1 \text{ has no @} & e_2 \rightsquigarrow [p_2,x_2.r_2] & e_3 \rightsquigarrow [p_3,x_3.r_3]} \\
% case with one @
\infer {
\mathtt{case}~e_1~\mathtt{of}~@=>e_2~|~v_3=>e_3 \rightsquigarrow  [p_2,
x_2.(\mathtt{case}~e_1~\mathtt{of}~\_=>r_2~|~v_3=>e_3)] 
}{e_1 \text{ has no @} & e_2 \rightsquigarrow [p_2,x_2.r_2]} \\
% function app
\infer {f~e \rightsquigarrow [p,x.f~r]}{e \rightsquigarrow [p,x.r]} \\
% function def
\infer {\mathtt{eval}~f~x=e~\mathtt{with}~e_2 \rightsquigarrow [\mathtt{eval}~f_2~x=e~\mathtt{with}~e_2, x.]}{e \rightsquigarrow [p,x.r]} 
\end{gather}

\end{document}
