\documentclass[]{article}

% The following \documentclass options may be useful:

% preprint      Remove this option only once the paper is in final form.
% 10pt          To set in 10-point type instead of 9-point.
% 11pt          To set in 11-point type instead of 9-point.
% authoryear    To obtain author/year citation style instead of numeric.

\usepackage{amsmath}
\usepackage{amssymb}
\usepackage{amsthm}
\usepackage{stmaryrd}
\usepackage{proof}
\usepackage{mathpartir}
\usepackage{bbm}
\usepackage{mathabx}
\usepackage{listings}
\usepackage{graphicx}
\usepackage[usenames,dvipsnames]{color}
\usepackage{framed}

\title{Typing the Outputs}
\author{Nico}

\newcommand {\next}{asdlfkj}
%!TEX root = paper.tex

\newcommand {\TODO} {\textbf{TODO:} }

\definecolor{needsfixcolor}{rgb}{0.8,0.0,0.0}
\newcommand{\needsfix}[1]{{\color{needsfixcolor}\textbf{#1}}}

% core conventions and language names
\newcommand {\bbone} {\ensuremath{\mathbbmss{1}}}
\newcommand {\bbonep} {\ensuremath{\mathbbmss{1G}}}
\newcommand {\bbonem} {\ensuremath{\mathbbmss{1M}}}
\newcommand {\bbtwo} {\ensuremath{\mathbbmss{2}}}
\newcommand {\ellStaged} {$\mathrm{L}^{\bbone\bbtwo}$}
\newcommand {\ellResid} {$\mathrm{L^R}$}
\newcommand {\ellTarget} {$\mathrm{L^T}$}
\newcommand {\lamStaged} {$\lambda^{\bbone\bbtwo}$}
\newcommand {\lamCircle} {$\lambda^{\fut}$}
\newcommand {\lamResid} {$\mathrm{\lambda^R}$}
\newcommand {\lamTarget} {$\mathrm{\lambda^T}$}
\newcommand {\lang} {\lamStaged}
\newcommand {\langTwo} {$\lambda^\bbtwo$}
\newcommand {\langS} {$\lambda^\&$}
\newcommand {\langmono} {$\lambda^{\textrm{U}}$}
\newcommand {\rmint} {{\rm int}}
\newcommand {\rmunit} {{\rm unit}}
\newcommand {\rmbool} {{\rm bool}}
\newcommand {\exv}[1]{\underline{#1}}

% metatheory
\newcommand{\wfsym}{\ensuremath{\mathsf{wf}}}
\newcommand{\pvalsym}{\ensuremath{\mathsf{pval}}}
\newcommand{\ressym}{\ensuremath{\mathsf{res}}}
\newcommand{\valsym}{\ensuremath{\mathsf{val}}}
\newcommand{\donesym}{\ensuremath{\mathsf{done}}}
\newcommand{\fvalsym}{\ensuremath{\mathsf{fval}}}
\newcommand{\cpvalsym}{\ensuremath{\mathsf{cpval}}}
\newcommand{\wf}{\ \wfsym}
\newcommand{\pval}{\ \pvalsym}
\newcommand{\res}{\ \ressym}
\newcommand{\val}{\ \valsym}
\newcommand{\fval}{\ \fvalsym}
\newcommand{\cpval}{\ \cpvalsym}

% bnf stuff
\newcommand {\myit} [1]{\operatorname{\it{#1}}}
\newcommand {\stage} {\langle\mathit{stage}\rangle}
\newcommand {\type} {\tau}
\newcommand {\typeo} {\langle\bbone\text{-}\mathit{type}\rangle}
\newcommand {\typet} {\langle\bbtwo\text{-}\mathit{type}\rangle}
\newcommand {\expr} {e}
\newcommand {\brancho} {\langle\bbone\text{-}\mathit{brn}\rangle}
\newcommand {\brancht} {\langle\bbtwo\text{-}\mathit{brn}\rangle}
\newcommand {\expro} {\langle\bbone\text{-}\mathit{exp}\rangle}
\newcommand {\exprt} {\langle\bbtwo\text{-}\mathit{exp}\rangle}
%\newcommand {\val} {\langle\mathit{val}\rangle}
\newcommand {\valo} {\langle\bbone\text{-}\mathit{val}\rangle}
\newcommand {\valt} {\langle\bbtwo\text{-}\mathit{val}\rangle}
\newcommand {\world} {w}
\newcommand {\var} {x}
\newcommand {\emptyC} {\bullet}
\newcommand {\context} {\Gamma}
\newcommand {\inte} {i}
\newcommand {\bool} {b}
\newcommand {\resi} {\langle\mathit{res}\rangle}
\newcommand {\gbar} {~~|~~}

% nodes
\newcommand {\yhat} {\ensuremath{\mathtt{\hat y}}}
\newcommand {\curr} {\nabla}
\newcommand {\fut} {\bigcirc}


\newcommand {\pause} {{\tt hold}}
\renewcommand {\next} {{\tt next}}
\newcommand {\prev} {{\tt prev}}
\newcommand {\pure} {{\tt gr}}

\newenvironment{abstrsyn}
{ 
	\newcommand {\mytt} [1] {\texttt{##1}}
	\renewcommand {\pause} [1] {\mytt{hold(}##1\mytt{)}}
	\renewcommand {\next} [1] {\mytt{next(}##1\mytt{)}}
	\renewcommand {\prev} [1] {\mytt{prev(}##1\mytt{)}}
	\renewcommand {\pure} [1] {\mytt{gr(}##1\mytt{)}}
	\newcommand {\inl} [1] {\mytt{inl}\mytt{(}{##1}\mytt{)}}
	\newcommand {\inr} [1] {\mytt{inr}\mytt{(}{##1}\mytt{)}}
	\newcommand {\tup} [1] {\mytt{<} ##1 \mytt{>}}
	\newcommand {\pio} [1] {\mytt{pi1(}##1\mytt{)}}
	\newcommand {\pit} [1] {\mytt{pi2(}##1\mytt{)}}
	\newcommand {\roll} [1] {\mytt{roll}\mytt{(}##1\mytt{)}}
	\newcommand {\unroll} [1] {\mytt{unroll(}##1\mytt{)}}
	\newcommand {\lifttag} [1] {\mytt{lifttag(}##1\mytt{)}}
	\newcommand {\push} {\mytt{unroll}}
	\newcommand {\letp} [3] {\mytt{letg(}{##2}\mytt{;}{##1}.{##3}\mytt{)}}

	\newcommand {\letin} [3] {\mytt{let(}{##2}\mytt{;}{##1}.{##3}\mytt{)}}
	\newcommand {\talllet} [3] {\begin{array}{@{}l@{}} \mytt{let(}{##2}\mytt{;}{##1}\mytt{.}\\{##3}\end{array}}

	\newcommand {\caseof} [3] {\mytt{case(}{##1}\mytt{;}{##2}\mytt{;}{##3}\mytt{)}}
	\newcommand {\tallcase} [3] {\begin{array}{@{}l@{}} \mytt{case(}{##1}\mytt{;}\\~{##2}\mytt{;}\\~{##3}\mytt{)} \end{array}}
	\newcommand {\mediumcase} [3] {\begin{array}{@{}l@{}} \mytt{case(}{##1}\mytt{;}\\~{##2}\mytt{;}{##3}\mytt{)} \end{array}}

	\newcommand {\rtab}  [2] {\llbracket{##1}\rrbracket{##2}}
	\newcommand {\caseP} [3] {\mytt{caseg(}{##1}\mytt{;}{##2}\mytt{;}{##3}\mytt{)}}
	\newcommand {\pipeM} [3] {\mytt{\{}{##1}\mytt{|}{##2}\mytt{.}{##3}\mytt{\}}}
	\newcommand {\pipeS} [3] {\mytt{\{}{##1}\mytt{|}{##2}\mytt{.}{##3}\mytt{\}}}
	\newcommand {\mval}  [2] {\mytt{\{}{##1}\mytt{;}{##2}\mytt{\}}}

	\newcommand {\app} [2] {\mytt{app}\mytt{(}{##1}\mytt{;}{##2}\mytt{)}}
	\newcommand {\lam} [2] {\mytt{fn}\mytt{(}{##1}\mytt{.}{##2}\mytt{)}}
	\newcommand {\fix} [3] {\mytt{fix}\mytt{(}{##1}\mytt{.}{##2}\mytt{.}{##3}\mytt{)}}
	\newcommand {\valprod} [2] {({##1},{##2})}
	\newcommand {\projbind} [1] {\letin{l_{##1}}{\pi_{##1} l}{r_{##1}}}

	\newcommand {\mualphatau} {\mu \alpha.\tau}

	\newcommand {\scriptC} {\mathcal C}
	\newcommand {\scriptCapp} [1] {\scriptC\mytt{(}##1\mytt{)}}
}
{ }


\newcommand {\myatsign} {\hspace{-0.15em}\text{@}\hspace{-0.2em}}

%judgments
\newcommand {\coltwo} [2] {{#1}:{#2}~\myatsign~\bbtwo}
\newcommand {\colmix} [2] {{#1}:{#2}~\myatsign~\bbonem}
\newcommand {\colpure} [2] {{#1}:{#2}~\myatsign~\bbonep}
\newcommand {\colwor} [2] {{#1}:{#2}~\myatsign~w}
\newcommand {\typesone} [3] [\Gamma] { {#1} \vdash \colmix{#2}{#3}}
\newcommand {\typespure} [3] [\Gamma] { {#1} \vdash \colpure{#2}{#3}}
\newcommand {\typestwo} [3] [\Gamma] { {#1} \vdash \coltwo{#2}{#3}}
\newcommand {\typeswor} [3] [] { \Gamma{#1} \vdash \colwor{#2}{#3}}
\newcommand {\types} [3] [\Gamma] {{#1} \vdash {#2}:{#3}}
\newcommand {\istypemix} {\ \mathsf{type}~\myatsign~\bbonem}
\newcommand {\istypetwo} {\ \mathsf{type}~\myatsign~\bbtwo}
\newcommand {\istypepure} {\ \mathsf{type}~\myatsign~\bbonep}
\newcommand {\istypewor} {\ \mathsf{type}~\myatsign~w}

\newcommand {\typeslangTwo} [3] [\Gamma] {{#1} \vdash_\bbtwo {#2}:{#3}}

\newcommand {\reify} [3] {[{#1};{#2}]\searrow{#3}}
\newcommand {\erasone} {\Downarrow^e_\bbone}
\newcommand {\erastwo} {\Downarrow^e_\bbtwo}
\newcommand {\isvalone} [1] {\Gamma \vdash {#1}~{\rm val}~\myatsign~\bbone}
\newcommand {\isvaltwo} {~{\rm val}~\myatsign~\bbtwo}
\newcommand {\isvalwor} {~{\rm val}~\myatsign~w}
\newcommand {\isseppval} {~{\rm pval^S}}
%\newcommand {\reducexpl} [4] {{#3} \Downarrow_{#1}^{#2} {#4}}
%\newcommand {\redone} [2] {{#1} \Downarrow_\bbone^L {#2}}
%\newcommand {\redtwo} [2] {{#1} \Downarrow_\bbtwo^\bbtwo {#2}}
%\newcommand {\reducewor} [2] {{#1} \Downarrow_w^L {#2}}
\newcommand {\specwor} [2] [] {{#2} \downarrow_{#1}}
%\newcommand {\diaone} [3] [\Gamma] {{#1} \vdash {#2} \Downarrow_\bbone [{#3}]}
%\newcommand {\diatwo} [3] [\Gamma] {{#1} \vdash {#2} \Downarrow_\bbtwo {#3}}
\newcommand {\diaone} [4] [\Gamma] {{#2} \Downarrow_{\bbonem} \rtab{#3}{#4}}
\newcommand {\diatwo} [3] [\Gamma] {{#2} \Downarrow_\bbtwo {#3}}
\newcommand {\sepredonesym} {{\Downarrow_\bbone^S}}
\newcommand {\sepredtwosym} {{\Downarrow_\bbtwo^S}}
\newcommand {\sepredone} [2] {{#1} \sepredonesym {#2}}
\newcommand {\sepredtwo} [2] {{#1} \sepredtwosym {#2}}
\newcommand {\reduce} [2] {{#1} \Downarrow {#2}}
\newcommand {\reifysym} {\overset{R}\Rightarrow}
\newcommand {\redsym} {{\Downarrow}}
\newcommand {\redonesym} {{\Downarrow_{\bbonem}}}
\newcommand {\redtwosym} {{\Downarrow_\bbtwo}}
\newcommand {\tworedsym} {{\downarrow_\bbtwo}}

%\newcommand {\reduceonesub} [1] [] {\reduceone{e_{#1}}{v_{#1}}}
%\newcommand {\reducetwosub} [1] [] {\reducetwo{e_{#1}}{v_{#1}}}
\newcommand {\reduceworsub} [1] [] {\reducewor{e_{#1}}{v_{#1}}}
\newcommand {\specworsub} [1] [] {\specwor{e_{#1}}}
\newcommand {\diaonesub} [1] [] {\diaone{e_{#1}}{\xi_{#1}}{v_{#1}}}
\newcommand {\diatwosub} [1] [] {\diatwo{e_{#1}}{q_{#1}}}

\newcommand{\stepsym}[1]{\overset{#1}\hookrightarrow}
\newcommand{\stepmix}[2]{{#1}\stepsym \bbonem{#2}}
\newcommand{\steppure}[2]{{#1}\stepsym \bbonep{#2}}
\newcommand{\steptwo}[2]{{#1}\stepsym \bbtwo {#2}}
\newcommand{\steptwosub}[1]{e_{#1}\stepsym \bbtwo e'_{#1}}
\newcommand{\stepwor}[2]{{#1}\stepsym w {#2}}
\newcommand{\stepworsub}[1]{e_{#1}\stepsym w e'_{#1}}
\newcommand{\done}[1][w]{\ \donesym^{#1}}
\newcommand{\lift}[2]{{#1}\nearrow \llbracket q/y \rrbracket{#2}}
\newcommand{\liftsub}[1]{e_{#1}\nearrow \llbracket q/y \rrbracket e'_{#1}}

\newcommand {\masko} [1] {|#1|_{\bbone}}
\newcommand {\maskt} [1] {|#1|_{\bbtwo}}

\newcommand {\vsplito} {\curvearrowbotright}
\newcommand {\vsplitt} {\overset{\bbtwo}\curvearrowbotright}
\newcommand {\tsplito} {\overset{\bbone}\curvearrowright}
\newcommand {\tsplits} {\overset{\bbtwo}\curvearrowright}
\newcommand {\csplit} {\curvearrowright}
\newcommand {\ssplit} {\curvearrowbotright}

\newcommand {\splitonesym} {\overset{\bbonem}\rightsquigarrow}
\newcommand {\splittwosym} {\overset{\bbtwo}\rightsquigarrow}
\newcommand {\splitone} 	[5] [\Gamma] {{#2} \splitonesym \mytt{\{}{#4}\mytt{|}{#5}\mytt{\}}}
\newcommand {\splitonetall} [5] [\Gamma] {{#2} \splitonesym \left\{\begin{array}{@{}l@{}}{#4}\\\mytt{|}{#5}\end{array}\right\}}
\newcommand {\splittwo} 	[6] [\Gamma] {{#2} \splittwosym \pipeS{#4}{#5}{#6} }
\newcommand {\splittwotall} [6] [\Gamma] {{#2} \splittwosym \left\{\begin{array}{@{}l@{}}{#4}\\\mytt{|}{#5}.{#6}\end{array}\right\}}
\newcommand {\splittwotaller}[6][\Gamma] {{#2} \splittwosym \left\{\begin{array}{@{}l@{}}{#4}\\\mytt{|}{#5}.\\\hspace{1em}{#6}\end{array}\right\}}
\newcommand {\splitonesub}	[3] [\Gamma] {\splitone [{#1}] {e_{#2}}{#3}{c_{#2}}{l_{#2}.r_{#2}}}
\newcommand {\splittwosub}	[3] [\Gamma] {\splittwo [{#1}] {e_{#2}}{#3}{p_{#2}}{l_{#2}}{r_{#2}}}

%misc
\newcommand{\dom}[1]{\mathsf{dom}(#1)}
\newcommand {\gcomp} [2] {\xi_{#1}, \xi_{#2}}
\newcommand {\daviesz} {\overset 0 \hookrightarrow}
\newcommand {\davieso} {\overset 1 \hookrightarrow}
\newcommand {\ttrpar} {\texttt{)}}
\newcommand {\ttlpar} {\texttt{(}}
\newcommand {\ttsemi} {\texttt{;}}

%default stuff
\newcommand \ty \typesone
\newcommand \red \reduce
\newcommand \sub \reduceonesub
\newcommand \spl \splitone
\newcommand \col \colone

%inference extension
\newcommand {\infertypeswor}  [3] 
		[NONAME]{
			\renewcommand \ty \typeswor
			\renewcommand \col \colwor
			\infer %[\mathrm{#1}] 
      {#2}{#3}}
\newcommand {\inferreducewor} [3] 
		[NONAME]{
			\renewcommand \red \reducewor
			\renewcommand \sub \reduceworsub
			\infer [\mathrm{#1\Downarrow}] 
      {#2}{#3}}
\newcommand {\inferreducespc} [3] 
		[NONAME]{
			\renewcommand \red \specwor
			\renewcommand \sub \specworsub
			\infer [\mathrm{#1\downarrow}] {#2}{#3}}
\newcommand {\inferdiaone} [3] 
		[NONAME]{
			\renewcommand \red \diaone
			\renewcommand \sub \diaonesub
			\infer %[\mathrm{#1\Downarrow_\bbone}] 
      {#2}{#3}}
\newcommand {\inferdiaspc} [3] 
		[NONAME]{
			\renewcommand \red \diatwo
			\renewcommand \sub \diatwosub
			\infer %[\mathrm{#1\Downarrow_\bbtwo}] 
      {#2}{#3}}
\newcommand {\infersplitone}  [3] 
		[NONAME]{
			\renewcommand \spl \splitone
			\renewcommand \sub \splitonesub
			\renewcommand \col \colone
			\infer %[\mathrm{#1 \overset{\bbone}\rightsquigarrow}]
      {#2}{#3}}
\newcommand {\infersplittwo}  [3] 
		[NONAME]{
			\renewcommand \spl \splittwo
			\renewcommand \sub \splittwosub
			\renewcommand \col \coltwo
			\infer %[\mathrm{#1 \overset{\bbtwo}\rightsquigarrow}]
      {#2}{#3}}
\newcommand {\inferstepw} [2]
		{
			\renewcommand \red {\stepwor}
			\renewcommand \sub {\stepworsub}
			\infer {#1}{#2}
		}
\newcommand {\infersteptwo} [2]
		{
			\renewcommand \red {\steptwo}
			\renewcommand \sub {\steptwosub}
			\infer {#1}{#2}
		}
\newcommand {\infersteppure} [2]
		{
			\renewcommand \red {\steppure}
			\renewcommand \sub {\steppuresub}
			\infer {#1}{#2}
		}
\newcommand {\inferstepone} [2]
		{
			\renewcommand \red {\stepwor}
			\renewcommand \sub {\stepworsub}
			\infer {#1}{#2 & w \in \{\bbonem,\bbonep\}}
		}
\newcommand {\inferlift} [2]
		{
			\renewcommand \red {\lift}
			\renewcommand \sub {\liftsub}
			\infer {#1}{#2}
		}


\begin{document}
\maketitle

\begin{abstrsyn}

\newcommand {\colfunsym} {:>}
\newcommand {\colfun} [3] {{#1}\colfunsym{#2}~\myatsign~{#3}}

\newcommand {\ibksym} {\rotatebox[origin=c]{270}{$\therefore$}}
\newcommand {\btsym} {\rotatebox[origin=c]{270}{$\triangle$}}
\newcommand {\ibtsym} {\mathrel{\ooalign{$\btsym$\cr \hidewidth\hbox{$\bullet\mkern3mu$}\cr}}}
\newcommand {\ktype} {\rm Type}
\newcommand {\ibk} [2] {{#1}~\ibksym~{#2}}
\newcommand {\btj} [4] {{#1}:{#2}~\myatsign~{#3}~\btsym~{#4}}
\newcommand {\ibtj} [3] {{#1}\colfunsym{#2}~\myatsign~\bbone~\ibtsym~{#3}}
\newcommand {\btyo} [4] [\Gamma] {{#1}\vdash \btj{#2}{#3}\bbone{#4}}
\newcommand {\btyt} [4] [\Gamma] {{#1}\vdash \btj{#2}{#3}\bbtwo{#4}}
\newcommand {\btyw} [4] [\Gamma] {{#1}\vdash \btj{#2}{#3}w{#4}}
\newcommand {\btysub} [2] [] {\bty{e_{#1}:{#2}}{\tau_{#1}}{\sigma_{#1}}}
\newcommand {\ibtyo} [4] [\Gamma] {{#1}\vdash \ibtj{#2}{#3}{#4}}
\newcommand {\ibtyt} [4] [\Gamma] {{#1}\vdash \ibtj{#2}{#3}\bbtwo{#4}}

\section {Intro}

The purpose of this document is to explain an approach to typing the outputs produced by our splitting algorithm.
As we will see, this will become somewhat convoluted, eventually requiring a dependent type system.
In general, the problem of typing the output has two subparts:
\begin{enumerate}
\item \textbf{Typing the boundary.}  The most interesting and important contribution of this document is characterizing the 
\emph{boundary type}, which describes the communication between the two stages.  
Figuring out the boundary type is a self-contained problem, 
in that the boundary type of a term depends only on the boundary types of its subterms.

\item \textbf{Typing the rest.}  With a good characterization of the boundary type,
we can then consider the broader question of assigning types to the outputs of splitting.
\end{enumerate}

For the most part, all of the complication lies in typing the boundary.
Relative to that, the second part is straight-forward 
(with the exception of mixed-stage sums, which add some details that are independent of the boundary type).

Most of the complication in boundary typing arises from two sources: divergence and higher-order functions,
with the combination of these features introducing even further complexities.
Accordingly, this document proceeds by first considering a fragment of the language with
divergence but only first-order recursive functions (Section \ref{sec:firstorder}),
before considering a language with order functions (Section \ref{sec:higherorder}),
and the finally putting it all together.

\section{Various Monostage Formalisms}

In this section, we'll cover a few background formalisms in the context of a monostage language.
They will be extended to a staged language in the next section.
This section is primarily intended to be a reference.

\subsection{First-Order Recursive Functions}

We implement first order recursive functions by introducing a new typing judgement \mbox{$f \colfunsym A \to B$},
which says that $f$ is a function from $A$ to $B$.  
We use a different typing judgement here in order to keep functions and other values distinct.
That is, the $\to$ connective only ever appears to the right of ``$:>$'', and never to the right of a ``$:$".
We also allow let statements to work for functions.
Note how the only way to derive a function type is to directly declare a function, or to assume it as a variable.
\begin{framed}
\noindent\textbf{First-Order Recursive Functions:}
\begin{mathpar}
\infertypeswor [\to I]		{\Gamma \vdash {\fix fx{e_1}} \colfunsym {A \to B}}	{\types [\Gamma, f \colfunsym {A \to B},\col x A] {e_1} B} 							\and
\infertypeswor [\to E]		{\types {\app {e_1}{e_2}} {B}}							{\Gamma \vdash {e_1} \colfunsym {A \to B} & \ty {e_2} A} 					\and
\infertypeswor [\fut E]		{\types {\letin f {e_1} {e_2}} B}				{\Gamma \vdash {e_1} \colfunsym {A \to B} 
																			&\types [\Gamma, f \colfunsym {A \to B}] {e_2} B } 					\and
\end{mathpar}
\end{framed}

\subsection{Values and Computations}

It turns out that, for the purposes of splitting, it will be handy to differentiate terms into two classes: 
\emph{values} which are fully reduced, and \emph{computations} which may have some reduction remaining.
Importantly, we want to \emph{explicitly note} when subterms of a computation are actually values,
and so we introduce a new piece of syntax to mark just that inclusion: the underline constructor, written $\exv v$
where $v$ is a term (more precisely, a value).  
The type of the underline is just the type of it's contents, which avoids interaction with the type system.

Because staging and the value/computation distinction are both related to reduction, they will interact non-trivially.
For this reason, we'll first devise a value system for a monostage language.

Prior work has observed the difference between values and computations
via a syntactic distinction (i.e. saying there are two kinds of terms, value-terms and computation-terms).
We instead observe the difference judgementally: there's only one kind of term, 
but there are two judgements ($\valsym$ and $\compsym$) which classify terms as one or the other.

For instance, the aforementioned underline constructor always yields a computation,
but its contents are a value.

\begin{framed}
\noindent\textbf{Underlines:}
\begin{mathpar}
\infer {\Gamma \vdash \exv v \comp} {\Gamma \vdash v \val} \and
\infer {\Gamma \vdash \exv v : A} {\Gamma \vdash v : A} 
\end{mathpar}
\end{framed}

Int, bool, and unit base terms are always values.  
Intro forms, (tuples, injections, and rolls) are values so long as all their subterms are.

\begin{framed}
\noindent\textbf{Standard Values:}
\begin{mathpar}
\infer {\Gamma \vdash \tup{} \val}						{\cdot} 											\and
\infer {\Gamma \vdash i \val}							{\cdot}												\and
\infer {\Gamma \vdash b \val}							{\cdot}												\and
\infer {\Gamma \vdash \tup{e_1,e_2} \val}				{\Gamma \vdash e_1 \val & \Gamma \vdash e_2 \val} 	\and
\infer {\Gamma \vdash \inl e \val}						{\Gamma \vdash e \val} 								\and
\infer {\Gamma \vdash \inr e \val}						{\Gamma \vdash e \val} 								\and
\infer {\Gamma \vdash \roll e \val}						{\Gamma \vdash e \val} 										
\end{mathpar}
\end{framed}

Inversely, intro forms are also computations whenever all of the subterms are.
The elim forms, {\tt let}s, and {\tt if}s are always computations.
Note that because we're studying a call-by-value language, all variables are introduced as values.

\begin{framed}
\noindent\textbf{Standard Computations:}
\begin{mathpar}
\infer {\Gamma \vdash \tup{e_1,e_2} \comp}				{\Gamma \vdash e_1 \comp & \Gamma \vdash e_2 \comp} 	\and
\infer {\Gamma \vdash \inl e \comp}						{\Gamma \vdash e \comp} 								\and
\infer {\Gamma \vdash \inr e \comp}						{\Gamma \vdash e \comp} 								\and
\infer {\Gamma \vdash \roll e \comp}					{\Gamma \vdash e \comp} 								\and
\infer {\Gamma \vdash \pio e \comp}						{\Gamma \vdash e \comp} 								\and
\infer {\Gamma \vdash \pit e \comp}						{\Gamma \vdash e \comp} 								\and
\infer {\Gamma \vdash \unroll e \comp}					{\Gamma \vdash e \comp} 								\and
\infer {\Gamma \vdash \ifthen{e_1}{e_2}{e_3} \comp}		{\Gamma \vdash e_1 \comp 
														&\Gamma \vdash e_2 \comp 
														&\Gamma \vdash e_3 \comp}	\and
\infer {\Gamma \vdash \letin x {e_1} {e_2} \comp}		{\Gamma \vdash e_1 \comp 
														& \Gamma, x \val \vdash e_2 \comp } 				
\end{mathpar}
\end{framed}

Higher-order functions are the only constructs where value terms have computation subterms.

\begin{framed}
\noindent\textbf{Higher-Order Functions:}
\begin{mathpar}			
\infer {\Gamma \vdash \lam x e \val}		{\Gamma, x \val \vdash e \comp} \and
\infer {\Gamma \vdash \app{e_1}{e_2} \comp}	{\Gamma \vdash e_1 \comp & \Gamma \vdash e_2 \comp}
\end{mathpar}
\end{framed}

A term with a first-order function type, $e \colfunsym A \to B$, can only be a lambda itself or a variable declaration.
Ergo, they are always values and never computations.
For this reason, first-order functions don't really participate in the value/computation system.

\begin{framed}
\noindent\textbf{First-Order Recursive Functions:}
\begin{mathpar}
\infer {\Gamma \vdash \app{e_1}{e_2} \comp}	{\Gamma \vdash e_2 \comp}
\end{mathpar}
\end{framed}


\section{Input Language Fragments}

This section defines a few input language fragments for later reference.
Those fragments include ``core'' features (base types, products, $\fut$ types, and lets, which present no difficulty for output typing), 
if expressions, first-order recursive functions, higher-order functions, and finally sum/recursive types.
The two function fragments will be used exclusively.

\begin{framed}
\noindent\textbf{Core Feature Typing:}
\begin{mathpar}
\infer {\Gamma \vdash \exv e : A} {\Gamma \vdash e : A} 																									\and
\infertypeswor [\rmunit] 	{\ty {\tup{}}\rmunit}								{\cdot} 																	\and
\infertypeswor [int]		{\ty {i} \rmint}									{\cdot}																		\and
\infertypeswor [hold]		{\typestwo {\pause e} \rmint}						{\typesone e \rmint}				 										\and
\infertypeswor [int]		{\ty {b} \rmbool}									{\cdot}																		\and
\infertypeswor [hold]		{\typestwo {\pause e} \rmbool}						{\typesone e \rmbool}				 										\and
\infertypeswor [\times I]	{\ty {\tup{e_1,e_2}}{A\times B}}					{\ty {e_1} A & \ty {e_2} B} 												\and
\infertypeswor [\times E_1]	{\ty {\pio e} A}									{\ty e {A\times B}} 														\and
\infertypeswor [\times E_2]	{\ty {\pit e} B}									{\ty e {A\times B}} 														\and
\infertypeswor [\fut I]		{\typesone {\next e}{\fut A}}						{\typestwo e A} 															\and
\infertypeswor [\fut E]		{\typestwo {\prev e} A}								{\typesone e {\fut A}} 														\and
\infertypeswor [\fut E]		{\ty {\letin x {e_1} {e_2}} B}						{\ty {e_1} A & \ty [,\col x A] {e_2} B } 									\and
\end{mathpar}
\end{framed}

\begin{framed}
\noindent\textbf{Core Feature Values/Computations:}
\begin{mathpar}
\infer {\Gamma \vdash \exv e \compo} 					{\Gamma \vdash e \valo} 								\and
\infer {\Gamma \vdash \tup{} \valo}						{\cdot} 												\and
\infer {\Gamma \vdash i \valo}							{\cdot}													\and
\infer {\Gamma \vdash b \valo}							{\cdot}													\and
\infer {\Gamma \vdash \tup{e_1,e_2} \valo}				{\Gamma \vdash e_1 \valo & \Gamma \vdash e_2 \valo} 	\and
\infer {\Gamma \vdash \tup{e_1,e_2} \compo}				{\Gamma \vdash e_1 \compo & \Gamma \vdash e_2 \compo} 	\and
\infer {\Gamma \vdash \pio e \compo}					{\Gamma \vdash e \compo} 								\and
\infer {\Gamma \vdash \pit e \compo}					{\Gamma \vdash e \compo} 								\and
\end{mathpar}
\end{framed}

\begin{framed}
\noindent\textbf{If Expressions:}
\begin{mathpar}
\infertypeswor [+ E]		{\ty {\ifthen{e_1}{e_2}{e_3}} A}	{\ty {e_1}\rmbool & \ty{e_2} A & \ty{e_3} A} \and
\infer {\Gamma \vdash \ifthen{e_1}{e_2}{e_3} \compo}			{\Gamma \vdash e_1 \compo 
																&\Gamma \vdash e_2 \compo 
																&\Gamma \vdash e_3 \compo}
\end{mathpar}
\end{framed}

We implement first order recursive functions by introducing a new typing judgement \mbox{$\colfun f {A\to B} w$},
which says that $f$ is a function from $A$ to $B$ at world $w$.  
We use a different typing judgement here in order to keep functions and other values distinct.
That is, the $\to$ connective only ever appears to the right of ``$:>$'', and never to the right of a ``$:$".
We also allow let statements to work for second stage functions 
(for reasons explained later, first stage function lets can be safely eliminated \emph{a priori} by substitution).
Note how the only way to derive a function type is to directly declare a function, or to assume it.
\begin{framed}
\noindent\textbf{First-Order Recursive Functions:}
\begin{mathpar}
\infertypeswor [\to I]		{\Gamma \vdash \colfun {\fix fx{e_1}} {A \to B} w}	{\ty [,\colfun f {A \to B} w,\col x A] {e_1} B} 							\and
%\infertypeswor [fix]		{\ty {\fix fxe} {A \to B}}							{\ty [,\col f {A \to B},\col x A] e B} 										\and
\infertypeswor [\to E]		{\ty {\app {e_1}{e_2}} {B}}							{\Gamma \vdash \colfun {e_1} {A \to B} w & \ty {e_2} A} 					\and
\infertypeswor [\fut E]		{\typestwo {\letin f {e_1} {e_2}} B}				{\Gamma \vdash \colfun {e_1} {A \to B} \bbtwo 
																				&\typestwo [\Gamma,\colfun f {A \to B} \bbtwo] {e_2} B } 					\and
\end{mathpar}
\end{framed}

\begin{framed}
\noindent\textbf{Higher-Order Functions:}
\begin{mathpar}
\infertypeswor [\to I]		{\ty {\lam {x}{e}} {A \to B}}						{\ty [,\col x A] e B} 														\and
%\infertypeswor [fix]		{\ty {\fix fxe} {A \to B}}							{\ty [,\col f {A \to B},\col x A] e B} 										\and
\infertypeswor [\to E]		{\ty {\app {e_1}{e_2}} {B}}							{\ty {e_1} {A \to B} & \ty {e_2} A} 										\and
\end{mathpar}
\end{framed}

\begin{framed}
\noindent\textbf{Sums and Recursion:}
\begin{mathpar}
\infertypeswor [+ I_1]		{\ty {\inl e} {A + B}}								{\ty e A} 																	\and
\infertypeswor [+ I_2]		{\ty {\inr e} {A + B}}								{\ty e B} 																	\and
\infertypeswor [+ E]		{\ty {\caseof{e_1}{x_2.e_2}{x_3.e_3}} C}			{\ty {e_1}{A+B} & \ty[,\col {x_2} A]{e_2} C & \ty[,\col {x_3} B]{e_3} C}	\and
\infertypeswor [\mu I]		{\ty {\roll e} {\mu \alpha.\tau}}					{\ty e {[\mualphatau / \alpha]\tau}} 										\and
\infertypeswor [\mu E]		{\ty {\unroll e} {[\mualphatau / \alpha]\tau}}		{\ty e \mualphatau} 														\and
\infer {\Gamma \vdash \inl e \valo}						{\Gamma \vdash e \valo} 								\and
\infer {\Gamma \vdash \inl e \compo}					{\Gamma \vdash e \compo} 								\and
\infer {\Gamma \vdash \inr e \valo}						{\Gamma \vdash e \valo} 								\and
\infer {\Gamma \vdash \inr e \compo}					{\Gamma \vdash e \compo} 								\and
\infer {\Gamma \vdash \roll e \valo}					{\Gamma \vdash e \valo} 								\and
\infer {\Gamma \vdash \roll e \compo}					{\Gamma \vdash e \compo} 								\and
\infer {\Gamma \vdash \unroll e \compo}					{\Gamma \vdash e \compo} 								\and
\infer {\Gamma \vdash \letin x {e_1} {e_2} \compo}		{\Gamma \vdash e_1 \compo 
														& \Gamma, x \valo \vdash e_2 \compo } 	
\end{mathpar}
\end{framed}


\section{First Order Recursive Functions}
\label{sec:firstorder}

In this section, we develop a boundary-type system for first-order recursive functions
with {\tt if}s, but no sum types.

\subsection {Core Features and {\tt if}s}

For the most part, we'll do all of our work with a single new judgment \mbox{$\btj e A w \tau$}, 
which indicates that the term $e$ (where $\colwor e A$ and $e \comp$) has the boundary type $\tau$.
The goal is for the splitting rule to agree with the boundary type given to us by this judgment.
That is, if $\btj e A \bbtwo \tau$ and $\splittwo{e}{A}{p}{l}{r}$, 
then $p:\tau$ and $l:\tau \vdash r:A$ (and a similar relationship at stage \bbone, covered later).

For core features at either stage, the boundary type of any term is 
just the product of the boundary types of its computation subterms.
The only exception to this pattern is the $\pause$ rule,
which also multiplies an integer into the boundary type.

\begin{framed}
\noindent\textbf{Core Feature Boundary Types:}
\begin{mathpar}
\infer {\btyw {\exv v} A\rmunit} {\cdot} \and
\infer {\btyw {\tup{e_1,e_2}} {A \times B} {\tau_1 \times \tau_2}} 
	{\btyw {e_1} A {\tau_1} 
	&\btyw {e_2} B {\tau_2}} \and
\infer {\btyw {\pio e} A \tau} 
	{\btyw {e} {A \times B} {\tau}} \and
\infer {\btyw {\pit e} B \tau} 
	{\btyw {e} {A \times B} {\tau}} \and
\infer {\btyo {\next e} {\fut A} \tau} 
	{\btyt {e} A {\tau}} \and
\infer {\btyt {\prev e} A \tau} 
	{\btyo {e} {\fut A} {\tau}} \and
\infer {\btyt {\pause e} \rmint {\tau \times \rmint}} 
	{\btyo e \rmint \tau} \and
\infer {\btyw {\letin x {e_1} {e_2}} {B} {\tau_1 \times \tau_2}} 
	{\btyw {e_1} A {\tau_1} 
	&\btyw {e_2} B {\tau_2}}
\end{mathpar}
\end{framed}

Note that in the {\tt let} rule, we needn't declare a boundary type for the $x$ variable,
since variables are values, and only computations have a boundary type.
Rules for \texttt{if} terms are similarly easy: 

\begin{framed}
\noindent\textbf{If Expression Boundary Types:}
\begin{mathpar}
\infer {\btyo {\ifthen {e_1}{e_2}{e_3}} {A} {\tau_1 \times (\tau_2 + \tau_3)}} 
	{\btyo {e_1} \rmbool {\tau_1} 
	&\btyo {e_2} A {\tau_2}
	&\btyo {e_3} A {\tau_3}} \and
\infer {\btyt {\ifthen {e_1}{e_2}{e_3}} {A} {\tau_1 \times \tau_2 \times \tau_3}} 
	{\btyt {e_1} \rmbool {\tau_1} 
	&\btyt {e_2} A {\tau_2}
	&\btyt {e_3} A {\tau_3}}
\end{mathpar}
\end{framed}

The only particular care we took here was to observe speculative behavior in the stage \bbtwo\ case,
which induces a product (whereas stage \bbone\ had a sum).

\subsection {Functions}

Consider the following example, which is open on the variable \mbox{$\colfun g {\rmint \to \rmint} \bbone$}.
\[
\app g {\app g 5}
\]
Clearly, the boundary type of this computation depends on the particular function $g$ is bound to.
Thus, in order for a boundary type system to be complete,
the variable $g$ must be associated with information about the boundary type that is produced where $g$ is applied.
This is called the \emph{internal boundary type} of $g$.

Our system works by associating every function derivation \mbox{$\colfun g {A \to B} \bbone$} with an internal boundary type $\sigma$
via the judgement $\ibtj g {A \to B} \sigma$.  
Functions delcared at world $\bbtwo$ do not need internal boundar types.

\begin{framed}
\noindent\textbf{First-Order Recursive Function Boundary Types:}
\begin{mathpar}
\infer {\btyo {\app {f}{e}} {A} {\tau \times \sigma}} 
	{\ibtyo {f} {A \to B} {\sigma} 
	&\btyo {e} A {\tau}} \and
% \infer {\btyt {\fix fxe} {A} {\tau}} 
% 	{\btyt {e} A {\tau}} \and
% \infer {\btyt {\app {f}{e}} {A} {\tau}} 
% 	{\btyt {e} A {\tau}} \and
% \infer {\btyw {\letin x {e_1} {e_2}} {B} {\tau_1 \times \tau_2}} 
% 	{\btyw {e_1} A {\tau_1} 
% 	&\btyw {e_2} B {\tau_2}}
\end{mathpar}
\end{framed}



% \section{Adding Functions and Fixed Points}

% To get a taste of why the problem is hard,
% we consider adding lots of features, namely \texttt{if}s, functions, and fixed points.
% The astute reader will notice that these are enough to make the language Turing-complete.
% Before getting into the rules, consider the following example, 
% where $E$ is some purely stage \bbone\ expression:

% \[
% \left(
% \tallif {E}
% 	{\lam x {\next{\pause{x}+1}}}
% 	{\lam x {\next{\pause{x}+\pause{x*x}}}}
% \right ) 5
% \]

% What do we expect the eventual boundary value for this whole piece of code to be?
% Well, it depends. 
% If $E$ evaluates to true, then we'd expect the boundary value to be an integer (that is, have type $\rmint$),
% but if $E$ evaluates to false, we'd expect the boundary value to be the tuple of two integers (that is, have type $\rmint \times \rmint$).
% Ergo, the only way to have a boundary type for this whole piece of code is to either,
% \begin{itemize}
% \item figure out what $E$ evaluates, or
% \item use a description for boundary types that mentions both $\rmint$ and $\rmint \times \rmint$, 
% without committing to one.
% \end{itemize}

% Of course, since our language is Turing complete, figuring out what $E$ evaluates to is not decidable,
% so we'll go with the second option.

% \section{Higher-Order Functions}
% \label{sec:higherorder}

% Lets now take a different stab at this, by adding functions to our straight-line code,
% but leaving \texttt{if}s out for now.
% Again, lets consider an example:
% \[
% L1 := \lam x {\next{5}}
% \]
% The boundary type of $L1$ is obviously unit, right?
% After all, lambdas are values, and values have unit boundary.
% So then what about this example:
% \[
% L2 := \lam x {\next{\pause{x}+1}}
% \]
% I'd like to think it's the same story: unit boundary.
% But what if we apply a value to these functions?
% Consider
% \[
% \app {\lam x {\next{5}}} 3
% \]
% versus
% \[
% \app {\lam x {\next{\pause{x}+1}}} 3
% \]
% The boundary type of the first expression should still be unit 
% (even after substitution it has no {\tt hold}s),
% but the bound type of the second expression should definitely be int.

% There's a problem here: $L1$ and $L2$ had the same boundary type before application, but different ones after.
% This suggests that if we want the boundary type to be a compositional property,
% then we'll need to keep track of more information than we do above.

% \subsection{Adding Boundary Annotations}

% Our first attempt at a solution will be to add more information to function types in our input language.
% This new information simply tells us what the boundary type will be yielded when the function is applied.

% [solution where we write the boundary type above the arrows]

% \subsection{Separating Internal and External Boundaries}

% There's something unsatisfying about this solution. 
% The type system of the input language was perfectly functional before all this mess about boundary types.

% [show how to separate the internal from external boundaries]

% \subsection{Higher order funcitons}

% [Take the system from the previous section and consider higher-order functions.]

% % We do this by making a distinction between the {\em internal} boundary of an expression,
% % and its {\em external} boundary.  
% % $L1$ and $L2$ have the same external boundary, namely unit,
% % but they have different internal boundaries.  
% % In particular, the interal boundary of $L1$ should be int (or isomorphic to it), 
% % whereas the internal boundary of $L2$ should be unit.

% % Consider a language with products and functions.  The internal boundary kind is given by:

% % \begin{mathpar}
% % \infer {\ibk {\rm int}{\rm Unit}} {\cdot} \and 
% % \infer {\ibk {\rm bool}{\rm Unit}} {\cdot} \and 
% % \infer {\ibk {A \times B}{\kappa_A \times \kappa_B}} {\ibk A {\kappa_A} & \ibk B {\kappa_B}} \and 
% % %\infer {\ibk {A \to B}{\kappa_A \to \ktype \times \kappa_B}} {\ibk A {\kappa_A} & \ibk B {\kappa_B}} 
% % \end{mathpar}


\end{abstrsyn}
\end{document}
