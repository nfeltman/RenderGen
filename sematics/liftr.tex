\documentclass[11pt]{article}
\usepackage{amsmath}
\usepackage{bbm}
\usepackage{stmaryrd}
\usepackage{proof}
\usepackage{mathpartir}
\usepackage{amssymb,amsthm}
\usepackage[margin=1in,nohead]{geometry}
\parindent 0pt
\parskip .75em

% bnf stuff
\newcommand {\myit} [1]{\operatorname{\it{#1}}}
\newcommand {\stage} {\langle\mathit{stage}\rangle}
\newcommand {\typeo} {\langle\bbone\text{-}\mathit{type}\rangle}
\newcommand {\expro} {\langle\bbone\text{-}\mathit{exp}\rangle}
\newcommand {\typet} {\langle\bbtwo\text{-}\mathit{type}\rangle}
\newcommand {\exprt} {\langle\bbtwo\text{-}\mathit{exp}\rangle}
\newcommand {\var} {\langle\mathit{var}\rangle}
\newcommand {\context} {\langle\mathit{cont}\rangle}
\newcommand {\valo} {\langle\bbone\text{-}\mathit{val}\rangle}
\newcommand {\valt} {\langle\bbtwo\text{-}\mathit{val}\rangle}
\newcommand {\resi} {\langle\mathit{res}\rangle}
\newcommand {\gbar} {~~|~~}

% nodes
\newcommand {\bbone} {\mathbbm 1}
\newcommand {\bbtwo} {\mathbbm 2}
\newcommand {\pause} {{\tt save}}
\newcommand {\next} {{\tt next}}
\newcommand {\prev} {{\tt prev}}
\newcommand {\fut} {\bigcirc}
\newcommand {\transfers} {\nearrow}
\newcommand {\gds} {{\Gamma \vdash^\sigma}}
\newcommand {\gdo} {{\Gamma \vdash^\bbone}}
\newcommand {\gdt} {{\Gamma \vdash^\bbtwo}}
\newcommand {\letin} [3] {{\tt let}~{#1} = {#2}~{\tt in}~{#3}}
\newcommand {\caseof} [3] {{\tt case}~{#1} ~{\tt of}~{#2}~{\tt |}~{#3}}
\newcommand {\tallcase} [3] {\begin{array}{l} {\tt case}~{#1} ~{\tt of}\\~~{#2}\\{\tt |}~{#3} \end{array}}
\newcommand {\lam} [3] {\lambda{#1}\mathrm{:}{#2}.{#3}}
\newcommand {\valprod} [2] {{\bf (}{#1},{#2}{\bf )}}
\newcommand {\valleft} {{\bf \iota}_1}
\newcommand {\valright} {{\bf \iota}_2}
\newcommand {\vallam} [2] {{\bf \lambda}{#1}.{#2}}
\newcommand {\valres} {{\bf res}}

% transitions
\newcommand {\splito} {\overset{\bbone}\rightsquigarrow}
\newcommand {\splits} {\overset{\bbtwo}\rightsquigarrow}

% Equation spacing
%\newcommand {\inferenceSpacing}{\setlength{\jot}{1.8ex}}
%\newcommand {\normalSpacing}{\setlength{\jot}{1ex}}
\newcommand {\tab}{~~~~~~}

%language names
\newcommand {\lang} {$\lambda^{\bbone\bbtwo}$}
\newcommand {\wstage} {\textsc{Stag}}
\newcommand {\wostage} {\textsc{Doe}}

\title{\Large\textbf{Staging as a Mechanism for Algorithm Derivation}}
\author{Nicolas Feltman}
\begin{document}
\maketitle

\section{Introduction}

[Not done.]

\section{\lang~Definition}
\label{sec:def}

In this section I define \lang, a two-stage language.  Throughout this document, I will use $\bbone$ and $\bbtwo$ to refer to the two stages.  

\subsection{Syntax}

A grammer for the terms, types, and contexts of \lang\ is shown in Figure \ref{fig:gram}.  Although both stages of \lang\ contain products, sums, and functions, I have chosen to seperate the stages syntactically to emphasize that this need not be the case.

You'll notice that there are three seperate mechanisms by which one stage can refer to another: $\next$, $\prev$, and $\pause$.  Specifically, $\next$ allows a stage $\bbone$ term to reference computation that will occur in the future, at stage $\bbtwo$, whereas $\prev$ and $\pause$ allow a stage $\bbtwo$ computation to refer back to stage $\bbone$.  The precise meaning of these constructs will be explored in future sections.

\begin{figure}
\caption{\lang~Syntax}
\label{fig:gram}
\centering
\begin{tabular}{ll} 
$\begin{aligned}
\typeo &::= \text{unit} \\
&\gbar \typeo \times \typeo \\
&\gbar \typeo + \typeo \\
&\gbar \typeo \to \typeo \\
&\gbar \fut \typet 
\end{aligned} $  
& 
$\begin{aligned}
\typet &::= \text{unit} \\
&\gbar \typet \times \typeo \\
&\gbar \typet + \typet \\
&\gbar \typet \to \typet 
\end{aligned} $  
\\ 
$\begin{aligned}
\expro &::= \lam{\var}{\typeo}{\expro} \\
&\gbar \var \\
&\gbar \expro~\expro \\
&\gbar () \\
&\gbar (\expro, \expro) \\
&\gbar \pi_1~\expro \gbar \pi_2~\expro \\
&\gbar \iota_1~\expro \gbar \iota_2~\expro \\
&\gbar \begin{array}{l} {\tt case}~\expro ~{\tt of}\\~~~\var.\expro~{\tt `|}\text{'}~\var.\expro \end{array} \\
&\gbar \next~\exprt 
\end{aligned} $ 
& 
$\begin{aligned}
\exprt &::= \lam{\var}{\typet}{\exprt} \\
&\gbar \var \\
&\gbar \exprt~\exprt \\
&\gbar () \\
&\gbar (\exprt, \exprt) \\
&\gbar \pi_1~\exprt \gbar \pi_2~\exprt \\
&\gbar \iota_1~\exprt \gbar \iota_2~\exprt \\
&\gbar \begin{array}{l} {\tt case}~\exprt ~{\tt of}\\~~~\var.\exprt~{\tt `|}\text{'}~\var.\exprt \end{array} \\
&\gbar \pause~\expro \\
&\gbar \prev~\expro 
\end{aligned} $ 
\\ 
$\begin{aligned}
\valo &::=() \\
&\gbar \valprod {\valo} {\valo} \\
&\gbar \valleft~\valo \\
&\gbar \valright~\valo \\
&\gbar \vallam {\var} {\expro} \\
&\gbar \valres~\resi
\end{aligned} $  
& 
$\begin{aligned}
\valt &::=() \\
&\gbar \valprod {\valt} {\valt} \\
&\gbar \valleft~\valt \\
&\gbar \valright~\valt \\
&\gbar \vallam {\var} {\exprt} 
\end{aligned} $
\\
$\begin{aligned}
\context &::= \mathrm{empty} \\
&\gbar \context, \var : \typeo ^\bbone \\
&\gbar \context, \var : \typet ^\bbtwo
\end{aligned} $
\end{tabular}
\end{figure}

\subsection{Static Semantics}

[Not done.]

\begin{figure}
\caption{\lang~Static Semantics}
\label{fig:staging}
\begin{mathpar}
\infer [\mathrm{hyp}] {\gds x : A}{x : A^\sigma \in \Gamma} \and
\infer [\to\mathrm{I}] {\gds (\lam{x}{A}{e}) : A \to B}{\Gamma,x:A^\sigma \vdash e : B} \and
\infer [\to\mathrm{E}] {\gds e_1~e_2 : B}{\gds e_1 : A \to B & \gds e_2 : A} \and
\infer [\mathrm{unit}] {\gds () : \text{unit}}{\cdot} \and
\infer [\times\mathrm{I}] {\gds (e_1,e_2) : A\times B}{\gds e_1 : A & \gds e_2 : B} \and
\infer [\times\mathrm{E_1}] {\gds \pi_1~e : A}{\gds e : A\times B} \and
\infer [\times\mathrm{E_2}] {\gds \pi_2~e : B}{\gds e : A\times B} \and
%\infer {\gds \letin {x}{e_1}{e_2} : B}{\gds e_1 : A & \Gamma,x:A^\sigma \vdash B : e_2} \\
\infer [+\mathrm{I_1}] {\gds \iota_1~e : A+B}{\gds e : A} \and
\infer [+\mathrm{I_2}] {\gds \iota_2~e : A+B}{\gds e : B} \and
\infer [+\mathrm{E}] {\gds \caseof {e_1}{x_2.e_2}{x_3.e_3} : C}{\gds e_1 : A+B & \Gamma,x_2:A^\sigma \vdash^\sigma e_2 : C & \Gamma,x_3:B^\sigma \vdash^\sigma e_3 : C} \\
\infer [\fut\mathrm{I}] {\Gamma \vdash^\bbone \next~e : \fut A}{\Gamma \vdash^\bbtwo e : A} \and
\infer [\fut\mathrm{E}] {\Gamma \vdash^\bbtwo \prev~e : A}{\Gamma \vdash^\bbone e : \fut A} \and
\infer [\mathrm{save}] {\Gamma \vdash^\bbtwo \pause~e : A'}{\Gamma \vdash^\bbone e : A & A \transfers A' } \and
\infer [\mathrm{unit}\transfers] {\text{unit} \transfers \text{unit}}{\cdot } \and
\infer [\times\transfers] {A \times B \transfers A' \times B'}{A \transfers A' & B \transfers B'} \and
\infer [+\transfers] {A + B \transfers A' + B'}{A \transfers A' & B \transfers B'}
\end{mathpar}
\end{figure}

\subsection{Dynamic Semantics}
The evaluation is given in Figure \ref{fig:eval}.  Were it complete, it would contain a definition of values (which we predict to be unchanged by the staging system), and a big-step sematics relating terms to values.  The big-step evaluation should be indexed by the stage at which it completes.  That is, we have both $\Downarrow_\bbone$ and $\Downarrow_\bbtwo$.  Also, the bigstep semantics will reflect that speculation occurs down both branches of a case as part of  $\Downarrow_\bbone$ reduction.

\newcommand {\downone} {\Downarrow_\bbone} 
\newcommand {\downtwo} {\Downarrow_\bbtwo} 
\newcommand {\downsig} {\Downarrow_\sigma} 
\newcommand {\spec} {\downarrow} 
\begin{figure}
\caption{\lang~Dynamic Semantics}
\label{fig:eval}
\begin{mathpar}
\infer [\lambda \downsig] {\lam{x}{\tau}{e} \downsig \vallam{x}{e}}{\cdot} \and
\infer [\mathrm{app} \downsig] {e_1~e_2 \downsig [N/x]M}{e_1 \downsig \vallam{x}{M} & e_2 \downsig N} \\
\infer [\mathrm{unit} \downsig] {() \downsig ()}{\cdot} \and
\infer [(,) \downsig] { (e_1,e_2) \downsig \valprod{v_1}{v_2}}{e_1 \downsig v_1 & e_2 \downsig v_2} \\
\infer [\pi_1 \downsig] {\pi_1~e \downsig v_1 }{e \downsig \valprod{v_1}{v_2}} \and
\infer [\pi_2 \downsig] {\pi_2~e \downsig v_2 }{e \downsig \valprod{v_1}{v_2}} \and
\infer [\iota_1 \downsig] {\iota_1~e \downsig \valleft~v}{e \downsig v} \and
\infer [\iota_2 \downsig] {\iota_2~e \downsig \valright~v}{e \downsig v} \and
\infer [\mathrm{case}_1 \downone] {\caseof {e}{x_1.e_1}{x_2.e_2} \downone v'}{e \downone \valleft~v & [v/x_1]e_1 \downone v'} \and
\infer [\mathrm{case}_1 \downtwo] {\caseof {e}{x_1.e_1}{x_2.e_2} \downtwo v'}{e \downtwo \valleft~v & [v/x_1]e_1 \downtwo v' & e_2 \spec} \and
\infer [\mathrm{case}_2 \downone] {\caseof {e}{x_1.e_1}{x_2.e_2} \downone v'}{e \downone \valright~v & [v/x_2]e_2 \downone v'} \and
\infer [\mathrm{case}_2 \downtwo] {\caseof {e}{x_1.e_1}{x_2.e_2} \downtwo v'}{e \downtwo \valright~v & e_1 \spec & [v/x_2]e_2 \downtwo v'} \and
\infer {\next~e \downone \valres~v}{e \downtwo v} \and
\infer {\prev~e \downtwo v}{e \downone \valres~v} \and 
\infer {\pause~e \downtwo v}{e \downone v} \\
\infer [\mathrm{var} \spec]{x \spec }{\cdot} \and
\infer [\lambda \spec]{\lam{x}{\tau}{e} \spec }{e \spec} \and
\infer [\mathrm{app} \spec] {e_1~e_2 \spec}{e_1 \spec & e_2 \spec} \and
\infer [\mathrm{unit} \spec]{() \spec }{\cdot} \and
\infer [\mathrm{app} \spec] {(e_1,e_2) \spec}{e_1 \spec & e_2 \spec} \and
\infer [\pi_i \spec]{\pi_i~e \spec }{e \spec} \and
\infer [\iota_i \spec]{\iota_i~e \spec }{e \spec} \and
\infer [\mathrm{case} \spec] {\caseof {e}{x_1.e_1}{x_2.e_2} \spec}{e_1 \spec & e_2 \spec} \and
\infer {\prev~e \spec}{e \downone \valres~v} \and 
\infer {\pause~e \spec}{e \downone v}
\end{mathpar}
\end{figure}

\section{Stage Splitting}

The core operation of interest is the process of ``stage splitting," wherein a term is split into its constituent stage $\bbone$ and stage $\bbtwo$ parts, each expressed in a simpler language.  Specifically, we introduce two judgements.  The simpler judgement is 
\[\gdt e : A \splits [p,x.r]\] 
which can be read ``under the context $\Gamma$, $e$ (which has type $A$ at stage $\bbtwo$) stage-splits into a precomputation $p$, and a residual $r$ which is open on $x$''.  The idea is that $p$ contains the parts of $e$ that are stage $\bbone$, $r$ contains the parts of $e$ that are stage $\bbtwo$, and the reduced value of $p$ is bound to $x$ when evaluating $r$.  There's also the slightly more complicated version for stage 1:
\[\gdo e : A \splito [c,x.r]\]
which can be read as [\ldots]

\subsection {Goals}

There are a few theorems that should hold true of stage splitting.  Firstly, that good types in lead to good types out.  Explicitly:
[Took this out for now.]
%
%\begin{center}
%\begin{tabular}{lll}
%If $\gdo e:\tau $ &~~~~~~~~~~~~~~ & If $\gdt e:\tau$ \\
%then $\gdo e \splito e'$ && then $\gdt e \splits [p,x.r]$ \\
%and $\Gamma \vdash e':\tau$ && and $\Gamma_\bbone \vdash p:\tau'$\\
% && and $\Gamma_\bbtwo, x:\tau' \vdash r:\tau$
%\end{tabular}
%\end{center}
%Note that I'm implicitly abusing the identical structures of pretypes in \wstage~and types in \wostage.  Also, for any stage $\sigma$, we define $\Gamma_\sigma$ as,
%\begin{align}
%(\cdot)_\sigma &= \cdot \\
%(\Gamma,x:\tau^\sigma)_\sigma &= \Gamma_\sigma, \tau \\
%(\Gamma,x:\tau^{\sigma'})_\sigma &= \Gamma_\sigma &\text{where } \sigma \not= \sigma'
%\end{align}

\subsection {Splitting}
The full rules for splitting are shown in Figures \ref{fig:splitBasic}, \ref{fig:splitProduct}, \ref{fig:splitFunction}, and \ref{fig:splitSum}.

\begin{figure*}
\caption{Basic Splitting}
\label{fig:splitBasic}
\begin{mathpar}
\infer [\fut\mathrm{I}\splito] {\gdo \next~e : \fut A \splito [((),p),l.r]}{\gdt e : A\splits [p,l.r] } \and
\infer [\fut\mathrm{E}\splits] {\gdt \prev~e : A\splits [\pi_2~c,l.r] }{\gdo e : \fut A \splito [c,l.r]} \and
\infer [\mathrm{save}\splits] {\gdt \pause~e : A' \splits [e,l.l]}{\gdo e : A& A \transfers A'} 
\end{mathpar}
\end{figure*}


\begin{figure*}
\caption{Product Splitting}
\label{fig:splitProduct}
\begin{mathpar}
\infer [\mathrm{unit}\splito] 
	{\gdo () : \text{unit} \splito [((),()),\_.()]}{\cdot} \and
\infer [\mathrm{unit}\splits]
	 {\gdt () : \text{unit} \splits [(),\_.()]}{\cdot} \and
\infer [\times\mathrm{I}\splito] 
	{\gdo (e_1,e_2) : A\times B \splito [((\pi_1 c, \pi_1 c),(\pi_2 c_1, \pi_2 c_2)), l.(\letin{l_1}{\pi_1 l}{r_1},\letin{l_2}{\pi_2 l}{r_2})] }
	{ \gdo e_1 : A \splito [c_1,l_1.r_1] 
	&\gdo e_2 : B \splito [c_2,l_2.r_2]} \and
\infer [\times\mathrm{I}\splits] 
	{\gdt (e_1,e_2) : A\times B \splits [(p_1,p_2), l.(\letin{l_1}{\pi_1~l}{r_1},\letin{l_2}{\pi_2~l}{r_2})] }
	{ \gdt e_1 : A \splits [p_1,l_1.r_1] 
	&\gdt e_2 : B \splits [p_2,l_2.r_2]} \and
\infer [\times\mathrm{E_1}\splito] 
	{\gdo \pi_1~e : A \splito [(\pi_1(\pi_1c),\pi_2 c),l.\pi_1~r] }{\gdo e : A\times B \splito [c,l.r] } \and
\infer [\times\mathrm{E_1}\splits] 
	{\gdt \pi_1~e : A \splits [p,l.\pi_1~r] }{\gdt e : A\times B \splits [p,l.r] } \and
\infer [\times\mathrm{E_2}\splito] 
	{\gdo \pi_2~e : B \splito [(\pi_2(\pi_1c),\pi_2 c),l.\pi_2~r] }{\gdo e : A\times B \splito [c,l.r]} \and
\infer [\times\mathrm{E_2}\splits] 
	{\gdt \pi_2~e : B \splits [p,l.\pi_2~r] }{\gdt e : A\times B \splits [p,l.r]} 
\end{mathpar}
\end{figure*}

\begin{figure*}
\caption{Function Splitting}
\label{fig:splitFunction}
\begin{mathpar}
\infer [\mathrm{hyp}\splito] 
	{\gdo x : A\splito [(x,()),\_.x]}{x : A^\bbone \in \Gamma} \and
\infer [\mathrm{hyp}\splits] 
	{\gdt x : A\splits [(),\_.x]}{x : A^\bbtwo \in \Gamma} \and
\infer [\to\mathrm{I}\splito] 
	{\gdo (\lam{x}{A}{e}) : A \to B \splito \left[\begin{array}{l}(\lam{x}{|A|_\bbone}{c},()),\\ \_.(\lam{(x,l)}{|A|_\bbtwo\times \tau}{r})\end{array}\right]}
	{\Gamma,x:A^\bbone \vdash^\bbone e : B \splito [c,l.r]} \and
\infer [\to\mathrm{I}\splits] 
	{\gdt (\lam{x}{A}{e}) : A \to B \splits [p,l.\lam{x}{A}{r}]}
	{\Gamma,x:A^\bbtwo \vdash^\bbtwo e : B \splits [p,l.r]} \and
\infer [\to\mathrm{E}\splito] 
	{\gdo e_1~e_2 : B \splito 
	\left[ \begin{array}{l}\letin{y}{(\pi_1 c_1) (\pi_1 c2)}{(\pi_1 y,(\pi_2 c_1,\pi_2 c_2, \pi_2 y))}, \\
	l.(\letin{x_1}{\pi_1~l}{r_1})(\letin{x_2}{\pi_2~l}{r_2},\pi_3~l) \end{array} \right ] }
	{\gdo e_1 : A \to B \splito [c_1,l_1.r_1] 
	& \gdo e_2 : A \splito [c_2,l_2.r_2]} \and
\infer [\to\mathrm{E}\splits] 
	{\gdt e_1~e_2 : B \splits [(p_1,p_2), l.(\letin{x_1}{\pi_1~l}{r_1})(\letin{x_2}{\pi_2~l}{r_2})]}
	{\gdt e_1 : A \to B \splits [p_1,l_1.r_1] 
	& \gdt e_2 : A \splits [p_2,l_2.r_2]}
\end{mathpar}
\end{figure*}

\begin{figure*}
\caption{Sum Splitting}
\label{fig:splitSum}
\begin{mathpar}
\infer [+\mathrm{I_1}\splito] 
	{\gdo \iota_1~e : A+B \splito [(\iota_1(\pi_1c),\pi_2 c),l.\iota_1~r] }
	{\gdo e : A \splito [c,l.r] } \and
\infer [+\mathrm{I_1}\splits] 
	{\gdt \iota_1~e : A+B \splits [p,l.\iota_1~r] }
	{\gdt e : A \splits [p,l.r] } \and
\infer [+\mathrm{I_2}\splito] 
	{\gdo \iota_2~e : A+B \splito [(\iota_2(\pi_1c),\pi_2 c),l.\iota_2~r] }
	{\gdo e : A \splito [c,l.r] } \and
\infer [+\mathrm{I_2}\splits] 
	{\gdt \iota_2~e : A+B \splits [p,l.\iota_2~r] }
	{\gdt e : A \splits [p,l.r] } \and
\infer [+\mathrm{E}\splito]
	{\gdo \left( \tallcase {e_1}{x_2.e_2}{x_3.e_3} \right) :C \splito 
	\left[\left(\tallcase {\pi_1c_2}{x_2.\letin{y}{c_2}{(\pi_1 y,\iota_1(\pi_2 y))}}{x_3.\letin{y}{c_3}{(\pi_1 y,\iota_2(\pi_2 y))}}\right), 
	l.\left(\tallcase{l}{l_2.r_2}{l_3.r_3}\right)\right]}
	{\gdo e_1 : A+B \splito [c_2, l_2.r_2]
	& \Gamma,x_2:A^\bbone \vdash^\bbone e_2 : C \splito [c_2,l_2.r_2] 
	& \Gamma,x_3:B^\bbone \vdash^\bbone e_3 : C \splito [c_3,l_3.r_3]} \and
\infer [+\mathrm{E}\splits]
	{\gdt \left( \tallcase {e_1}{x_2.e_2}{x_3.e_3}\right) : C \splits 
	\left[(p_1,(p_2,p_3)), 
	l.\left(\tallcase{(\letin {l_1}{\pi_1 l}{r_1})}{x_2.\letin {l_2}{\pi_1(\pi_2 l)}{r_2}} {x_3.\letin {l_3}{\pi_2(\pi_2 l)}{r_3}} \right)
	\right] }
	{ \gdt e_1 : A+B \splits [p_1,l_1.r_1]
	& \Gamma,x_2:A^\bbtwo \vdash^\bbtwo e_2 : C \splits [p_2,l_2.r_2] 
	& \Gamma,x_3:B^\bbtwo \vdash^\bbtwo e_3 : C \splits [p_3,l_3.r_3]} 
\end{mathpar}
\end{figure*}
%\infer {\Gamma\vdash \letin {x}{e_1}{e_2} \splits [\letin {x}{e'_1}{p_2}, y_2.r_2] }{\Gamma\vdash e_1 \translates e_1' & \Gamma, x:\bbone \vdash e_2 \splits [p_2,y_2.r_2]} \\
%\infer { \Gamma\vdash \letin {x}{e_1}{e_2}  \splits [(p_1,p_2), l.\letin {x}{(\letin {y_1}{\pi_1~l}{r_1})}{\letin {y_2}{\pi_2~l}{r_2}}] }{\Gamma\vdash e_1 \splits [p_1,y_1.r_1] & \Gamma, x:\sigma_1 \vdash e_2 \splits [p_2,y_2.r_2]} \\

\section{Implementation and Examples}

[Work in progress.]

\section{Related Work}
\label{sec:lit}
\subsection{Partial Evaluation}
One can immediately see a connection between this work and partial evaluation. Both involve the idea of specializing a piece of code to some of it's inputs, leaving a residual that depends only on the remaining inputs.  But stage splitting is actually only part of partial evaluation.  At it's core, stage splitting is ``factor out the first stage, reduce it to a value, and express the second stage abstractly over that value," whereas partial evaluation is ``factor out the first stage, reduce it to a value, and specialize the code of the second stage to that value."  Expressed equationally,
\begin{center}
\begin{tabular}{rll}
{\bf Partial Evaluation}:& $p(f,a)=f_a$ &s.t. $f_a(b) = f(a,b)$ \\
{\bf Stage Splitting}:& $s(f)=(f_1,f_2)$ &s.t. $f_2(f_1(a),b) = f(a,b)$
\end{tabular}
\end{center}
Immediately, we note that stage splitting has broader application than partial evaluation, since the latter requires that $p$ (specifically, some code-specializing apparatus) and $a$ be available at the same time.  If this requirement is satisfied, then we can compare apples to apples by creating a partial evaluator out of a stage splitter:
	\[f_a = (\letin {x}{f_1(a)}{\lam{b}{?}{f_2(x,b)}})\]
From this view, we see that partial evaluators are more powerful because they can avoid memeory loads, prune unused branches, and even duplicate recursive code for further specialization.  These are largely free wins, except for the recursive code generation, which might have large space costs.

\subsection{Speculation}

I can find nothing in the literature that looks like our speculation.  That's probably because speculation is unsafe (especially around side effects), and so it would be a terrible idea in a system without lots of programmer direction.

\end{document}
