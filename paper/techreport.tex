\documentclass[preprint]{sigplanconf}

% The following \documentclass options may be useful:

% preprint      Remove this option only once the paper is in final form.
% 10pt          To set in 10-point type instead of 9-point.
% 11pt          To set in 11-point type instead of 9-point.
% authoryear    To obtain author/year citation style instead of numeric.

\usepackage{amsmath}
%\usepackage{amssymb}
\usepackage{amsthm}
\usepackage{bbm}
\usepackage{stmaryrd}
\usepackage{proof}
\usepackage{mathpartir}
\usepackage{mathabx}
\usepackage{listings}
\usepackage{hyperref}
\usepackage{cprotect}

\usepackage{cleveref}
\AtBeginDocument{\renewcommand\ref\cref}
\crefname{figure}{Figure}{Figures}
\crefname{theorem}{Theorem}{Theorems}


\newtheorem{theorem}{Theorem}[section]
\newtheorem{lemma}[theorem]{Lemma}
\newtheorem{definition}[theorem]{Definition}
\newtheorem{remark}[theorem]{Remark}

\newcommand {\TODO} {\textbf{TODO:} }

\definecolor{needsfixcolor}{rgb}{0.8,0.0,0.0}
\newcommand{\needsfix}[1]{{\color{needsfixcolor}\textbf{#1}}}

% core conventions and language names
\newcommand {\bbone} {\ensuremath{\mathbbm{1}}}
\newcommand {\bbtwo} {\ensuremath{\mathbbm{2}}}
\newcommand {\bbmono} {\ensuremath{\mathbb{M}}}
\newcommand {\bbem} {\ensuremath{\mathbb{M}}}
\newcommand {\ellStaged} {$\mathrm{L}^{\bbone\bbtwo}$}
\newcommand {\ellResid} {$\mathrm{L^R}$}
\newcommand {\ellTarget} {$\mathrm{L^T}$}
\newcommand {\lamStaged} {$\lambda^{\bbone\bbtwo}$}
\newcommand {\lamCircle} {$\lambda^{\fut}$}
\newcommand {\lamResid} {$\mathrm{\lambda^R}$}
\newcommand {\lamTarget} {$\mathrm{\lambda^T}$}
\newcommand {\lang} {\lamStaged}
\newcommand {\langTwo} {$\lambda^\bbtwo$}
\newcommand {\langmono} {$\lambda^{\textrm{U}}$}
\newcommand {\rmint} {{\rm int}}
\newcommand {\rmunit} {{\rm unit}}
\newcommand {\rmbool} {{\rm bool}}

% metatheory
\newcommand{\wfsym}{\ensuremath{\mathsf{wf}}}
\newcommand{\pvalsym}{\ensuremath{\mathsf{pval}}}
\newcommand{\ressym}{\ensuremath{\mathsf{res}}}
\newcommand{\valsym}{\ensuremath{\mathsf{val}}}
\newcommand{\fconsym}{\ensuremath{\mathsf{rtab}}}
\newcommand{\wf}{\ \wfsym}
\newcommand{\pval}{\ \pvalsym}
\newcommand{\res}{\ \ressym}
\newcommand{\val}{\ \valsym}
\newcommand{\fcon}{\ \fconsym}

% bnf stuff
\newcommand {\myit} [1]{\operatorname{\it{#1}}}
\newcommand {\stage} {\langle\mathit{stage}\rangle}
\newcommand {\type} {\tau}
\newcommand {\typeo} {\langle\bbone\text{-}\mathit{type}\rangle}
\newcommand {\typet} {\langle\bbtwo\text{-}\mathit{type}\rangle}
\newcommand {\expr} {e}
\newcommand {\brancho} {\langle\bbone\text{-}\mathit{brn}\rangle}
\newcommand {\brancht} {\langle\bbtwo\text{-}\mathit{brn}\rangle}
\newcommand {\expro} {\langle\bbone\text{-}\mathit{exp}\rangle}
\newcommand {\exprt} {\langle\bbtwo\text{-}\mathit{exp}\rangle}
%\newcommand {\val} {\langle\mathit{val}\rangle}
\newcommand {\valo} {\langle\bbone\text{-}\mathit{val}\rangle}
\newcommand {\valt} {\langle\bbtwo\text{-}\mathit{val}\rangle}
\newcommand {\world} {w}
\newcommand {\var} {x}
\newcommand {\emptyC} {\bullet}
\newcommand {\context} {\Gamma}
\newcommand {\inte} {i}
\newcommand {\bool} {b}
\newcommand {\resi} {\langle\mathit{res}\rangle}
\newcommand {\gbar} {~~|~~}

% nodes
\newcommand {\yhat} {\ensuremath{\mathtt{\hat y}}}
\newcommand {\pause} {{\tt hold}}
\renewcommand {\next} {{\tt next}}
\newcommand {\prev} {{\tt prev}}
\newcommand {\monoTerm} {{\tt mono}}
\newcommand {\monoType} {\nabla}
\newcommand {\curr} {\nabla}
\newcommand {\fut} {\bigcirc}
\newcommand {\inl} {{\tt inl}}
\newcommand {\inr} {{\tt inr}}
\newcommand {\pio} {{\tt pi1}}
\newcommand {\pit} {{\tt pi2}}
\newcommand {\roll} {{\tt roll}}
\newcommand {\unroll} {{\tt unroll}}
\newcommand {\lifttag} {{\tt liftTag}}
\newcommand {\liftint} {{\tt lift}}
\newcommand {\push} {{\tt unroll}}
\newcommand {\letin} [3] {{\tt let}~{#1} = {#2}~{\tt in}~{#3}}
\newcommand {\letmono} [3] {{\tt let~mono~\{}{#1}{\tt \} =~}{#2}~{\tt in}~{#3}}
\newcommand {\talllet} [3] {\begin{array}{@{}l@{}} {\tt let}~{#1} ={#2}~{\tt in} \\ {#3}\end{array}}
\newcommand {\caseof} [3] {{\tt case}~{#1} ~{\tt of}~{#2}~{\tt |}~{#3}}
\newcommand {\tallcase} [3] {\begin{array}{@{}l@{}} {\tt case}~{#1} ~{\tt of}\\~~{#2}\\{\tt |}~{#3} \end{array}}
\newcommand {\ifthen} [3] {{\tt if}~{#1} ~{\tt then}~{#2}~{\tt else}~{#3}}
\newcommand {\tallif} [3] {\begin{array}{@{}l@{}} {\tt if}~{#1} \\{\tt then}~{#2}\\{\tt else}~{#3} \end{array}}
\newcommand {\lam} [3] {\lambda{#1}\mathrm{:}{#2}.{#3}}
\newcommand {\valprod} [2] {({#1},{#2})}
\newcommand {\projbind} [1] {\letin{l_{#1}}{\pi_{#1} l}{r_{#1}}}

%judgments
\newcommand {\colone} [2] {{#1}:{#2}~@~\bbone}
\newcommand {\coltwo} [2] {{#1}:{#2}~@~\bbtwo}
\newcommand {\colmono} [2] {{#1}:{#2}~@~\bbmono}
\newcommand {\colm} [2] {{#1}:{#2}~@~\bbmono}
\newcommand {\colwor} [2] {{#1}:{#2}~@~w}
\newcommand {\typesone} [3] [\Gamma] { {#1} \vdash \colone{#2}{#3}}
\newcommand {\typestwo} [3] [\Gamma] { {#1} \vdash \coltwo{#2}{#3}}
\newcommand {\typesmono} [3] [\Gamma] { {#1} \vdash \colmono{#2}{#3}}
\newcommand {\typesm} [3] [\Gamma] { {#1} \vdash \colm{#2}{#3}}
\newcommand {\typeswor} [3] [] { \Gamma{#1} \vdash \colwor{#2}{#3}}
\newcommand {\types} [3] [\Gamma] {{#1} \vdash {#2}:{#3}}
\newcommand {\istypeone} {~{\rm type}~@~\bbone}
\newcommand {\istypetwo} {~{\rm type}~@~\bbtwo}
\newcommand {\istypewor} {~{\rm type}~@~w}

\newcommand {\typeslangTwo} [3] [\Gamma] {{#1} \vdash_\bbtwo {#2}:{#3}}

\newcommand {\reify} [3] {[{#1};{#2}]\overset{R}\Rightarrow{#3}}
\newcommand {\erasone} {\Downarrow^e_\bbone}
\newcommand {\erastwo} {\Downarrow^e_\bbtwo}
\newcommand {\isvalone} [1] {\Gamma \vdash {#1}~{\rm val}~@~\bbone}
\newcommand {\isvaltwo} {~{\rm val}~@~\bbtwo}
\newcommand {\isvalwor} {~{\rm val}~@~w}
%\newcommand {\reducexpl} [4] {{#3} \Downarrow_{#1}^{#2} {#4}}
%\newcommand {\redone} [2] {{#1} \Downarrow_\bbone^L {#2}}
%\newcommand {\redtwo} [2] {{#1} \Downarrow_\bbtwo^\bbtwo {#2}}
%\newcommand {\reducewor} [2] {{#1} \Downarrow_w^L {#2}}
\newcommand {\specwor} [2] [] {{#2} \downarrow_{#1}}
%\newcommand {\diaone} [3] [\Gamma] {{#1} \vdash {#2} \Downarrow_\bbone [{#3}]}
%\newcommand {\diatwo} [3] [\Gamma] {{#1} \vdash {#2} \Downarrow_\bbtwo {#3}}
\newcommand {\diaone} [3] [\Gamma] {{#2} \Downarrow_\bbone [{#3}]}
\newcommand {\diatwo} [3] [\Gamma] {{#2} \Downarrow_\bbtwo {#3}}
\newcommand {\reduce} [2] {{#1} \Downarrow {#2}}
\newcommand {\reifysym} {\overset{R}\Rightarrow}
\newcommand {\redsym} {{\Downarrow}}
\newcommand {\redonesym} {{\Downarrow_\bbone}}
\newcommand {\redtwosym} {{\Downarrow_\bbtwo}}
\newcommand {\tworedsym} {{\downarrow_\bbtwo}}

%\newcommand {\reduceonesub} [1] [] {\reduceone{e_{#1}}{v_{#1}}}
%\newcommand {\reducetwosub} [1] [] {\reducetwo{e_{#1}}{v_{#1}}}
\newcommand {\reduceworsub} [1] [] {\reducewor{e_{#1}}{v_{#1}}}
\newcommand {\specworsub} [1] [] {\specwor{e_{#1}}}
\newcommand {\diaonesub} [1] [] {\diaone{e_{#1}}{\xi_{#1};v_{#1}}}
\newcommand {\diatwosub} [1] [] {\diatwo{e_{#1}}{q_{#1}}}

\newcommand {\masko} [1] {|#1|_{\bbone}}
\newcommand {\maskt} [1] {|#1|_{\bbtwo}}

\newcommand {\vsplito} {\overset{\bbone}\curvearrowbotright}
\newcommand {\vsplitt} {\overset{\bbtwo}\curvearrowbotright}
\newcommand {\tsplito} {\overset{\bbone}\curvearrowright}
\newcommand {\tsplits} {\overset{\bbtwo}\curvearrowright}
\newcommand {\csplit} {\curvearrowright}
\newcommand {\ssplit} {\curvearrowbotright}

\newcommand {\splitonesym} {\overset{\bbone}\rightsquigarrow}
\newcommand {\splittwosym} {\overset{\bbtwo}\rightsquigarrow}
\newcommand {\splitone} [4] [\Gamma] {{#2} \splitonesym \left[{#4}\right]}
\newcommand {\splittwo} [4] [\Gamma] {{#2} \splittwosym \left[{#4}\right]}
\newcommand {\splitonesub} [3] [\Gamma] {\splitone [{#1}] {e_{#2}}{#3}{c_{#2}, l_{#2}.r_{#2}}}
\newcommand {\splittwosub} [3] [\Gamma] {\splittwo [{#1}] {e_{#2}}{#3}{p_{#2}, l_{#2}.r_{#2}}}

%misc
\newcommand{\dom}[1]{\mathsf{dom}(#1)}
\newcommand {\gcomp} [2] {\xi_{#1}, \xi_{#2}}
\newcommand {\daviesz} {\overset 0 \hookrightarrow}
\newcommand {\davieso} {\overset 1 \hookrightarrow}

%default stuff
\newcommand \ty \typesone
\newcommand \red \reduce
\newcommand \sub \reduceonesub
\newcommand \spl \splitone
\newcommand \col \colone

%inference extension
\newcommand {\infertypeswor}  [3] 
		[NONAME]{
			\renewcommand \ty \typeswor
			\renewcommand \col \colwor
			\infer %[\mathrm{#1}] 
      {#2}{#3}}
\newcommand {\inferreducewor} [3] 
		[NONAME]{
			\renewcommand \red \reducewor
			\renewcommand \sub \reduceworsub
			\infer [\mathrm{#1\Downarrow}] 
      {#2}{#3}}
\newcommand {\inferreducespc} [3] 
		[NONAME]{
			\renewcommand \red \specwor
			\renewcommand \sub \specworsub
			\infer [\mathrm{#1\downarrow}] {#2}{#3}}
\newcommand {\inferdiaone} [3] 
		[NONAME]{
			\renewcommand \red \diaone
			\renewcommand \sub \diaonesub
			\infer %[\mathrm{#1\Downarrow_\bbone}] 
      {#2}{#3}}
\newcommand {\inferdiaspc} [3] 
		[NONAME]{
			\renewcommand \red \diatwo
			\renewcommand \sub \diatwosub
			\infer %[\mathrm{#1\Downarrow_\bbtwo}] 
      {#2}{#3}}
\newcommand {\infersplitone}  [3] 
		[NONAME]{
			\renewcommand \spl \splitone
			\renewcommand \sub \splitonesub
			\renewcommand \col \colone
			\infer %[\mathrm{#1 \overset{\bbone}\rightsquigarrow}]
      {#2}{#3}}
\newcommand {\infersplittwo}  [3] 
		[NONAME]{
			\renewcommand \spl \splittwo
			\renewcommand \sub \splittwosub
			\renewcommand \col \coltwo
			\infer %[\mathrm{#1 \overset{\bbtwo}\rightsquigarrow}]
      {#2}{#3}}


\begin{document}

\lstset{language=ML,
        columns=fullflexible,
        basicstyle=\ttfamily,
        tabsize=4,
        escapeinside={"*}{*"},
        literate={|->}{{\ $\mapsto$\ }}1
                 {xhat}{{$\hat{\texttt{x}}$}}1
                 {yhat}{{$\hat{\texttt{y}}$}}1
                 {zhat}{{$\hat{\texttt{z}}$}}1
                 {murec}{{$\mu$}}1
                 {alpha}{{$\alpha$}}1
                 {beta}{{$\beta$}}1
                 {$}{{$\fut$}}1}

\special{papersize=8.5in,11in}
\setlength{\pdfpageheight}{\paperheight}
\setlength{\pdfpagewidth}{\paperwidth}

\conferenceinfo{CONF 'yy}{Month d--d, 20yy, City, ST, Country} 
\copyrightyear{20yy} 
\copyrightdata{978-1-nnnn-nnnn-n/yy/mm} 
\doi{nnnnnnn.nnnnnnn}

% Uncomment one of the following two, if you are not going for the 
% traditional copyright transfer agreement.

%\exclusivelicense                % ACM gets exclusive license to publish, 
                                  % you retain copyright

%\permissiontopublish             % ACM gets nonexclusive license to publish
                                  % (paid open-access papers, 
                                  % short abstracts)

%\titlebanner{banner above paper title}        % These are ignored unless
%\preprintfooter{short description of paper}   % 'preprint' option specified.

\title{Stage-Splitting a Modal Language}

\authorinfo{Nicolas Feltman \and Carlo Angiuli \and Umut Acar \and Kayvon Fatahalian}
           {Carnegie Mellon University}
           {\{nfeltman,cangiuli,umut,kayvonf\}@cs.cmu.edu}

\maketitle

\begin{abstract}
Many algorithms can perform useful work before receiving all of their inputs.
If we regard those inputs as arriving at different \emph{stages} of the
computation, then we can \emph{split} those algorithms into, for each stage, a
function performing all the work dependent only on the available inputs.

In this paper, we provide a theoretical understanding of this splitting process
by defining it as a program transformation on \lang, a typed lambda calculus
equipped with a modal staging operator. This approach extends splitting to
language features absent in prior work, including first-class functions and
disjoint unions.

These new features allow us to express some familiar algorithms which, when
split, yield asymptotic improvements; for example, quickselect of the $k$th
smallest element of a list $l$ splits into (1) sorting $l$ into a binary search
tree, then (2) finding the tree's $k$th leftmost element.

\TODO (Is this true?)
We also show that splitting a staged implementation of an interpreter yields a
partial evaluator.

%into one function per stage, performing all the work depending only on the
%available inputs.

%as much work as possible given the inputs available at each stage. 

%In this way, splitting is essentially an algorithm generation technique.

% amortized

\end{abstract}

\category{CR-number}{subcategory}{third-level}

% general terms are not compulsory anymore, 
% you may leave them out
\terms
term1, term2

\keywords
keyword1, keyword2

%%%%%%%%%%%%%%%%%%%%%%%%%%%%%%%%%%%%%%%%%%%%%%%%%%%%%%%%%%%%%%%%%%%%%%%%%%%%%%%%

\section{Introduction}

This document is a tech report on current research into a program transformation
technique which we call {\em splitting}.  This report has three parts, with
a section dedicated to each,
\begin{itemize}
\item \Cref{sec:semantics} describes a language, called \lang, which unambiguously specifies the inputs to 
our splitting transformation.  It includes both a type system, which is unimaginatively based on previous work, 
and a semantics, which is more novel.
\item \Cref{sec:splitting} describes the splitting transformation itself.  This takes programs, specified in \lang, and converts them to terms in a standard unstaged language.  This section also describes the more formal definition of the correctness for this process.
\item \Cref{sec:examples} gives examples of how this splitting transformation can be used
to derive data structures, and discusses the effectiveness and limitations of this technique.
\end{itemize}

\section {Staged Programming in \lang}
\label{sec:staging}
\begin{figure*}
\label{fig:quickselect}
\begin{minipage}{0.5\textwidth}
\begin{lstlisting} 
datatype list = Empty | Cons of int * list
fun partition (p : int, l : list) 
  : (int*list*list) =
  case unroll l of 
    Empty => (0,Empty, Empty) 
  | Cons (h,t) =>
      let (s,left,right) = partition (p,t) in
      if h<p 
      then (s+1,Cons(h,left),right)
      else (s,left,Cons(h,right))

fun qs (l : list, k : int) : int = 
  case l of
    Empty => 0
  | Cons (h,t) => 
      let (left,right,n) = partition h t in
        case compare k n of
          LT => qs (left, k)
        | EQ => h
        | GT => qs (right, k-n-1)
\end{lstlisting}
\caption{Unstaged Code}
\end{minipage}
\begin{minipage}{0.5\textwidth}
\begin{lstlisting} 
datatype list = Empty | Cons of int * list
fun partition (p : int, l : list) = ...

fun qs (l : list, k : $int) : $int = 
  case l of
    1`Empty` => next {2`0`}
  | Cons (h,t) => 
      let (left,right,n) = partition h t in
      next{
        let n = hold{n} in
          case compare prev{k} n of
            LT => prev {qs (left k)}
          | EQ => hold {h}
          | GT => prev {qs (right, next{prev{k}-n-1)}}
      }	
\end{lstlisting}
\caption{Staged Code}
\end{minipage}
\begin{minipage}{0.5\textwidth}
\begin{lstlisting} 
datatype list = Empty | Cons of int * list
fun partition (p : int, l : list) = ...
	
datatype tree = Branch of int * int * tree * tree
                | Leaf

fun qs1 (l : list) : tree =
  case l of
    Empty => Leaf
  | Cons (h,t) => 
      let (left,right,n) = partition h t in
      Branch (n, h, qs1 left, qs1 right)

fun qs2 (p : tree, k : int) : int = 
  case unroll p of
    Leaf => 0
  | Branch (n,h,p1,p2) => 
      case compare k n of
        LT => qs2 (p1,k)
      | EQ => h
      | GT => qs2 (p2,k-n-1)
\end{lstlisting}
\caption{Split Code}
\end{minipage}
\caption{Caption place holder}
\end{figure*}

For an example of an algorithm amenable to staging techniques,
consider the quickselect algorithm in [figure part (a)],
which finds the $k$th largest element of a list.
It operates by inspecting the head of the list, 
partitioning the rest of the list by the head,
comparing the size of the partition to the desired index,
and the recursively selecting on the correct sublist.
Assuming that the list is randomly sorted, 
quickselect will take expected $O(n)$ time, where $n$ is the size of the list.

Importantly, quickselect has property that the partitioning of the list does not depend on $k$, 
even through the recursive boundary.\footnote{There is a control-dependence on $k$, but we can ignore that for now.}
We can leverage this dependency pattern to split quickselect into two other functions:
one which precomputes all of the partitions of the input list to build a binary search tree,
and another which uses this tree to perform an accelerated index lookup.
This split version of quickselect is shown in [figure part (c)].
One advantage of this form is that the tree can be built once per list
and then used to accelerate many lookups.

The goal of our work is to automate this splitting transformation.
Our first step is to define a language which clearly and unambiguously specifies how its terms should be split.
Ideally, only minimal refactoring should be required to write terms within this language,
and from there, the type system should be enough to prove that a valid splitting exists.

To this end, we present \lang.
The main idea of \lang\ is that all parts of terms defined in it can be identified with one of two {\em stages}, namely \bbone\ or \bbtwo.
Intuitively, the stage of a term expresses \emph{when} to evaluate it---all stage-\bbone\
subexpressions are evaluated before stage-\bbtwo\ ones.
Correspondingly, after splitting those parts of the \lang\ term in \bbone\ will end up in the precomputation, 
and those parts in \bbtwo\ will end up in the residual.
Within well-typed terms, information can flow from stage \bbone\ to stage \bbtwo\ portions,
but never from \bbtwo\ to \bbone.  
Indeed this is a necessary property if we want to be able to split the term.

[Figure part (b)] shows a staged implementation of quickselect in \lang.
As before, the inputs to the function are the list to select from and the index to select,
but now the latter is represented with the type $\fut\rmint$ rather than $\rmint$.
The difference here is that an $\rmint$ is an integer available at the current stage (stage \bbone), 
whereas a $\fut\rmint$ is an integer available only at the next stage (stage \bbtwo).
Naturally the output type, representing the $k$th largest element of the list,
is also $\fut\rmint$, since it cannot be known until the next stage.

Each construct in the body of quickselect is now associated with a stage via an interleaving of $\next$ and $\prev$ blocks.  
Specifically, $\next$ occurs in a stage \bbone\ context and indicates that the contents of its block are stage \bbtwo, 
whereas $\prev$ occurs in a stage \bbtwo\ context and indicates that the contents of its block is stage \bbone.
(We adapt the convention convention that the top-level context is stage \bbone.)
The output type of a $\next$ block is the $\fut$'d version of the type of its stage \bbtwo\ contents.  
For example, \verb|next{0}| from above has the type $\fut\rmint$.
Correspondingly, $\prev$ requires that its stage \bbone\ contents have a $\fut$ type, and it eliminates the wrapper.
For example, \verb|prev{k}| from above has type $\rmint$ at stage \bbtwo, since $k$ is bound to a $\fut\rmint$ at stage \bbone.
These type restrictions essentially enforce that ``later stage content" is always treated hygienically at stage \bbone,
which is necessary to admit a properly staged implementation.

The code also contains two $\pause$ blocks.  
This construct has same stage signature as $\prev$,
but instead of ``unwrapping" $\fut$ types it simply promotes integers from stage \bbone\ to stage \bbtwo.
\footnote{It will turn out that $\pause$ is implementable---though it takes some effort---from our other language features.
We instead provide it as a primitive to shorten examples.  
Furthermore, it would be wise to extend $\pause$ to other base types, if we had them, and to products and sums thereof.
This is related to the notion of {\em mobility} in \cite{murphy05} and {\em stability} in \cite{krishnaswami13}.}

But for the $\fut$, $\next$, $\prev$, and $\pause$ constructs, 
the staged version of quickselect is virtually identical to the unstaged version.
The constructs that were added were placed in order to maximize the work done at stage \bbone\ while still conforming to the type signature.
It would have also been valid to simply move the whole input list unchanged into stage \bbtwo\ at the very beginning, 
but that would not be particularly interesting since it would result in an effectively trivial split 
that's just the identity at stage \bbone\ and plus quickselect at stage \bbtwo.
There has been the extensive research into the question of how to automatically add staging annotations to unstaged code.
This process is known as {\em binding time analysis}, and we do not consider it here.
Instead, we assume that all input programs are properly staged according to some programmer's intent.

\section{\lang\ Statics and Dynamics}
\label{sec:semantics}

\begin{figure}
\centering
$\begin{aligned}
\world &::= \bbone \gbar \bbtwo \\
\type &::= \text{unit}~|~\text{int}~|~\text{bool}
 \gbar \type \times \type 
 \gbar \type + \type \\
&\gbar \type \to \type
 \gbar \fut \type
 \gbar \alpha \gbar \mu \alpha.\tau \\
\expr &::= ()\gbar\inte\gbar\bool\gbar \var
 \gbar \lam{\var}{\type}{\expr} 
 \gbar \expr~\expr \\
&\gbar (\expr, \expr) 
 \gbar \pio~\expr 
 \gbar \pit~\expr
 \gbar \inl~\expr 
 \gbar \inr~\expr \\
&\gbar \caseof {\expr}{\var.\expr}{\var.\expr}
 \gbar \ifthen {\expr}{\expr}{\expr} \\
&\gbar \roll 
 \gbar \unroll
 \gbar \letin{\var}{\expr}{\expr} \\
&\gbar \next~\expr 
 \gbar \prev~\expr 
 \gbar \pause~\expr \\
\context &::=\emptyC \gbar \context, \colwor \var \type
\end{aligned} $
\caption{\lang~Syntax}
\label{fig:grammar}
\end{figure}

%
%\begin{figure}
%\caption{\ellStaged~Syntax}
%\label{fig:ellStagedSyntax}
%\centering
%\begin{tabular}{ll} 
%$\begin{aligned}
%\typeo &::= \text{unit}~|~\text{int}~|~\text{bool} \\
%&\gbar \typeo \times \typeo \\
%&\gbar \fut \typet \\
%\expro &::= ()~|~\inte~|~\bool  \\
%&\gbar \letin{\var}{\expro}{\expro} \\
%&\gbar \var \\
%&\gbar (\expro, \expro) \\
%&\gbar \pi_1~\expro \gbar \pi_2~\expro \\
%&\gbar \ifthen {\expro}{\expro}{\expro} \\
%&\gbar \next~\exprt \\
%\contextot &::=\emptyC \\
%&\gbar \contextot, \var : \typeo ^\bbone \\
%&\gbar \contextot, \var : \typet ^\bbtwo
%\end{aligned} $ 
%& 
%$\begin{aligned}
%\typet &::=  \text{unit}~|~\text{int}~|~\text{bool} \\
%&\gbar \typet \times \typet \\
%\\
%\exprt &::= ()~|~\inte~|~\bool \\
%&\gbar \letin{\var}{\exprt}{\exprt} \\
%&\gbar \var \\
%&\gbar (\exprt, \exprt) \\
%&\gbar \pi_1~\exprt \gbar \pi_2~\exprt \\
%&\gbar \ifthen {\exprt}{\exprt}{\exprt} \\
%&\gbar \prev~\expro \\
%\\
%\\
%\\
%\end{aligned} $
%\end{tabular}
%\end{figure}

%
%\begin{figure}
%\caption{\ellTarget~Syntax}
%\label{fig:ellTargetSyntax}
%\centering
%\begin{tabular}{ll} 
%$\begin{aligned}
%\expr &::= ()~|~\inte~|~\bool \\
%&\gbar \letin{\var}{\expr}{\expr} \\
%&\gbar \var \\
%&\gbar (\expr, \expr) \\
%&\gbar \pi_1~\expr \gbar \pi_2~\expr \\
%&\gbar \inl~\expr \gbar \inr~\expr \\
%&\gbar \ifthen {\expr}{\expr}{\expr}  \\
%&\gbar \caseof {\expr}{x_1.\expr}{x_2.\expr} 
%\end{aligned} $
%& 
%$\begin{aligned}
%\type &::=  \rmunit~|~\text{int}~|~\text{bool} \\
%&\gbar \type \times \type  \\
%&\gbar \type + \type 
%\\
%\context &::= \emptyC \\
%&\gbar \context, \var : \type
%\\ \\ \\ \\
%\end{aligned} $
%\end{tabular}
%\end{figure}

%!TEX root = ../paper.tex

%\begin{figure}
%\begin{mathpar}
%\infertypeswor [\rmunit] 	{\Delta \vdash {\tt unit} \istypewor}						{\cdot} 																			\and
%\infertypeswor [int]			{\Delta \vdash {\tt int} \istypewor}						{\cdot} 																			\and
%%\infertypeswor [bool]			{\Delta \vdash {\tt bool} \istypewor}						{\cdot} 																			\and
%\infertypeswor [\times]		{\Delta \vdash A \times B \istypewor}						{\Delta \vdash A \istypewor & \Delta \vdash B \istypewor} 					\and
%\infertypeswor [+]			{\Delta \vdash A + B \istypewor}							{\Delta \vdash A \istypewor & \Delta \vdash B \istypewor} 					\and
%\infertypeswor [\to]			{\Delta \vdash A \to B \istypewor}							{\Delta \vdash A \istypewor & \Delta \vdash B \istypewor} 					\and
%\infertypeswor [\mu]			{\Delta \vdash \mu \alpha. A \istypewor}					{\Delta, \alpha \istypewor \vdash A \istypewor} 								\and
%\infertypeswor [var]			{\Delta \vdash \alpha \istypewor}							{\alpha \istypewor \in \Delta} 													\and
%\infertypeswor [\fut]			{\Delta \vdash \fut A \istypeone}							{\Delta \vdash A \istypetwo} 																
%\end{mathpar}
%\caption{\lang~Valid Types}
%\label{fig:validTypes}
%\end{figure}

\begin{figure*}
\begin{abstrsyn}
\begin{mathpar}
\infertypeswor [\rmunit] 	{\ty {()}\rmunit}									{\cdot} 																	\and
\infertypeswor [int]		{\ty {i} \rmint}									{\cdot} 																	\and
\infertypeswor [hyp]		{\ty x A}											{\col x A \in \Gamma} 														\and
\infertypeswor [\to I]		{\ty {\lam {x}{e}} {A \to B}}						{A \istypewor & \ty [,\col x A] e B} 										\and
\infertypeswor [\to E]		{\ty {e_1~e_2} {B}}									{\ty {e_1} {A \to B} & \ty {e_2} A} 										\and
\infertypeswor [\times I]	{\ty {(e_1,e_2)}{A\times B}}						{\ty {e_1} A & \ty {e_2} B} 												\and
\infertypeswor [\times E_1]	{\ty {\pio e} A}									{\ty e {A\times B}} 														\and
\infertypeswor [\times E_2]	{\ty {\pit e} B}									{\ty e {A\times B}} 														\and
\infertypeswor [+ I_1]		{\ty {\inl e} {A + B}}								{\ty e A} 																	\and
\infertypeswor [+ I_2]		{\ty {\inr e} {A + B}}								{\ty e B} 																	\and
\infertypeswor [+ E]		{\ty {\caseof{e_1}{x_2.e_2}{x_3.e_3}} C}			{\ty {e_1}{A+B} & \ty[,\col {x_2} A]{e_2} C & \ty[,\col {x_3} B]{e_3} C} 	\and
\infertypeswor [\mu I]		{\ty {\roll e} {\mu \alpha.\tau}}					{\ty e {[\mualphatau / \alpha]\tau}} 										\and
\infertypeswor [\mu E]		{\ty {\unroll e} {[\mualphatau / \alpha]\tau}}		{\ty e \mualphatau} 														\and
\infertypeswor [\fut I]		{\typesone {\next e}{\fut A}}						{\typestwo e A} 															\and
\infertypeswor [\fut E]		{\typestwo {\prev e} A}								{\typesone e {\fut A}} 														\and
\infertypeswor [\curr I]	{\typesone {\pure e} {\curr A}}						{\typesmono e A} 															\and
\infertypeswor [\curr E]	{\typesone {\letp x {e_1} {e_2}} B}					{\typesone {e_1} {\curr A} & \typesone [\Gamma,\colmono x A] {e_2} B} 		\and
\infertypeswor [lift] 		{\typesone {\lifttag e} {\curr A + \curr B}}		{\typesone e {\curr (A+B)}} 												\and
\infertypeswor [hold]		{\typesone {\pause e} {\fut \rmint}}				{\typesone e {\curr \rmint}}		 										\and
\end{mathpar}
\end{abstrsyn}
\caption{\lang~Static Semantics}
\label{fig:statics}
\end{figure*}


As stated in \cref{sec:staging}, we operate on a language called \lang,
which is a typed two-stage lambda calculus featuring products, sums, and isorecursive types.

\subsection{Statics}
The grammar and type system of \lang\ are given in \ref{fig:grammar,fig:statics}. 
They are a simple adaptation of those of \cite{davies96}, 
restricted to two stages\footnote{This restriction is not a big deal...} and extended with general sums and recursion.

Typing judgments and context variables are annotated with stages after an $@$ symbol.
Only $\fut$ and its introductory and eliminatory forms $\next$ and $\prev$ affect the stage
of a term or type.

\footnote{It is possible for different stages to have different sets of
features, but for simplicity we do not consider this.}
Every valid expression has both a type and a stage, either \bbone~or \bbtwo. 

The stage-\bbone\ type $\fut A$ contains encapsulated
stage-\bbtwo\ expressions of type $A$. Terms of type $\fut A$ are treated
opaquely by stage-\bbone\ code, as (by the requirements of staging) they cannot be evaluated until stage \bbtwo.
While $\fut$ allows us to embed stage-\bbtwo\ types within stage-\bbone\ types,
there is no way to embed a stage-\bbone\ type within a stage-\bbtwo\ type.
That is, there is a one-way dependence between stages at the term level.

Alternatively, the stages are mutually dependent at the term level.
The $\next$ constructor embeds stage-\bbtwo\ expressions into stage \bbone,
while $\prev$ embeds stage-\bbone\ expressions into stage \bbtwo.  $\next$ and $\prev$ are the
only ways in \lang\ to alter the stage of a term; we surround their arguments
with braces in \lang\ syntax to clearly indicate stage boundaries within a
program.

%We formulated our typing judgments in the style of \cite{davies96}, where the
%whole judgment is annotated with a stage.  
%The grammar and type system for \lang\ is given in
%\ref{fig:grammar,fig:statics}.
% We annotate typing judgments and context variables with stages;
%This is made manifest as rules which are entirely abstract over stage.
%In addition to determining the stage, $\next$ and $\prev$ are the introduction and elimination forms for $\fut$ types.
Specifically, given an argument with type $A$ at stage \bbtwo, $\next$ forms a $\fut A$ at stage \bbone.  
%That is, it forms the promise of a future $A$ out of a construction for an $A$ at the next timestep.
Stage \bbtwo\ expressions can obtain the original stage \bbtwo\ argument via the $\prev$ construct.  
Since $\prev$ operates at stage \bbtwo, this ensures no violation of causality,\cite{cave14}.
The $\pause$ construct serves to wrap stage \bbone\ integers for use in stage \bbtwo.  
It is possible to implement $\pause$ from other \lang\ features, but
we provide it as a core primitive to simplify our examples. 

Recall the fast exponent function from above.
We present a valid staging of it here:
\begin{lstlisting} 
let fexp (b : $int, p : int) : $int =
	if p == 0 then
		next{1}
	else if (p mod 2) == 0 then
		next{let x = prev{fexp(b,p/2)} in x*x}
	else
		next{prev{b} * prev{fexp(b,p-1)}}		
\end{lstlisting}

\TODO explain why we staged this the way we did

The function receives a stage-\bbone\ exponent {\tt e} and a stage-\bbtwo\ base {\tt b} and returns a stage-\bbtwo\ result. 
The {\tt if} predicates and exponent decomposition are all stage-\bbone\ terms, since they occur within $\prev$ blocks.

\subsection{Dynamics}
\label{sec:stagedsemantics}

A key attribute of the quickselect example is that stage-\bbone\ and stage-\bbtwo\ expressions are nested. 
Ordinary term evaluation eliminates outermost redexes first, 
however in the case that stage \bbone\ expressions are contained inside stage \bbtwo\ ones 
(such as the recurive call to {\tt fexp} above), 
this strategy conflicts with the precept of staged execution: 
that all stage-\bbone\ code be evaluated before the evaluation of stage-\bbtwo\ code. 

Thus, our dynamic semantics for \lang\ evaluates all of a term's stage \bbone\
subexpressions before any of its stage \bbtwo\ subexpressions. This results in a
stage \bbtwo\ term with no stage \bbone\ content; we say this is a term in a
monostaged language called \langTwo. Then, we perform the remainder of the evaluation with 
$\tworedsym$, an ordinary dynamic semantics for \langTwo\ (the rules for this
judgment are not shown, but they are standard).

\subsection{Non-Duplicating First-Stage Evaluation}

To gain intuition about the challenges of implementing this staged dynamic
semantics, consider the following example:
\begin{lstlisting}
#2 (next {fib 20}, 2+3)
\end{lstlisting}
This is a stage-\bbone\ expression of type $\rmint$; the pair inside it is a
stage-\bbone\ expression of type $(\fut\rmint)\times\rmint$. 
In this example, ${\tt fib} : \rmint \to \rmint$ is a stage-\bbtwo\ reference to the Fibonacci function.
A conventional call-by-value evaluation strategy demands that we evaluate both components of
the pair to values before we project from the pair. The problem is that
while \verb|next {fib 20}| is not a value (in the sense that additional
stage-\bbtwo\ computation steps are necessary to produce the result \verb|6765|), 
evaluating the contents of \verb|next| cannot occur as part of stage-\bbone\ evaluation.
Intuitively, the solution is to designate \verb|next {fib 20}| as a value \emph{in
stage \bbone}, even though it requires additional evaluation in stage \bbtwo.
Therefore, we evaluate the pair to
\begin{lstlisting}
#2 (next {fib 20}, 5)
\end{lstlisting}
then obtain \verb|5| as the result of projection.

Now consider a more complex example where 
stage-\bbone\ evaluation must substitute such a incompletely-evaluated
expression. The following stage-\bbtwo\ term has type $\rmint$:
\begin{lstlisting} 
prev{
  let x = (next {fib 20}, 3+4) in
  next{ prev{#1 x} * prev{#1 x} * hold{#2 x} }
}
\end{lstlisting}
Again, stage-\bbone\ evaluation will not fully reduce this term, because the answer
depends on the value of \verb|fib 20|, which is not reduced to \verb|3| until
stage \bbtwo.

If we simply treat \verb|(next {fib 20}, 7)| as a value during stage~\bbone, and
substitute it for the three occurrences of \verb|x| in the body of the
\verb|let| expression, the result of stage \bbone\ computation is
\begin{lstlisting} 
prev{
  next{ 
    prev{next {fib 20}} * prev{next {fib 20}} * 7 
  }
}
\end{lstlisting}
Finally, the $\prev$s eliminate the $\next$s to yield the final residual:
\begin{lstlisting} 
(fib 20) * (fib 20) * 7
\end{lstlisting}
Note how stage-\bbtwo\ evaluation of this expression will compute \verb|fib 20| twice.  
To avoid duplicating computations, we take a different approach.  Instead of
duplicating the contents of the $\next$ expression in the tuple, we bind the contained stage-\bbtwo\ expression to
a variable (here, $\mathtt{y}$) and duplicate only a reference to that variable.
This produces:
\begin{lstlisting} 
let y = fib 20 in y * y * 7
\end{lstlisting}

Achieving this behavior mechanically requires us to resolve a contradiction:
we must substitute for \texttt{x} in stage \bbone, but we cannot evaluate inside the $\next$ block within the tuple. 
Our solution is to replace the contents of the $\next$ with a new stage-\bbtwo\ variable and create an explicit substitution (shown with a $\mapsto$) binding that stage-\bbtwo\ variable to the $\next$'s old contents.  
This substitution then floats up to the top of the containing $\prev$ block:
\begin{lstlisting} 
prev {
[yhat|->fib 20]
  let x = (next{yhat}, 7) in
  next{prev{#1 x} * prev{#1 x} *  hold{#2 x}}
}
\end{lstlisting}
As a convention, we render the new variable with a %stylish and fashionable
hat.  We're now free to perform the stage-\bbone~substitution for {\tt x} without duplicating stage-\bbtwo\ work.
\begin{lstlisting} 
prev {
[yhat|->fib 20]
  next{
    prev{#1 (next {yhat}, 7)} * 
    prev{#1 (next {yhat}, 7)} *
    hold{#2 (next {yhat}, 7)}
  }
}
\end{lstlisting}
To evaluate the outermost $\next$, we must first partially evaluate within its body by finding all of the contained stage-\bbone~terms and reducing them. 
As a rule, these will reduce to $\next$ expressions, which the $\prev$ eliminates, leaving the variable in place:
\begin{lstlisting} 
prev {
[yhat|->fib 20]
    next{ yhat * yhat * 7 }
}
\end{lstlisting}
Once again, we lift the contents of the $\next$ into a substitution:
\begin{lstlisting} 
prev {
[yhat|->fib 20]
[zhat|->yhat*yhat*7]
    next{ zhat }
}
\end{lstlisting}
Finally, when evaluating the outer $\prev$, we must {\em reify} the contained substitutions into let statements, yielding
\begin{lstlisting} 
let yhat = fib 20 in
let zhat = yhat * yhat * 7 in z
\end{lstlisting}

Thus we have evaluated all of the stage \bbone\ expressions of this program without duplicating the contents of $\next$ blocks.

\subsection{Dynamics}
\label{ssec:dynamics}

%!TEX root = ../paper.tex

\begin{figure*}
\begin{abstrsyn}
\begin{mathpar}
\fbox {Values} \and
\infertypeswor[\rmunit]		{() \pval}									{\cdot}			\and
\infertypeswor[int]			{i \pval}									{\cdot}			\and
\infertypeswor[\times]		{(v_1, v_2) \pval}							{v_1 \pval 
																		&v_2 \pval}		\and
\infertypeswor[+_1]			{\inl v \pval}								{v \pval}		\and
\infertypeswor[+_2]			{\inr v \pval}								{v \pval}		\and
\infertypeswor[\to]			{\lam x e \pval}							{\cdot}			\and
\infertypeswor[\fut]		{\next {\hat y} \pval}						{\cdot}			\and
\infertypeswor[]			{\pure {v} \pval}							{v \val}		\and
\infertypeswor[]			{[\cdot;v] \cpval}							{v \pval}		\and
\infertypeswor[]			{[\hat y \mapsto q,\xi;v] \cpval}			{q \res 
																		&[\xi;v] \pval}
\end{mathpar}
\hrule
\begin{mathpar}
\fbox{1-Evaluation}
\and
\inferdiaone[\rmunit]
{\red {()}{\cdot;()}}
{\cdot}
\and
\inferdiaone [\to I]
{\red {\lam{x}{e}} {\cdot;\lam{x}{e}}}
{\cdot}
\and
\inferdiaone [\to E]
{\red {\app {e_1}{e_2}} {\gcomp 1 2, \xi';v'}}
{\red {e_1} {\xi_1;\lam{x}{e'}} & \sub [2] & \red  [\Gamma,\dom{\xi_1},\dom{\xi_2}] {[v_2/x]e'}{\xi';v'}}
\and
%
\inferdiaone [\times I]
{\red {(e_1,e_2)}{\gcomp 1 2;\valprod{v_1}{v_2}}}
{\sub [1] & \sub [2]}
\and
\inferdiaone [\times E_1]
{\red {\pio{e}}{\xi;v_1}}
{\red{e}{\xi;\valprod{v_1}{v_2}}}
\and
\inferdiaone [\times E_2]
{\red {\pit{e}}{\xi;v_2}}
{\red{e}{\xi;\valprod{v_1}{v_2}}}
\and
\inferdiaone [+ I_1]
{\red {\inl{e}} {\xi;\inl{v}}}
{\sub}
\and
\inferdiaone [+ I_2]
{\red {\inr{e}} {\xi;\inr{v}}}
{\sub}
\and
%
\inferdiaone [+ E_1]
{\red {\caseof{e_1}{x_2.e_2}{x_3.e_3}}{\gcomp 1 2;v_2}}
{\red {e_1}{\xi_1;\inl{v}} & \red
  [\Gamma,\dom{\xi_1}]{[v/x_2]e_2}{\xi_2;v_2}}
\and
\inferdiaone [+ E_2]
{\red {\caseof{e_1}{x_2.e_2}{x_3.e_3}}{\gcomp 1 3;v_3}}
{\red {e_1}{\xi_1;\inr{v}} & \red
  [\Gamma,\dom{\xi_1}]{[v/x_3]e_3}{\xi_3;v_3}}
\and
\inferdiaone[\mu I]
{\red {\roll{e}}{\xi;\roll{v}}}
{\sub}
\and
\inferdiaone[\mu E]
{\red {\unroll{e}}{\xi; v}}
{\red {e}{\xi; \roll{v}}}
\end{mathpar}

\hrule
\begin{mathpar}
\fbox{Residualization}
\and
\inferdiaspc[\rmunit]
{\red {()}{()}}
{\cdot}
\and
\inferdiaspc[int]
{\red {i}{i}}
{\cdot}
\and
%% %\inferdiaspc [bool]{\red {b}{b}}
%% {\cdot}
%% \and
\inferdiaspc[hyp]
{\red {x}{x}}
{\cdot}
\and
\inferdiaspc[\to I]
{\red {\lam{x}{A}{e}}{\lam{x}{A}{q}}}
{\diatwo [\Gamma,x] e q} 
\and
\inferdiaspc [\to E]
{\red {e_1~e_2}{q_1~q_2}}
{\sub [1] & \sub [2]}
\and
\inferdiaspc [\times I]
{\red {(e_1,e_2)}{(q_1,q_2)}}
{\sub [1] & \sub [2]} 
\and
\inferdiaspc [C]
{\red {\mathcal{C}(e)}{\mathcal{C}(q)}}
{\sub}
%\inferdiaspc[+ I_2]
%{\red {\inr~e}{\inr~q}}
%{\sub}
\and
\inferdiaspc[+ E_1]
{\red {\caseof{e_1}{x_2.e_2}{x_3.e_3}}
{\caseof{q_1}{x_2.q_2}{x_3.q_3}}}
{\sub [1] & \diatwo [\Gamma,x_2] {e_2} {q_2} & \diatwo [\Gamma,x_3] {e_3} {q_3}} 
%\inferdiaspc [\mu E]   		{\red {\unroll~e}{\unroll~v}}
%	{\sub}										\and
%\inferdiaspc [let]			{\red {\letin{x}{e_1}{e_2}}{\letin{x}{q_1}{q_2}}}			{\sub [1] & \diatwo [\Gamma,x] {e_2} {q_2}} 					\and
%\inferdiaspc [if_T] 			{\red {\ifthen{e_1}{e_2}{e_3}}{\ifthen{q_1}{q_2}{q_3}}}	{\sub [1] & \sub [2] & \sub [3]} 		\and
%
\end{mathpar}
\hrule
\begin{mathpar}
\fbox{Staging Features} \and
\inferdiaone [\fut I]	{\red {\next{e}}{\hat y \mapsto q;\next{\hat y}}}			{\diatwo e q}														\and
\inferdiaspc [\fut E]	{\red {\prev{e}} q}											{\diaone e {\xi; \next{\hat y}} & \reify{\xi}{\hat y}q}				\and
\inferdiaone [hold]		{\red {\pause e} {\xi, \hat y \mapsto i; \next {\hat y}}}	{\red e {\xi; \pure i}}												\and
\infer					{\reify {\cdot}{q}{q}}										{\cdot}																\and
\infer					{\reify {(y \mapsto q_1), \xi}{q_2}{\letin{y}{q_1}{q'}}}	{\reify{\xi}{q_2}{q'}}												\and
\inferdiaone			{\red {\pure e} {\cdot; \pure~v}}							{\reduce e v} 														\and
\inferdiaone			{\red {\letp x {e_1} {e_2}} {\xi_1, \xi_2; v_2}}			{\red {e_1} {\xi_1;\pure {v_1}} & \red {[v_1/x]e_2} {\xi_2;v_2}} 	\and
\inferdiaone			{\red {\lifttag e} {\xi;\inl{\pure v}}}						{\red e {\xi; \pure{\inl v}}}										\and
\inferdiaone			{\red {\lifttag e} {\xi;\inr{\pure v}}}						{\red e {\xi; \pure{\inr v}}}										
\end{mathpar}

\end{abstrsyn}
\caption{\lang~Dynamic Semantics.  In residualization rules, $\mathcal{C}$ can stand for \texttt{pi1}/\texttt{pi2}/\texttt{inl}/\texttt{inr}/\texttt{roll}/\texttt{unroll}.}
\label{fig:diaSemantics}
\end{figure*}


The algorithm described above creates three different kinds of expressions which
cannot be evaluated further at a particular stage:
\begin{itemize}
\item 
Partial values ($\pvalsym$s) are stage \bbone\ terms that have been fully evaluated, 
but which may contain stage-\bbtwo\ variables wrapped in $\next$ blocks. 
In the example above, 
\verb|(|$\next \{\mathtt{\hat y}\}$\verb|,7)|.

\item Residuals ($\ressym$es) are \langTwo\ terms---stage \bbtwo\ terms whose
stage \bbone\ subexpressions have all been fully evaluated. In the example
above,
\verb|(1+2)| and \verb|(let z = y*y*7 in z)|.

\item Values ($\valsym$s) are \langTwo\ terms which are fully evaluated; these
are the results of a computation after both stages have been completed. In the
example above, the term evaluates to \verb|63|.
\end{itemize}

% would like some kind of intro here that anticipates all the complexity.
% The dynamics of \lang\ consists of three types of judgements: $\redonesym$ (stage-\bbone reduction), $\redtwosym$ (speculation), and $\reifysym$ (reification).

The $\redonesym$ judgment takes an open stage \bbone\ term to a {\em future
environment} $\xi$ and a partial value $v$.  The future environment is a mapping
from fresh stage \bbtwo\ variables (which may appear inside $\next$ blocks in
$v$) to residuals---in our example above, we represented this environment with
explicit substitutions. For all of the normal features of \lang\ 
(\ref{fig:diaSemanticsCore}), first stage evaluation has the same behavior and
effect on (partial) values as does standard evaluation, and the final future environment is
gotten by merging the future environments of the subterms.

When the \bbone\ judgment encounters a $\next$ block (\ref{fig:diaSemanticsNP}), it
searches into the block's stage \bbtwo\ content to find any contained stage
\bbone\ subexpressions and evaluate them in place.  This search process, called
\emph{speculation}, is implemented by the $\redtwosym$ judgment, which takes
a stage \bbtwo\ term to a residual.  Once the contents of the $\next$ block are
speculated into a residual ($q$), the output of $\redonesym$ is a fresh variable wrapped in a
$\next$ block ($\next~\hat y$), along with a future environment which maps that
variable to the residual ($\hat y \mapsto q$).

At all of the normal features (\ref{fig:diaSemanticsSpec}), speculation does
nothing but recursively speculate into every subexpression.  Once speculation
finds a $\prev$ block, it resumes stage \bbone\ evaluation of the contents, which
produces a future context and (by canonical forms) a $\next$-wrapped variable
($\next\{\hat y\}$).  The context is then reified (using the $\reifysym$
judgment) into a series of let bindings with $\mathtt{\hat y}$, stripped of
its $\next$, at the bottom.

Within speculation lies a subtle---if perhaps unintuitive---feature.  
Observe that speculation will traverse into both branches of a stage-\bbtwo\ {\tt if} or {\tt case} 
statement in its search for stage-\bbone\ code. 
Thus the evaluation of that stage one code will occur {\em regardless of the eventual value of the predicate},
and so a term like 
\begin{lstlisting} 
next{
  if true 
  then hold{1+2} 
  else prev{spin() (* loops forever *)}
}
\end{lstlisting}
will fail to evaluate at stage \bbone.
This behavior is why the judgment is named ``speculation."

The context ($\Gamma$) keeps track of stage \bbtwo\ variables in the input term. 
These both appear in the original program at stage \bbtwo\ and are inserted by the semantics.

As an optimization, we can include the special-case rule,
\begin{mathpar}
\inferdiaone [hat] {\red {\next~\hat y}{\cdot,\next~\hat y}}{\cdot}
\end{mathpar}
to avoid one-for-one variable bindings in the residual.
We used this implicitly in the example in the previous section.

If we change the $\next$ and $\prev$ rules to 
\begin{mathpar}
\infer {\diaone {\next~e}{\cdot,\next~q}} {\diatwo e q} \and
\infer {\diatwo{\prev~e}{v}} {\diaone e {\cdot,\next~v}} 
\end{mathpar}
and treat $\next~q$ as a partial value for any residual $q$,
then our semantics becomes rule-for-rule isomorphic to that from \cite{davies96}. This essentially bypasses the
environment bookkeeping in $\redonesym$, by inlining residuals instead of
hoisting them in \verb|let|-bindings.
From this it's clear that the semantics of \cite{davies96} and ours always produce the same value when they both terminate.
{\em Because \lang's semantics dictate that a reified residual will always be evaluated, regardless of whether its result is consumed, \lang\ programs terminate strictly less often than that of \cite{davies96}}.

Returning to our {\tt fexp} example, we can speculate on the stage \bbtwo\ term
$\verb|fn b => prev{fexp(next{b},10)}|$
to get the residual
\begin{lstlisting} 
fn b : int =>
  let x0 =
    (let x1 = 
      let x2 = 1 * b
      in x2 * x2
    in x1 * x1) * b
  in x0 * x0
\end{lstlisting}

Note how the recursion causes duplicate code in the residual.

\subsection {A Partial Evaluation System}
\label{sec:partialeval}

The dynamics described in the previous section provide a formal description of partial evaluation.
Recall that in partial evaluation systems, we start with a multivariate function $f$ for which some input is labeled {\em static} and 
some labeled {\em dynamic}.  Once the static input is provided, partial evaluation of $f$ produces
a residual that depends on only the dynamic input.  Equationally, a partial evaluator is any $p$ such that
\[
	\forall f,x. \exists f_x. [p(f,x) = f_x \text{ and } \forall y.\llbracket f \rrbracket(x,y)=\llbracket f_x \rrbracket(y)]
\]
where $\llbracket \cdot \rrbracket$ translates the text of a function to it's mathematical interpretation (a la \cite{jones96}).
Here, $x$ is the static input, $y$ is the dynamic input, and $f_x$ is the residual, also called ``f specialized to x."

% KAYVONF: good statement, but hold out for now
%The hope of partial evaluation is that $f_x$ is cheaper to execute than $f$, meaning that we can save work if we must %evaluate it many times.

By identifying static with stage \bbone\ and dynamic with stage \bbtwo, 
our dynamics can serve as a partial evaluator.   
Specifically, we encode $f$ as a \lang\ expression with a function type of the form $A\to\fut(B\to C)$
\cprotect\footnote{We can rewrite \texttt{fexp} in this form, or simply apply
the following higher-order function which makes the adjustment:
\begin{lstlisting} 
let adjust (f : $int * int -> $int) =
  fn (p : int) => 
    next{
      fn (b : int) => 
        prev{f (next {b}, p)}
    }
\end{lstlisting}}.
%
Here $A$ represents the static input, $B$ represents the dynamic input, and $C$ represents the output.

Once a stage \bbone\ argument $a:A$ is provided, we can evaluate the partially-applied
function:
$\cdot\vdash f~a \mathop{\redonesym} [\xi,v]$.
The result is an environment $\xi$ and a partial value $v$ of type $\fut(B\to
C)$, which by canonical forms must have the form $v = \next~\hat y$. 
Next, we reify this environment into a sequence of \verb|let|-bindings
enclosing $\hat y$, via $\reify\xi{\hat y}{f_a}$. 
Because reification preserves types, the resulting residual $f_a$ has type $B\to C$ in \langTwo, so we can apply it to some $b:B$
and compute the final result of the function, $f_a~b \mathop{\tworedsym} c$.

That this sequence of evaluations is in fact staged follows from our
characterizations of partial values, residuals, and values, that $\redonesym$
outputs a partial value, and that $\reifysym$ outputs an expression in \langTwo.

%\begin{remark}
%For any $\colone{e}{A}$ containing no $\next$ subexpressions, $\redonesym$ will
%always compute an empty environment, and a partial value identical to the result
%of call-by-value evaluation of $e$.
%%derivationally equivalent to standard call-by-value evaluation.
%\end{remark}

%\subsection{Metatheory}
%
%Recall that residuals live in \langTwo; we will indicate typing judgments in
%\langTwo\ with $\vdash_\bbtwo$.
%
%%\begin{definition}
%%Context $\Gamma$ is well-formed ($\Gamma\wf$) if it
%%contains only stage-2 variables.
%%\end{definition}
%
%\begin{definition}
%An environment $\xi$ is well-formed ($\Gamma\vdash\xi\wf$) if either:
%\begin{enumerate}
%\item $\xi = \cdot$; or
%\item $\xi = \xi',x:B\mapsto e$ where
%$\Gamma\vdash\xi'\wf$ and
%$\typeslangTwo[\Gamma,\dom{\xi'}] e B$
%%$\Gamma,\dom{\xi'}\vdash \coltwo{e}{B}$ and
%%$\Gamma,\dom{\xi'}\vdash e \res$.
%\end{enumerate}
%\end{definition}
%
%\begin{theorem}
%If $\typeswor e A$ then $\Gamma\wf$ and $A\istypewor$.
%\end{theorem}
%
%\begin{theorem}
%If $\diaonesub$ and $\typesone e A$ then
%\begin{enumerate}
%\item $\Gamma\vdash\xi\wf$;
%\item $\Gamma,\dom\xi\vdash \colone{v}{A}$; and
%\item $\Gamma,\dom\xi\vdash v\pval$.
%\end{enumerate}
%\end{theorem}
%
%\begin{theorem}
%If $\diatwosub$ and $\typestwo e A$ then
%\begin{enumerate}
%\item $\typeslangTwo q A$; and
%\item $\Gamma\vdash_\bbtwo q\val$.
%\end{enumerate}
%\end{theorem}
%
%\begin{theorem}\label{thm:reify-type}
%If $\Gamma\vdash\xi\wf$ and
%$\Gamma,\dom\xi\vdash \colone{\next\ \hat y}{\fut A}$
%then 
%$\reify{\xi}{\hat y}{q}$ and
%$\typeslangTwo q A$.
%\end{theorem}

%\TODO
%Note somewhere how to run stage-one non-$\fut A$ terms. For example, a stage-one
%integer term is guaranteed not to depend on the table, although one might be
%produced. One may either discard the table, or evaluate everything in the table
%(and terminating with the partial value iff everything in the table terminates).




\section{Splitting Algorithm}
\label{sec:splitting}

The basic idea is that we want to take a \lang\ term which contains interleaved stage \bbone\ and stage \bbtwo\ code
and split it into two separate terms: one with all of the stage \bbone\ code and one with all the stage \bbtwo\ code.
To facilitate communication between the stages, there is a data structure passed between the stages, 
which is an output of the stage \bbone\ term and and input to the stage \bbtwo\ term.

As with the dynamics, the form of the splitting statement depends on the external stage of the term.
We consider splitting for stage \bbtwo\ terms first, as it has a simpler form.  
For example, take \verb|hold{1+2} < 5|.
We can split this into two separate terms, \verb|1+2| and \verb|fn l => l < 5|, 
which are called the {\em precomputation} and {\em residual}\footnote{We also used the term {\em residual} for the result of partial evaluation.
Where necessary, we will distinguish between these concepts by using the terms {\em splitting-residual}
and {\em evaluation-residual}.}, respectively.  

The form of splitting for stage \bbone\ terms is more complicated.
To see why, consider the term \verb|(next{hold{1+2} < 5},3*4)|.
As before, the comparison operation is part of stage \bbtwo, 
and the addition operation is a stage \bbone\ precomputation, the result of which will eventually become input to the residual.
The multiplication operation, however, is neither part of the residual nor a precomputation to support it.
Instead it is the {\em immediate result} of stage \bbone, 
because it is available to the stage \bbone\ context around our original term.
For example:
\begin{lstlisting}
let x = (next{hold{1+2} < 5},3*4) in
next{ if prev{#1 x} then 7 else hold {#2 x} }
\end{lstlisting}
To support this, stage \bbone\ splitting produces two outputs: the {\em combined result} and the residual,
where the combined result reduces to a tuple containing the immediate result and the precomputation.

We must also develop a notion of what it means for splitting to be correct.  
Our general approach is to say that the dynamic semantics from \ref{sec:semantics} and splitting method are equivalent in some way.  
In general, if a multi-stage program $P$ splits into stage-\bbone\ part $P_1$ and stage-\bbtwo\ part $P_2$, then
\begin{enumerate}
\item Partial evaluation of $P$ will terminate if and only if evaluation of $P_1$ terminates, and if so then
\item the immediate results of each method will be equivalent, and
\item the evaluation-residual and the splitting-residual ($P_2$ bound under the precomputation) are equivalent.
\end{enumerate}

This is straight-forward, given a careful definition of equivalence.


\subsection{Value Splitting}

Before getting into the formal presentation, 
we first consider the problem of splitting stage-\bbone\ partial values.
This serves as a didactic stepping-stone to full term-splitting, 
and it is a necessary component of the correctness theorem.

The rules for value splitting are given in \cref{fig:valSplit}.  It consists of a single judgment, $v \vsplito [v_1,v_2]$,
which sends a partial value $v$ to two single-stage values $v_1$ and $v_2$. 
Those respectively represent the stage-\bbone\ and stage-\bbtwo\ content of the original value.  
At base types, value splitting assigns all content to stage \bbone, and puts in a nullary tuple for stage two,
in accordance with the precept that base types have only trivial content at stage \bbtwo.
Splitting essentially distributes into values at product, sum, and recursive types.
Functions split into two other functions.  Since the body of a function is an open term, this rule must rely on 
general term splitting, which is covered momentarily.

Lastly, we consider value splitting at the $\fut$ type.  The only partial value with this type
is a $\next$ block containing a reference into the future environment.  This splits into a 
nullary tuple and the same variable, again aligning with the precept that $\fut$ types contain trivial stage-\bbone\
content.

\subsection {Formal Setup}

The splitting algorithm comprises two judgments, $\splitonesym$ and $\splittwosym$.
For both, the input is a term in \lang, and the output is two terms in \langmono, an
unstaged language.  The grammar for \langmono\ is given in \ref{fig:monoGrammar}.  It has no
staging features (we say that implicitly it has only one stage), and it is untyped.  
Modulo these differences, \langmono\ has all of the same features as \lang, 
although this is a matter of taste
\footnote{Since \langmono\ is untyped, we could encode everything with just functions.}.

Specifically, the $\splitonesym$ judgment sends a stage \bbone\ term ($e$) to a combined term ($c$) and residual ($l.r$),
while $\splittwosym$ sends a stage \bbtwo\ term ($e$) to a precomputation ($p$) and a residual ($l.r$).
For concision, we represent residuals as a term open on a single variable, rather than as functions.
For both judgments, we use a context ($\Gamma$) to keep track of the open variables of $e$.

The rules\footnote{The rules are written using patterns, including the open variable of the residual.} 
for splitting $\next$, $\prev$, and $\pause$ are given in \ref{fig:termSplitOne}.
The rule for $\next$ simply tuples up the precomputation of its subexpression with a trivial immediate result,
while the rule for $\prev$ projects the combined result of its subexpression to just the precomputation.
The $\pause$ rule treats the entire combined result of its subexpression as a precomputation, 
and projects out the integer result in the residual\footnote{The residual of an integer expression is usually trivial, 
but we have to include it here for termination purposes.}

The rules of stage \bbtwo\ splitting are given in \ref{fig:termSplitTwo}.  
Every rule works by bundling the precomputations of the constituent parts, and then unbundling them with a pattern.
The rules of stage \bbone\ splitting are given in \ref{fig:termSplitOne}.  


\subsection {Speculation}

Notice in the stage \bbtwo\ rules for {\tt if}s, {\tt case}s, and functions, the precomputation is lifted out from within branches.
This is the manifestation of the speculation behavior from the semantics.

\subsection {Stage \bbone\ Divergence}
Consider the splitting rules for stage-\bbone\ {\tt if} and {\tt case} expressions.
In both cases we have enough information at stage \bbone\ to know what branch to take, so there's no need to speculate.
Instead, we evaluate the stage \bbone\ portion of only the active branch, and then inject precomputation into a sum type.
Then in the residual, we case on that sum and resume the stage \bbtwo\ portion of the correct branch.

\subsection {Stage \bbone\ Functions}

The trick when splitting stage \bbone\ functions is that the contents themselves may be multi-stage.
We handle this by splitting them into two functions:
one in the immediate result which handles all the stage \bbone\ content of the original,
and one in the residual which handles all of the stage \bbtwo\ content.
Note that stage \bbone\ $\lambda$-expressions themselves have only a trivial precomputation.

%\subsection {Boundary Type Worst Case}
%
%\TODO add the example which is the worst case for figuring out boundary types
%

%!TEX root = ../paper.tex

\begin{figure*}
\begin{abstrsyn}
\begin{mathpar}
\fbox{Partial Values} \and 
\infer {x 					\vsplito \mval x x}														{\cdot}												\and
\infer {\tup{} 				\vsplito \mval {\tup{}} {\tup{}}}										{\cdot}												\and
\infer {\pure m 			\vsplito \mval m {\tup{}}}												{\cdot}												\and
\infer {\next y 			\vsplito \mval {\tup{}} y}												{\cdot}												\and
\infer {\tup{v_1,v_2} 		\vsplito \mval {\tup{i_1,i_2}} {\tup{q_1,q_2}}}							{v_1 \vsplito \mval {i_1} {q_1} 
																									&v_2 \vsplito \mval {i_2} {q_2}}					\and
\infer {\fix f x e			\vsplito \mval {\fix f x {\letin {\tup{x,y}} c {\tup{x,\roll y}}}} 
							{\fix f {\tup{x,\roll l}} r}}											{\splitone e A c {l.r}}								\and
\end{mathpar}
\hrule
\begin{mathpar}
\fbox{\bbonem\ Terms} 
\and
%\infersplitone [unit]			{\spl {\tup{}}{\rmunit}{\tup{\tup{},\tup{}}}{\_.\tup{}}}											{\cdot} \and
%\infersplitone [int]			{\spl {i}{\rm int}{\tup{i,\tup{}};\_.\tup{}}}														{\cdot} \and
%\infersplitone [hyp]			{\spl {x}{A}{\tup{x,\tup{}}}{\_.x}}																	{\cdot} \and
% \infersplitone [\to I]			{\splitonetall {\lam{x}{e}}{A \to B}{\tup{\lam x c, \tup{}}}{\_.\tup{\lam {\tup{x,l}} r}}}			{\sub [\Gamma,\col x A] {} B} 							\and
% \infersplitone [fix]			{\splitonetall 	
% 									{\fix{f}{x}{e}} {} 
% 									{\tup{\fix f x {\letin{\tup{y,z}}c{\tup{y,\roll z}}}, \tup{}}}
% 									{\_.\fix f {\tup {x,l_0}} {\letin{l}{\unroll {l_0}}{r}}}}										{\sub [\Gamma,\col x A] {} B} 							\and
\infersplitone 					{\spl {\exv v} A {\exv{\tup{i,\tup{}}}} {\_.\exv q}} {v \vsplito \mval i q} \and 
\infersplitone [common1]		{\spl {\scriptCapp e}{A}{\letin{\tup{y,z}}{c}
									{\tup{\scriptCapp {\exv y},\exv z}}}{l.\scriptCapp r}}	{\sub {} {A} 
																																	& \scriptC \in \{\mathtt{pi1},\mathtt{pi2}\}} 			\and
\infersplitone [\to E]			{\splitonetall {\app {e_1}{e_2}}{B}{\left(
									\talllet{\tup{y_1,z_1}}{c_1}{
									\talllet{\tup{y_2,z_2}}{c_2}{
									\talllet{\tup{y_3,z_3}}{\app{y_1}{y_2}}{\exv{\tup{y_3,\tup{z_1,z_2,z_3}}}\ttrpar\ttrpar\ttrpar}}}
									\right)}
									{\tup{l_1,l_2,l_3}.\app{\exv {r_1}}{\exv{\tup{r_2,l_3}}}}}										{\sub 1 {A \to B} & \sub 2 A}							\and
% \infersplitone [common2]		{\spl {\scriptCapp e}{A}{\letin{\tup{y,z}}{c}{\tup{\scriptCapp y,z}}}{l.r}}							{\sub {} {A} 
% 																																	& \scriptC \in \{\mathtt{inl},\mathtt{inr},
% 																																	\mathtt{roll},\mathtt{unroll}\}} 						\and
%\infersplitone [\times E_2]	{\spl {\pit~e}{B}{\talllet{\tup{y,z}}{c}{\tup{\pit~y,z}};l.\pit~r}}								{\sub {} {A\times B}} \and
%\infersplitone [let]			{\spl {\letin{x}{e_1}{e_2}}{B}{
%								\begin{array}{l}
%									\left(\talllet{\tup{x,z_1}}{c_1}{
%									\talllet{\tup{y,z_2}}{c_2}{
%									\tup{y, \tup{z_1, z_2}}}}\right);
%									\tup{l_1,l_2}.\letin{x}{r_1}{r_2}
%								\end{array}}}																					{\sub 1 A & \sub [\Gamma,\col{x}{A}] 2 B} 		\and
%\infersplitone [+ I_1]	{\spl {\inl~e}{A+B}{\talllet{\tup{y,z}}{c}{\tup{\inl~y,z}};l.r}}			{\sub {} A}												\and
%\infersplitone [+ I_2]	{\spl {\inr~e}{A+B}{\talllet{\tup{y,z}}{c}{\tup{\inr~y,z}};l.r}}			{\sub {} A}												\and
\infersplitone [\times I] 		{\splitonetall {\tup{e_1,e_2}}{A\times B}
									{\left(
										\talllet{\tup{y_1,z_1}}{c_1}{
										\talllet{\tup{y_2,z_2}}{c_2}{
										\exv{\tup{\tup{y_1, y_2},\tup{z_1, z_2}}}\ttrpar\ttrpar}}
									\right)}
									{\tup{l_1,l_2}.\tup{r_1,r_2}}}																	{\sub 1 A & \sub 2 B} \and
\infersplitone [+ E]		{\splitonetall {\caseP{e_1} {x_2.e_2} {x_3.e_3}}{C}
								{\left(
								\talllet{\tup{y_1,z_1}}{c_1}{
									\tallcase{y_1}
									{x_2.\letin{\tup{y_2,z_2}}{c_2}{\exv {\tup{y_2,\tup{z_1,\inl{z_2}}}}}}
									{x_3.\letin{\tup{y_3,z_3}}{c_3}{\exv {\tup{y_3,\tup{z_1,\inr{z_3}}}}}\ttrpar}
								}\right)}
								{\tup{l_1,l_b}.{\ttlpar r_1 \ttsemi \caseof{l_b}{\exv{l_2}.[\tup{}/x_2]{r_2}}{l_3.[\tup{}/x_3]{r_3}\ttrpar}
								}}}																									{\sub 1 {A+B} 
																																	& \sub [\Gamma,\col{x_2} A] 2 C 	
																																	& \sub [\Gamma,\col{x_3} B] 3 C} 	
%
%\infersplitone [\mu I]		{\spl {\roll~e}{\mu\alpha.\tau}{\talllet{\tup{y,z}}{c}{\tup{\roll~y,z}};l.r}}						{\sub {} {[\mu\alpha.\tau/\alpha]\tau}}							\and
%\infersplitone [\mu E]		{\spl {\unroll~e}{[\mu\alpha.\tau/\alpha]\tau}{\talllet{\tup{y,z}}{c}{\tup{\unroll~y,z}};l.r}}	{\sub {} {\mu\alpha.\tau}}										\and
%
%\infersplitone [if]			{\spl {\left( \tallif {e_1}{e_2}{e_3} \right)}{A} 
%								{\begin{array}{l}
%									\left(\talllet{\tup{y_1,z_1}}{c_1}{\tallif {y_1}
%										{\letin{\tup{y_2,z_2}}{c_2}{\tup{y_2, \tup{z_1, \inl~z_2}}}}
%										{\letin{\tup{y_3,z_3}}{c_2}{\tup{y_3, \tup{z_1, \inl~z_3}}}}}
%									\right); \\
%									\tup{l_1,l_b}.\tup{r_1;\caseof{l_b}{l_2.r_2}{l_3.r_3}}
%								\end{array}}}																					{\sub 1 \rmbool & \sub 2 A & \sub 3 A }		\and
%
%\infersplitone	{\spl {\liftint~e} ? {c;l.r}}											{\sub {} ?}					
%
\end{mathpar}
\hrule
\begin{mathpar}
\fbox{\bbtwo\ Terms} 
\and
%\infersplittwo [unit]			{\spl {\tup{}} {\rmunit} {\tup{}} {\_}{\tup{}}}												{\cdot} 														\and
%\infersplittwo [int]			{\spl {i}{\rm int}{\tup{};\_}{i}}															{\cdot} 														\and
%\infersplittwo [hyp]			{\spl {x}{A}{\tup{}}{\_}{x}}																{\cdot} 														\and
\infersplitone 					{\spl {\exv q} A {\exv {\tup{}}} {\_. q}} 													{v \vsplito \mval i q} 											\and 
\infersplittwo [\times E_1]		{\spl {\scriptCapp e}{A}{p}{l}{\scriptCapp r}}												{\sub {} {A\times B} 
																															& \scriptC \in \{\mathtt{pi1},\mathtt{pi2},\mathtt{inl},
																															  \mathtt{inr},\mathtt{roll},\mathtt{unroll}\}} 				\and
%\infersplittwo [\to I]			{\spl {\lam{x}{e}}{A \to B}{p}{l}{\lam{x}{r}}}												{\sub [\Gamma,\col x A] {} B} 									\and
\infersplittwo [\to I]			{\spl {\fix fxe}{A \to B}{p}{l}{\fix fxr}}													{\sub [\Gamma,\col x A] {} B} 									\and
\infersplittwo [\to E]			{\spl {\app{e_1}{e_2}}{B}{\tup{p_1,p_2}}{\tup{l_1,l_2}}{\app{r_1}{r_2}}}					{\sub 1 {A \to B} & \sub 2 A}									\and			
\infersplittwo [let]			{\spl {\letin{x}{e_1}{e_2}}{B}{\tup{p_1,p_2}} {\tup{l_1,l_2}} {\letin{x}{r_1}{r_2}}}		{\sub 1 A & \sub [\Gamma,\col{x}{A}] 2 B} 						\and
\infersplittwo [\times I]		{\spl {\tup{e_1,e_2}}{A\times B}{\tup{p_1,p_2}}{\tup{l_1,l_2}}{\tup{r_1,r_2}}}				{\sub 1 A & \sub 2 B} 											\and
%\infersplittwo [\times E_2]	{\spl {\pit~e}{B}{p;l}{\pit~r}}																{\sub {} {A\times B}} 											\and
%\infersplittwo [+ I_1]			{\spl {\inl~e}{A+B}{p;l}{\inl~r}}															{\sub {} A}														\and
%\infersplittwo [+ I_1]			{\spl {\inr~e}{A+B}{p;l}{\inr~r}}															{\sub {} A}														\and
%\infersplittwo [\mu I]			{\spl {\roll~e}{\mu\alpha.\tau}{p;l.\roll~r}}												{\sub {} {[\mu\alpha.\tau/\alpha]\tau}}							\and
%\infersplittwo [\mu E]			{\spl {\unroll~e}{[\mu\alpha.\tau/\alpha]\tau}{p;l.\unroll~r}}								{\sub {} {\mu\alpha.\tau}}										\and
\infersplittwo [+ E]			{\spl {\caseof{e_1}{x_2.e_2}{x_3.e_3}}{C}
								{\tup{p_1,p_2,p_3}}{ \tup{l_1,l_2,l_3}}{\caseof{r_1}{x_2.r_2}{x_3.r_3}}}					{\sub 1 {A+B} & \sub [\Gamma,\col{x_2} A] 2 C 
																															& \sub [\Gamma,\col{x_3} B] 3 C} 								
%\infersplittwo [if]			{\spl {\left(\tallif {e_1}{e_2}{e_3}\right)}{A} 
%								{\tup{p_1,p_2,p_3}}{\tup{l_1,l_2,l_3}.\left(\tallif{r_1}{r_2}{r_2}\right)}}						{\sub 1 \rmbool & \sub 2 A & \sub 3 A }		\and
\end{mathpar} 
\hrule
\begin{mathpar}
\fbox{Staging features}
\and
\infersplittwo [\fut E]		{\spl {\prev e}{A} {\pit c} l r }													{\splitonesub {} {\fut A}} 		\and
\infersplitone [\fut I]		{\spl {\next e}{\fut A}{\tup{\tup{},p}}{l.r}}										{\splittwosub {} A} 			\and
%\infersplittwo [hold]		{\spl {\pause e}{\rmint}{c,\tup{a,l}.\tup{r;a}}}									{\splitonesub {} \rmint}		\and
\infersplitone				{\spl {\pure e} {\curr A} {\tup{e,\tup{}}}{\_.\tup{}}}								{\cdot}							\and
\infersplitone				{\splitonetall {\letp x{e_1}{e_2}} {?} 
							{\letin {\tup{x,z_1}} {c_1} {
							 \letin {\tup{y_2,z_2}} {c_2} {\tup{y_2,\tup{z_1,z_2}}}}}
							 {\tup{l_1,l_2}.\letin{\_}{r_1}{r_2}}}												{\sub 1 ? & \sub 2 ?}			
%%\infersplitone				{\spl {\lifttag e} ? {c}{l.r}}													{\sub {} ?}						
\end{mathpar}
\end{abstrsyn}
\caption{Splitting rules.}
\label{fig:termSplit}
\end{figure*}



\subsection{Metatheory}

We start with two mutually dependent definitions of equivalence.  
Both relate evaluation-residuals on the left with splitting residuals on the right,
but $\equiv$ operates on terms, whereas $\sim$ on values.

\begin{definition}
For evaluation-residual $q$ and splitting-residual $r$, define $q \equiv r : A$ to mean that 
$q \tworedsym v_q$ iff $\reduce {r} {v_r}$ where $v_q \sim v_r : A$, 
\end{definition}

\begin{definition}
For evaluation-residual value $v_1$ and splitting-residual value $v_2$, define $v_1 \sim v_2 : A$ by the following cases:
\begin{itemize}
\item $i \sim i : \rmint$
\item $(v_1,u_1) \sim (v_2,u_2) : A \times B$ where $v_1 \sim v_2 : A$ and $u_1 \sim u_2 : B$
\item $\inl~v_1 \sim \inl~v_2 : A + B$ where $v_1 \sim v_2 : A$
\item $\inr~v_1 \sim \inr~v_2 : A + B$ where $v_1 \sim v_2 : B$
\item $\lam {x_1} A {e_1} \sim \lambda x_2.e_2 : A \to B$ where \\ $\forall (v_1 \sim v_2 : A). [v_1/x_1]e_1 \equiv [v_2/x_2]e_2 : B$
\end{itemize}
\end{definition}

Essentially, we can read these as saying that two terms are equivalent if they evaluate to the same value,
where "same" for functions means that those functions always evaluate to the same thing given equivalent inputs.

We give the following end-to-end correctness lemmas for open terms. 
It's a bit of a mess currently, but the $\Gamma$ is supposed to be all of the stage \bbone\ bindings, 
whereas $\Gamma'$ is the stage \bbtwo\ bindings.
Substitution splitting works just like value splitting, which is why they use the same symbol.

The jury is still out on how to make these strong enough to prove that the partitioning between stages is correct.

\begin{lemma}
If $\typesone e A$ then
$\Gamma\vdash e : A \splitonesym [c,l.r]$.
If $\typestwo e A$ then
$\Gamma\vdash e : A \splittwosym [p,l.r]$.
\end{lemma}

\begin{lemma}
If $\Gamma, \Gamma'\vdash e : A \splittwosym [p,l.r]$ then for all substitutions $\gamma : \Gamma$,
\begin{itemize}
\item $\gamma \vsplito [\gamma_1, \gamma_2]$
\item $\diatwo [\Gamma'] {\gamma(e)} q$ iff $\reduce {\gamma_1(p)} u$ where
\item $\Gamma' \vdash q \equiv (\letin{l}{u}{\gamma_2(r)})$
\end{itemize}
\end{lemma}

\begin{lemma}
If $\Gamma, \Gamma'\vdash e : A \splitonesym [c,l.r]$ then for all $\gamma : \Gamma$,
\begin{itemize}
\item $\gamma \vsplito [\gamma_1, \gamma_2]$
\item $\diaone [\Gamma'] {\gamma(e)} {\xi;v}$ iff $\reduce {\gamma_1(c)} {(v_1,u)}$ where
\item $\Gamma',\dom \xi \vdash v \vsplito [v_1,v_2]$
\item $\reify \xi {v_2} q$
\item $\Gamma'\vdash q \equiv (\letin{l}{u}{\gamma_2(r)})$
\end{itemize}
\end{lemma}

%You should think of these theorems as saying that 
%splitting commutes with evaluation.
These lemmas are rather technical, but they ultimately imply that evaluating a
closed term by splitting or by the dynamic semantics of \ref{ssec:dynamics} are
equivalent. 
We state this result for closed terms at each stage.

\begin{theorem}[Correctness of splitting at $\bbone$]
If $\vdash e:A~@~\bbone$, then (by splitting)
\begin{itemize}
\item $\vdash e : A \splitonesym [c,l.r]$
\item $\reduce {c} {(v_1,u)}$
\item $\reduce {(\letin{l}{u}{r})} v_S$
\end{itemize}
if and only if (by the staged dynamic semantics)
\begin{itemize}
\item $\diaone [] e {\xi;v}$
\item $\dom \xi \vdash v \vsplito [v_1,v_2]$
\item $\reify \xi{v_2}q$
\item $q \mathbin{\tworedsym} v_D$
\end{itemize}
and if so, then $v_D \sim v_S$.
\end{theorem}

\begin{theorem}[Correctness of splitting at $\bbtwo$]
If $\vdash e:A~@~\bbtwo$, then (by splitting)
\begin{itemize}
\item $\vdash e : A \splittwosym [p,l.r]$
\item $\reduce p u$
\item $\reduce{(\letin{l}{u}{r})}{v_S}$
\end{itemize}
if and only if (by the staged dynamic semantics)
\begin{itemize}
\item $\diatwo [] e q$
\item $q \mathbin{\tworedsym} v_D$
\end{itemize}
and if so, then $v_D \sim v_S$.
\end{theorem}

If we apply the former theorem to \verb|next{e}| of type $\fut A$, we
essentially obtain the latter theorem at \verb|e|.

The latter theorem implies that, given a multi-stage function $f:A\to\fut(B\to
C)~@~\bbone$, the two methods of evaluating \verb|prev{f a} b| agree.
However, we also expect that splitting $f$ directly will give us two functions,
one which accepts an $A$ and outputs an intermediate value and boundary data,
and another which takes in that boundary data and a $B$ and outputs a $C$.
Moreover, the composition of these two functions should be extensionally equal
to the staged dynamic semantics.

\TODO write the theorem for $\vdash f:A\to\fut(B\to C)~@~\bbone$.

%
%\subsubsection{Simple Types}
%
%\begin{theorem}
%If $\cdot\vdash e : A \splittwosym [p,l:\tau.r]$ then,
%\begin{itemize}
%\item $\cdot \vdash e : A~@~\bbtwo$ 
%\item $\types [\cdot] p \tau$ and $\types [l:\tau] r A$ 
%\item $\diatwo [\cdot] e q$ iff $\reduce p u$ and if so
%\item $q \equiv (\letin{l}{u}{r})$
%\end{itemize}
%\end{theorem}
%
%\begin{theorem}
%If $\cdot\vdash e : A \splitonesym [c,l:\tau.r]$ then,
%\begin{itemize}
%\item $\typesone [\cdot] e A$ 
%\item $A \tsplito [A_1,A_2]$
%\item $\types [\cdot] c {A_1 \times \tau}$ and $\types [l:\tau] r A_2$ 
%\item $\diaone [\cdot] e {\xi;v}$ iff $\reduce c {(v_1,u)}$ and if so
%\item $\dom \xi \vdash v \vsplito [v_1,v_2]$
%\item $\reify \xi {v_2} q$
%\item $q \equiv (\letin{l}{u}{r}) : A_2$
%\end{itemize}
%\end{theorem}
%
%\subsubsection{\bbone-Dependent Types}
%\begin{theorem}
%If $\cdot\vdash e : A \splitonesym [c,l:\tau.r]$ then,
%\begin{itemize}
%\item $\typesone [\cdot] e A$ 
%\item $A \tsplito [A_1,a.A_2]$
%\item $\types [\cdot] c {A_1 \times \tau}$ and $\types [l:\tau,a:A_1] r A_2$ 
%\item $\diaone [\cdot] e {\xi;v}$ iff $\reduce c {(v_1,u)}$ and if so
%\item $\dom \xi \vdash v \vsplito [v_1,v_2]$
%\item $\reify \xi {v_2} q$
%\item $q \equiv (\letin{l}{u}{r}) : A_2$
%\end{itemize}
%\end{theorem}

%!TEX root = ../paper.tex

%\begin{figure*}
%\caption{Type Splitting}
%\label{fig:typeSplit}
%\begin{mathpar}
%\infer [\mathrm{unit}\tsplito]	{\rmunit \tsplito [\rmunit,\rmunit]}{\cdot} \and
%\infer [\mathrm{int}\tsplito]	{{\rm int} \tsplito [{\rm int}, \rmunit]}{\cdot} \and
%\infer [\mathrm{bool}\tsplito]	{{\rm bool} \tsplito [{\rm bool}, \rmunit]}{\cdot} \and
%\infer [\times\tsplito]
%	{A\times B \tsplito [A_1 \times B_1, A_2 \times B_2]}
%	{A \tsplito [A_1, A_2] & B \tsplito [B_1,B_2]} \and
%\infer [\fut\tsplito]
%	{\fut~A \tsplito [{\rm unit}, A']}
%	{A \tsplits A'} \and
%\end{mathpar}
%\end{figure*}
%

% \[\begin{array}{l|l|l}
% [\xi;v] & \masko{\xi;v} & \maskt{\xi;v} \\ \hline
% [\hat y \mapsto q,\xi;v]
%   & \masko{\cdot;v}
%   & \letin {\hat y} q {\maskt{\xi;v}} \\\relax
% [\cdot;()] 
%   & ()
%   & () \\\relax
% [\cdot;\monoTerm~v] 
%   & v
%   & () \\\relax
% [\cdot;\next~\hat y] 
%   & ()
%   & \hat y \\\relax
% [\cdot;(v_1,v_2)] 
%   & (\masko {\cdot;v_1},\masko {\cdot;v_2})
%   & (\maskt {\cdot;v_1},\maskt {\cdot;v_2}) \\\relax
% [\cdot;\inl~v] 
%   & \inl~\masko{\cdot;v}
%   & \maskt {\cdot;v} \\\relax
% [\cdot;\inr~v] 
%   & \inr~\masko{\cdot;v}
%   & \maskt {\cdot;v} \\\relax
% [\cdot;\roll~v] 
%   & \roll~\masko{\cdot;v}
%   & \maskt {\cdot;v} \\\relax
% [\cdot;\lam xAe]
%   & \lambda x.c 
%   & \lambda (x,l).r \\
% & \text{ for } \splitone e A {c,l.r}
% & \text{ for } \splitone e A {c,l.r}
% \end{array}\]

\begin{figure}
\begin{abstrsyn}
\[\begin{array}{rll}
[\hat y \mapsto q,\xi;v]
  &\vsplito& \mval 
  {i}
  {\letin {\hat y} q {r}} \\\relax
[\cdot;()]
  &\vsplito& \mval 
  {()}
  {()} \\\relax
[\cdot;\monoTerm{m}]
  &\vsplito& \mval 
  {m}
  {()} \\\relax
[\cdot;\next{\hat y}]
  &\vsplito& \mval 
  {()}
  {\hat y} \\\relax
[\cdot;(v_1,v_2)]
  &\vsplito& \mval 
  {(i_1,i_2)}
  {(r_1,r_2)} \\\relax
[\cdot;\inl{v}]
  &\vsplito& \mval 
  {\inl{i}}
  {r} \\\relax
[\cdot;\inr{v}]
  &\vsplito& \mval 
  {\inr{i}}
  {r} \\\relax
[\cdot;\roll{v}]
  &\vsplito& \mval 
  {\roll{i}}
  {r} \\\relax
[\cdot;\lam xe]
  &\vsplito& \mval 
  {\lam {x} {c}}
  {\lam {(x,l)} {r}}
\end{array}\]
\end{abstrsyn}
\caption{The masking relations takes a \lang\ partial value a masked value
comprising the seperated stage~\bbone\ and stage~\bbtwo\ parts. 
[needs some where clauses above]}
\label{fig:valMask}
\end{figure}

%\begin{figure*}
%\begin{mathpar}
%\infer {i \sim i : \rmint} {\cdot} \and
%\infer {(u_1,v_1) \sim (u_2,v_2) : A \times B} {u_1 \sim u_2 : A & v_1 \sim v_2 : B} \and
%\infer {\inl~v_1 \sim \inl~v_2 : A+B} {v_1 \sim v_2 : A} \and
%\infer {\inr~v_1 \sim \inr~v_2 : A+B} {v_1 \sim v_2 : A} \and
%\infer [{\rm PROBLEM}] {\lam {x_1} A {e_1} \sim \lam {x_2} A {e_2} : A \to B} 
%		{x_1 \equiv x_2 : A \vdash e_1 \equiv e_2 : B}
%\end{mathpar}
%\caption{Auxiliary Splitting Judgments}
%\label{fig:auxSplit}
%\end{figure*}
%
%\begin{figure*}
%\caption{Context Splitting}
%\label{fig:contSplit}
%\begin{mathpar}
%\infer {\emptyC \csplit [\emptyC~;~\emptyC]}{\cdot} \and
%\infer {\Gamma, v : A^\bbone \csplit [\Gamma_1,v:A_1; \Gamma_2,v:A_2]}{\Gamma \csplit [\Gamma_1; \Gamma_2] & A \tsplito [A_1, A_2]} \and
%\infer {\Gamma, v : A^\bbtwo \csplit [\Gamma_1; \Gamma_2,v:A']}{\Gamma \csplit [\Gamma_1; \Gamma_2] & A \tsplits A'} 
%\end{mathpar}
%\end{figure*}





\section{Examples of Algorithm Derivation}
\label{sec:examples}

In this section, I show a few example of how splitting can be used to derive
standard data structures a relatively simple multistage code.

\subsection{Fast Exponent}

Fast exponent example.  

\begin{lstlisting} 
fun fexp (b : $int, e : int) : $int =
	if e == 0 then
		next{1}
	else if (e mod 2) == 0 then
		next{let x = prev{fexp(b,e/2)} in x*x}
	else
		next{prev{b} * prev{fexp (b,e-1)}}		
\end{lstlisting}

splits into

\begin{lstlisting} 
fun fexp (b, e) =
  ((), roll (
    if e == 0 then inL ()
    else
      inR (
        if (e mod 2) == 0
        then inL (#2 (fexp (b,e/2)))
        else InR (#2 (fexp (b,e-1)))
      )
  ))
\end{lstlisting}

and

\begin{lstlisting} 
fun fexp ((b, e), p) =
	case unroll p of
	  inL () => 1
	| inR d =>
		case d of
		  inL r => let x = fexp ((b,()),r) in x*x
		| inR r => b * fexp ((b,()),r)
\end{lstlisting}

\subsection{Quickselect}

\begin{lstlisting} 
datatype list = Empty | Cons of int * list
fun partition ((p,l) : int*list) : (int*list*list) =
  case unroll l of 
    Empty => (0,Empty, Empty) 
  | Cons (h,t) =>
      let (s,left,right) = partition (p,t) in
      if h<p 
      then (s+1,Cons(h,left),right)
      else (s,left,Cons(h,right))
fun qs (l : list, i: $int) = 
  case l of
    Empty => next {0}
  | Cons (h,t) => 
      let (left,right,n) = partition h t in
      next{
        let n = hold{n} in
          case compare prev{i} n of
            LT => prev {qs left i}
          | EQ => hold {h}
          | GT => prev {qs right next{prev{i}-n-1}}
      }	
\end{lstlisting}

\subsection{Trie}

\begin{lstlisting} 
datatype letter = A | B | C
datatype string = EmptyS | ConsS of letter * string
datatype list = EmptyL | ConsL of string * list
fun partition (l : list) : (bool * list * list * list) =
  case l of
    EmptyL => (false,EmptyL,EmptyL,EmptyL)
  | ConsL (s,ss) =>
      let (anyEmpty,a,b,c) = partition ss in
      case unroll s of
        EmptyS => (true,a,b,c)
      | ConsS (z,zs) =>
          case z of 
            A => (anyEmpty, ConsL(zs,a), b, c) 
          | B => (anyEmpty, a, ConsL(zs,b), c) 
          | C => (anyEmpty, a, b, ConsL(zs, c))
fun exists ((l,s) : list * $string) : $bool =
  case l of 
    EmptyL => next{false} 
  | ConsL _ => 
    let (anyEmpty, a, b, c) = partition l in
    next {
      case prev{s} of
        EmptyS => prev{
          if anyEmpty then next{true} else next{false}
          }
      | ConsS (z,zs) =>
          case unroll z of
            A => prev{exists (a,next{zs})}
          | B => prev{exists (b,next{zs})} 
          | C => prev{exists (c,next{zs})}
    }
\end{lstlisting}

\subsection {Partial Evaluation vs. Splitting}
asdffsa



%\appendix
%\section{Appendix Title}
%This is the text of the appendix, if you need one.

%\acks
%Acknowledgments, if needed.

% We recommend abbrvnat bibliography style.

\bibliography{paper}
\bibliographystyle{abbrvnat}

\end{document}
