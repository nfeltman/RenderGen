Staged programming languages assign a stage to each expression and
evaluate each expression in its assigned stage.
%For example, in a
%two-staged program, some expressions are evaluated in the first stage
%and others in the second stage.
%Programming languages, evaluation
%strategies, and type systems for staged programs where stages roughly
%correspond to the arrival order of inputs have been studied
%extensively in the areas of partial evaluation and meta programming.
While staged programming has been studied extensively in the areas of
partial evaluation and meta programming, in this paper we present an
algorithm for statically {\em splitting} programs written in a
two-stage typed lambda calculus (with a $\fut$ modality denoting
computation in a later stage) into multi-pass programs that are
structured as two dependent, unstaged programs and are evaluated using
two distinct execution passes.  While splitting has been studied in the past
under the names \emph{pass separation} and \emph{data specialization},
such techniques were either manual or applicable only to simple programs
that lack important features such as recursion and first-class
functions.

Since our work builds on a lambda calculus, our splitting algorithm
enables the programmer to write staged programs by using powerful
programming abstractions in a composable and modular fashion. Our
splitting transformation maps such programs into multi-pass programs
that can run significantly more efficiently by assigning frequent
computations to earlier passes so that they can be reused (cheaply)
later.  Our experiments based on a prototype implementation show that
our splitting algorithm is able to perform highly non-trivial algorithmic
transformations, sometimes automatically improving efficiency
asymptotically by nearly a linear factor.  Since the splitting
transformation takes place at compile time, these benefits are
achieved without significant run-time penalty.

% The motivations for pass separation are 1) improvements in efficiency
% by assigning frequent computations to earlier passes so that they can
% be re-used (cheaply) later and, 2) performing such scheduling
% statically at compile time without loss of any generality.  To assess
% the effectiveness of our techniques for pass separation, we consider
% several examples and demonstrate that our implementation is able to
% perform pass separation statically as expected.  In a particularly
% interesting example, involving recursion and recursive data, we show
% that our techniques perform highly non-trivial algorithmic
% transformations, improving efficiency asymptotically by nearly a
% linear factor.




% Many algorithms can perform useful work before receiving all of their inputs.
% If we regard those inputs as arriving at different \emph{stages} of the
% computation, then we can \emph{split} those algorithms into, for each stage, a
% function performing all the work dependent only on the available inputs.

% Prior work on splitting has been limited to simple, imperative languages; in
% this paper, we extend splitting to \lang, a typed lambda calculus equipped with
% a modal staging operator, allowing us to split recursive and higher-order
% functions.

% These new features allow us to express some familiar algorithms which, when
% split, yield asymptotic improvements; for example, quickselect of the $k$th
% smallest element of a list $l$ splits into (1) sorting $l$ into a binary search
% tree, then (2) finding the tree's $k$th leftmost element.

% % something about compositional, cross-cutting code
