Many algorithms can perform useful work before receiving all of their inputs.
If we regard those inputs as arriving at different \emph{stages} of the
computation, then we can \emph{split} those algorithms into, for each stage, a
function performing all the work dependent only on the available inputs.

In this paper, we provide a theoretical understanding of this splitting process
by defining it as a program transformation on \lang, a typed lambda calculus
equipped with a modal staging operator. This approach extends splitting to
language features absent in prior work, including first-order functions and
disjoint unions.

These new features allow us to express some familiar algorithms which, when
split, yield asymptotic improvements; for example, quickselect of the $k$th
smallest element of a list $l$ splits into (1) sorting $l$ into a binary search
tree, then (2) finding the tree's $k$th leftmost element.

\TODO (Is this true?)
We also show that splitting a staged implementation of an interpreter yields a
partial evaluator.

%into one function per stage, performing all the work depending only on the
%available inputs.

%as much work as possible given the inputs available at each stage. 

%In this way, splitting is essentially an algorithm generation technique.

% amortized
