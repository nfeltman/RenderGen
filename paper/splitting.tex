%!TEX root = paper.tex

\section{Splitting Algorithm}
\label{sec:splitting}

\begin{abstrsyn}

The primary strength of \lang\ is the expressive power afforded by
its ability to syntactically interleave work intended for stage \bbone\
and work intended for stage \bbtwo, as well as its ability to form
mixed-stage abstractions. 

In this section, we provide a {\em splitting algorithm} to translate the multistage terms of \lang\
into a simpler but equivalent form.
The output of splitting is simpler because it is {\em syntactically separated}; that is, 
stage \bbone\ and stage \bbtwo\ are represented by distinct subterms and are only connected at the top syntactic level.
The output of splitting is equivalent (to its input) because it can be evaluated to the same residual 
that would be produced by the semantics of \ref{sec:semantics}.

More precisely, for any $\coltwo e A$ which reduces to a residual $q$ (via $\diatwo e q$),
splitting translates $e$ into some syntactically separated program $\mathcal{P}$ 
(via the new relation $e \splittwosym \mathcal{P}$), which
can also be reduced to $q$ (via the new relation $\mathcal{P} \sepredtwosym q$).
We enforce the ``syntactically separated'' condition by saying that $\mathcal{P}$ 
can only have one form, namely $\pipeS p l r$, where  
$p$ (called the {\em precomputation}) is a monostage term encoding all the stage \bbone\ computation that was in $e$, 
and $l.r$ (called the {\em resumer}) is a monostage term encoding all the stage \bbtwo\ computation that was in $e$.  
Likewise, $\sepredtwosym$ is defined by a single rule, which evaluates the precomputation and plugs it in for $l$:
\[
\infer{\sepredtwo {\pipeS p l r} {[b/l]r}} {\reduce p b}
\]
Here we call $b$ the {\em boundary value}, as it represents the communication at the boundary between stages.
A summary of these operations and the relationship between them is provided on the right side of \ref{fig:splittingSummary}.

Since terms at world \bbtwo\ can depend on terms at \bbonem\ (via \texttt{prev}),
we also provide a way to split those as well.
This is given by the new relation $e \splitonesym \mathcal{P}$,
where $\mathcal{P}$ must have the form $\pipeM c l r$ where  
$c$ (called the {\em combined term}) is a monostage term encoding all the stage \bbone\ computation that was in $e$, 
and $l.r$ (again called the {\em resumer}) is a monostage term encoding all the stage \bbtwo\ computation that was in $e$.

Similar to above, we also define a $\sepredtwosym$ relation which can be used to evaluate \texttt{sepM} terms,
with one complication arising from the fact that \lang\ terms at world \bbonem\ produces multistage output.
For example, consider the term
\begin{lstlisting}
(next {1+2}, grnd{3+4})
\end{lstlisting}
which reduces (via $\redonesym$) to the residual table and partial value
\begin{lstlisting}
[yhat |-> 1+2] (next {yhat}, grnd{7})
\end{lstlisting}
Since this output has interleaved stages, theres no way a syntactically seperated term could directly reduce to it.

We resolve this mismatch by introducing one more translation, called {\em masking} and written $[\xi;v] \vsplito \mathcal{V}$,
which converts residual tables and partial values like the those above into a syntactically seperated version $\mathcal{V}$.
As before, we enforce the ``syntactically seperated'' property structurally,
saying that $\mathcal{V}$ must have the form $\mval i q$ (called a {\em masked value}) where 
$i$ is a monostage value respresenting the stage \bbone\ part of $v$,
and $q$ is a monostage term representing all of the stage \bbtwo\ computation in $\xi$ and $v$.
Note that masking exists only for the purposes of stating correctness; it is not part of the splitting algorithm itself.

With this machinery in place, we can define $\sepredtwosym$ as just creating masked values with the single rule:
\[
\infer{\sepredone {\pipeM c l r} {\mval i {[b/l]r}}} {\reduce c {(i,b)}}
\]
Here we see the primary difference between terms at world \bbtwo\ and world \bbone.
Since $\coltwo e A$ reduces to an entirely stage \bbtwo\ result, 
its stage \bbone\ subcomputations only exist to internally pass their value to stage \bbtwo;
but since $\colmix e A$ reduces to a multistage result, 
its stage \bbone\ subcomputations exist {\em both} to pass their value to stage \bbtwo\ {\em and} to be present in the result.
Correspondingly, in the $\sepredtwosym$ rule, the precomptuation $p$ reduces only to a boundary value which is passed to the resumer;
but in the $\sepredonesym$ rule, the combined term $c$ reduces to a tuple containing both the immediate result and the boundary value,
with only the latter being passed to the resumer.

A summary of these operations and the relationship between them is provided on the left side of \ref{fig:splittingSummary}.

This section proceeds by defining masking ($\vsplito$), 
then using masking to motivate the definition of \bbonem-translation ($\splitonesym$),
and finally coming back to define \bbtwo-translation ($\splittwosym$).

\subsection{Masking}

The point of the masking operation is simply to convert a partial value into a masked value $\mval i q$
by assigning all of the stage \bbone\ content to $i$ and all of the stage \bbtwo\ content to $q$.
The rules of the masking relation are given in \ref{fig:valMask}.

Masking operates by first inducting on the entries of the residual table.  
Being purely stage~\bbtwo\ content, these are reified into let statements at the top of the resumer.
Once the table is empty, masking inducts on value itself.

Masking assigns ground stage~\bbone\ values to the immediate value
and likewise assigns references into the residual table to the resumer.
In both cases, the alternate component is assigned to $\tup{}$, to represent trivial information.

Masking distributes into tuples, injections, and rolls, since their subvalues may have content at both stages.
However, the tags of injections and rolls are replicated only in the immediate value, 
since they represent stage~\bbone\ information.

Since lambdas may represent multi-stage computations, 
masking splits the body of lambdas as general stage~\bbone\ terms (as described in \ref{sec:split-one}), 
and packages the resulting terms as functions.

\subsection{Term Splitting at \bbonem}
\label{sec:split-one}

We now show how to translate terms $\colmix e A$ into the form $\pipeM c l r$,
pursuant to the correctness condition given in \ref{fig:splittingSummary}.
The algorithm is specified by the $\splitonesym$ relation (\cref{fig:termSplit}), 
which proceeds recursively on the structure of~$e$.

When $e$ is a terminal (a variable or unit)
splitting yields a combined term formed by tupling $e$ with a \texttt{()} precomputation, 
and the trivial resumer \texttt{()}. 
(Stage~\bbone\ terminals, by definition, contain no stage~\bbtwo\ subcomputations.)  
For example, the integer constant \texttt{3} splits into the combined term \texttt{(3,())} and resumer \texttt{\_=>()}.

For all non-terminals (except \texttt{next}),
splitting descends into $e$, recursively splitting its $n$ subterms
to produce their respective combined terms $c_1,\ldots,c_n$ and resumers $r_1, \ldots, r_n$.
The combined term of $e$ is formed by binding $c_1,\ldots,c_n$
to the patterns $(y_1,z_1),\ldots,(y_n,z_n)$
to isolate immediate results from boundary values. Then,
the immediate result of $e$ is formed by replacing $e$'s subterms with $y_1,\ldots,y_n$.
The resumer binds the boundary values $b_1,\ldots,b_n$ to an
argument $(l_1,\ldots,l_n)$ in a term that has the same structure
of~$e$ but where each subterm is replaced by its resumer ($r_i$'s).

Splitting {\tt case} yields a combined term that executes one of the branches' combined terms based on the immediate result $y_1$ of the predicate.
The boundary value $b_i$ for this branch is injected and bundled with that of the predicate ($b_1$).   
$b_i$ is cased in the resumer to determine which branch's resumer should be executed.
{\tt case} and \texttt{hold} are the only two rules where splitting adds non-trivial logic is added to the precomputation.

Function introduction has a \texttt{()} boundary value,
since functions are already fully reduced in our semantics.
However, since the body of a function may itself be multi-stage, splitting must continue into it.
The immediate result is a new function formed from the stage~\bbone\ part of the original body.
The resumer is a new function formed out of the stage~\bbtwo\ part of the original body.
It is the responsibility of the application site to save the precomputation of the function body
and pass it to the resumer version of the function.

Since the results of splitting \texttt{next} terms depend on the output of splitting its stage~\bbtwo\ subterm,
we defer description of \texttt{next} until after describing stage~\bbtwo\ term splitting.

\subsection{Term Splitting at \bbtwo}

Because stage~\bbtwo\ terms in \lang\ reduce to monostage residuals (as opposed to partial values),
term splitting at \bbtwo\ assumes a simpler form than the version at \bbonem\ does. 
The algorithm is specified by the $\splittwosym$ relation in \cref{fig:termSplit}.

In the terminal cases of
constants and variables, splitting generates trivial precomputations that are \texttt{()}, and resumers consisting of the original term.
For example, the integer constant \texttt{3} splits into the
precomputation \texttt{()} and resumer \texttt{\_=>3}.

More generally, for all (except \texttt{prev} and \texttt{hold}) 
$n$-ary terms $e = \mathcal{C}\ttlpar e_1 \ttsemi \ldots \ttsemi e_n \ttrpar$ 
the precomputation is the tupled precomputations of $e$'s $n$ subterms:
$p=(p_1,\ldots,p_n)$.  The resumer binds each boundary value to an
argument $(l_1,\ldots,l_n)$ in a term that has the same structure
of~$e$ but where each subterm is replaced by its corresponding resumer:
$r = \mathcal{C}\ttlpar r_1 \ttsemi \ldots \ttsemi r_n \ttrpar$ .
Notably, at \texttt{case}s and functions the
precomputation of subterms is lifted out from underneath stage \bbtwo\ binders.  
% TODO: should probably draw a parallel to the same behavior in dynamics

Splitting \texttt{prev} generates a precomputation that projects the immediate result of its stage~\bbone\ subterm.
Since the argument to \texttt{prev} is of $\fut$ type, its immediate result reduces to $\tup{}$, justifying why it can be thrown away.
\texttt{hold} treats the entire combined term of its stage~\bbone\ subexpression as a precomputation, 
and projects out the integer result in the resumer. 
\footnote{The resumer of an integer expression is usually trivial, 
but we have to include it here for termination purposes.} 
Finally, splitting \texttt{next} simply tuples up the precomputation of its stage~\bbtwo\ subterm with a trivial immediate result $\tup{}$.

\subsection {Three World System?}

Why do we need a three world system?

\end{abstrsyn}

%!TEX root = ../paper.tex

\begin{figure*}
\begin{abstrsyn}
\begin{mathpar}
\fbox{Partial Values} \and 
\infer {x 					\vsplito \mval x x}														{\cdot}												\and
\infer {\tup{} 				\vsplito \mval {\tup{}} {\tup{}}}										{\cdot}												\and
\infer {\pure m 			\vsplito \mval m {\tup{}}}												{\cdot}												\and
\infer {\next y 			\vsplito \mval {\tup{}} y}												{\cdot}												\and
\infer {\tup{v_1,v_2} 		\vsplito \mval {\tup{i_1,i_2}} {\tup{q_1,q_2}}}							{v_1 \vsplito \mval {i_1} {q_1} 
																									&v_2 \vsplito \mval {i_2} {q_2}}					\and
\infer {\fix f x e			\vsplito \mval {\fix f x {\letin {\tup{x,y}} c {\tup{x,\roll y}}}} 
							{\fix f {\tup{x,\roll l}} r}}											{\splitone e A c {l.r}}								\and
\end{mathpar}
\hrule
\begin{mathpar}
\fbox{\bbonem\ Terms} 
\and
%\infersplitone [unit]			{\spl {\tup{}}{\rmunit}{\tup{\tup{},\tup{}}}{\_.\tup{}}}											{\cdot} \and
%\infersplitone [int]			{\spl {i}{\rm int}{\tup{i,\tup{}};\_.\tup{}}}														{\cdot} \and
%\infersplitone [hyp]			{\spl {x}{A}{\tup{x,\tup{}}}{\_.x}}																	{\cdot} \and
% \infersplitone [\to I]			{\splitonetall {\lam{x}{e}}{A \to B}{\tup{\lam x c, \tup{}}}{\_.\tup{\lam {\tup{x,l}} r}}}			{\sub [\Gamma,\col x A] {} B} 							\and
% \infersplitone [fix]			{\splitonetall 	
% 									{\fix{f}{x}{e}} {} 
% 									{\tup{\fix f x {\letin{\tup{y,z}}c{\tup{y,\roll z}}}, \tup{}}}
% 									{\_.\fix f {\tup {x,l_0}} {\letin{l}{\unroll {l_0}}{r}}}}										{\sub [\Gamma,\col x A] {} B} 							\and
\infersplitone 					{\spl {\exv v} A {\exv{\tup{i,\tup{}}}} {\_.\exv q}} {v \vsplito \mval i q} \and 
\infersplitone [common1]		{\spl {\scriptCapp e}{A}{\letin{\tup{y,z}}{c}
									{\tup{\scriptCapp {\exv y},\exv z}}}{l.\scriptCapp r}}	{\sub {} {A} 
																																	& \scriptC \in \{\mathtt{pi1},\mathtt{pi2}\}} 			\and
\infersplitone [\to E]			{\splitonetall {\app {e_1}{e_2}}{B}{\left(
									\talllet{\tup{y_1,z_1}}{c_1}{
									\talllet{\tup{y_2,z_2}}{c_2}{
									\talllet{\tup{y_3,z_3}}{\app{y_1}{y_2}}{\exv{\tup{y_3,\tup{z_1,z_2,z_3}}}\ttrpar\ttrpar\ttrpar}}}
									\right)}
									{\tup{l_1,l_2,l_3}.\app{\exv {r_1}}{\exv{\tup{r_2,l_3}}}}}										{\sub 1 {A \to B} & \sub 2 A}							\and
% \infersplitone [common2]		{\spl {\scriptCapp e}{A}{\letin{\tup{y,z}}{c}{\tup{\scriptCapp y,z}}}{l.r}}							{\sub {} {A} 
% 																																	& \scriptC \in \{\mathtt{inl},\mathtt{inr},
% 																																	\mathtt{roll},\mathtt{unroll}\}} 						\and
%\infersplitone [\times E_2]	{\spl {\pit~e}{B}{\talllet{\tup{y,z}}{c}{\tup{\pit~y,z}};l.\pit~r}}								{\sub {} {A\times B}} \and
%\infersplitone [let]			{\spl {\letin{x}{e_1}{e_2}}{B}{
%								\begin{array}{l}
%									\left(\talllet{\tup{x,z_1}}{c_1}{
%									\talllet{\tup{y,z_2}}{c_2}{
%									\tup{y, \tup{z_1, z_2}}}}\right);
%									\tup{l_1,l_2}.\letin{x}{r_1}{r_2}
%								\end{array}}}																					{\sub 1 A & \sub [\Gamma,\col{x}{A}] 2 B} 		\and
%\infersplitone [+ I_1]	{\spl {\inl~e}{A+B}{\talllet{\tup{y,z}}{c}{\tup{\inl~y,z}};l.r}}			{\sub {} A}												\and
%\infersplitone [+ I_2]	{\spl {\inr~e}{A+B}{\talllet{\tup{y,z}}{c}{\tup{\inr~y,z}};l.r}}			{\sub {} A}												\and
\infersplitone [\times I] 		{\splitonetall {\tup{e_1,e_2}}{A\times B}
									{\left(
										\talllet{\tup{y_1,z_1}}{c_1}{
										\talllet{\tup{y_2,z_2}}{c_2}{
										\exv{\tup{\tup{y_1, y_2},\tup{z_1, z_2}}}\ttrpar\ttrpar}}
									\right)}
									{\tup{l_1,l_2}.\tup{r_1,r_2}}}																	{\sub 1 A & \sub 2 B} \and
\infersplitone [+ E]		{\splitonetall {\caseP{e_1} {x_2.e_2} {x_3.e_3}}{C}
								{\left(
								\talllet{\tup{y_1,z_1}}{c_1}{
									\tallcase{y_1}
									{x_2.\letin{\tup{y_2,z_2}}{c_2}{\exv {\tup{y_2,\tup{z_1,\inl{z_2}}}}}}
									{x_3.\letin{\tup{y_3,z_3}}{c_3}{\exv {\tup{y_3,\tup{z_1,\inr{z_3}}}}}\ttrpar}
								}\right)}
								{\tup{l_1,l_b}.{\ttlpar r_1 \ttsemi \caseof{l_b}{\exv{l_2}.[\tup{}/x_2]{r_2}}{l_3.[\tup{}/x_3]{r_3}\ttrpar}
								}}}																									{\sub 1 {A+B} 
																																	& \sub [\Gamma,\col{x_2} A] 2 C 	
																																	& \sub [\Gamma,\col{x_3} B] 3 C} 	
%
%\infersplitone [\mu I]		{\spl {\roll~e}{\mu\alpha.\tau}{\talllet{\tup{y,z}}{c}{\tup{\roll~y,z}};l.r}}						{\sub {} {[\mu\alpha.\tau/\alpha]\tau}}							\and
%\infersplitone [\mu E]		{\spl {\unroll~e}{[\mu\alpha.\tau/\alpha]\tau}{\talllet{\tup{y,z}}{c}{\tup{\unroll~y,z}};l.r}}	{\sub {} {\mu\alpha.\tau}}										\and
%
%\infersplitone [if]			{\spl {\left( \tallif {e_1}{e_2}{e_3} \right)}{A} 
%								{\begin{array}{l}
%									\left(\talllet{\tup{y_1,z_1}}{c_1}{\tallif {y_1}
%										{\letin{\tup{y_2,z_2}}{c_2}{\tup{y_2, \tup{z_1, \inl~z_2}}}}
%										{\letin{\tup{y_3,z_3}}{c_2}{\tup{y_3, \tup{z_1, \inl~z_3}}}}}
%									\right); \\
%									\tup{l_1,l_b}.\tup{r_1;\caseof{l_b}{l_2.r_2}{l_3.r_3}}
%								\end{array}}}																					{\sub 1 \rmbool & \sub 2 A & \sub 3 A }		\and
%
%\infersplitone	{\spl {\liftint~e} ? {c;l.r}}											{\sub {} ?}					
%
\end{mathpar}
\hrule
\begin{mathpar}
\fbox{\bbtwo\ Terms} 
\and
%\infersplittwo [unit]			{\spl {\tup{}} {\rmunit} {\tup{}} {\_}{\tup{}}}												{\cdot} 														\and
%\infersplittwo [int]			{\spl {i}{\rm int}{\tup{};\_}{i}}															{\cdot} 														\and
%\infersplittwo [hyp]			{\spl {x}{A}{\tup{}}{\_}{x}}																{\cdot} 														\and
\infersplitone 					{\spl {\exv q} A {\exv {\tup{}}} {\_. q}} 													{v \vsplito \mval i q} 											\and 
\infersplittwo [\times E_1]		{\spl {\scriptCapp e}{A}{p}{l}{\scriptCapp r}}												{\sub {} {A\times B} 
																															& \scriptC \in \{\mathtt{pi1},\mathtt{pi2},\mathtt{inl},
																															  \mathtt{inr},\mathtt{roll},\mathtt{unroll}\}} 				\and
%\infersplittwo [\to I]			{\spl {\lam{x}{e}}{A \to B}{p}{l}{\lam{x}{r}}}												{\sub [\Gamma,\col x A] {} B} 									\and
\infersplittwo [\to I]			{\spl {\fix fxe}{A \to B}{p}{l}{\fix fxr}}													{\sub [\Gamma,\col x A] {} B} 									\and
\infersplittwo [\to E]			{\spl {\app{e_1}{e_2}}{B}{\tup{p_1,p_2}}{\tup{l_1,l_2}}{\app{r_1}{r_2}}}					{\sub 1 {A \to B} & \sub 2 A}									\and			
\infersplittwo [let]			{\spl {\letin{x}{e_1}{e_2}}{B}{\tup{p_1,p_2}} {\tup{l_1,l_2}} {\letin{x}{r_1}{r_2}}}		{\sub 1 A & \sub [\Gamma,\col{x}{A}] 2 B} 						\and
\infersplittwo [\times I]		{\spl {\tup{e_1,e_2}}{A\times B}{\tup{p_1,p_2}}{\tup{l_1,l_2}}{\tup{r_1,r_2}}}				{\sub 1 A & \sub 2 B} 											\and
%\infersplittwo [\times E_2]	{\spl {\pit~e}{B}{p;l}{\pit~r}}																{\sub {} {A\times B}} 											\and
%\infersplittwo [+ I_1]			{\spl {\inl~e}{A+B}{p;l}{\inl~r}}															{\sub {} A}														\and
%\infersplittwo [+ I_1]			{\spl {\inr~e}{A+B}{p;l}{\inr~r}}															{\sub {} A}														\and
%\infersplittwo [\mu I]			{\spl {\roll~e}{\mu\alpha.\tau}{p;l.\roll~r}}												{\sub {} {[\mu\alpha.\tau/\alpha]\tau}}							\and
%\infersplittwo [\mu E]			{\spl {\unroll~e}{[\mu\alpha.\tau/\alpha]\tau}{p;l.\unroll~r}}								{\sub {} {\mu\alpha.\tau}}										\and
\infersplittwo [+ E]			{\spl {\caseof{e_1}{x_2.e_2}{x_3.e_3}}{C}
								{\tup{p_1,p_2,p_3}}{ \tup{l_1,l_2,l_3}}{\caseof{r_1}{x_2.r_2}{x_3.r_3}}}					{\sub 1 {A+B} & \sub [\Gamma,\col{x_2} A] 2 C 
																															& \sub [\Gamma,\col{x_3} B] 3 C} 								
%\infersplittwo [if]			{\spl {\left(\tallif {e_1}{e_2}{e_3}\right)}{A} 
%								{\tup{p_1,p_2,p_3}}{\tup{l_1,l_2,l_3}.\left(\tallif{r_1}{r_2}{r_2}\right)}}						{\sub 1 \rmbool & \sub 2 A & \sub 3 A }		\and
\end{mathpar} 
\hrule
\begin{mathpar}
\fbox{Staging features}
\and
\infersplittwo [\fut E]		{\spl {\prev e}{A} {\pit c} l r }													{\splitonesub {} {\fut A}} 		\and
\infersplitone [\fut I]		{\spl {\next e}{\fut A}{\tup{\tup{},p}}{l.r}}										{\splittwosub {} A} 			\and
%\infersplittwo [hold]		{\spl {\pause e}{\rmint}{c,\tup{a,l}.\tup{r;a}}}									{\splitonesub {} \rmint}		\and
\infersplitone				{\spl {\pure e} {\curr A} {\tup{e,\tup{}}}{\_.\tup{}}}								{\cdot}							\and
\infersplitone				{\splitonetall {\letp x{e_1}{e_2}} {?} 
							{\letin {\tup{x,z_1}} {c_1} {
							 \letin {\tup{y_2,z_2}} {c_2} {\tup{y_2,\tup{z_1,z_2}}}}}
							 {\tup{l_1,l_2}.\letin{\_}{r_1}{r_2}}}												{\sub 1 ? & \sub 2 ?}			
%%\infersplitone				{\spl {\lifttag e} ? {c}{l.r}}													{\sub {} ?}						
\end{mathpar}
\end{abstrsyn}
\caption{Splitting rules.}
\label{fig:termSplit}
\end{figure*}


%!TEX root = ../paper.tex

%\begin{figure*}
%\caption{Type Splitting}
%\label{fig:typeSplit}
%\begin{mathpar}
%\infer [\mathrm{unit}\tsplito]	{\rmunit \tsplito [\rmunit,\rmunit]}{\cdot} \and
%\infer [\mathrm{int}\tsplito]	{{\rm int} \tsplito [{\rm int}, \rmunit]}{\cdot} \and
%\infer [\mathrm{bool}\tsplito]	{{\rm bool} \tsplito [{\rm bool}, \rmunit]}{\cdot} \and
%\infer [\times\tsplito]
%	{A\times B \tsplito [A_1 \times B_1, A_2 \times B_2]}
%	{A \tsplito [A_1, A_2] & B \tsplito [B_1,B_2]} \and
%\infer [\fut\tsplito]
%	{\fut~A \tsplito [{\rm unit}, A']}
%	{A \tsplits A'} \and
%\end{mathpar}
%\end{figure*}
%

% \[\begin{array}{l|l|l}
% [\xi;v] & \masko{\xi;v} & \maskt{\xi;v} \\ \hline
% [\hat y \mapsto q,\xi;v]
%   & \masko{\cdot;v}
%   & \letin {\hat y} q {\maskt{\xi;v}} \\\relax
% [\cdot;()] 
%   & ()
%   & () \\\relax
% [\cdot;\monoTerm~v] 
%   & v
%   & () \\\relax
% [\cdot;\next~\hat y] 
%   & ()
%   & \hat y \\\relax
% [\cdot;(v_1,v_2)] 
%   & (\masko {\cdot;v_1},\masko {\cdot;v_2})
%   & (\maskt {\cdot;v_1},\maskt {\cdot;v_2}) \\\relax
% [\cdot;\inl~v] 
%   & \inl~\masko{\cdot;v}
%   & \maskt {\cdot;v} \\\relax
% [\cdot;\inr~v] 
%   & \inr~\masko{\cdot;v}
%   & \maskt {\cdot;v} \\\relax
% [\cdot;\roll~v] 
%   & \roll~\masko{\cdot;v}
%   & \maskt {\cdot;v} \\\relax
% [\cdot;\lam xAe]
%   & \lambda x.c 
%   & \lambda (x,l).r \\
% & \text{ for } \splitone e A {c,l.r}
% & \text{ for } \splitone e A {c,l.r}
% \end{array}\]

\begin{figure}
\begin{abstrsyn}
\[\begin{array}{rll}
[\hat y \mapsto q,\xi;v]
  &\vsplito& \mval 
  {i}
  {\letin {\hat y} q {r}} \\\relax
[\cdot;()]
  &\vsplito& \mval 
  {()}
  {()} \\\relax
[\cdot;\monoTerm{m}]
  &\vsplito& \mval 
  {m}
  {()} \\\relax
[\cdot;\next{\hat y}]
  &\vsplito& \mval 
  {()}
  {\hat y} \\\relax
[\cdot;(v_1,v_2)]
  &\vsplito& \mval 
  {(i_1,i_2)}
  {(r_1,r_2)} \\\relax
[\cdot;\inl{v}]
  &\vsplito& \mval 
  {\inl{i}}
  {r} \\\relax
[\cdot;\inr{v}]
  &\vsplito& \mval 
  {\inr{i}}
  {r} \\\relax
[\cdot;\roll{v}]
  &\vsplito& \mval 
  {\roll{i}}
  {r} \\\relax
[\cdot;\lam xe]
  &\vsplito& \mval 
  {\lam {x} {c}}
  {\lam {(x,l)} {r}}
\end{array}\]
\end{abstrsyn}
\caption{The masking relations takes a \lang\ partial value a masked value
comprising the seperated stage~\bbone\ and stage~\bbtwo\ parts. 
[needs some where clauses above]}
\label{fig:valMask}
\end{figure}

%\begin{figure*}
%\begin{mathpar}
%\infer {i \sim i : \rmint} {\cdot} \and
%\infer {(u_1,v_1) \sim (u_2,v_2) : A \times B} {u_1 \sim u_2 : A & v_1 \sim v_2 : B} \and
%\infer {\inl~v_1 \sim \inl~v_2 : A+B} {v_1 \sim v_2 : A} \and
%\infer {\inr~v_1 \sim \inr~v_2 : A+B} {v_1 \sim v_2 : A} \and
%\infer [{\rm PROBLEM}] {\lam {x_1} A {e_1} \sim \lam {x_2} A {e_2} : A \to B} 
%		{x_1 \equiv x_2 : A \vdash e_1 \equiv e_2 : B}
%\end{mathpar}
%\caption{Auxiliary Splitting Judgments}
%\label{fig:auxSplit}
%\end{figure*}
%
%\begin{figure*}
%\caption{Context Splitting}
%\label{fig:contSplit}
%\begin{mathpar}
%\infer {\emptyC \csplit [\emptyC~;~\emptyC]}{\cdot} \and
%\infer {\Gamma, v : A^\bbone \csplit [\Gamma_1,v:A_1; \Gamma_2,v:A_2]}{\Gamma \csplit [\Gamma_1; \Gamma_2] & A \tsplito [A_1, A_2]} \and
%\infer {\Gamma, v : A^\bbtwo \csplit [\Gamma_1; \Gamma_2,v:A']}{\Gamma \csplit [\Gamma_1; \Gamma_2] & A \tsplits A'} 
%\end{mathpar}
%\end{figure*}


