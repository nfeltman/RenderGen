\section{Splitting Algorithm}
\label{sec:splitting}


\begin{figure}
\textbf{Languages:}
\begin{itemize}
\item \lang: two-staged lambda calculus with product, sum, and
  recursive types
\item \langmono: an unstaged lambda calculus with products.
\end{itemize}

\textbf{\lang\ Evaluation Relations:}
\begin{itemize}
\item 
\bbone-Evaluation: $e\mathbin{\redonesym}[\xi;v]$, where $\xi$ is a \emph{residual table} and $v$ is a \emph{partial value}. 

\item
\bbtwo-Evaluation: $e\mathbin{\redtwosym}q$, where $q$ is a \emph{residual} in the unstaged language \langTwo

\item 
Table reification: \ur{Fill in...}
\end{itemize}



% $red_2$: 2-eval,  which we think of as being identical to specialization. 

\vspace{.75em}
\textbf{\lang\ Splitting:}

\hspace{2em}\bbone-Splitting Structure: $e \splitonesym [c,l.r]$, where:

\hspace{4em}$c$ is the \emph{combined term} (representing all stage~\bbone\ subcomputations in $e$)

\hspace{4em}$l.r$ is the \emph{resumer} (representating all stage~\bbtwo\ subcomputations in $e$)

\hspace{4em}$c~\redsym~(y,b)$, where $y$ is the \emph{\bbone-result} of $e$, and $b$ is the \emph{boundary value} of $e$

 
\hspace{2em}\bbone-Splitting Correctness: If $e\mathbin{\redonesym}[\xi;v]$, then:

\hspace{4em}$y$ is identical to $\masko{\xi;v}$

\hspace{4em}$\maskt{\xi;v}$ and \texttt{(let l=b in r)} reduce (via $\redsym$) to identical values.  

\hspace{2em}\bbtwo-Splitting Structure: $e \splittwosym [p,l.r]$, where:

\hspace{4em}$p$ is the \emph{precomputation} (representing all stage~\bbone\ subcomputations in $e$)

\hspace{4em}$l.r$ is the \emph{resumer} (representating all stage~\bbtwo\ subcomputations in $e$)

\hspace{2em}\bbtwo-Splitting Correctness: If $e~\redtwosym~q$, and $q~\redsym~v$, then \texttt{(let l=b in r)}~$\redsym~v$

\caption{Summary of \lang\ evaluation and splitting.}
\label{fig:terminology}
\label{fig:termSplitSummary}
\end{figure}



Given a term $e$ in the source language \lang, where terms of
stage~\bbone\ and stage~\bbtwo\ may be interleaved, \emph{(stage)
  splitting} separates the stage~\bbone\ and stage~\bbtwo\
subcomputations into two terms,~$e_1$ and~$e_2$, in an unstaged target language \langmono\ 
%each performing the
%subcomputations corresponding to each stage.  More precisely, given a
%\lang\ term~$e$, splitting yields two terms $e_1$ and $e_2$ in a
%unstaged target language \langmono\
such that~$e_1$ performs all the
stage~\bbone\ subcomputations in~$e$ and~$e_2$~performs all of the
stage~\bbtwo\ subcomputations. Splitting preserves correctness in the sense that evaluating~$e$ in
the staged language produces the same result as evaluating~$e_1$ in the target language,
then further evaluating~$e_2$ in the target language with the result from~$e_1$.

In this section, we present a
splitting algorithm for the source language \lang, specifying
precisely the algorithm and its correctness conditions.  
We begin by defining a masking relation that serves to extract stage~\bbone\ and
stage~\bbtwo\ components of \lang\ partial values into \langmono\ terms that the outputs of splitting are designed to match. We then 
move on to the specification of the splitting algorithm itself.


\subsection{Masking}

Recall that partial values and residual tables produced by \bbone-evaluation may
contain content at \emph{both} stages. For example the term:
\begin{lstlisting}
1`(1+2, next{`2`7 + 5`1`})`
\end{lstlisting}
evaluates via $\redonesym$ to:
\begin{lstlisting}
2`[yhat|->7+5]` 1`(3,next{`2`yhat`1`})`
\end{lstlisting}
To match the results of \bbone-evaluation,
splitting the term should generate an~$e_1$ that reduces to the stage~\bbone\ content of the partial value, \texttt{3},
and an~$e_2$ corresponding to the stage~\bbtwo\ computation: \\
$\letin \yhat {\mathtt{7+5}} \yhat$.

to precisely specific the stage~\bbone\ and stage~\bbtwo\ components of \bbone-evaluation results, we define 
\emph{masking} functions, written $\masko{{-};{-}}$ and $\maskt{{-};{-}}$, 
which take a residual table and partial value as input and emit a term in \langmono\
that corresponds to only the content belonging to a single stage.
\ref{fig:valMask} defines these masking functions.

\emph{Stage~\bbone\ masking} of the partial value~$v$ (written $\masko{v}$) produces a term in \langmono\ containing exactly
the stage~\bbone\ content of~$v$. Since \next\ terms only contain stage~\bbtwo\
content, $\masko{v}$ replaces all $\next$ subterms of~$v$ with \texttt{()}.  Since lambdas may represent multi-stage computations, stage~\bbone\ masking splits the body of lambdas as a general stage~\bbone\ term (as described in \ref{sec:split-one}), and returns
the stage~\bbone\ component. $\masko{v}$ always produces a value in \langmono\ that has
the same structure as~$v$.  For example, the stage~\bbone\ mask of the partial value given above is \texttt{(3,())}. 

\emph{Stage~\bbtwo\ masking} of the partial value $v$ and residual context~$\xi$ (written $\maskt{\xi;v}$) produces a term in \langmono\ containing
exactly the stage~\bbtwo\ content of $[\xi;v]$. Unlike stage~\bbone\ masking, stage~\bbtwo\ masking 
requires access to the residual table $\xi$ because after \bbone-evaluation, all stage~\bbtwo\
has been lifted out of~$v$. Stage~\bbtwo\ masking discards
stage~\bbone\ content in~$v$ by replacing base constants with~\texttt{()},
replacing $\next~\hat{y}$ with~$\hat{y}$, and replacing all $n$-ary term
constructors with an $n$-tuple of masked subterms. Like stage~\bbone\ masking, stage~\bbtwo\
masking splits the bodies of lambdas as general stage~\bbone\ terms, and
returns the stage~\bbtwo\ component. After discarding all stage~\bbone\ content from~$v$, stage~\bbtwo\ masking reifies $\xi$ around the result,
producing a term in \langmono. This term reduces to a value that with the same structure as~$v$.
For example, the stage~\bbtwo\ mask of the partial value and residual table given above is:
\begin{lstlisting}
let yhat=7+5 in ((),yhat)
\end{lstlisting}

\subsection{Stage \bbone\ Term Splitting}
\label{sec:split-one}

Splitting a stage~\bbone\ term $e$ in \lang\ yields a term 
$e_1$ that performs all stage~\bbone\ subcomputations in $e$ and a \emph{resumer term} $e_2 = l.r$ that performs all stage~\bbtwo\ subcomputations.
To ensure the results of splitting produce the same output as the evaluation $e\mathbin{\redonesym}[\xi;v]$,
we require that $e_1$ reduce to the tuple $(y,b)$,
where $y$, called the \emph{\bbone-result} of $e$, is identical to $\masko{\xi;v}$, and that
$\letin l b r$ reduces to the same value as $\maskt{\xi;v}$. (We refer to $b$ as the \emph{boundary value} of the split since it carries information between stage~\bbone\ and stage~\bbtwo\ subcomputations.) Since~$e_1$ performs both stage~\bbone\ subcomputations needed to produce the \bbone-result and subcomputations (refered to as \emph{precomputations}) that produce the boundary value, we refer to this term as the \emph{combined term}, and hereafter denote it as~$c$.

The splitting algorithm for stage~\bbone\ terms, as specified by the
judgment $e \splitonesym [c,l.r]$ in \cref{fig:termSplit}, proceeds
recursively on the structure of~$e$.
When $e$ is a terminal (a variable or base constant)
splitting yields a combined term formed by tupling $e$ with a \texttt{()} precomputation, and the trivial resumer \texttt{()}. (Stage~\bbone\ terminals, by definition, contain no stage~\bbtwo\ subcomputations.)  For example, the integer constant \texttt{3} splits into the combined term \texttt{(3,())} and resumer \texttt{\_=>()}.

% Note the direct correspondence of these splitting outputs to the output of the stage~\bbone\ and~\bbtwo\ masking functions.

For all non-terminals (except $\next$),
splitting descends into $e$, recursively splitting its $n$ subterms
to produce their respective combined terms $c_1,\ldots,c_n$ and resumers $r_1, \ldots, r_n$.
The combined term of $e$ is formed by binding $c_1,\ldots,c_n$
to the patterns $(y_1,b_1),\ldots,(y_n,b_n)$
to isolate \bbone-results from boundary values. Then,
the \bbone-result of $e$ is formed by replacing $e$'s subterms with $y_1,\ldots,y_n$.
The resumer binds the boundary values $b_1,\ldots,b_n$ to an
argument $(l_1,\ldots,l_n)$ in a term that has the same structure
of~$e$ but where each subterm is replaced by its resumer ($r_i$'s).

Splitting {\tt case} yields a combined term that executes one of the branches' combined terms based on the \bbone-result $y_1$ of the predicate.
The boundary value $b_i$ for this branch is injected and bundled with that of the predicate ($b_1$).   
$b_i$ is cased in the resumer to determine which branch's resumer should be executed.
{\tt case} and $\pause$ are the only two rules where splitting adds non-trivial logic is added to the precomputation.

Function introduction has a \texttt{()} boundary value,
since functions are already fully reduced in our semantics.
However, since the body of a function may itself be multi-stage, splitting must continue into it.
The \bbone-result is a new function formed from the stage~\bbone\ part of the original body.
The resumer is a new function formed out of the stage~\bbtwo\ part of the original body.
It is the responsibility of the application site to save the precomputation of the function body
and pass it to the resumer version of the function.

Since the results of splitting $\next$ terms depend on the output of splitting its stage~\bbtwo\ subterm,
we defer description of $\next$ until after describing stage~\bbtwo\ term splitting.

\subsection{Stage \bbtwo\ Term Splitting}

Because stage~\bbtwo\ terms in \lang\ fully reduce to values (as opposed to partial values),
splitting stage~\bbtwo\ terms in \lang\ assumes a simpler form than that of stage~\bbone\ term splitting. 
Specifically, splitting a stage~\bbtwo\ term $e$ in \lang\ generates a precomputation term $p$
(performing all stage~\bbone\ subcomputations in $e$) and a
resumer term $e_2=l.r$ (performing all stage~\bbtwo\ subcomputations) such that when 
$e$ reduces to the value~$v$, $\letin l p r$ reduces to an identical value in \langmono.
Note that in contrast to the combined term produced during stage~\bbone\ splitting,
the precomputation $p$ generated by stage~\bbtwo\ splitting exists entirely
to compute the boundary value expected by the resumer.

% \ur{I don't quite understand the following.}
% (Thus, the resumer, bound to the value output by $p$, is equivalent to
% the residual term defined in~\ref{ssec:dynamics})

As with stage~\bbone\ terms, the splitting algorithm for stage~\bbtwo\ terms,
as specified by the judgment $e \splittwosym [p,l.r]$ in \cref{fig:termSplit}, is defined on the local structure of $e$.
In the terminal cases of
constants and variables, splitting generates trivial precomputations that are \texttt{()}, and resumers consisting of the original term.
For example, the integer constant \texttt{3} splits into the
precomputation \texttt{()} and resumer \texttt{\_=>3}.

For all non-terminal terms~$e$ in \lang\ (except $\prev$
and $\pause$)
% recursively splitting
% subterms and generating precomputations $p_1, \ldots, p_n$ and
% resumers $r_1, \ldots, r_n$ for a term with~$n$ immediate subterms.
the precomputation of~$e$ is defined as the combined precomputations of $e$'s $n$ subterms:
$p=(p_1,\ldots,p_n)$.  The resumer binds each boundary value to an
argument $(l_1,\ldots,l_n)$ in a term that has the same structure
of~$e$ but where each subterm is replaced by its corresponding resumer $r_i$.

A notable properly of splitting stage~\bbtwo\ \texttt{case}s and functions is that the
precomputation of subterms is lifted out from underneath stage
\bbtwo\
binders.  % TODO: should probably draw a parallel to the same behavior in dynamics
\ur{This last point seems important.  It would be important to come
  back to it.}

Splitting $\prev$ generates a precomputation that projects the \bbone-result of its stage~\bbone\ subterm.
Since the argument to $\prev$ is of $\fut$ type, its \bbone-result reduces to \texttt{()}, justifying why it can be thrown away.
$\pause$ treats the entire combined term of its stage~\bbone\ subexpression as a precomputation, 
and projects out the integer result in the resumer. 
\footnote{The resumer of an integer expression is usually trivial, 
but we have to include it here for termination purposes.} 
Finally, splitting $\next$ simply tuples up the precomputation of its stage~\bbtwo\ subterm with a trivial \bbone-result \texttt{()}.

%!TEX root = ../paper.tex

\begin{abstrsyn}
\begin{figure}[t]
\begin{mathpar}
\infer {x 					\vsplito \mval x x}														{\cdot}												\and
\infer {\tup{} 				\vsplito \mval {\tup{}} {\tup{}}}										{\cdot}												\and
\infer {\pure m 			\vsplito \mval m {\tup{}}}												{\cdot}												\and
\infer {\next y 			\vsplito \mval {\tup{}} y}												{\cdot}												\and
\infer {\tup{v_1,v_2} 		\vsplito \mval {\tup{i_1,i_2}} {\tup{q_1,q_2}}}							{v_1 \vsplito \mval {i_1} {q_1} 
																									&v_2 \vsplito \mval {i_2} {q_2}}					\and
\infer {\scont v 			\vsplito \mval {\scont i} q}											{v \vsplito \mval i q
																									& \scont \dash \in \{ 
																										\inl \dash , 
																										\inr \dash,
																										\roll \dash 
																									\}}								\and
\infer {\fix f x e			\vsplito \mval {\fix f x {\letin {\tup{x,y}} c {\tup{x,\roll y}}}} 
							{\fix f {\tup{x,\roll l}} r}}											{\splitone e A c {l.r}}								\and
\end{mathpar}
\caption{Value splitting rules.}
\label{fig:valueSplit}
\end{figure}

\begin{figure}
\begin{mathpar}
\infersplitone 					{\spl {\exv v} A {\exv{\tup{i,\tup{}}}} {\_.\exv q}} {v \vsplito \mval i q} \and 
\infersplitone [common2]		{\spl {\scont e}{A}{\letin{\tup{y,z}}{c}
									{\tup{\scont {\exv y},\exv z}}}{l. r}}															{\sub {} {A} 
																																	& \scont \dash \in \{ 
																																	\inl \dash , 
																																	\inr \dash , 
																																	\roll \dash,
																																	\unroll \dash
																																	\}} 		\and
\infersplitone [\fut I]			{\spl {\next e}{\fut A}{\tup{\tup{},p}}{l.r}}														{\splittwosub {} A} 			\and
\infersplitone [common1]		{\spl {\scont e}{A}{\letin{\tup{y,z}}{c}
									{\tup{\scont {\exv y},\exv z}}}{l.\scont r}}													{\sub {} {A} 
																																	& \scont \dash \in \{ 
																																	\pio \dash , 
																																	\pit \dash
																																	\}} 		\and
\infersplitone [\times I] 		{\splitonetall {\tup{e_1,e_2}}{A\times B}
									{\left(
										\talllet{\tup{y_1,z_1}}{c_1}{
										\talllet{\tup{y_2,z_2}}{c_2}{
										\exv{\tup{\tup{y_1, y_2},\tup{z_1, z_2}}}\ttrpar\ttrpar}}
									\right)}
									{\tup{l_1,l_2}.\tup{r_1,r_2}}}																	{\sub 1 A & \sub 2 B} \and
\infersplitone [\to E]			{\splitoneTall {\app {e_1}{e_2}}{B}{\left(
									\talllet{\tup{y_1,z_1}}{c_1}{
									\talllet{\tup{y_2,z_2}}{c_2}{
									\talllet{\tup{y_3,z_3}}{\app{y_1}{y_2}}{\exv{\tup{y_3,\tup{z_1,z_2,z_3}}}\ttrpar\ttrpar\ttrpar}}}
									\right)}
									{\tup{l_1,l_2,l_3}.\app{\exv {r_1}}{\exv{\tup{r_2,l_3}}}}}										{\sub 1 {A \to B} & \sub 2 A}							\and
\infersplitone				{\spl {\pure e} {\curr A} {\tup{e,\tup{}}}{\_.\tup{}}}								{\cdot}							\and
\infersplitone				{\splitoneTall {\letp x{e_1}{e_2}} {?} 
							{\letin {\tup{x,z_1}} {c_1} {
							 \letin {\tup{y_2,z_2}} {c_2} {\tup{y_2,\tup{z_1,z_2}}}}}
							 {\tup{l_1,l_2}.\letin{\_}{r_1}{r_2}}}												{\sub 1 ? & \sub 2 ?}			\and
\infersplitone [+ E]		{\splitoneTall {\caseP{e_1} {x_2.e_2} {x_3.e_3}}{C}
								{\left(
								\talllet{\tup{y_1,z_1}}{c_1}{
									\tallcase{y_1}
									{x_2.\letin{\tup{y_2,z_2}}{c_2}{\exv {\tup{y_2,\tup{z_1,\inl{z_2}}}}}}
									{x_3.\letin{\tup{y_3,z_3}}{c_3}{\exv {\tup{y_3,\tup{z_1,\inr{z_3}}}}}\ttrpar}
								}\right)}
								{\tup{l_1,l_b}.{\ttlpar r_1 \ttsemi \caseof{\exv{l_b}}{l_2.[\tup{}/x_2]{r_2}}{l_3.[\tup{}/x_3]{r_3}\ttrpar}
								}}}																									{\sub 1 {A+B} 
																																	& \sub [\Gamma,\col{x_2} A] 2 C 	
																																	& \sub [\Gamma,\col{x_3} B] 3 C} 		
\end{mathpar}
\hrule
\begin{mathpar}
\infersplittwo 					{\spl {\exv q} A {\exv {\tup{}}} \_ q} 														{\cdot} 						\and 
\infersplittwo [\fut E]			{\spl {\prev e}{A} {\pit c} l r }															{\splitonesub {} {\fut A}} 		\and 
\infersplittwo [\times E_1]		{\spl {\scont e}{A}{p}{l}{\scont r}}														{\sub {} {A\times B} 
																															& \scont \dash \in \{ 
																																\pio \dash , 
																																\pit \dash,  
																																\inl \dash,  
																																\inr \dash,  
																																\roll \dash, 
																																\unroll \dash, 
																																\fix fx \dash 
																															\}} 				\and  
\infersplittwo [\to E]			{\spl {\scont{e_1,e_2}}{B}{\tup{p_1,p_2}}{\tup{l_1,l_2}}{\scont{r_1,r_2}}}				{\sub 1 {A \to B} & \sub 2 A
																															& \scont {\dash, \dash} \in \{ 
																																\tup {\dash,\dash} , 
																																\app \dash \dash,
																																\letin x \dash \dash 
																															\}}									\and			
\infersplittwo [+ E]			{\splittwo {\caseof{e_1}{x_2.e_2}{x_3.e_3}}{C}
								{\tup{p_1,p_2,p_3}}{ \tup{l_1,l_2,l_3}}{\caseof{r_1}{x_2.r_2}{x_3.r_3}}}					{\sub 1 {A+B} & \sub [\Gamma,\col{x_2} A] 2 C 
																															& \sub [\Gamma,\col{x_3} B] 3 C} 								
\end{mathpar} 
\caption{Splitting rules for terms at \bbonem\ and \bbtwo.}
\label{fig:termSplit}
\end{figure}
\end{abstrsyn}


%!TEX root = ../paper.tex

\begin{figure*}
\begin{abstrsyn}
\begin{mathpar}
\infer {\rtab {\hat y \mapsto q,\xi} v	\vsplito \mval i {\letin {\hat y} q r}}				{\rtab \xi v \vsplito \mval i r}					\and
\infer {\rtab \cdot {\tup{}} 			\vsplito \mval {\tup{}} {\tup{}}}					{\cdot}												\and
\infer {\rtab \cdot {\pure m} 			\vsplito \mval m {\tup{}}}							{\cdot}												\and
\infer {\rtab \cdot {\next{\hat y}} 	\vsplito \mval {\tup{}} {\hat y}}					{\cdot}												\and
\infer {\rtab \cdot {\tup{v_1,v_2}} 	\vsplito \mval {\tup{i_1,i_2}} {\tup{r_1,r_2}}}		{\rtab \cdot {v_1} \vsplito \mval {i_1} {r_1} 
																							&\rtab \cdot {v_2} \vsplito \mval {i_2} {r_2}}		\and
\infer {\rtab \cdot {\scriptCapp v} 	\vsplito \mval {\scriptCapp i} r}					{\rtab \cdot v \vsplito \mval i r
																							&\scriptC \in 
																							\{\mathtt{inl},\mathtt{inr},\mathtt{roll}\}}		\and
\infer {\rtab \cdot {\lam x e}			\vsplito \mval {\lam x c} {\lam {\tup{x,l}} r}}		{\splitone e A c {l.r}}								\and
\end{mathpar}
\end{abstrsyn}
\caption{Masking separates a residual table and its associated partial value into its first- and second-stage components.}
\label{fig:valMask}
\end{figure*}


\subsection{Improved Definition}
\label{sec:resumers}

The definitions of masking and splitting can be augmented to better handle special cases.
One such special case is the behavior of stage \bbone\ splitting at stage \bbone\ types that contain no $\fut$s or $\to$s.
We call such types {\em monostage} (judgmentally, $A~mono$), because a partial value $v$ with a monostage type can contain no $\next$s.

Intuitively, for $[\xi;v] : A$ and $A~mono$, we would expect that $\maskt{\xi;v}$ only reduce to something ``trivial."
There are two reasonable definitions of trivial here: 
\begin{enumerate}
\item a value is trivial if it is exactly \texttt{()},
\item a value is trivial if it is in the closure of \texttt{()} and \texttt{(-,-)}
\end{enumerate}
Indeed, as $\maskt{-;-}$ is currently defined, this property holds for the latter definition of trivial,
but we can make it hold for the former definition by adjusting just the tuple rule:
\[
\maskt{\cdot;(v_1,v_2)} = \left\{ \begin{array}{ll} 
\mathtt{()} &\text{if } (v_1,v_2):A~mono\\ 
(\maskt{\cdot;v_1},\maskt{\cdot;v_2}) &\text{otherwise} \end{array} \right.
\]
From here, the splitting rules can be made to comport with this new 
definition adding special-case tuple, projection, and application 
rules\footnote{The old tuple, projection, and application rules 
must also be augmented to apply only when the $mono$ premises don't hold.
Also, all splitting rules need to be augmented to carry types, though
we omit the contexts here.}:
\begin{mathpar}
\infer
	{(e_1,e_2):A\times B\splitonesym [\ldots,(l_1,l_2).(r_1;r_2;())]}
	{e_1 : A \splitonesym [c_1,l_1.r_1] 
	& e_2 : B \splitonesym [c_2,l_2.r_2] 
	& A \times B~mono} 
\and
\infer
	{\pio~e:A\splitonesym [\ldots,l.(r;())]}
	{e : A\times B \splitonesym [c,l.r] 
	& A \times B~mono} 
\and
\infer
	{e_1~e_2: B\splitonesym [\ldots,(l_1,l_2,l_3).(r_1 (r_2,l_3); ())]}
	{e_1 : A \to B \splitonesym [c_1,l_1.r_1] 
	& e_2 : A \splitonesym [c_2,l_2.r_2] 
	& B~mono} 
\end{mathpar}