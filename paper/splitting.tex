%!TEX root = paper.tex

\section{Splitting Algorithm}
\label{sec:splitting}

\begin{figure}
\textbf{Languages:}
\begin{itemize}
\item \lang: two-staged lambda calculus with product, sum, and
  recursive types
\item \langmono: an unstaged lambda calculus with products.
\end{itemize}

\textbf{\lang\ Evaluation Relations:}
\begin{itemize}
\item 
\bbone-Evaluation: $e\mathbin{\redonesym}[\xi;v]$, where $\xi$ is a \emph{residual table} and $v$ is a \emph{partial value}. 

\item
\bbtwo-Evaluation: $e\mathbin{\redtwosym}q$, where $q$ is a \emph{residual} in the unstaged language \langTwo

\item 
Table reification: \ur{Fill in...}
\end{itemize}



% $red_2$: 2-eval,  which we think of as being identical to specialization. 

\vspace{.75em}
\textbf{\lang\ Splitting:}

\hspace{2em}\bbone-Splitting Structure: $e \splitonesym [c,l.r]$, where:

\hspace{4em}$c$ is the \emph{combined term} (representing all stage~\bbone\ subcomputations in $e$)

\hspace{4em}$l.r$ is the \emph{resumer} (representating all stage~\bbtwo\ subcomputations in $e$)

\hspace{4em}$c~\redsym~(y,b)$, where $y$ is the \emph{\bbone-result} of $e$, and $b$ is the \emph{boundary value} of $e$

 
\hspace{2em}\bbone-Splitting Correctness: If $e\mathbin{\redonesym}[\xi;v]$, then:

\hspace{4em}$y$ is identical to $\masko{\xi;v}$

\hspace{4em}$\maskt{\xi;v}$ and \texttt{(let l=b in r)} reduce (via $\redsym$) to identical values.  

\hspace{2em}\bbtwo-Splitting Structure: $e \splittwosym [p,l.r]$, where:

\hspace{4em}$p$ is the \emph{precomputation} (representing all stage~\bbone\ subcomputations in $e$)

\hspace{4em}$l.r$ is the \emph{resumer} (representating all stage~\bbtwo\ subcomputations in $e$)

\hspace{2em}\bbtwo-Splitting Correctness: If $e~\redtwosym~q$, and $q~\redsym~v$, then \texttt{(let l=b in r)}~$\redsym~v$

\caption{Summary of \lang\ evaluation and splitting.}
\label{fig:terminology}
\label{fig:termSplitSummary}
\end{figure}



\begin{abstrsyn}

Although the type system of \lang\ ensures that we can perform all first-stage
computations before any second-stage ones, actually evaluating terms in this
fashion is rather involved, because stages are syntactically interleaved
within terms. In this section, we present a \emph{splitting algorithm} which
statically converts a term in \lang\ into a pair of 
syntactically % I don't like this word but temporarily here for continuity
separate programs corresponding to the two stages of computation in that term.

%The output of splitting is simpler because it is {\em syntactically separated};
%The output of splitting is equivalent (to its input) because it can be
%evaluated to the same residual that would be produced by the semantics of
%\ref{sec:semantics}.

More precisely, for any $\coltwo e A$ which reduces to a residual $q$ (via $\diatwo e q$),
$e$ splits into a syntactically separated program $\mathcal{P}$ 
(via the new relation $e \splittwosym \mathcal{P}$), which
can also be reduced to $q$ (via the new relation $\mathcal{P} \sepredtwosym q$).
We enforce the ``syntactically separated'' condition by saying that $\mathcal{P}$ 
can only have one form, namely $\pipeS p l r$, where  
$p$ (called the {\em precomputation}) is a monostage term encoding all the
first-stage computation that was in $e$, 
and $l.r$ (called the {\em resumer}) is a monostage term encoding all the
second-stage computation that was in $e$.  
Likewise, $\sepredtwosym$ is defined by a single rule, which evaluates the precomputation and plugs it in for $l$:
\[
\infer{\sepredtwo {\pipeS p l r} {[b/l]r}} {\reduce p b}
\]
Here we call $b$ the {\em boundary value}, as it represents the communication at the boundary between stages.
A summary of these operations and the relationship between them is provided on the right side of \ref{fig:splittingSummary}.

Since terms at world \bbtwo\ can depend on terms at \bbonem\ (via \texttt{prev}),
we must also provide a way to split those as well.
This is given by the new relation $e \splitonesym \mathcal{P}$,
where $\mathcal{P}$ must have the form $\pipeM c l r$ where  
$c$ (called the {\em combined term}) is a monostage term encoding all the first-stage computation that was in $e$, 
and $l.r$ (again called the {\em resumer}) is a monostage term encoding all the
second-stage computation that was in $e$.

Similar to above, we also define a $\sepredtwosym$ relation 
which can be used to evaluate these seperated world \bbonep terms.
One complication arises from the fact that \lang\ terms at world \bbonem\ produce multistage output.
For example, consider the term
\begin{lstlisting}
(next {1+2}, grnd{3+4})
\end{lstlisting}
which reduces (via $\redonesym$) to the residual table and partial value
\begin{lstlisting}
[yhat |-> 1+2] (next {yhat}, grnd{7})
\end{lstlisting}
Since this output has interleaved stages, theres no way a syntactically separated term could directly reduce to it.

We resolve this mismatch by introducing a translation, called {\em masking} and written $\rtab \xi v \vsplito \mathcal{V}$,
which converts residual tables and partial values---like the those above---into a syntactically separated version $\mathcal{V}$.
As before, we enforce the ``syntactically separated'' property structurally,
saying that $\mathcal{V}$ must have the form $\mval i q$ (called a {\em masked value}) where 
$i$ is a monostage value representing the first-stage part of $v$,
and $q$ is a monostage term representing all of the second-stage computation in $\xi$ and $v$.
Note that masking is a technical device for the purpose of stating correctness of splitting at \bbonem; 
it is not part of the splitting algorithm itself.

With this machinery in place, we can define $\sepredonesym$ as just creating masked values with the single rule:
\[
\infer{\sepredone {\pipeM c l r} {\mval i {[b/l]r}}} {\reduce c {(i,b)}}
\]
Here we see the primary difference between world \bbtwo\ terms and world \bbonem\ terms.
Since $\coltwo e A$ reduces to an entirely second-stage result, 
its first-stage subcomputations only exist to internally pass their value to the
second stage;
but since $\colmix e A$ reduces to a multistage result, 
its first-stage subcomputations exist {\em both} to pass their value to the
second stage {\em and} to be present in the result.
Correspondingly, in the $\sepredtwosym$ rule, the precomptuation $p$ reduces only to a boundary value which is passed to the resumer;
but in the $\sepredonesym$ rule, the combined term $c$ reduces to a tuple containing both the immediate result and the boundary value,
with only the latter being passed to the resumer.

A summary of these operations and the relationship between them is provided on the left side of \ref{fig:splittingSummary}.

This section proceeds by defining masking ($\vsplito$), 
then using masking to motivate the definition of \bbonem-translation ($\splitonesym$),
and finally coming back to define \bbtwo-translation ($\splittwosym$).

\subsection{Masking}

The point of the masking operation is simply to convert a partial value into a masked value $\mval i q$
by assigning all of the first-stage content to $i$ and all of the second-stage content to $q$.
The rules of the masking relation are given in \ref{fig:valMask}.

Masking operates by first inducting on the entries of the residual table.  
Being purely second-stage content, these are reified into let statements at the top of the resumer.
Once the table is empty, masking inducts on value itself.

Masking assigns values in \bbonep\ to the immediate value
and likewise assigns references into the residual table to the resumer.
In both cases, the alternate component is assigned to $\tup{}$, to represent trivial information.

Masking distributes into tuples, injections, and rolls, since their subvalues may have content at both stages.
However, the tags of injections and rolls are replicated only in the immediate value, 
since they represent first-stage information.

Since lambdas may represent multistage computations, 
masking splits the body of lambdas as general world \bbonem\ terms (as described in \ref{sec:split-one}), 
and packages the resulting terms as functions.

\subsection{Term Splitting at \bbonem}
\label{sec:split-one}

We now show how to translate terms $\colmix e A$ into the form $\pipeM c l r$,
pursuant to the correctness condition given in \ref{fig:splittingSummary}.
The algorithm is specified by the $\splitonesym$ relation (\cref{fig:termSplit}), 
which proceeds recursively on the structure of~$e$.

When $e$ is unit or a variable,
splitting yields a combined term formed by tupling $e$ with a $\tup{}$ precomputation.
These expressions, as they contain no second-stage subcomputations, have a trivial resumer, $\tup{}$.
Similarly, the contents of a \texttt{grnd} block are assigned entirely to the immediate value,
with trivial precomputation and resumer.

For all non-terminals (except \texttt{next}),
splitting descends into $e$, recursively splitting its $n$ subterms
to produce their respective combined terms $c_1,\ldots,c_n$ and resumers $r_1, \ldots, r_n$.
The combined term of $e$ is formed by binding $c_1,\ldots,c_n$
to the patterns $(y_1,z_1),\ldots,(y_n,z_n)$
to isolate immediate results from boundary values. Then,
the immediate result of $e$ is formed by replacing $e$'s subterms with $y_1,\ldots,y_n$.
The resumer binds the boundary values $b_1,\ldots,b_n$ to an
argument $(l_1,\ldots,l_n)$ in a term that has the same structure
of~$e$ but where each subterm is replaced by its resumer ($r_i$'s).

Splitting {\tt case} yields a combined term that executes one of the branches' combined terms based on the immediate result $y_1$ of the predicate.
The boundary value $b_i$ for this branch is injected and bundled with that of the predicate ($b_1$).   
$b_i$ is cased in the resumer to determine which branch's resumer should be executed.
{\tt case} is the only rule where splitting adds non-trivial logic is added to the precomputation.

Function introduction has a $\tup{}$ boundary value,
since functions are already fully reduced in our semantics.
However, since the body of a function may itself be multistage, splitting must continue into it.
The immediate result is a new function formed from the first-stage part of the original body.
The resumer is a new function formed out of the second-stage part of the original body.
It is the responsibility of the application site to save the precomputation of the function body
and pass it to the resumer version of the function.

Since the results of splitting \texttt{next} terms depend on the output of
splitting its world \bbtwo\ subterm,
we defer description of \texttt{next} until after describing world \bbtwo\ term splitting.

\subsection{Term Splitting at \bbtwo}

Because world \bbtwo\ terms in \lang\ reduce to monostage residuals (as opposed to partial values),
term splitting at world \bbtwo\ assumes a simpler form than the version at \bbonem\ does. 
The algorithm is specified by the $\splittwosym$ relation in \cref{fig:termSplit}.

In the terminal cases of
constants and variables, splitting generates trivial precomputations that are \texttt{()}, and resumers consisting of the original term.
For example, the integer constant \texttt{3} splits into the
precomputation \texttt{()} and resumer \texttt{\_=>3}.

More generally, for all (except \texttt{prev}) 
$n$-ary terms $e = \mathcal{C}\ttlpar e_1 \ttsemi \ldots \ttsemi e_n \ttrpar$ 
the precomputation is the tupled precomputations of $e$'s $n$ subterms:
$p=(p_1,\ldots,p_n)$.  The resumer binds each boundary value to an
argument $(l_1,\ldots,l_n)$ in a term that has the same structure
of~$e$ but where each subterm is replaced by its corresponding resumer:
$r = \mathcal{C}\ttlpar r_1 \ttsemi \ldots \ttsemi r_n \ttrpar$ .
Notably, at \texttt{case}s and functions the
precomputation of subterms is lifted out from underneath world \bbtwo\ binders.  
% TODO: should probably draw a parallel to the same behavior in dynamics

Splitting \texttt{prev} generates a precomputation that projects the immediate
result of its world \bbone\ subterm.
Since the argument to \texttt{prev} is of $\fut$ type, its immediate result reduces to $\tup{}$, justifying why it can be thrown away.
Finally, splitting \texttt{next} simply tuples up the precomputation of its
world \bbtwo\ subterm with a trivial immediate result $\tup{}$.

\subsection {Necessity of \bbonep}
\label{sec:needGround}

We have not yet given a justification for why first stage code must be
partitioned between the worlds \bbonem\ and \bbonep.
Certainly one could imagine a simpler system without all the \texttt{grnd} annotations.  
Why wouldn't this work?

In order to be correct, any code at \bbonem\ must be split with the 
pessimistic assumption that it may result in work at the second stage.
When this assumption turns out not to be true---that is, purely monostage code---then 
splitting may produce second stage code which is needlessly costly 
(an example of this will occur in \ref{sec:exampleQS}).
Detecting and optimizing this case is in general a global program analysis,
since functions can be passed around as values.  
So instead of relying on hefty analysis, 
we take the approach of adding enough structure to the input language's type system to
allow the input code to {\em prove} itself to be---in some parts---monostage.

We find that giving input with \texttt{grnd} annotations is not especially cumbersome in practice.
Yet even if a language designer desired a different implicit/explicit trade-off,
then the three-world \lang\ would still be useful as a typed intermediate representation.

\end{abstrsyn}

%!TEX root = ../paper.tex

\begin{figure*}
\begin{abstrsyn}
\begin{mathpar}
\infer {\rtab {\hat y \mapsto q,\xi} v	\vsplito \mval i {\letin {\hat y} q r}}				{\rtab \xi v \vsplito \mval i r}					\and
\infer {\rtab \cdot {\tup{}} 			\vsplito \mval {\tup{}} {\tup{}}}					{\cdot}												\and
\infer {\rtab \cdot {\pure m} 			\vsplito \mval m {\tup{}}}							{\cdot}												\and
\infer {\rtab \cdot {\next{\hat y}} 	\vsplito \mval {\tup{}} {\hat y}}					{\cdot}												\and
\infer {\rtab \cdot {\tup{v_1,v_2}} 	\vsplito \mval {\tup{i_1,i_2}} {\tup{r_1,r_2}}}		{\rtab \cdot {v_1} \vsplito \mval {i_1} {r_1} 
																							&\rtab \cdot {v_2} \vsplito \mval {i_2} {r_2}}		\and
\infer {\rtab \cdot {\scriptCapp v} 	\vsplito \mval {\scriptCapp i} r}					{\rtab \cdot v \vsplito \mval i r
																							&\scriptC \in 
																							\{\mathtt{inl},\mathtt{inr},\mathtt{roll}\}}		\and
\infer {\rtab \cdot {\lam x e}			\vsplito \mval {\lam x c} {\lam {\tup{x,l}} r}}		{\splitone e A c {l.r}}								\and
\end{mathpar}
\end{abstrsyn}
\caption{Masking separates a residual table and its associated partial value into its first- and second-stage components.}
\label{fig:valMask}
\end{figure*}

%!TEX root = ../paper.tex

\begin{abstrsyn}
\begin{figure}[t]
\begin{mathpar}
\infer {x 					\vsplito \mval x x}														{\cdot}												\and
\infer {\tup{} 				\vsplito \mval {\tup{}} {\tup{}}}										{\cdot}												\and
\infer {\pure m 			\vsplito \mval m {\tup{}}}												{\cdot}												\and
\infer {\next y 			\vsplito \mval {\tup{}} y}												{\cdot}												\and
\infer {\tup{v_1,v_2} 		\vsplito \mval {\tup{i_1,i_2}} {\tup{q_1,q_2}}}							{v_1 \vsplito \mval {i_1} {q_1} 
																									&v_2 \vsplito \mval {i_2} {q_2}}					\and
\infer {\scont v 			\vsplito \mval {\scont i} q}											{v \vsplito \mval i q
																									& \scont \dash \in \{ 
																										\inl \dash , 
																										\inr \dash,
																										\roll \dash 
																									\}}								\and
\infer {\fix f x e			\vsplito \mval {\fix f x {\letin {\tup{x,y}} c {\tup{x,\roll y}}}} 
							{\fix f {\tup{x,\roll l}} r}}											{\splitone e A c {l.r}}								\and
\end{mathpar}
\caption{Value splitting rules.}
\label{fig:valueSplit}
\end{figure}

\begin{figure}
\begin{mathpar}
\infersplitone 					{\spl {\exv v} A {\exv{\tup{i,\tup{}}}} {\_.\exv q}} {v \vsplito \mval i q} \and 
\infersplitone [common2]		{\spl {\scont e}{A}{\letin{\tup{y,z}}{c}
									{\tup{\scont {\exv y},\exv z}}}{l. r}}															{\sub {} {A} 
																																	& \scont \dash \in \{ 
																																	\inl \dash , 
																																	\inr \dash , 
																																	\roll \dash,
																																	\unroll \dash
																																	\}} 		\and
\infersplitone [\fut I]			{\spl {\next e}{\fut A}{\tup{\tup{},p}}{l.r}}														{\splittwosub {} A} 			\and
\infersplitone [common1]		{\spl {\scont e}{A}{\letin{\tup{y,z}}{c}
									{\tup{\scont {\exv y},\exv z}}}{l.\scont r}}													{\sub {} {A} 
																																	& \scont \dash \in \{ 
																																	\pio \dash , 
																																	\pit \dash
																																	\}} 		\and
\infersplitone [\times I] 		{\splitonetall {\tup{e_1,e_2}}{A\times B}
									{\left(
										\talllet{\tup{y_1,z_1}}{c_1}{
										\talllet{\tup{y_2,z_2}}{c_2}{
										\exv{\tup{\tup{y_1, y_2},\tup{z_1, z_2}}}\ttrpar\ttrpar}}
									\right)}
									{\tup{l_1,l_2}.\tup{r_1,r_2}}}																	{\sub 1 A & \sub 2 B} \and
\infersplitone [\to E]			{\splitoneTall {\app {e_1}{e_2}}{B}{\left(
									\talllet{\tup{y_1,z_1}}{c_1}{
									\talllet{\tup{y_2,z_2}}{c_2}{
									\talllet{\tup{y_3,z_3}}{\app{y_1}{y_2}}{\exv{\tup{y_3,\tup{z_1,z_2,z_3}}}\ttrpar\ttrpar\ttrpar}}}
									\right)}
									{\tup{l_1,l_2,l_3}.\app{\exv {r_1}}{\exv{\tup{r_2,l_3}}}}}										{\sub 1 {A \to B} & \sub 2 A}							\and
\infersplitone				{\spl {\pure e} {\curr A} {\tup{e,\tup{}}}{\_.\tup{}}}								{\cdot}							\and
\infersplitone				{\splitoneTall {\letp x{e_1}{e_2}} {?} 
							{\letin {\tup{x,z_1}} {c_1} {
							 \letin {\tup{y_2,z_2}} {c_2} {\tup{y_2,\tup{z_1,z_2}}}}}
							 {\tup{l_1,l_2}.\letin{\_}{r_1}{r_2}}}												{\sub 1 ? & \sub 2 ?}			\and
\infersplitone [+ E]		{\splitoneTall {\caseP{e_1} {x_2.e_2} {x_3.e_3}}{C}
								{\left(
								\talllet{\tup{y_1,z_1}}{c_1}{
									\tallcase{y_1}
									{x_2.\letin{\tup{y_2,z_2}}{c_2}{\exv {\tup{y_2,\tup{z_1,\inl{z_2}}}}}}
									{x_3.\letin{\tup{y_3,z_3}}{c_3}{\exv {\tup{y_3,\tup{z_1,\inr{z_3}}}}}\ttrpar}
								}\right)}
								{\tup{l_1,l_b}.{\ttlpar r_1 \ttsemi \caseof{\exv{l_b}}{l_2.[\tup{}/x_2]{r_2}}{l_3.[\tup{}/x_3]{r_3}\ttrpar}
								}}}																									{\sub 1 {A+B} 
																																	& \sub [\Gamma,\col{x_2} A] 2 C 	
																																	& \sub [\Gamma,\col{x_3} B] 3 C} 		
\end{mathpar}
\hrule
\begin{mathpar}
\infersplittwo 					{\spl {\exv q} A {\exv {\tup{}}} \_ q} 														{\cdot} 						\and 
\infersplittwo [\fut E]			{\spl {\prev e}{A} {\pit c} l r }															{\splitonesub {} {\fut A}} 		\and 
\infersplittwo [\times E_1]		{\spl {\scont e}{A}{p}{l}{\scont r}}														{\sub {} {A\times B} 
																															& \scont \dash \in \{ 
																																\pio \dash , 
																																\pit \dash,  
																																\inl \dash,  
																																\inr \dash,  
																																\roll \dash, 
																																\unroll \dash, 
																																\fix fx \dash 
																															\}} 				\and  
\infersplittwo [\to E]			{\spl {\scont{e_1,e_2}}{B}{\tup{p_1,p_2}}{\tup{l_1,l_2}}{\scont{r_1,r_2}}}				{\sub 1 {A \to B} & \sub 2 A
																															& \scont {\dash, \dash} \in \{ 
																																\tup {\dash,\dash} , 
																																\app \dash \dash,
																																\letin x \dash \dash 
																															\}}									\and			
\infersplittwo [+ E]			{\splittwo {\caseof{e_1}{x_2.e_2}{x_3.e_3}}{C}
								{\tup{p_1,p_2,p_3}}{ \tup{l_1,l_2,l_3}}{\caseof{r_1}{x_2.r_2}{x_3.r_3}}}					{\sub 1 {A+B} & \sub [\Gamma,\col{x_2} A] 2 C 
																															& \sub [\Gamma,\col{x_3} B] 3 C} 								
\end{mathpar} 
\caption{Splitting rules for terms at \bbonem\ and \bbtwo.}
\label{fig:termSplit}
\end{figure}
\end{abstrsyn}


