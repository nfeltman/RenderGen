
\section{Splitting Algorithm}
\label{sec:splitting}

%As with the dynamics, our stage-splitting algorithm for expressions in \lang\
%takes the form of two mutually-dependent judgments, $\splitonesym$ and
%$\splittwosym$, which respectively split terms that type $@ \bbone$ and $@
%\bbtwo$.  Before diving into the full generality, 

The goal of splitting is to take a multistage \lang\ term,
which may feature interleaved stage \bbone\ and stage \bbtwo\ code,
and untangle it to produce two separate terms:
one with all of the stage \bbone\ CONTENT, and one with all of
the stage \bbtwo\ CONTENT.

Splitting emits terms in an unstaged language, \langmono, that is 
untyped and contains no staging features, but otherwise has the same term constructors as \lang. 
\langmono\ has a evaluation judgment $\redsym$, which works in the expected way to produce unstaged values.

Since splitting stage \bbtwo\ terms from \lang\ is a simpler operation than splitting stage 1 terms,
we first describe the mechanics of splitting stage \bbtwo, before moving to the more complex
case of stage \bbone\ term splitting.

%The splitting algorithm comes in several forms, depending on the stage of the input and whether it is a value.
%The stage \bbtwo\ versions will be addressed first,
%before the more complicated stage \bbone\ versions.


\subsection {Stage \bbtwo\ Splitting}

% We cover stage \bbtwo\ term splitting first, because it has the most familiar signature.

Splitting a stage-2 term $\coltwo e A$ produces a term
$p$ (called the {\em precomputation}) and a term $l.r$ (called the {\em resumer})
open on a single variable, such that $e$ is equivalent to $\letin l p r$.
Intuitively, the responsibility of $p$ is to produce an intermediate data structure, 
and the responsibility of $r$ is to consume that data structure and produce the same final value as $e$.

 For example, the stage \bbtwo\ term
\begin{lstlisting}
2`hold{`1` 1+2 `2`} < hold{`1` 3+4 `2`} - 5`
\end{lstlisting}
splits into a precomputation \verb|(1+2,3+4)| and a resumer
\verb|(L1,L2) => L1 < L2-5|, with the property that the resumer applied to
the precomputation must agree with the original term. This process of hoisting
all the stage \bbone\ subterms out of a term is called \emph{precomputation lifting}.

Splitting is implemented by the judgment $\splittwosym$,
the rules for which are given in \cref{fig:termSplitTwo}.  
It operates inductively on the structure of the term.  
As mentioned before, the only responsibility of $\splittwosym$ is to lift out the precomputation
from contained stage \bbone\ code into a single expression $p$, 
leaving a resumer $l.r$ to interpret the results of that precomputation.
Since base types and variables have no stage \bbone\ content, their precomputation is trivial;
that is, $p$ is a unit value and $l$ is never used in $r$.
At every other expression except $\prev$, the $\splittwosym$ rule works by bundling 
the precomputations of the constituent parts, and then unbundling them with a pattern,
with the resumer precisely resembling the original expression.
The $\prev$ rule depends on splitting stage \bbone\ terms, which is covered later.

\subsection{(Partial) Value Splitting}

Observe that when applied to a stage \bbtwo\ values, 
splitting always produces an effectively trivial precomputation:
either \texttt{()}, or nested tuples thereof.
Intuitively, this makes sense---there is no way to embed stage \bbone\ code
in the internals of a stage \bbtwo\ value, 
thus they can have no stage \bbone\ content.
Indeed it would be almost nonsensical to talk of an algorithm for splitting stage \bbtwo\ values,
since the precondition, that the input is a stage \bbtwo\ value, is enough to imply the postcondition,
that no stage interleaving remains.

The same is not true of partial values---partial values, a restriction on stage \bbone\ terms,
{\em can} have stage \bbtwo\ content.
Thus, it is meaningful to define some notion of splitting for them.
This is useful to study, because the definition will later be used as a contract 
with which general stage \bbone\ term splitting must comply.

Recall that a partial value is externally a stage \bbone\ thing; 
it has a type $@~\bbone$, and along with a future context, it is the result of reducing a stage \bbone\ term.
But internally, a partial value might contain stage \bbtwo\ subexpressions.
There are two places where such embedding may occur: 
$\next$ blocks, which must contain a variable reference to some future context; 
and lambda values, which may contain a $\next$ in their body.
The goal of splitting for stage \bbone\ partial values is therefore 
to separate the contents of $\next$ blocks from the rest of the value, 
while maintaining some structure.

Splitting a partial value $v$ creates two unstaged values $v_1$ and $v_2$,
where $v_w$ contains only the information required for stage $w$ evaluation of
$v$. Splitting partial values is implemented by the judgment $v \vsplito
[v_1,v_2]$ defined in \ref{fig:valSplit}. 
At product, sum, and recursive types, splitting acts recursively, gathering all
subterms' stage $w$ content into $v_w$. At base types, splitting assigns $v$
entirely to stage \bbone\ (and \texttt{()} to stage \bbtwo); on the other hand,
at $\fut$ types, splitting assigns $v$ (which must have the form $\next~\hat y$)
entirely to stage \bbtwo\ (and \texttt{()} to stage \bbone). This process of
exposing only the relevant structure of a term is called \emph{masking}.

\ur{The case for ``roll'' seems missing in partial value splitting rules.}

Splitting a function is more complex, because function bodies are open terms,
and so are split by the judgments described in the next subsection. Essentially,
functions will split into two functions, whose bodies correspond to the stage
\bbone\ and stage \bbtwo\ components of the original function body.

\ur{Unclear whether the term ``masking'' includes the case for
  functions or not.  If it does, then it sounds rather
  inaccurate.  If not, then it should be stated clearly that it does
  not, but then what is the value of introducing one more term?}

\subsection{Stage \bbone\ Term Splitting}

Lifting and masking are two necessary ingredients for ensuring that splitting
observes the stages of \lang. When splitting a partial value, the stage \bbone\
result must \emph{mask} the stage \bbtwo\ computations in order to be
single-staged. When splitting a stage \bbtwo\ term, the precomputation must
\emph{lift} out all embedded stage \bbone\ computations, so they can be
performed first, even though their results are only used in stage \bbtwo.
%and it's accessible to whatever surrounding piece of code refers to the value being split.
%is not accessible externally.

Splitting a stage \bbone\ term, in contrast, requires both masking and splitting.
Consider the term 
\begin{lstlisting}
1`(next{`2`hold{`1` 1+2 `2`} < 5`1`},3*4)`
\end{lstlisting}
Masking yields the stage \bbone\ computation \verb|((),3*4)|, while lifting
yields the (stage \bbone) precomputation \verb|(1+2)|. While both computations
must occur at stage \bbone, the former is needed by stage \bbone, while the
latter is needed by stage \bbtwo.

The precomputation for this term is a \emph{combined result} which is
the pair of these components: \verb|(((),3*4),1+2)|. The resumer is
likewise the pair of the stage \bbtwo\ results of lifting and masking:
\verb|fn L => (L<5,())|.
%\footnote{In this example, the combined result is a 2-tuple, but more generally
%it only must be a term which reduces to a 2-tuple.}

Splitting is implemented by the judgment $\splitonesym$,
the rules for which are given in \cref{fig:termSplitOne}.  
It operates inductively on the structure of the term.  
The $\splitonesym$ judgment sends a stage \bbone\ term $e$ to a combined term
$c$ and resumer $l.r$. 

The rule for $\next$ simply tuples up the precomputation of its subexpression with a trivial immediate result,
while the rule for $\prev$ projects the combined result of its subexpression to isolate the precomputation.
It may seem like this $\prev$ rule is throwing information away, but we well show later that the first component 
of the evaluated form of $c$ in the $\prev$ rule must always be unit.
The $\pause$ rule treats the entire combined result of its subexpression as a precomputation, 
and projects out the integer result in the resumer.%
\footnote{The resumer of an integer expression is usually trivial, 
but we have to include it here for termination purposes.}

\subsection {Notable Properties}

\nr{These points will probably get lifted into the preceding sections.}

There are some particularly notable features of this definition of term
splitting.

\paragraph {Speculation}

In the stage \bbtwo\ rules for {\tt if}s, {\tt case}s, and functions, the
precomputation is lifted out from within branches. This is the manifestation of
the speculation behavior from the semantics.

\paragraph {Stage \bbone\ Divergence}

In the splitting rules for stage \bbone\ {\tt if} and {\tt case} expressions, we
have enough information at stage \bbone\ to know what branch to take, so there's
no need to speculate. Instead, we evaluate the stage \bbone\ portion of the
taken branch only, and inject the precomputation into a sum type. In the
resumer, we case on that injection, and resume the stage \bbtwo\ portion of the
correct branch.

\paragraph {Stage \bbone\ Functions}

The body of a stage \bbone\ function may contain stage \bbtwo\ computation. We
handle this by splitting the function into two functions: one in the immediate
result which handles all the stage \bbone\ content of the original, and one in
the resumer which handles all of the stage \bbtwo\ content. Note that stage
\bbone\ lambdas are partial values, and so have only a trivial precomputation.

\subsection {Boundary Type Worst Case}

\TODO add the example which shows that we need dynamic types


%!TEX root = ../paper.tex

\begin{abstrsyn}
\begin{figure}[t]
\begin{mathpar}
\infer {x 					\vsplito \mval x x}														{\cdot}												\and
\infer {\tup{} 				\vsplito \mval {\tup{}} {\tup{}}}										{\cdot}												\and
\infer {\pure m 			\vsplito \mval m {\tup{}}}												{\cdot}												\and
\infer {\next y 			\vsplito \mval {\tup{}} y}												{\cdot}												\and
\infer {\tup{v_1,v_2} 		\vsplito \mval {\tup{i_1,i_2}} {\tup{q_1,q_2}}}							{v_1 \vsplito \mval {i_1} {q_1} 
																									&v_2 \vsplito \mval {i_2} {q_2}}					\and
\infer {\scont v 			\vsplito \mval {\scont i} q}											{v \vsplito \mval i q
																									& \scont \dash \in \{ 
																										\inl \dash , 
																										\inr \dash,
																										\roll \dash 
																									\}}								\and
\infer {\fix f x e			\vsplito \mval {\fix f x {\letin {\tup{x,y}} c {\tup{x,\roll y}}}} 
							{\fix f {\tup{x,\roll l}} r}}											{\splitone e A c {l.r}}								\and
\end{mathpar}
\caption{Value splitting rules.}
\label{fig:valueSplit}
\end{figure}

\begin{figure}
\begin{mathpar}
\infersplitone 					{\spl {\exv v} A {\exv{\tup{i,\tup{}}}} {\_.\exv q}} {v \vsplito \mval i q} \and 
\infersplitone [common2]		{\spl {\scont e}{A}{\letin{\tup{y,z}}{c}
									{\tup{\scont {\exv y},\exv z}}}{l. r}}															{\sub {} {A} 
																																	& \scont \dash \in \{ 
																																	\inl \dash , 
																																	\inr \dash , 
																																	\roll \dash,
																																	\unroll \dash
																																	\}} 		\and
\infersplitone [\fut I]			{\spl {\next e}{\fut A}{\tup{\tup{},p}}{l.r}}														{\splittwosub {} A} 			\and
\infersplitone [common1]		{\spl {\scont e}{A}{\letin{\tup{y,z}}{c}
									{\tup{\scont {\exv y},\exv z}}}{l.\scont r}}													{\sub {} {A} 
																																	& \scont \dash \in \{ 
																																	\pio \dash , 
																																	\pit \dash
																																	\}} 		\and
\infersplitone [\times I] 		{\splitonetall {\tup{e_1,e_2}}{A\times B}
									{\left(
										\talllet{\tup{y_1,z_1}}{c_1}{
										\talllet{\tup{y_2,z_2}}{c_2}{
										\exv{\tup{\tup{y_1, y_2},\tup{z_1, z_2}}}\ttrpar\ttrpar}}
									\right)}
									{\tup{l_1,l_2}.\tup{r_1,r_2}}}																	{\sub 1 A & \sub 2 B} \and
\infersplitone [\to E]			{\splitoneTall {\app {e_1}{e_2}}{B}{\left(
									\talllet{\tup{y_1,z_1}}{c_1}{
									\talllet{\tup{y_2,z_2}}{c_2}{
									\talllet{\tup{y_3,z_3}}{\app{y_1}{y_2}}{\exv{\tup{y_3,\tup{z_1,z_2,z_3}}}\ttrpar\ttrpar\ttrpar}}}
									\right)}
									{\tup{l_1,l_2,l_3}.\app{\exv {r_1}}{\exv{\tup{r_2,l_3}}}}}										{\sub 1 {A \to B} & \sub 2 A}							\and
\infersplitone				{\spl {\pure e} {\curr A} {\tup{e,\tup{}}}{\_.\tup{}}}								{\cdot}							\and
\infersplitone				{\splitoneTall {\letp x{e_1}{e_2}} {?} 
							{\letin {\tup{x,z_1}} {c_1} {
							 \letin {\tup{y_2,z_2}} {c_2} {\tup{y_2,\tup{z_1,z_2}}}}}
							 {\tup{l_1,l_2}.\letin{\_}{r_1}{r_2}}}												{\sub 1 ? & \sub 2 ?}			\and
\infersplitone [+ E]		{\splitoneTall {\caseP{e_1} {x_2.e_2} {x_3.e_3}}{C}
								{\left(
								\talllet{\tup{y_1,z_1}}{c_1}{
									\tallcase{y_1}
									{x_2.\letin{\tup{y_2,z_2}}{c_2}{\exv {\tup{y_2,\tup{z_1,\inl{z_2}}}}}}
									{x_3.\letin{\tup{y_3,z_3}}{c_3}{\exv {\tup{y_3,\tup{z_1,\inr{z_3}}}}}\ttrpar}
								}\right)}
								{\tup{l_1,l_b}.{\ttlpar r_1 \ttsemi \caseof{\exv{l_b}}{l_2.[\tup{}/x_2]{r_2}}{l_3.[\tup{}/x_3]{r_3}\ttrpar}
								}}}																									{\sub 1 {A+B} 
																																	& \sub [\Gamma,\col{x_2} A] 2 C 	
																																	& \sub [\Gamma,\col{x_3} B] 3 C} 		
\end{mathpar}
\hrule
\begin{mathpar}
\infersplittwo 					{\spl {\exv q} A {\exv {\tup{}}} \_ q} 														{\cdot} 						\and 
\infersplittwo [\fut E]			{\spl {\prev e}{A} {\pit c} l r }															{\splitonesub {} {\fut A}} 		\and 
\infersplittwo [\times E_1]		{\spl {\scont e}{A}{p}{l}{\scont r}}														{\sub {} {A\times B} 
																															& \scont \dash \in \{ 
																																\pio \dash , 
																																\pit \dash,  
																																\inl \dash,  
																																\inr \dash,  
																																\roll \dash, 
																																\unroll \dash, 
																																\fix fx \dash 
																															\}} 				\and  
\infersplittwo [\to E]			{\spl {\scont{e_1,e_2}}{B}{\tup{p_1,p_2}}{\tup{l_1,l_2}}{\scont{r_1,r_2}}}				{\sub 1 {A \to B} & \sub 2 A
																															& \scont {\dash, \dash} \in \{ 
																																\tup {\dash,\dash} , 
																																\app \dash \dash,
																																\letin x \dash \dash 
																															\}}									\and			
\infersplittwo [+ E]			{\splittwo {\caseof{e_1}{x_2.e_2}{x_3.e_3}}{C}
								{\tup{p_1,p_2,p_3}}{ \tup{l_1,l_2,l_3}}{\caseof{r_1}{x_2.r_2}{x_3.r_3}}}					{\sub 1 {A+B} & \sub [\Gamma,\col{x_2} A] 2 C 
																															& \sub [\Gamma,\col{x_3} B] 3 C} 								
\end{mathpar} 
\caption{Splitting rules for terms at \bbonem\ and \bbtwo.}
\label{fig:termSplit}
\end{figure}
\end{abstrsyn}



\subsection{Correctness of Splitting}

We can now develop an notion of what it means for splitting to be correct.  
Our general approach is to say that the dynamic semantics from \ref{sec:semantics} and splitting method are equivalent in some way.  
Thus, we start with two mutually dependent definitions of equivalence.  
Both relate residuals on the left with resumers on the right,
but $\equiv$ equates closed expressions, whereas $\sim$ equates values.

\begin{definition}
For residual $q$ and resumer $r$, define $q \equiv r : A$ to mean that 
$q \tworedsym v_q$ iff $\reduce {r} {v_r}$ where $v_q \sim v_r : A$, 
\end{definition}

\begin{definition}
For residual value $v_1$ and resumer value $v_2$, define $v_1 \sim v_2 : A$ by the following cases:
\begin{itemize}
\item $i \sim i : \rmint$
\item $(v_1,u_1) \sim (v_2,u_2) : A \times B$ where $v_1 \sim v_2 : A$ and $u_1 \sim u_2 : B$
\item $\inl~v_1 \sim \inl~v_2 : A + B$ where $v_1 \sim v_2 : A$
\item $\inr~v_1 \sim \inr~v_2 : A + B$ where $v_1 \sim v_2 : B$
\item $\lam {x_1} A {e_1} \sim \lambda x_2.e_2 : A \to B$ where \\ $\forall (v_1 \sim v_2 : A). [v_1/x_1]e_1 \equiv [v_2/x_2]e_2 : B$
\end{itemize}
\end{definition}

Essentially, we can read these as saying that two terms are equivalent if they evaluate to the same value,
where "same" for functions means that those functions always evaluate to the same thing given equivalent inputs.

We give the following end-to-end correctness lemmas for open terms. 
Those readers not interested in the gritty details are advised to skip ahead to the theorems for closed terms, which give the important gist.

In some of the lemmas, we sort the usually heterogeneous context into $\Gamma$ and $\Gamma'$, 
which respectively hold all of the stage \bbone\ and stage \bbtwo\ variable bindings.
Substitution splitting works just like value splitting, which is why they use the same symbol.

\begin{lemma}
If $\typesone e A$ then
$\Gamma\vdash e : A \splitonesym [c,l.r]$.
If $\typestwo e A$ then
$\Gamma\vdash e : A \splittwosym [p,l.r]$.
\end{lemma}

\begin{lemma}
If $\Gamma, \Gamma'\vdash e : A \splittwosym [p,l.r]$ then for all substitutions $\gamma : \Gamma$,
\begin{itemize}
\item $\gamma \vsplito [\gamma_1, \gamma_2]$
\item $\diatwo [\Gamma'] {\gamma(e)} q$ iff $\reduce {\gamma_1(p)} u$ where
\item $\Gamma' \vdash q \equiv (\letin{l}{u}{\gamma_2(r)})$
\end{itemize}
\end{lemma}

\begin{lemma}
If $\Gamma, \Gamma'\vdash e : A \splitonesym [c,l.r]$ then for all $\gamma : \Gamma$,
\begin{itemize}
\item $\gamma \vsplito [\gamma_1, \gamma_2]$
\item $\diaone [\Gamma'] {\gamma(e)} {\xi;v}$ iff $\reduce {\gamma_1(c)} {(v_1,u)}$ where
\item $\Gamma',\dom \xi \vdash v \vsplito [v_1,v_2]$
\item $\reify \xi {v_2} q$
\item $\Gamma'\vdash q \equiv (\letin{l}{u}{\gamma_2(r)})$
\end{itemize}
\end{lemma}

%You should think of these theorems as saying that 
%splitting commutes with evaluation.
These lemmas are rather technical, but they ultimately imply that evaluating a
closed term by splitting or by the dynamic semantics of \ref{ssec:dynamics} are
equivalent. 
We state this result for closed terms at each stage.

\begin{theorem}[Correctness of splitting at $\bbone$]
If $\vdash e:A~@~\bbone$, then (by splitting)
\begin{itemize}
\item $\vdash e : A \splitonesym [c,l.r]$
\item $\reduce {c} {(v_1,u)}$
\item $\reduce {(\letin{l}{u}{r})} v_S$
\end{itemize}
if and only if (by the staged dynamic semantics)
\begin{itemize}
\item $\diaone [] e {\xi;v}$
\item $\dom \xi \vdash v \vsplito [v_1,v_2]$
\item $\reify \xi{v_2}q$
\item $q \mathbin{\tworedsym} v_D$
\end{itemize}
and if so, then $v_D \sim v_S$.
\end{theorem}

\begin{theorem}[Correctness of splitting at $\bbtwo$]
If $\vdash e:A~@~\bbtwo$, then (by splitting)
\begin{itemize}
\item $\vdash e : A \splittwosym [p,l.r]$
\item $\reduce p u$
\item $\reduce{(\letin{l}{u}{r})}{v_S}$
\end{itemize}
if and only if (by the staged dynamic semantics)
\begin{itemize}
\item $\diatwo [] e q$
\item $q \mathbin{\tworedsym} v_D$
\end{itemize}
and if so, then $v_D \sim v_S$.
\end{theorem}

If we apply the former theorem to \verb|next{e}| of type $\fut A$, we
essentially obtain the latter theorem at \verb|e|.

The latter theorem implies that, given a multi-stage function $f:A\to\fut(B\to
C)~@~\bbone$, the two methods of evaluating \verb|prev{f a} b| agree.
However, we also expect that splitting $f$ directly will give us two functions,
one which accepts an $A$ and outputs an intermediate value and boundary data,
and another which takes in that boundary data and a $B$ and outputs a $C$.
Moreover, the composition of these two functions should be extensionally equal
to the staged dynamic semantics.

\TODO write the theorem for $\vdash f:A\to\fut(B\to C)~@~\bbone$.

%
%\subsubsection{Simple Types}
%
%\begin{theorem}
%If $\cdot\vdash e : A \splittwosym [p,l:\tau.r]$ then,
%\begin{itemize}
%\item $\cdot \vdash e : A~@~\bbtwo$ 
%\item $\types [\cdot] p \tau$ and $\types [l:\tau] r A$ 
%\item $\diatwo [\cdot] e q$ iff $\reduce p u$ and if so
%\item $q \equiv (\letin{l}{u}{r})$
%\end{itemize}
%\end{theorem}
%
%\begin{theorem}
%If $\cdot\vdash e : A \splitonesym [c,l:\tau.r]$ then,
%\begin{itemize}
%\item $\typesone [\cdot] e A$ 
%\item $A \tsplito [A_1,A_2]$
%\item $\types [\cdot] c {A_1 \times \tau}$ and $\types [l:\tau] r A_2$ 
%\item $\diaone [\cdot] e {\xi;v}$ iff $\reduce c {(v_1,u)}$ and if so
%\item $\dom \xi \vdash v \vsplito [v_1,v_2]$
%\item $\reify \xi {v_2} q$
%\item $q \equiv (\letin{l}{u}{r}) : A_2$
%\end{itemize}
%\end{theorem}
%
%\subsubsection{\bbone-Dependent Types}
%\begin{theorem}
%If $\cdot\vdash e : A \splitonesym [c,l:\tau.r]$ then,
%\begin{itemize}
%\item $\typesone [\cdot] e A$ 
%\item $A \tsplito [A_1,a.A_2]$
%\item $\types [\cdot] c {A_1 \times \tau}$ and $\types [l:\tau,a:A_1] r A_2$ 
%\item $\diaone [\cdot] e {\xi;v}$ iff $\reduce c {(v_1,u)}$ and if so
%\item $\dom \xi \vdash v \vsplito [v_1,v_2]$
%\item $\reify \xi {v_2} q$
%\item $q \equiv (\letin{l}{u}{r}) : A_2$
%\end{itemize}
%\end{theorem}

\subsection{Cost}

While these splitting rules are all correct in terms of producing the right values, applying them naively can result in terms with needlessly large cost.  Consider the following example involving the factorial function:
\begin{lstlisting}
2`prev{
  1`letfun fact (n : int) : int = 
    if n <= 0 then 1 else fact(n-1)*n
  in next{`2`hold {`1`fact 5`2`}-100`1`}`2`
}`
\end{lstlisting}

The boundary type\footnote{The splitting results are untyped, but we add in type annotations to aid the reader.} of this is give by the following datatype declaration:
\begin{lstlisting}
datatype prec = L | R of prec
\end{lstlisting}
The example then splits into the precomputation,
\begin{lstlisting}
1`letfun fact (n : int) : int * prec = 
  if n <= 0 then (1,L) 
  else let (y,z) = fact(n-1) in (y*n,R z)
in ((),fact 5)`
\end{lstlisting}
and the resumer
\begin{lstlisting}
2`l.
letfun fact (n : unit, l0 : prec) : unit = 
  case l0 of L => () | R l1 => fact ((),l0)
in (fact (#2 l); #1 l)-100`
\end{lstlisting}
The problem here is that the naive system doesn't realize that the \texttt{fact} function has no stage \bbtwo\ content,
and so it pessimistically retraces all of its steps in the resumer.  
Fixing this requires some recursive reasoning.  

We instead tried to solve the issue by adding a single new staging annotation to \lang, called \texttt{mono}.  
A \texttt{mono} block must appear in a stage \bbone\ context, and it indicates that its entire context is monostage.
This can be encoded in our statics and dynamics by adding a new stage \bbmono\ and the judgements:
\begin{mathpar}
\infer{\typesone{\monoSt~e}A}{\typesmono e A} \and
\infer{\diaone{\monoSt~e}{\cdot;v}}{\diaone{e}{\cdot;v}}
\end{mathpar}

The splitting rule for \texttt{mono} is more difficult.
...
transition to dummy.

The evaluation rules for dummy values are given by:
\begin{mathpar}
\infer{e_1~e_2~\redsym~\mathtt{dummy}}{e_1~\redsym~\mathtt{dummy} & e_2~\redsym~v} \and
\infer{\unroll~e~\redsym~\mathtt{dummy}}{e~\redsym~\mathtt{dummy}}\and
\infer{\pio~e~\redsym~\mathtt{dummy}}{e~\redsym~\mathtt{dummy}}\and
\infer{\pit~e~\redsym~\mathtt{dummy}}{e~\redsym~\mathtt{dummy}}
\end{mathpar}

The basic idea here is that dummy will dynamically be whatever data structure you want it to be.

This is safe because [nico is just throwing something out here] 
it can only appear in the resumers of terms that have no circles in their \lang\ type.
Since the $\prev$ rule requires a circle type, dummy values cannot leak from
the resumers of stage \bbone\ terms into the resumers of stage \bbtwo\ terms.

%!TEX root = ../paper.tex

\begin{figure*}
\begin{abstrsyn}
\begin{mathpar}
\infer {\rtab {\hat y \mapsto q,\xi} v	\vsplito \mval i {\letin {\hat y} q r}}				{\rtab \xi v \vsplito \mval i r}					\and
\infer {\rtab \cdot {\tup{}} 			\vsplito \mval {\tup{}} {\tup{}}}					{\cdot}												\and
\infer {\rtab \cdot {\pure m} 			\vsplito \mval m {\tup{}}}							{\cdot}												\and
\infer {\rtab \cdot {\next{\hat y}} 	\vsplito \mval {\tup{}} {\hat y}}					{\cdot}												\and
\infer {\rtab \cdot {\tup{v_1,v_2}} 	\vsplito \mval {\tup{i_1,i_2}} {\tup{r_1,r_2}}}		{\rtab \cdot {v_1} \vsplito \mval {i_1} {r_1} 
																							&\rtab \cdot {v_2} \vsplito \mval {i_2} {r_2}}		\and
\infer {\rtab \cdot {\scriptCapp v} 	\vsplito \mval {\scriptCapp i} r}					{\rtab \cdot v \vsplito \mval i r
																							&\scriptC \in 
																							\{\mathtt{inl},\mathtt{inr},\mathtt{roll}\}}		\and
\infer {\rtab \cdot {\lam x e}			\vsplito \mval {\lam x c} {\lam {\tup{x,l}} r}}		{\splitone e A c {l.r}}								\and
\end{mathpar}
\end{abstrsyn}
\caption{Masking separates a residual table and its associated partial value into its first- and second-stage components.}
\label{fig:valMask}
\end{figure*}

