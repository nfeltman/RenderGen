
\section{Splitting Algorithm}
\label{sec:splitting}

Much like partial evaluation, the goal of stage-splitting is to separate the evaluation of a multi-stage function into distinct phases.
However, where partial evaluation requires the value of some inputs to be known, stage splitting is a static transformation performed
before any arguments are available.
Taking a multivariate function $f$ as input, stage-splitting produces two functions $f_1$ and $f_2$,
where $f_1$ uses the stage-\bbone\ input to produce a data structure, 
and $f_2$ uses that data structure and the stage-\bbtwo\ input to produce the final output.  
More precisely, a stage splitter is any $s$ such that
\[
	\forall f. \exists f_1,f_2. 
	\left[
		\begin{array}{l}
		s(f) = (f_1,f_2) \text{ and } \\
		\forall x,y.\llbracket f \rrbracket(x,y)=\llbracket f_2 \rrbracket(\llbracket f_1 \rrbracket(x),y)
		\end{array}
	\right]
\]
with the same $\llbracket \cdot \rrbracket$ notation as before.
Importantly, the stage splitting operation does not depend on the stage-\bbone\ input, unlike partial evaluation.

As with the dynamics, our stage-splitting algorithm for expressions in \lang\ take 
the form of two mutually-dependent judgments, $\splitonesym$ and $\splittwosym$. 
Before diving into the full generality, we consider splitting for a simple fragment of the full language: partial values.
After that, we define splitting for full expressions, discuss some notable properties of the algorithm, and then prove correctness.

\subsection{Partial Value Splitting}

We now consider the problem of splitting stage-\bbone\ partial values.
This serves as a simple didactic stepping-stone to the full formulation, 
and we will see later that it is a necessary component of the correctness theorem.

The basic goal of partial value splitting is to take a partial value $v$ and create from it two single stage values $v_1$ and $v_2$,
where $v_1$ contains only the information from $v$ that is relevant to the first stage, 
and $v_2$ contains only the information from $v$ that is relevant to the second stage.
There are many reasonable ways to do this, 
and we simply chose one that preserves as much structure as possible from $v$.
Formally, partial value splitting comprises a single judgment, $v \vsplito [v_1,v_2]$, the rules for which are given in \cref{fig:valSplit}.
It essentially acts distributively at product, sum, and recursive types, 
doing little more than recuring into subvalues and duplicating the structure to both results.
At base types, partial value splitting assign all content ({\em i.e.} the integer or boolean) to stage \bbone, and uses a nullary tuple placeholder for stage \bbtwo.
Conversely at $\fut$ types, splitting assigns all content ({\em i.e.} the environment reference) to stage \bbtwo, and uses a nullary tuple for stage \bbone.

Functions are inherently more complicated. 
Since the body of a function is an open term, the partial value splitting rule for functions must rely on general expression splitting, which is covered momentarily.
The most we can say here is a function splits into two other functions, 
whose bodies correspond to the stage \bbone\ and stage \bbtwo\ components of the original function's body.

\subsection{Full Expression Splitting}

We consider splitting of stage \bbtwo\ terms first, as it has a simpler form.  
For example, consider \verb|hold{1+2} < 5|.
We can split this into two separate terms, \verb|1+2| and \verb|fn l => l < 5|, 
which are called the {\em precomputation} and {\em residual}\footnote{We also used the term {\em residual} for the result of partial evaluation.
Where necessary, we will distinguish between these concepts by using the terms {\em splitting-residual}
and {\em evaluation-residual}.}, respectively.  

The form of splitting for stage \bbone\ terms is more complicated.
To see why, consider the term \verb|(next{hold{1+2} < 5},3*4)|.
As before, the comparison operation is part of stage \bbtwo, 
and the addition operation is a stage \bbone\ precomputation, the result of which will eventually become input to the residual.
The multiplication operation, however, is neither part of the residual nor a precomputation to support it.
Instead it is the {\em immediate result} of stage \bbone, 
because it is available to the stage \bbone\ context around our original term.
For example:
\begin{lstlisting}
let x = (next{hold{1+2} < 5},3*4) in
next{ if prev{#1 x} then 7 else hold {#2 x} }
\end{lstlisting}
To support this, stage \bbone\ splitting produces two outputs: the {\em combined result} and the residual,
where the combined result reduces to a tuple containing the immediate result and the precomputation.

We must also develop a notion of what it means for splitting to be correct.  
Our general approach is to say that the dynamic semantics from \ref{sec:semantics} and splitting method are equivalent in some way.  
In general, if a multi-stage program $P$ splits into stage-\bbone\ part $P_1$ and stage-\bbtwo\ part $P_2$, then
\begin{enumerate}
\item Partial evaluation of $P$ will terminate if and only if evaluation of $P_1$ terminates, and if so then
\item the immediate results of each method will be equivalent, and
\item the evaluation-residual and the splitting-residual ($P_2$ bound under the precomputation) are equivalent.
\end{enumerate}

This is straight-forward, given a careful definition of equivalence.

\subsection {Formal Setup}

The splitting algorithm comprises two judgments, $\splitonesym$ and $\splittwosym$.
For both, the input is a term in \lang, and the output is two terms in \langmono, an
unstaged language.  The grammar for \langmono\ is given in \ref{fig:monoGrammar}.  It has no
staging features (we say that implicitly it has only one stage), and it is untyped.  
Modulo these differences, \langmono\ has all of the same features as \lang, 
although this is a matter of taste
\footnote{Since \langmono\ is untyped, we could encode everything with just functions.}.

Specifically, the $\splitonesym$ judgment sends a stage \bbone\ term ($e$) to a combined term ($c$) and residual ($l.r$),
while $\splittwosym$ sends a stage \bbtwo\ term ($e$) to a precomputation ($p$) and a residual ($l.r$).
For concision, we represent residuals as a term open on a single variable, rather than as functions.
For both judgments, we use a context ($\Gamma$) to keep track of the open variables of $e$.

The rules\footnote{The rules are written using patterns, including the open variable of the residual.} 
for splitting $\next$, $\prev$, and $\pause$ are given in \ref{fig:termSplitOne}.
The rule for $\next$ simply tuples up the precomputation of its subexpression with a trivial immediate result,
while the rule for $\prev$ projects the combined result of its subexpression to just the precomputation.
The $\pause$ rule treats the entire combined result of its subexpression as a precomputation, 
and projects out the integer result in the residual\footnote{The residual of an integer expression is usually trivial, 
but we have to include it here for termination purposes.}

The rules of stage \bbtwo\ splitting are given in \ref{fig:termSplitTwo}.  
Every rule works by bundling the precomputations of the constituent parts, and then unbundling them with a pattern.
The rules of stage \bbone\ splitting are given in \ref{fig:termSplitOne}.  


\subsection {Speculation}

Notice in the stage \bbtwo\ rules for {\tt if}s, {\tt case}s, and functions, the precomputation is lifted out from within branches.
This is the manifestation of the speculation behavior from the semantics.

\subsection {Stage \bbone\ Divergence}
Consider the splitting rules for stage-\bbone\ {\tt if} and {\tt case} expressions.
In both cases we have enough information at stage \bbone\ to know what branch to take, so there's no need to speculate.
Instead, we evaluate the stage \bbone\ portion of only the active branch, and then inject precomputation into a sum type.
Then in the residual, we case on that sum and resume the stage \bbtwo\ portion of the correct branch.

\subsection {Stage \bbone\ Functions}

The trick when splitting stage \bbone\ functions is that the contents themselves may be multi-stage.
We handle this by splitting them into two functions:
one in the immediate result which handles all the stage \bbone\ content of the original,
and one in the residual which handles all of the stage \bbtwo\ content.
Note that stage \bbone\ $\lambda$-expressions themselves have only a trivial precomputation.

%\subsection {Boundary Type Worst Case}
%
%\TODO add the example which is the worst case for figuring out boundary types
%

%!TEX root = ../paper.tex

\begin{abstrsyn}
\begin{figure}[t]
\begin{mathpar}
\infer {x 					\vsplito \mval x x}														{\cdot}												\and
\infer {\tup{} 				\vsplito \mval {\tup{}} {\tup{}}}										{\cdot}												\and
\infer {\pure m 			\vsplito \mval m {\tup{}}}												{\cdot}												\and
\infer {\next y 			\vsplito \mval {\tup{}} y}												{\cdot}												\and
\infer {\tup{v_1,v_2} 		\vsplito \mval {\tup{i_1,i_2}} {\tup{q_1,q_2}}}							{v_1 \vsplito \mval {i_1} {q_1} 
																									&v_2 \vsplito \mval {i_2} {q_2}}					\and
\infer {\scont v 			\vsplito \mval {\scont i} q}											{v \vsplito \mval i q
																									& \scont \dash \in \{ 
																										\inl \dash , 
																										\inr \dash,
																										\roll \dash 
																									\}}								\and
\infer {\fix f x e			\vsplito \mval {\fix f x {\letin {\tup{x,y}} c {\tup{x,\roll y}}}} 
							{\fix f {\tup{x,\roll l}} r}}											{\splitone e A c {l.r}}								\and
\end{mathpar}
\caption{Value splitting rules.}
\label{fig:valueSplit}
\end{figure}

\begin{figure}
\begin{mathpar}
\infersplitone 					{\spl {\exv v} A {\exv{\tup{i,\tup{}}}} {\_.\exv q}} {v \vsplito \mval i q} \and 
\infersplitone [common2]		{\spl {\scont e}{A}{\letin{\tup{y,z}}{c}
									{\tup{\scont {\exv y},\exv z}}}{l. r}}															{\sub {} {A} 
																																	& \scont \dash \in \{ 
																																	\inl \dash , 
																																	\inr \dash , 
																																	\roll \dash,
																																	\unroll \dash
																																	\}} 		\and
\infersplitone [\fut I]			{\spl {\next e}{\fut A}{\tup{\tup{},p}}{l.r}}														{\splittwosub {} A} 			\and
\infersplitone [common1]		{\spl {\scont e}{A}{\letin{\tup{y,z}}{c}
									{\tup{\scont {\exv y},\exv z}}}{l.\scont r}}													{\sub {} {A} 
																																	& \scont \dash \in \{ 
																																	\pio \dash , 
																																	\pit \dash
																																	\}} 		\and
\infersplitone [\times I] 		{\splitonetall {\tup{e_1,e_2}}{A\times B}
									{\left(
										\talllet{\tup{y_1,z_1}}{c_1}{
										\talllet{\tup{y_2,z_2}}{c_2}{
										\exv{\tup{\tup{y_1, y_2},\tup{z_1, z_2}}}\ttrpar\ttrpar}}
									\right)}
									{\tup{l_1,l_2}.\tup{r_1,r_2}}}																	{\sub 1 A & \sub 2 B} \and
\infersplitone [\to E]			{\splitoneTall {\app {e_1}{e_2}}{B}{\left(
									\talllet{\tup{y_1,z_1}}{c_1}{
									\talllet{\tup{y_2,z_2}}{c_2}{
									\talllet{\tup{y_3,z_3}}{\app{y_1}{y_2}}{\exv{\tup{y_3,\tup{z_1,z_2,z_3}}}\ttrpar\ttrpar\ttrpar}}}
									\right)}
									{\tup{l_1,l_2,l_3}.\app{\exv {r_1}}{\exv{\tup{r_2,l_3}}}}}										{\sub 1 {A \to B} & \sub 2 A}							\and
\infersplitone				{\spl {\pure e} {\curr A} {\tup{e,\tup{}}}{\_.\tup{}}}								{\cdot}							\and
\infersplitone				{\splitoneTall {\letp x{e_1}{e_2}} {?} 
							{\letin {\tup{x,z_1}} {c_1} {
							 \letin {\tup{y_2,z_2}} {c_2} {\tup{y_2,\tup{z_1,z_2}}}}}
							 {\tup{l_1,l_2}.\letin{\_}{r_1}{r_2}}}												{\sub 1 ? & \sub 2 ?}			\and
\infersplitone [+ E]		{\splitoneTall {\caseP{e_1} {x_2.e_2} {x_3.e_3}}{C}
								{\left(
								\talllet{\tup{y_1,z_1}}{c_1}{
									\tallcase{y_1}
									{x_2.\letin{\tup{y_2,z_2}}{c_2}{\exv {\tup{y_2,\tup{z_1,\inl{z_2}}}}}}
									{x_3.\letin{\tup{y_3,z_3}}{c_3}{\exv {\tup{y_3,\tup{z_1,\inr{z_3}}}}}\ttrpar}
								}\right)}
								{\tup{l_1,l_b}.{\ttlpar r_1 \ttsemi \caseof{\exv{l_b}}{l_2.[\tup{}/x_2]{r_2}}{l_3.[\tup{}/x_3]{r_3}\ttrpar}
								}}}																									{\sub 1 {A+B} 
																																	& \sub [\Gamma,\col{x_2} A] 2 C 	
																																	& \sub [\Gamma,\col{x_3} B] 3 C} 		
\end{mathpar}
\hrule
\begin{mathpar}
\infersplittwo 					{\spl {\exv q} A {\exv {\tup{}}} \_ q} 														{\cdot} 						\and 
\infersplittwo [\fut E]			{\spl {\prev e}{A} {\pit c} l r }															{\splitonesub {} {\fut A}} 		\and 
\infersplittwo [\times E_1]		{\spl {\scont e}{A}{p}{l}{\scont r}}														{\sub {} {A\times B} 
																															& \scont \dash \in \{ 
																																\pio \dash , 
																																\pit \dash,  
																																\inl \dash,  
																																\inr \dash,  
																																\roll \dash, 
																																\unroll \dash, 
																																\fix fx \dash 
																															\}} 				\and  
\infersplittwo [\to E]			{\spl {\scont{e_1,e_2}}{B}{\tup{p_1,p_2}}{\tup{l_1,l_2}}{\scont{r_1,r_2}}}				{\sub 1 {A \to B} & \sub 2 A
																															& \scont {\dash, \dash} \in \{ 
																																\tup {\dash,\dash} , 
																																\app \dash \dash,
																																\letin x \dash \dash 
																															\}}									\and			
\infersplittwo [+ E]			{\splittwo {\caseof{e_1}{x_2.e_2}{x_3.e_3}}{C}
								{\tup{p_1,p_2,p_3}}{ \tup{l_1,l_2,l_3}}{\caseof{r_1}{x_2.r_2}{x_3.r_3}}}					{\sub 1 {A+B} & \sub [\Gamma,\col{x_2} A] 2 C 
																															& \sub [\Gamma,\col{x_3} B] 3 C} 								
\end{mathpar} 
\caption{Splitting rules for terms at \bbonem\ and \bbtwo.}
\label{fig:termSplit}
\end{figure}
\end{abstrsyn}



\subsection{Metatheory}

We start with two mutually dependent definitions of equivalence.  
Both relate evaluation-residuals on the left with splitting residuals on the right,
but $\equiv$ operates on terms, whereas $\sim$ on values.

\begin{definition}
For evaluation-residual $q$ and splitting-residual $r$, define $q \equiv r : A$ to mean that 
$q \tworedsym v_q$ iff $\reduce {r} {v_r}$ where $v_q \sim v_r : A$, 
\end{definition}

\begin{definition}
For evaluation-residual value $v_1$ and splitting-residual value $v_2$, define $v_1 \sim v_2 : A$ by the following cases:
\begin{itemize}
\item $i \sim i : \rmint$
\item $(v_1,u_1) \sim (v_2,u_2) : A \times B$ where $v_1 \sim v_2 : A$ and $u_1 \sim u_2 : B$
\item $\inl~v_1 \sim \inl~v_2 : A + B$ where $v_1 \sim v_2 : A$
\item $\inr~v_1 \sim \inr~v_2 : A + B$ where $v_1 \sim v_2 : B$
\item $\lam {x_1} A {e_1} \sim \lambda x_2.e_2 : A \to B$ where \\ $\forall (v_1 \sim v_2 : A). [v_1/x_1]e_1 \equiv [v_2/x_2]e_2 : B$
\end{itemize}
\end{definition}

Essentially, we can read these as saying that two terms are equivalent if they evaluate to the same value,
where "same" for functions means that those functions always evaluate to the same thing given equivalent inputs.

We give the following end-to-end correctness lemmas for open terms. 
It's a bit of a mess currently, but the $\Gamma$ is supposed to be all of the stage \bbone\ bindings, 
whereas $\Gamma'$ is the stage \bbtwo\ bindings.
Substitution splitting works just like value splitting, which is why they use the same symbol.

The jury is still out on how to make these strong enough to prove that the partitioning between stages is correct.

\begin{lemma}
If $\typesone e A$ then
$\Gamma\vdash e : A \splitonesym [c,l.r]$.
If $\typestwo e A$ then
$\Gamma\vdash e : A \splittwosym [p,l.r]$.
\end{lemma}

\begin{lemma}
If $\Gamma, \Gamma'\vdash e : A \splittwosym [p,l.r]$ then for all substitutions $\gamma : \Gamma$,
\begin{itemize}
\item $\gamma \vsplito [\gamma_1, \gamma_2]$
\item $\diatwo [\Gamma'] {\gamma(e)} q$ iff $\reduce {\gamma_1(p)} u$ where
\item $\Gamma' \vdash q \equiv (\letin{l}{u}{\gamma_2(r)})$
\end{itemize}
\end{lemma}

\begin{lemma}
If $\Gamma, \Gamma'\vdash e : A \splitonesym [c,l.r]$ then for all $\gamma : \Gamma$,
\begin{itemize}
\item $\gamma \vsplito [\gamma_1, \gamma_2]$
\item $\diaone [\Gamma'] {\gamma(e)} {\xi;v}$ iff $\reduce {\gamma_1(c)} {(v_1,u)}$ where
\item $\Gamma',\dom \xi \vdash v \vsplito [v_1,v_2]$
\item $\reify \xi {v_2} q$
\item $\Gamma'\vdash q \equiv (\letin{l}{u}{\gamma_2(r)})$
\end{itemize}
\end{lemma}

%You should think of these theorems as saying that 
%splitting commutes with evaluation.
These lemmas are rather technical, but they ultimately imply that evaluating a
closed term by splitting or by the dynamic semantics of \ref{ssec:dynamics} are
equivalent. 
We state this result for closed terms at each stage.

\begin{theorem}[Correctness of splitting at $\bbone$]
If $\vdash e:A~@~\bbone$, then (by splitting)
\begin{itemize}
\item $\vdash e : A \splitonesym [c,l.r]$
\item $\reduce {c} {(v_1,u)}$
\item $\reduce {(\letin{l}{u}{r})} v_S$
\end{itemize}
if and only if (by the staged dynamic semantics)
\begin{itemize}
\item $\diaone [] e {\xi;v}$
\item $\dom \xi \vdash v \vsplito [v_1,v_2]$
\item $\reify \xi{v_2}q$
\item $q \mathbin{\tworedsym} v_D$
\end{itemize}
and if so, then $v_D \sim v_S$.
\end{theorem}

\begin{theorem}[Correctness of splitting at $\bbtwo$]
If $\vdash e:A~@~\bbtwo$, then (by splitting)
\begin{itemize}
\item $\vdash e : A \splittwosym [p,l.r]$
\item $\reduce p u$
\item $\reduce{(\letin{l}{u}{r})}{v_S}$
\end{itemize}
if and only if (by the staged dynamic semantics)
\begin{itemize}
\item $\diatwo [] e q$
\item $q \mathbin{\tworedsym} v_D$
\end{itemize}
and if so, then $v_D \sim v_S$.
\end{theorem}

If we apply the former theorem to \verb|next{e}| of type $\fut A$, we
essentially obtain the latter theorem at \verb|e|.

The latter theorem implies that, given a multi-stage function $f:A\to\fut(B\to
C)~@~\bbone$, the two methods of evaluating \verb|prev{f a} b| agree.
However, we also expect that splitting $f$ directly will give us two functions,
one which accepts an $A$ and outputs an intermediate value and boundary data,
and another which takes in that boundary data and a $B$ and outputs a $C$.
Moreover, the composition of these two functions should be extensionally equal
to the staged dynamic semantics.

\TODO write the theorem for $\vdash f:A\to\fut(B\to C)~@~\bbone$.

%
%\subsubsection{Simple Types}
%
%\begin{theorem}
%If $\cdot\vdash e : A \splittwosym [p,l:\tau.r]$ then,
%\begin{itemize}
%\item $\cdot \vdash e : A~@~\bbtwo$ 
%\item $\types [\cdot] p \tau$ and $\types [l:\tau] r A$ 
%\item $\diatwo [\cdot] e q$ iff $\reduce p u$ and if so
%\item $q \equiv (\letin{l}{u}{r})$
%\end{itemize}
%\end{theorem}
%
%\begin{theorem}
%If $\cdot\vdash e : A \splitonesym [c,l:\tau.r]$ then,
%\begin{itemize}
%\item $\typesone [\cdot] e A$ 
%\item $A \tsplito [A_1,A_2]$
%\item $\types [\cdot] c {A_1 \times \tau}$ and $\types [l:\tau] r A_2$ 
%\item $\diaone [\cdot] e {\xi;v}$ iff $\reduce c {(v_1,u)}$ and if so
%\item $\dom \xi \vdash v \vsplito [v_1,v_2]$
%\item $\reify \xi {v_2} q$
%\item $q \equiv (\letin{l}{u}{r}) : A_2$
%\end{itemize}
%\end{theorem}
%
%\subsubsection{\bbone-Dependent Types}
%\begin{theorem}
%If $\cdot\vdash e : A \splitonesym [c,l:\tau.r]$ then,
%\begin{itemize}
%\item $\typesone [\cdot] e A$ 
%\item $A \tsplito [A_1,a.A_2]$
%\item $\types [\cdot] c {A_1 \times \tau}$ and $\types [l:\tau,a:A_1] r A_2$ 
%\item $\diaone [\cdot] e {\xi;v}$ iff $\reduce c {(v_1,u)}$ and if so
%\item $\dom \xi \vdash v \vsplito [v_1,v_2]$
%\item $\reify \xi {v_2} q$
%\item $q \equiv (\letin{l}{u}{r}) : A_2$
%\end{itemize}
%\end{theorem}

%!TEX root = ../paper.tex

\begin{figure*}
\begin{abstrsyn}
\begin{mathpar}
\infer {\rtab {\hat y \mapsto q,\xi} v	\vsplito \mval i {\letin {\hat y} q r}}				{\rtab \xi v \vsplito \mval i r}					\and
\infer {\rtab \cdot {\tup{}} 			\vsplito \mval {\tup{}} {\tup{}}}					{\cdot}												\and
\infer {\rtab \cdot {\pure m} 			\vsplito \mval m {\tup{}}}							{\cdot}												\and
\infer {\rtab \cdot {\next{\hat y}} 	\vsplito \mval {\tup{}} {\hat y}}					{\cdot}												\and
\infer {\rtab \cdot {\tup{v_1,v_2}} 	\vsplito \mval {\tup{i_1,i_2}} {\tup{r_1,r_2}}}		{\rtab \cdot {v_1} \vsplito \mval {i_1} {r_1} 
																							&\rtab \cdot {v_2} \vsplito \mval {i_2} {r_2}}		\and
\infer {\rtab \cdot {\scriptCapp v} 	\vsplito \mval {\scriptCapp i} r}					{\rtab \cdot v \vsplito \mval i r
																							&\scriptC \in 
																							\{\mathtt{inl},\mathtt{inr},\mathtt{roll}\}}		\and
\infer {\rtab \cdot {\lam x e}			\vsplito \mval {\lam x c} {\lam {\tup{x,l}} r}}		{\splitone e A c {l.r}}								\and
\end{mathpar}
\end{abstrsyn}
\caption{Masking separates a residual table and its associated partial value into its first- and second-stage components.}
\label{fig:valMask}
\end{figure*}

