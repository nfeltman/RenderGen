
\section{Related Work}
There are several distinctive aspect of our work which we wish to compare and contrast.
Firstly, \lang\ is a simply typed lambda calculus 
with explicit stage annotations and a non-duplicating semantics.
Its statics and dynamics enforce no backward information flow,
which a is necessary property to define a splitting transformation
that is equivalent to the dynamics.
Finally, the splitting transformation itself operates on any well-typed open term in \lang.

Partial evaluation is a well-studied technique for specializing programs to some of their input.  
Often, the goal of partial evaluation systems is to go all the way from unstaged code to a specialized version.
Although this problem definition implicitly includes binding time analysis,
many presentations factor out that step and start with a stage-annotated representation, as we have done.
We've shown that our dynamic semantics can be used to perform partial evaluation 
[need to do more reading to know exactly how it compares].

One unifying feature of partial evaluation systems is that they operate on code which is closed at the first stage,
and so they must necessarily unfold recursive calls and wait for all stage \bbone\ input to operate.
There has been some work to avoid these issues by templating [need citations].
Stage-splitting, in contrast, goes all the way to operating on terms which are open on first stage variables,
and so these problems are avoided.


Stage-splitting itself has appeared in the literature---under 
the names {\em pass separation} and {\em data specialization}--- for various simpler input languages.

Most often, splitting is considered a compiler optimization (\cite{jorring86},\cite{knoblock96})
to minimize recomputation.
In this use case, only the arguments to the program are annotated with stages, 
and the splitter is left to perform an {\em ad hoc} binding time analysis to determine the 
internal staging of the program.
Because this occurs without programmer supervision, the binding time analyzers for these uses are 
reluctant to place stage \bbone\ code within stage \bbtwo\ conditionals, 
since that would result in speculative behavior that could increase the overall runtime.
We don't worry about this since the annotations of \lang\ are assumed to represent the will of a programmer.

Like the previous example, the Spark language (\cite{sparkThesis}) uses staging to minimize recomputation in real-time rendering applications.  
But instead of using a binding-time analysis to determine what parts of the program to put in what shader, 
Spark gives the programmer manual control of term staging using a language that is in spirit similar to \lang. 

We are also not the first to use stage splitting to derive algorithms.
\cite{malmkjaer89}, expanding on the work of \cite{barzdins88}, defined a stage-splitter
for a language of tail-recursive first-order equations with explicit staging annotations [double check that],
and considered applications to string matching.

Our work is distinct from all previous stage splitters in the following ways:
\begin{itemize}
\item Our input language, \lang, has a disciplined type theory.
\item \lang\ has full first-class functions (and by extension, recursion), as well as sum types.
\item Our stage splitting is proven correct.
\end{itemize}

The goal of {\em metaprogramming} (\cite{taha-thesis-99}, \cite{devito13}, \cite{davies01}) is 
to use code in a host language to write code in an object language.
Like partial evaluation and stage-splitting, this can be modeled by a staged language with explicit annotations.
Instead of identifying stages with points in time, stages are identified with host/object levels of the language,
and so type safety ensures the proper separation of levels of the language and safety of the constructed object code.

But the differences in goals of metaprogramming do cause a few differences in form.
For example, it's generally preferred in the metaprogramming context to use 
a dynamics that duplicates the contents of $\next$ blocks.
Moreover, in metaprogramming systems it makes sense to define an operator, \texttt{run}, 
which evaluates stage \bbtwo\ code to get an answer back at stage \bbone.
This is essentially a form a backwards information flow.
Thus, the \texttt{run} feature is directly incompatible with a temporal interpretation of stages,
which stage-splitting requires.

\cite{davies96} explored the connection between linear temporal logic and its corresponding type system (which is nearly identical to that for \lang), 
and showed the equivalence between that type system and existing one for binding time analysis. 
That work also provided a metaprogramming-style duplicating dynamic semantics, which we purposely avoided.
That said, the dynamics of \cite{davies96} is otherwise very similar to ours.
In fact, if we change the $\next$ and $\prev$ rules of ours to 
\begin{mathpar}
\infer {\diaone {\next~e}{\cdot,\next~q}} {\diatwo e q} \and
\infer {\diatwo{\prev~e}{v}} {\diaone e {\cdot,\next~v}} 
\end{mathpar}
and treat $\next~q$ as a partial value for any residual $q$,
then our semantics becomes rule-for-rule isomorphic to that from \cite{davies96}. This essentially bypasses the
environment bookkeeping in $\redonesym$, by inlining residuals instead of
hoisting them in \verb|let|-bindings.
From this it's clear that the semantics of \cite{davies96} and ours always produce the same value when they both terminate.
{\em However because \lang's semantics dictate that a reified residual will always be evaluated, regardless of whether its result is consumed, \lang\ programs terminate strictly less often than that of \cite{davies96}}.












Online and offline.  Both approaches have their merits but online
partial evaluation can be too slow and if stopped lead to an
inefficient residual.  Offline partial evaluation is faster.  Also,
full self-application has only been achieved by offline partial
evaluation.


Weaknesses of partial evaluation include~\cite{GJ05}:
1) speedups linear in the subject programs run time (excuse moi?)

2) good speedups require close knowledge of the program and some
binding-time improvements.

3) result of specialization can be hard to predict as it can lead to
slowdown, code explosion, and non-termination at specialization time.


\paragraph{Meta programming}

What happens when we try to write quickselect as a meta program? 


\ur{Umut: I am starting a paragraph on termination.  we might need to expand
  on this.}
\paragraph{Termination.}

There are two reasons for non-termination: infinite loops and
execution of errorneous code.  The splitting algorithm can cause the
first.  But can it also cause the second? (If all computations really
depend on stage 1 code, then would it be possible to for example see
something like 1/0?  i think so but only if there is a meta dependency
between the data in two stages?)


Is in possible for the residual program to not terminate (unexpectedly)? 

Termination analysis in partial evaluation: see paper~\cite{AH96,GJ05}.

Based on Glenstrup and Jones (pp. 1154, 1155), a partial evaluator
should be complete. That is every computation depending on the static
input should be reduced to a value (during specialization).
Unfortunately, completeness is difficult to achieve if we also require
that the specialization is always terminating.

One problem with their description is that it is not clear which
termination problem they are talking about termination of the residual
or the specializer. 

Nontermination during specialization: 
consider code

double x = doubleplus (x,0) 

doubleplus (u,v) = if u <= 0 then v else doubleplus (u-1,v+2)

this can lead to non-termination if v is considered static.  

Is this a problem for us? 



\paragraph{Supercompilation.}
~\cite{Turchin86}


\paragraph{Intro.}

A predominating theme in the design, implementatation, and
optimization of algorithm and software is frequency reduction and
precomputation.  The idea is to reduce the frequency of computations
by reducing their frequency and ideally by precomputing when possible.


The reason for why this theme is as predominant as it is that it is
naturally common in many actual applications.  For example, we may
have a collection of names or numbers that we frequently lookup, as
for example part of a membership test as part of a larger system.  In
some applications areas such as in graphics (the motivation for our
work), this is so common that there is a relatively large literature
on speeding up graphics computations by taking advantage of frequency
reduction and precomputation.

One way to express frequency of computations is to allow the
programmer to stage computation explicitly by making explicit the
stage of each computation and then by moving computations between
stages to improve efficiency by performing work as early (to save time
later when performance might be more critical) and  as infrequently
as possible (to reduce overall time).






For example, in meta-programming~\cite{}, the programmer can assign a
program expression a stage (stage 1, 2, 3, ...), which indicates the
stage at which that expression evaluates.  Combined with the ability
for earlier stages to construct expressions of later stages,
meta-programming techniques can be used to construct efficient
programs that improve efficiency over their non-staged counterparts by
reducing frequency reduction and precomputation.  For example, the
programmer can write a program that takes a list of numbers $\ell$ as
a first stage value and returns a second-stage function that performs
a fast membership test on that list $\ell$ by inlining all the
comparisons.  When given a key to lookup in the second stage, the
membership test function would quickly return the result.




