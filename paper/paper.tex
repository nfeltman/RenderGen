\documentclass{sigplanconf}

% The following \documentclass options may be useful:

% preprint      Remove this option only once the paper is in final form.
% 10pt          To set in 10-point type instead of 9-point.
% 11pt          To set in 11-point type instead of 9-point.
% authoryear    To obtain author/year citation style instead of numeric.

\usepackage{amsmath}
%\usepackage{amssymb}
\usepackage{bbm}
\usepackage{stmaryrd}
\usepackage{proof}
\usepackage{mathpartir}
\usepackage{mathabx}
\usepackage{listings}

\begin{document}

%!TEX root = paper.tex

\newcommand {\TODO} {\textbf{TODO:} }

\definecolor{needsfixcolor}{rgb}{0.8,0.0,0.0}
\newcommand{\needsfix}[1]{{\color{needsfixcolor}\textbf{#1}}}

% core conventions and language names
\newcommand {\bbone} {\ensuremath{\mathbbmss{1}}}
\newcommand {\bbonep} {\ensuremath{\mathbbmss{1G}}}
\newcommand {\bbonem} {\ensuremath{\mathbbmss{1M}}}
\newcommand {\bbtwo} {\ensuremath{\mathbbmss{2}}}
\newcommand {\ellStaged} {$\mathrm{L}^{\bbone\bbtwo}$}
\newcommand {\ellResid} {$\mathrm{L^R}$}
\newcommand {\ellTarget} {$\mathrm{L^T}$}
\newcommand {\lamStaged} {$\lambda^{\bbone\bbtwo}$}
\newcommand {\lamCircle} {$\lambda^{\fut}$}
\newcommand {\lamResid} {$\mathrm{\lambda^R}$}
\newcommand {\lamTarget} {$\mathrm{\lambda^T}$}
\newcommand {\lang} {\lamStaged}
\newcommand {\langTwo} {$\lambda^\bbtwo$}
\newcommand {\langS} {$\lambda^\&$}
\newcommand {\langmono} {$\lambda^{\textrm{U}}$}
\newcommand {\rmint} {{\rm int}}
\newcommand {\rmunit} {{\rm unit}}
\newcommand {\rmbool} {{\rm bool}}
\newcommand {\exv}[1]{\underline{#1}}

% metatheory
\newcommand{\wfsym}{\ensuremath{\mathsf{wf}}}
\newcommand{\pvalsym}{\ensuremath{\mathsf{pval}}}
\newcommand{\ressym}{\ensuremath{\mathsf{res}}}
\newcommand{\valsym}{\ensuremath{\mathsf{val}}}
\newcommand{\donesym}{\ensuremath{\mathsf{done}}}
\newcommand{\fvalsym}{\ensuremath{\mathsf{fval}}}
\newcommand{\cpvalsym}{\ensuremath{\mathsf{cpval}}}
\newcommand{\wf}{\ \wfsym}
\newcommand{\pval}{\ \pvalsym}
\newcommand{\res}{\ \ressym}
\newcommand{\val}{\ \valsym}
\newcommand{\fval}{\ \fvalsym}
\newcommand{\cpval}{\ \cpvalsym}

% bnf stuff
\newcommand {\myit} [1]{\operatorname{\it{#1}}}
\newcommand {\stage} {\langle\mathit{stage}\rangle}
\newcommand {\type} {\tau}
\newcommand {\typeo} {\langle\bbone\text{-}\mathit{type}\rangle}
\newcommand {\typet} {\langle\bbtwo\text{-}\mathit{type}\rangle}
\newcommand {\expr} {e}
\newcommand {\brancho} {\langle\bbone\text{-}\mathit{brn}\rangle}
\newcommand {\brancht} {\langle\bbtwo\text{-}\mathit{brn}\rangle}
\newcommand {\expro} {\langle\bbone\text{-}\mathit{exp}\rangle}
\newcommand {\exprt} {\langle\bbtwo\text{-}\mathit{exp}\rangle}
%\newcommand {\val} {\langle\mathit{val}\rangle}
\newcommand {\valo} {\langle\bbone\text{-}\mathit{val}\rangle}
\newcommand {\valt} {\langle\bbtwo\text{-}\mathit{val}\rangle}
\newcommand {\world} {w}
\newcommand {\var} {x}
\newcommand {\emptyC} {\bullet}
\newcommand {\context} {\Gamma}
\newcommand {\inte} {i}
\newcommand {\bool} {b}
\newcommand {\resi} {\langle\mathit{res}\rangle}
\newcommand {\gbar} {~~|~~}

% nodes
\newcommand {\yhat} {\ensuremath{\mathtt{\hat y}}}
\newcommand {\curr} {\nabla}
\newcommand {\fut} {\bigcirc}


\newcommand {\pause} {{\tt hold}}
\renewcommand {\next} {{\tt next}}
\newcommand {\prev} {{\tt prev}}
\newcommand {\pure} {{\tt gr}}

\newenvironment{abstrsyn}
{ 
	\newcommand {\mytt} [1] {\texttt{##1}}
	\renewcommand {\pause} [1] {\mytt{hold(}##1\mytt{)}}
	\renewcommand {\next} [1] {\mytt{next(}##1\mytt{)}}
	\renewcommand {\prev} [1] {\mytt{prev(}##1\mytt{)}}
	\renewcommand {\pure} [1] {\mytt{gr(}##1\mytt{)}}
	\newcommand {\inl} [1] {\mytt{inl}\mytt{(}{##1}\mytt{)}}
	\newcommand {\inr} [1] {\mytt{inr}\mytt{(}{##1}\mytt{)}}
	\newcommand {\tup} [1] {\mytt{<} ##1 \mytt{>}}
	\newcommand {\pio} [1] {\mytt{pi1(}##1\mytt{)}}
	\newcommand {\pit} [1] {\mytt{pi2(}##1\mytt{)}}
	\newcommand {\roll} [1] {\mytt{roll}\mytt{(}##1\mytt{)}}
	\newcommand {\unroll} [1] {\mytt{unroll(}##1\mytt{)}}
	\newcommand {\lifttag} [1] {\mytt{lifttag(}##1\mytt{)}}
	\newcommand {\push} {\mytt{unroll}}
	\newcommand {\letp} [3] {\mytt{letg(}{##2}\mytt{;}{##1}.{##3}\mytt{)}}

	\newcommand {\letin} [3] {\mytt{let(}{##2}\mytt{;}{##1}.{##3}\mytt{)}}
	\newcommand {\talllet} [3] {\begin{array}{@{}l@{}} \mytt{let(}{##2}\mytt{;}{##1}\mytt{.}\\{##3}\end{array}}

	\newcommand {\caseof} [3] {\mytt{case(}{##1}\mytt{;}{##2}\mytt{;}{##3}\mytt{)}}
	\newcommand {\tallcase} [3] {\begin{array}{@{}l@{}} \mytt{case(}{##1}\mytt{;}\\~{##2}\mytt{;}\\~{##3}\mytt{)} \end{array}}
	\newcommand {\mediumcase} [3] {\begin{array}{@{}l@{}} \mytt{case(}{##1}\mytt{;}\\~{##2}\mytt{;}{##3}\mytt{)} \end{array}}

	\newcommand {\rtab}  [2] {\llbracket{##1}\rrbracket{##2}}
	\newcommand {\caseP} [3] {\mytt{caseg(}{##1}\mytt{;}{##2}\mytt{;}{##3}\mytt{)}}
	\newcommand {\pipeM} [3] {\mytt{\{}{##1}\mytt{|}{##2}\mytt{.}{##3}\mytt{\}}}
	\newcommand {\pipeS} [3] {\mytt{\{}{##1}\mytt{|}{##2}\mytt{.}{##3}\mytt{\}}}
	\newcommand {\mval}  [2] {\mytt{\{}{##1}\mytt{;}{##2}\mytt{\}}}

	\newcommand {\app} [2] {\mytt{app}\mytt{(}{##1}\mytt{;}{##2}\mytt{)}}
	\newcommand {\lam} [2] {\mytt{fn}\mytt{(}{##1}\mytt{.}{##2}\mytt{)}}
	\newcommand {\fix} [3] {\mytt{fix}\mytt{(}{##1}\mytt{.}{##2}\mytt{.}{##3}\mytt{)}}
	\newcommand {\valprod} [2] {({##1},{##2})}
	\newcommand {\projbind} [1] {\letin{l_{##1}}{\pi_{##1} l}{r_{##1}}}

	\newcommand {\mualphatau} {\mu \alpha.\tau}

	\newcommand {\scriptC} {\mathcal C}
	\newcommand {\scriptCapp} [1] {\scriptC\mytt{(}##1\mytt{)}}
}
{ }


\newcommand {\myatsign} {\hspace{-0.15em}\text{@}\hspace{-0.2em}}

%judgments
\newcommand {\coltwo} [2] {{#1}:{#2}~\myatsign~\bbtwo}
\newcommand {\colmix} [2] {{#1}:{#2}~\myatsign~\bbonem}
\newcommand {\colpure} [2] {{#1}:{#2}~\myatsign~\bbonep}
\newcommand {\colwor} [2] {{#1}:{#2}~\myatsign~w}
\newcommand {\typesone} [3] [\Gamma] { {#1} \vdash \colmix{#2}{#3}}
\newcommand {\typespure} [3] [\Gamma] { {#1} \vdash \colpure{#2}{#3}}
\newcommand {\typestwo} [3] [\Gamma] { {#1} \vdash \coltwo{#2}{#3}}
\newcommand {\typeswor} [3] [] { \Gamma{#1} \vdash \colwor{#2}{#3}}
\newcommand {\types} [3] [\Gamma] {{#1} \vdash {#2}:{#3}}
\newcommand {\istypemix} {\ \mathsf{type}~\myatsign~\bbonem}
\newcommand {\istypetwo} {\ \mathsf{type}~\myatsign~\bbtwo}
\newcommand {\istypepure} {\ \mathsf{type}~\myatsign~\bbonep}
\newcommand {\istypewor} {\ \mathsf{type}~\myatsign~w}

\newcommand {\typeslangTwo} [3] [\Gamma] {{#1} \vdash_\bbtwo {#2}:{#3}}

\newcommand {\reify} [3] {[{#1};{#2}]\searrow{#3}}
\newcommand {\erasone} {\Downarrow^e_\bbone}
\newcommand {\erastwo} {\Downarrow^e_\bbtwo}
\newcommand {\isvalone} [1] {\Gamma \vdash {#1}~{\rm val}~\myatsign~\bbone}
\newcommand {\isvaltwo} {~{\rm val}~\myatsign~\bbtwo}
\newcommand {\isvalwor} {~{\rm val}~\myatsign~w}
\newcommand {\isseppval} {~{\rm pval^S}}
%\newcommand {\reducexpl} [4] {{#3} \Downarrow_{#1}^{#2} {#4}}
%\newcommand {\redone} [2] {{#1} \Downarrow_\bbone^L {#2}}
%\newcommand {\redtwo} [2] {{#1} \Downarrow_\bbtwo^\bbtwo {#2}}
%\newcommand {\reducewor} [2] {{#1} \Downarrow_w^L {#2}}
\newcommand {\specwor} [2] [] {{#2} \downarrow_{#1}}
%\newcommand {\diaone} [3] [\Gamma] {{#1} \vdash {#2} \Downarrow_\bbone [{#3}]}
%\newcommand {\diatwo} [3] [\Gamma] {{#1} \vdash {#2} \Downarrow_\bbtwo {#3}}
\newcommand {\diaone} [4] [\Gamma] {{#2} \Downarrow_{\bbonem} \rtab{#3}{#4}}
\newcommand {\diatwo} [3] [\Gamma] {{#2} \Downarrow_\bbtwo {#3}}
\newcommand {\sepredonesym} {{\Downarrow_\bbone^S}}
\newcommand {\sepredtwosym} {{\Downarrow_\bbtwo^S}}
\newcommand {\sepredone} [2] {{#1} \sepredonesym {#2}}
\newcommand {\sepredtwo} [2] {{#1} \sepredtwosym {#2}}
\newcommand {\reduce} [2] {{#1} \Downarrow {#2}}
\newcommand {\reifysym} {\overset{R}\Rightarrow}
\newcommand {\redsym} {{\Downarrow}}
\newcommand {\redonesym} {{\Downarrow_{\bbonem}}}
\newcommand {\redtwosym} {{\Downarrow_\bbtwo}}
\newcommand {\tworedsym} {{\downarrow_\bbtwo}}

%\newcommand {\reduceonesub} [1] [] {\reduceone{e_{#1}}{v_{#1}}}
%\newcommand {\reducetwosub} [1] [] {\reducetwo{e_{#1}}{v_{#1}}}
\newcommand {\reduceworsub} [1] [] {\reducewor{e_{#1}}{v_{#1}}}
\newcommand {\specworsub} [1] [] {\specwor{e_{#1}}}
\newcommand {\diaonesub} [1] [] {\diaone{e_{#1}}{\xi_{#1}}{v_{#1}}}
\newcommand {\diatwosub} [1] [] {\diatwo{e_{#1}}{q_{#1}}}

\newcommand{\stepsym}[1]{\overset{#1}\hookrightarrow}
\newcommand{\stepmix}[2]{{#1}\stepsym \bbonem{#2}}
\newcommand{\steppure}[2]{{#1}\stepsym \bbonep{#2}}
\newcommand{\steptwo}[2]{{#1}\stepsym \bbtwo {#2}}
\newcommand{\steptwosub}[1]{e_{#1}\stepsym \bbtwo e'_{#1}}
\newcommand{\stepwor}[2]{{#1}\stepsym w {#2}}
\newcommand{\stepworsub}[1]{e_{#1}\stepsym w e'_{#1}}
\newcommand{\done}[1][w]{\ \donesym^{#1}}
\newcommand{\lift}[2]{{#1}\nearrow \llbracket q/y \rrbracket{#2}}
\newcommand{\liftsub}[1]{e_{#1}\nearrow \llbracket q/y \rrbracket e'_{#1}}

\newcommand {\masko} [1] {|#1|_{\bbone}}
\newcommand {\maskt} [1] {|#1|_{\bbtwo}}

\newcommand {\vsplito} {\curvearrowbotright}
\newcommand {\vsplitt} {\overset{\bbtwo}\curvearrowbotright}
\newcommand {\tsplito} {\overset{\bbone}\curvearrowright}
\newcommand {\tsplits} {\overset{\bbtwo}\curvearrowright}
\newcommand {\csplit} {\curvearrowright}
\newcommand {\ssplit} {\curvearrowbotright}

\newcommand {\splitonesym} {\overset{\bbonem}\rightsquigarrow}
\newcommand {\splittwosym} {\overset{\bbtwo}\rightsquigarrow}
\newcommand {\splitone} 	[5] [\Gamma] {{#2} \splitonesym \mytt{\{}{#4}\mytt{|}{#5}\mytt{\}}}
\newcommand {\splitonetall} [5] [\Gamma] {{#2} \splitonesym \left\{\begin{array}{@{}l@{}}{#4}\\\mytt{|}{#5}\end{array}\right\}}
\newcommand {\splittwo} 	[6] [\Gamma] {{#2} \splittwosym \pipeS{#4}{#5}{#6} }
\newcommand {\splittwotall} [6] [\Gamma] {{#2} \splittwosym \left\{\begin{array}{@{}l@{}}{#4}\\\mytt{|}{#5}.{#6}\end{array}\right\}}
\newcommand {\splittwotaller}[6][\Gamma] {{#2} \splittwosym \left\{\begin{array}{@{}l@{}}{#4}\\\mytt{|}{#5}.\\\hspace{1em}{#6}\end{array}\right\}}
\newcommand {\splitonesub}	[3] [\Gamma] {\splitone [{#1}] {e_{#2}}{#3}{c_{#2}}{l_{#2}.r_{#2}}}
\newcommand {\splittwosub}	[3] [\Gamma] {\splittwo [{#1}] {e_{#2}}{#3}{p_{#2}}{l_{#2}}{r_{#2}}}

%misc
\newcommand{\dom}[1]{\mathsf{dom}(#1)}
\newcommand {\gcomp} [2] {\xi_{#1}, \xi_{#2}}
\newcommand {\daviesz} {\overset 0 \hookrightarrow}
\newcommand {\davieso} {\overset 1 \hookrightarrow}
\newcommand {\ttrpar} {\texttt{)}}
\newcommand {\ttlpar} {\texttt{(}}
\newcommand {\ttsemi} {\texttt{;}}

%default stuff
\newcommand \ty \typesone
\newcommand \red \reduce
\newcommand \sub \reduceonesub
\newcommand \spl \splitone
\newcommand \col \colone

%inference extension
\newcommand {\infertypeswor}  [3] 
		[NONAME]{
			\renewcommand \ty \typeswor
			\renewcommand \col \colwor
			\infer %[\mathrm{#1}] 
      {#2}{#3}}
\newcommand {\inferreducewor} [3] 
		[NONAME]{
			\renewcommand \red \reducewor
			\renewcommand \sub \reduceworsub
			\infer [\mathrm{#1\Downarrow}] 
      {#2}{#3}}
\newcommand {\inferreducespc} [3] 
		[NONAME]{
			\renewcommand \red \specwor
			\renewcommand \sub \specworsub
			\infer [\mathrm{#1\downarrow}] {#2}{#3}}
\newcommand {\inferdiaone} [3] 
		[NONAME]{
			\renewcommand \red \diaone
			\renewcommand \sub \diaonesub
			\infer %[\mathrm{#1\Downarrow_\bbone}] 
      {#2}{#3}}
\newcommand {\inferdiaspc} [3] 
		[NONAME]{
			\renewcommand \red \diatwo
			\renewcommand \sub \diatwosub
			\infer %[\mathrm{#1\Downarrow_\bbtwo}] 
      {#2}{#3}}
\newcommand {\infersplitone}  [3] 
		[NONAME]{
			\renewcommand \spl \splitone
			\renewcommand \sub \splitonesub
			\renewcommand \col \colone
			\infer %[\mathrm{#1 \overset{\bbone}\rightsquigarrow}]
      {#2}{#3}}
\newcommand {\infersplittwo}  [3] 
		[NONAME]{
			\renewcommand \spl \splittwo
			\renewcommand \sub \splittwosub
			\renewcommand \col \coltwo
			\infer %[\mathrm{#1 \overset{\bbtwo}\rightsquigarrow}]
      {#2}{#3}}
\newcommand {\inferstepw} [2]
		{
			\renewcommand \red {\stepwor}
			\renewcommand \sub {\stepworsub}
			\infer {#1}{#2}
		}
\newcommand {\infersteptwo} [2]
		{
			\renewcommand \red {\steptwo}
			\renewcommand \sub {\steptwosub}
			\infer {#1}{#2}
		}
\newcommand {\infersteppure} [2]
		{
			\renewcommand \red {\steppure}
			\renewcommand \sub {\steppuresub}
			\infer {#1}{#2}
		}
\newcommand {\inferstepone} [2]
		{
			\renewcommand \red {\stepwor}
			\renewcommand \sub {\stepworsub}
			\infer {#1}{#2 & w \in \{\bbonem,\bbonep\}}
		}
\newcommand {\inferlift} [2]
		{
			\renewcommand \red {\lift}
			\renewcommand \sub {\liftsub}
			\infer {#1}{#2}
		}

%
%
%\begin{figure}
%\caption{\ellStaged~Syntax}
%\label{fig:ellStagedSyntax}
%\centering
%\begin{tabular}{ll} 
%$\begin{aligned}
%\typeo &::= \text{unit}~|~\text{int}~|~\text{bool} \\
%&\gbar \typeo \times \typeo \\
%&\gbar \fut \typet \\
%\expro &::= ()~|~\inte~|~\bool  \\
%&\gbar \letin{\var}{\expro}{\expro} \\
%&\gbar \var \\
%&\gbar (\expro, \expro) \\
%&\gbar \pi_1~\expro \gbar \pi_2~\expro \\
%&\gbar \ifthen {\expro}{\expro}{\expro} \\
%&\gbar \next~\exprt \\
%\contextot &::=\emptyC \\
%&\gbar \contextot, \var : \typeo ^\bbone \\
%&\gbar \contextot, \var : \typet ^\bbtwo
%\end{aligned} $ 
%& 
%$\begin{aligned}
%\typet &::=  \text{unit}~|~\text{int}~|~\text{bool} \\
%&\gbar \typet \times \typet \\
%\\
%\exprt &::= ()~|~\inte~|~\bool \\
%&\gbar \letin{\var}{\exprt}{\exprt} \\
%&\gbar \var \\
%&\gbar (\exprt, \exprt) \\
%&\gbar \pi_1~\exprt \gbar \pi_2~\exprt \\
%&\gbar \ifthen {\exprt}{\exprt}{\exprt} \\
%&\gbar \prev~\expro \\
%\\
%\\
%\\
%\end{aligned} $
%\end{tabular}
%\end{figure}

%
%\begin{figure}
%\caption{\ellTarget~Syntax}
%\label{fig:ellTargetSyntax}
%\centering
%\begin{tabular}{ll} 
%$\begin{aligned}
%\expr &::= ()~|~\inte~|~\bool \\
%&\gbar \letin{\var}{\expr}{\expr} \\
%&\gbar \var \\
%&\gbar (\expr, \expr) \\
%&\gbar \pi_1~\expr \gbar \pi_2~\expr \\
%&\gbar \inl~\expr \gbar \inr~\expr \\
%&\gbar \ifthen {\expr}{\expr}{\expr}  \\
%&\gbar \caseof {\expr}{x_1.\expr}{x_2.\expr} 
%\end{aligned} $
%& 
%$\begin{aligned}
%\type &::=  \rmunit~|~\text{int}~|~\text{bool} \\
%&\gbar \type \times \type  \\
%&\gbar \type + \type 
%\\
%\context &::= \emptyC \\
%&\gbar \context, \var : \type
%\\ \\ \\ \\
%\end{aligned} $
%\end{tabular}
%\end{figure}


\begin{figure}
\centering
$\begin{aligned}
\world &::= \bbone \gbar \bbtwo \\
\type &::= \text{unit}~|~\text{int}~|~\text{bool} \\
&\gbar \type \times \type 
 \gbar \type + \type \\
&\gbar \type \to \type
 \gbar \fut \type \\
&\gbar \alpha \gbar \mu \alpha.\tau \\
\expr &::= ()\gbar\inte\gbar\bool\gbar \var  \\
&\gbar \lam{\var}{\type}{\expr} 
 \gbar \expr~\expr \\
&\gbar (\expr, \expr) 
 \gbar \pio~\expr 
 \gbar \pit~\expr \\
&\gbar \inl~\expr 
 \gbar \inr~\expr \\
&\gbar \caseof {\expr}{\var.\expr}{\var.\expr} \\
&\gbar \ifthen {\expr}{\expr}{\expr} \\
&\gbar \letin{\var}{\expr}{\expr} \\
&\gbar \next~\expr 
 \gbar \prev~\expr 
 \gbar \pause~\expr \\
\context &::=\emptyC \gbar \context, \colwor \var \type
\end{aligned} $
\caption{\lamStaged~Syntax}
\label{fig:lamStagedSyntax}
\end{figure}
\section{Terminology and Overview}

\begin{figure}
\textbf{Languages:}
\begin{itemize}
\item \lang: two-staged lambda calculus with product, sum, and
  recursive types
\item \langmono: an unstaged lambda calculus with products.
\end{itemize}

\textbf{\lang\ Evaluation Relations:}
\begin{itemize}
\item 
\bbone-Evaluation: $e\mathbin{\redonesym}[\xi;v]$, where $\xi$ is a \emph{residual table} and $v$ is a \emph{partial value}. 

\item
\bbtwo-Evaluation: $e\mathbin{\redtwosym}q$, where $q$ is a \emph{residual} in the unstaged language \langTwo

\item 
Table reification: \ur{Fill in...}
\end{itemize}



% $red_2$: 2-eval,  which we think of as being identical to specialization. 

\vspace{.75em}
\textbf{\lang\ Splitting:}

\hspace{2em}\bbone-Splitting Structure: $e \splitonesym [c,l.r]$, where:

\hspace{4em}$c$ is the \emph{combined term} (representing all stage~\bbone\ subcomputations in $e$)

\hspace{4em}$l.r$ is the \emph{resumer} (representating all stage~\bbtwo\ subcomputations in $e$)

\hspace{4em}$c~\redsym~(y,b)$, where $y$ is the \emph{\bbone-result} of $e$, and $b$ is the \emph{boundary value} of $e$

 
\hspace{2em}\bbone-Splitting Correctness: If $e\mathbin{\redonesym}[\xi;v]$, then:

\hspace{4em}$y$ is identical to $\masko{\xi;v}$

\hspace{4em}$\maskt{\xi;v}$ and \texttt{(let l=b in r)} reduce (via $\redsym$) to identical values.  

\hspace{2em}\bbtwo-Splitting Structure: $e \splittwosym [p,l.r]$, where:

\hspace{4em}$p$ is the \emph{precomputation} (representing all stage~\bbone\ subcomputations in $e$)

\hspace{4em}$l.r$ is the \emph{resumer} (representating all stage~\bbtwo\ subcomputations in $e$)

\hspace{2em}\bbtwo-Splitting Correctness: If $e~\redtwosym~q$, and $q~\redsym~v$, then \texttt{(let l=b in r)}~$\redsym~v$

\caption{Summary of \lang\ evaluation and splitting.}
\label{fig:terminology}
\label{fig:termSplitSummary}
\end{figure}



We present an overview of the formal development along with our
terminology, as summarized in \figref{terminology}.

\figref{terminology}

Our starting point is a two-staged lambda calculus, whose syntax and
types are very similar to the multi-stage languages considered in the
literature on partial evaluation and meta programming.  Specifically,
we consider a two-stage, which we call \lang\, which is variant of
Davies`s multi-stage language.  Apart from the restriction to two
stages instead of multiple stages, there are no significant
differences between our syntax and the static semantics and those of
Davies'. We restrict our attention to two stages, because it
simplifies the (many) details of the splitting problem, without
altering the essence of the problem.  We expect that the techniques
presented here can be extended to multi-stage languages.

The dynamic semantics of \lang\ differs from Davies's approach and
similar approaches in meta programming literature, because it avoids
duplication of staged work. Duplication takes place in meta
programming when a second-stage computation is silently duplicated by
the semantics.  In addition to leading to increases in run time, a
duplicating semantics makes it harder to reason about the splitting
algorithm, whose development and exposition is our main goal.  Apart
from the non-duplication, the dynamic semantics of \lang\ is
essentially identical to  Davies'. 


We define the dynamics semantics of \lang\ not as an explicit
evaluation relation but as a relation that produces a term in an
unstaged, completely standard language, which we refer to as
\langmono\, because it is essentially lambda calculus extended with
product types. Given a \lang\ term, the dynamic semantics can be used
to reduce the term to a \langmono\ program, called {\em residual} by
using one application of stage-1 evaluation, one application of
reification, followed by one application of stage-2 evaluation.

\ur{Splitting...}



\section{\texorpdfstring{\lang}{λ12} Statics and Dynamics}
\label{sec:semantics}

%
%
%\begin{figure}
%\caption{\ellStaged~Syntax}
%\label{fig:ellStagedSyntax}
%\centering
%\begin{tabular}{ll} 
%$\begin{aligned}
%\typeo &::= \text{unit}~|~\text{int}~|~\text{bool} \\
%&\gbar \typeo \times \typeo \\
%&\gbar \fut \typet \\
%\expro &::= ()~|~\inte~|~\bool  \\
%&\gbar \letin{\var}{\expro}{\expro} \\
%&\gbar \var \\
%&\gbar (\expro, \expro) \\
%&\gbar \pi_1~\expro \gbar \pi_2~\expro \\
%&\gbar \ifthen {\expro}{\expro}{\expro} \\
%&\gbar \next~\exprt \\
%\contextot &::=\emptyC \\
%&\gbar \contextot, \var : \typeo ^\bbone \\
%&\gbar \contextot, \var : \typet ^\bbtwo
%\end{aligned} $ 
%& 
%$\begin{aligned}
%\typet &::=  \text{unit}~|~\text{int}~|~\text{bool} \\
%&\gbar \typet \times \typet \\
%\\
%\exprt &::= ()~|~\inte~|~\bool \\
%&\gbar \letin{\var}{\exprt}{\exprt} \\
%&\gbar \var \\
%&\gbar (\exprt, \exprt) \\
%&\gbar \pi_1~\exprt \gbar \pi_2~\exprt \\
%&\gbar \ifthen {\exprt}{\exprt}{\exprt} \\
%&\gbar \prev~\expro \\
%\\
%\\
%\\
%\end{aligned} $
%\end{tabular}
%\end{figure}

%
%\begin{figure}
%\caption{\ellTarget~Syntax}
%\label{fig:ellTargetSyntax}
%\centering
%\begin{tabular}{ll} 
%$\begin{aligned}
%\expr &::= ()~|~\inte~|~\bool \\
%&\gbar \letin{\var}{\expr}{\expr} \\
%&\gbar \var \\
%&\gbar (\expr, \expr) \\
%&\gbar \pi_1~\expr \gbar \pi_2~\expr \\
%&\gbar \inl~\expr \gbar \inr~\expr \\
%&\gbar \ifthen {\expr}{\expr}{\expr}  \\
%&\gbar \caseof {\expr}{x_1.\expr}{x_2.\expr} 
%\end{aligned} $
%& 
%$\begin{aligned}
%\type &::=  \rmunit~|~\text{int}~|~\text{bool} \\
%&\gbar \type \times \type  \\
%&\gbar \type + \type 
%\\
%\context &::= \emptyC \\
%&\gbar \context, \var : \type
%\\ \\ \\ \\
%\end{aligned} $
%\end{tabular}
%\end{figure}


\begin{figure}
\centering
$\begin{aligned}
\world &::= \bbone \gbar \bbtwo \\
\type &::= \text{unit}~|~\text{int}~|~\text{bool} \\
&\gbar \type \times \type 
 \gbar \type + \type \\
&\gbar \type \to \type
 \gbar \fut \type \\
&\gbar \alpha \gbar \mu \alpha.\tau \\
\expr &::= ()\gbar\inte\gbar\bool\gbar \var  \\
&\gbar \lam{\var}{\type}{\expr} 
 \gbar \expr~\expr \\
&\gbar (\expr, \expr) 
 \gbar \pio~\expr 
 \gbar \pit~\expr \\
&\gbar \inl~\expr 
 \gbar \inr~\expr \\
&\gbar \caseof {\expr}{\var.\expr}{\var.\expr} \\
&\gbar \ifthen {\expr}{\expr}{\expr} \\
&\gbar \letin{\var}{\expr}{\expr} \\
&\gbar \next~\expr 
 \gbar \prev~\expr 
 \gbar \pause~\expr \\
\context &::=\emptyC \gbar \context, \colwor \var \type
\end{aligned} $
\caption{\lamStaged~Syntax}
\label{fig:lamStagedSyntax}
\end{figure}
%!TEX root = ../paper.tex

%\begin{figure}
%\caption{\lang~Valid Types}
%\label{fig:validTypes}
%\end{figure}

\begin{figure*}
\begin{abstrsyn}
\begin{mathpar}
\fbox {Valid Types} \and
\infertypeswor [\rmunit] 	{\Delta \vdash {\tt unit} \istypewor}						{\cdot} 																	\and
\infertypeswor [int]		{\Delta \vdash {\tt int} \istypewor}						{w \in \{\bbonep,\bbtwo\}}													\and
\infertypeswor [\times]		{\Delta \vdash A~\mathcal{O}~B \istypewor}					{\Delta \vdash A \istypewor 
																						& \Delta \vdash B \istypewor
																						& \mathcal O \in \{+,\times,\to \}} 										\and
\infertypeswor [\mu]		{\Delta \vdash \mu \alpha. A \istypewor}					{\Delta, \alpha \istypewor \vdash A \istypewor} 							\and
\infertypeswor [var]		{\Delta \vdash \alpha \istypewor}							{\alpha \istypewor \in \Delta} 												\and
\infertypeswor [\fut]		{\Delta \vdash \fut A \istypemix}							{\Delta \vdash A \istypetwo} 												\and
\infertypeswor [\fut]		{\Delta \vdash \curr A \istypemix}							{\Delta \vdash A \istypepure} 															
\end{mathpar}
\hrule
\begin{mathpar}
\fbox {Standard Typing} \and
\infertypeswor [\rmunit] 	{\ty {\tup{}}\rmunit}								{\cdot} 																	\and
\infertypeswor [int]		{\ty {i} \rmint}									{w \in \{\bbonep,\bbtwo\}}													\and
\infertypeswor [hyp]		{\ty x A}											{\col x A \in \Gamma} 														\and
\infertypeswor [\to I]		{\ty {\lam {x}{e}} {A \to B}}						{A \istypewor & \ty [,\col x A] e B} 										\and
\infertypeswor [\to E]		{\ty {\app {e_1}{e_2}} {B}}							{\ty {e_1} {A \to B} & \ty {e_2} A} 										\and
\infertypeswor [\times I]	{\ty {\tup{e_1,e_2}}{A\times B}}					{\ty {e_1} A & \ty {e_2} B} 												\and
\infertypeswor [\times E_1]	{\ty {\pio e} A}									{\ty e {A\times B}} 														\and
\infertypeswor [\times E_2]	{\ty {\pit e} B}									{\ty e {A\times B}} 														\and
\infertypeswor [+ I_1]		{\ty {\inl e} {A + B}}								{\ty e A} 																	\and
\infertypeswor [+ I_2]		{\ty {\inr e} {A + B}}								{\ty e B} 																	\and
\infertypeswor [\mu I]		{\ty {\roll e} {\mu \alpha.\tau}}					{\ty e {[\mualphatau / \alpha]\tau}} 										\and
\infertypeswor [\mu E]		{\ty {\unroll e} {[\mualphatau / \alpha]\tau}}		{\ty e \mualphatau} 														\and
\infertypeswor [+ E]		{\ty {\caseof{e_1}{x_2.e_2}{x_3.e_3}} C}			{\ty {e_1}{A+B} & \ty[,\col {x_2} A]{e_2} C & \ty[,\col {x_3} B]{e_3} C}
\end{mathpar}
\hrule
\begin{mathpar}
\fbox {Staging Features} \and
\infertypeswor [hold]		{\typesone {\pause e} {\fut \rmint}}				{\typesone e {\curr \rmint}}		 										\and
\infertypeswor [\fut I]		{\typesone {\next e}{\fut A}}						{\typestwo e A} 															\and
\infertypeswor [\fut E]		{\typestwo {\prev e} A}								{\typesone e {\fut A}} 														\and
\infertypeswor [\curr I]	{\typesone {\pure e} {\curr A}}						{\typespure e A} 															\and
\infertypeswor [\curr E]	{\typesone {\letp x {e_1} {e_2}} B}					{\typesone {e_1} {\curr A} & \typesone [\Gamma,\colpure x A] {e_2} B} 		\and
%\infertypeswor [lift] 		{\typesone {\lifttag e} {\curr A + \curr B}}		{\typesone e {\curr \tup{A+B}}} 											\and
\infertypeswor [+ E]		{\typesone {\caseP{e_1}{x_2.e_2}{x_3.e_3}} C}		{\typesone {e_1}{\curr(A+B)} 
																				&\typesone[\Gamma,\colmix {x_2} {\curr A}]{e_2} C 
																				&\typesone[\Gamma,\colmix {x_3} {\curr B}]{e_3} C} 							
\end{mathpar}
% \hrule
% \begin{mathpar}
% \fbox {Derivable $\curr$ Rules} \and
% \infertypeswor [\to E]		{\ty {\app {e_1}{e_2}} {\curr B}}					{\ty {e_1} {\curr\tup{A \to B}} & \ty {e_2} {\curr A}} 							\and
% \infertypeswor [\times E_1]	{\ty {\pio e} {\curr A}}							{\ty e {\curr\tup{A\times B}}}													\and
% \infertypeswor [\times E_2]	{\ty {\pit e} {\curr B}}							{\ty e {\curr\tup{A\times B}}}													\and
% \infertypeswor []			{\typesone {x} {\curr A}}							{\colpure x A \in \Gamma}													\and
% \infertypeswor []			{\typespure x A}									{\colone x {\curr A} \in \Gamma}													
% \end{mathpar}
\end{abstrsyn}
\caption{\lang~Static Semantics}
\label{fig:statics}
\end{figure*}


\subsection{Statics}

In \lang, the $\next$ constructor includes a stage \bbtwo\ term in a
stage~\bbone\ term. To ensure that the staging features are only applied to
valid terms, we index our judgments not only by a type, but also a stage $w$.

The type validity judgment $\Delta \vdash A \istypewor$, defined in
\cref{fig:validTypes}, ensures that a type $A$ exists at stage $w$; for
example, any type $\fut A$ only exists at stage \bbone.  The typing judgment
$\typeswor e A$ says that $e$ has type $A$ at stage $w$, in the context
$\Gamma$, and is defined in \cref{fig:statics}.  As one would expect, the
typing judgment only produces valid types.
%(by induction on the derivation of the typing judgment).
% under the assumptions in $\Delta$. This context is only augmented by recursive types.

Except for the staging features $\next$, $\prev$, and $\pause$, the ordinary
language features (i.e., the introductory and eliminatory forms for product,
sum, and function types) preserve the stage of their subterms, and work at both
stages \bbone\ and \bbtwo. Thus, if we were to remove the staging features,
\lang\ would simply consist of two non-interacting copies of a standard lambda
calculus. Variables in the context are annotated with the stage at which they
were introduced and can only be used at the same stage.

Note that there is no way to eliminate a term of type $\fut A$ into a stage
\bbone\ term---the only elimination form is $\prev$, which results in a stage
\bbtwo\ term. This is why well-typed terms cannot have any information flow
from stage \bbtwo\ to stage \bbone, and what makes splitting possible.

\subsection{Dynamics}
\label{sec:stagedsemantics}

As is the case in essentially all staged language, the dynamic
semantics evaluates a term to a residual (a term in the target
language \langmono) in two stages. In the first stage, all stage-1
terms are evaluated by a relation, called {\em 1-evaluation}; in the
second stage all stage-2 terms are evaluated by a relation
$\redtwosym$, called {\em 2-evaluation}.  To eliminate duplication,
our semantics lifts out each stage-2 term nested inside of a stage-1
expressions, into a table which makes it possible for the stage-2
term---more precisely the corresponding residual obtained by reducing
all stage-1 subterms---to be referred to by use a dynamically
generated variable (a label).

Since stage-1 evaluation does not evaluate stage-2 terms and since
stage-2 terms may be nested inside stage-1 terms, 1-evaluation cannot
reduce \lang\ terms to values.  Instead 1-evaluation yields a {\em
  partial value}, which which is a stage~\bbone\ term that have been
evaluated except for its stage-2 subterms, which can only
be~\bbtwo\ variables wrapped in $\next$ blocks.  \figref{validPvalues}
defines partial values.

In order to avoid duplication of stage-2 computations, 1-evaluation
labels stage-2 computations by generating variable symbols for them as
needed and stores their residuals in a {\em residual table} keyed by
the variable.


The relation 1-evaluation, defined by $\redonesym$ judgment as shown
in \ref{fig:diaSemantics}, takes an stage \bbone\ term%
\footnote{ The input to $\redonesym$ may be open on stage \bbtwo\ variables, but
the type system ensures that those must occur under a $\next$. Consequently
$\redonesym$ will never directly encounter a variable.}
%
to a residual table $\xi$ and a partial value $v$.  For non-staging
features of \lang, $\redonesym$ is essentially standard call-by-value
evaluation but in addition gathers the subterms' residual tables into
a single table. \ur{The syntax of residual tables should be defined in
  the terminology figure} Residual table entries are created by the
evaluation of \next: 1-evaluation reduces a \next expression by
reducing the body of the \next with 2-evaluation to obtain a
residual $q$ and returns a residual table that maps a freshly
generated variable to the residual.  \ur{Variable should be fresh in
  the rule.}


The relation 2-evaluation, defined by $\redtwosym$ judgment, as shows
in \figref{diaSemantics}, takes a stage \bbtwo\ term and reduces it to
a residual.  Apart from the evaluation of \prev and \pause, the
evaluation is straightforward. To reduce a \prev term, 2-evaluation
reduces the body by 1-evaluation to obtain a residual table and a
partial value.  It then reifies the residual table and the partial
value into an residual term, by creating for each variable-residual
pair in the table a corresponding variable binding in the residual,
finally ending with the partial value.
\ur{There seems to be errors in the semantics in the handling of
  recursive types.  I don't see any reason to do any reductions here.
If you want to, the description above needs to be updated.}

...UMUT: UP TO HERE...




 In addition, 1-evaluation {\em reifies} the contents of
the residual table into a residual when changing from stage-2 to stage



%% Note that partial values and residuals are both terms for which stage \bbone\
%% computation has completed; however, partial values are stage~\bbone\ terms,
%% while residuals are stage~\bbtwo\ terms.



 The residual
table implements the explicit substitutions in our example---it maps fresh
stage \bbtwo\ variables (which may appear inside $\next$ blocks in $v$) to
residuals.


In order to avoid duplicating subterms, evaluation collects explicit
substitutions into a single {\em residual table} that is also the output of
evaluation.




 the auxiliary $\reifysym$ (reification). 
Along with these are three judgments identifying forms of values:
$\pvalsym$ (partial values), $\ressym$ (residuals), 
and the auxiliary $\fcon$ (residual table).
These judgments are related as follows:
\begin{itemize}
\item Evaluation sends a stage~\bbone\ term to a partial value.

\item Specialization sends a stage~\bbtwo\ term to a residual.

\item Reification sends a single residual and a residual table to another residual.  


\item Evaluation depends on specialization for $\next$ terms, which have
stage~\bbtwo\ subterms.
\item Specialization depends on evaluation for $\prev$ and $\pause$, which have
stage~\bbone\ subterms. The residual table produced by evaluation is also
reified here.
\end{itemize}











The relation 2-evaluation takes a stage-2 term and maps to a residual.

To generate the residual table, the dynamics semantics relies on a
{\em reification} relation, $\reifysym$, that maps a table to a
residual, essentially generating a target expression representing the
computations in the table.  

We can evaluate a \lang\ term as follows: we first apply 1-evaluation
to obtain a partial value and a residual table, we then construct a
2-term from the partial value and the residual table by using
reification, and then reduce this term to a residual via 2-evaluation.

Thus, the dynamic semantics for \lang\ evaluates all of a term's stage \bbone\
subexpressions before any of its stage~\bbtwo\ subexpressions. This results in
a stage \bbtwo\ term with no stage~\bbone\ content, 
which can be described as a term in a monostaged language called \langTwo. 
Then, at stage~\bbtwo, we complete
evaluating this term with a standard dynamic semantics, $\tworedsym$. (The
rules for this judgment are not shown, but they are standard.)





\crem{Can we cut \ref{fig:diaSemanticsSpec} from the paper? Perhaps just put a
few representative rules into \ref{fig:diaSemantics}?}

At $\next~e$, in contrast, $\redonesym$ specializes into $e$ for stage~\bbone\ 
subexpressions to evaluate.  This is implemented by the $\redtwosym$ judgment,
which takes a stage \bbtwo\ term to a residual. 
For the most part (\ref{fig:diaSemanticsSpec}), specialization simply recurs into
all subexpressions; at $\prev$, however, it resumes $\redonesym$ evaluation,
which produces a residual table $\xi$ and (by canonical forms) an expression
$\next\{\hat y\}$. $\xi$ is then reified using $\reifysym$ into a series of let
bindings enclosing $\mathtt{\hat y}$.
Once specialization reduces $e$ to a residual $q$, the output of $\redonesym$ is a
fresh variable wrapped in a $\next$ block ($\next~\hat y$), along with a residual
table which maps that variable to the residual ($\hat y \mapsto q$).

Within specialization lies a subtle---if perhaps unintuitive---feature.  
Observe that specialization will traverse into both branches of a stage-\bbtwo\ {\tt if} or {\tt case} 
statement in its search for stage~\bbone\ code. 
Thus the evaluation of that stage~\bbone\ code will occur {\em regardless of the eventual value of the predicate},
and so a term like 
\begin{lstlisting} 
1`next{
  `2`if true 
  then hold{`1`3+4`2`} 
  else prev{`1`spin() (* loops forever *)`2`}`1`
}`
\end{lstlisting}
will fail to evaluate at stage \bbone.
Although it may seem undesirable, this behavior is a deliberate feature.
Indeed it is critical to the quickselect example, 
wherein all three branches of the $n$-$k$ comparison are evaluated at stage~\bbone.
As always, it is the programmer's responsibility to ensure that her staging annotations 
do not cause a program to loop forever.  

The context ($\Gamma$) keeps track of stage \bbtwo\ variables in the input term. 
These both appear in the original program at stage \bbtwo\ and are inserted by the semantics.

As an optimization, we can include the special-case rule,
\begin{mathpar}
\inferdiaone [hat] {\red {\next~\hat y}{\cdot,\next~\hat y}}{\cdot}
\end{mathpar}
to avoid one-for-one variable bindings in the residual.

\subsubsection {Top-Level Evaluation}
\label{sec:partialeval}

% KAYVONF: good statement, but hold out for now
%The hope of partial evaluation is that $f_x$ is cheaper to execute than $f$, meaning that we can save work if we must %evaluate it many times.

These dynamics allow us to define a partial evaluator for \lang\ by identifying
static with stage \bbone\ and dynamic with stage \bbtwo.  Specifically, we
encode $f$ as a \lang\ expression with a function type of the form
$A\to\fut(B\to C)$.%
%\cprotect\footnote{We can rewrite \texttt{fexp} in this form, or simply apply
%the following higher-order function which makes the adjustment:
%\begin{lstlisting} 
%let adjust (f : $int * int -> $int) =
%  fn (p : int) => 
%    next{
%      fn (b : int) => 
%        prev{f (next {b}, p)}
%    }
%\end{lstlisting}}
%
Here $A$ is the static input, $B$ is the dynamic input, and $C$ is the output.

Once a stage \bbone\ argument $a:A$ is provided, we can evaluate the partially-applied
function:
$\cdot\vdash f~a \mathop{\redonesym} [\xi,v]$.
The result is an environment $\xi$ and a partial value $v$ of type $\fut(B\to
C)$, which by canonical forms must have the form $v = \next~\hat y$. 
Next, we reify this environment into a sequence of \verb|let|-bindings
enclosing $\hat y$, via $\reify\xi{\hat y}{f_a}$. 
Because reification preserves types, the resulting residual $f_a$ has type $B\to C$ in \langTwo, so we can apply it to some $b:B$
and compute the final result of the function, $f_a~b \mathop{\tworedsym} c$.

That this sequence of evaluations is in fact staged follows from our
characterizations of partial values and residuals, that $\redonesym$
outputs a partial value, and that $\reifysym$ outputs an expression in \langTwo.

For example, to specialize quickselect to a particular list, 
\begin{lstlisting}
qsStaged [5,2,7,4,1]
\end{lstlisting}
\TODO finish example

%% Umut: superseded by what is above.
%% Version Nov 7, 2014
%% \section{\texorpdfstring{\lang}{λ12} Statics and Dynamics}
%% \label{sec:semantics}

%% %
%
%\begin{figure}
%\caption{\ellStaged~Syntax}
%\label{fig:ellStagedSyntax}
%\centering
%\begin{tabular}{ll} 
%$\begin{aligned}
%\typeo &::= \text{unit}~|~\text{int}~|~\text{bool} \\
%&\gbar \typeo \times \typeo \\
%&\gbar \fut \typet \\
%\expro &::= ()~|~\inte~|~\bool  \\
%&\gbar \letin{\var}{\expro}{\expro} \\
%&\gbar \var \\
%&\gbar (\expro, \expro) \\
%&\gbar \pi_1~\expro \gbar \pi_2~\expro \\
%&\gbar \ifthen {\expro}{\expro}{\expro} \\
%&\gbar \next~\exprt \\
%\contextot &::=\emptyC \\
%&\gbar \contextot, \var : \typeo ^\bbone \\
%&\gbar \contextot, \var : \typet ^\bbtwo
%\end{aligned} $ 
%& 
%$\begin{aligned}
%\typet &::=  \text{unit}~|~\text{int}~|~\text{bool} \\
%&\gbar \typet \times \typet \\
%\\
%\exprt &::= ()~|~\inte~|~\bool \\
%&\gbar \letin{\var}{\exprt}{\exprt} \\
%&\gbar \var \\
%&\gbar (\exprt, \exprt) \\
%&\gbar \pi_1~\exprt \gbar \pi_2~\exprt \\
%&\gbar \ifthen {\exprt}{\exprt}{\exprt} \\
%&\gbar \prev~\expro \\
%\\
%\\
%\\
%\end{aligned} $
%\end{tabular}
%\end{figure}

%
%\begin{figure}
%\caption{\ellTarget~Syntax}
%\label{fig:ellTargetSyntax}
%\centering
%\begin{tabular}{ll} 
%$\begin{aligned}
%\expr &::= ()~|~\inte~|~\bool \\
%&\gbar \letin{\var}{\expr}{\expr} \\
%&\gbar \var \\
%&\gbar (\expr, \expr) \\
%&\gbar \pi_1~\expr \gbar \pi_2~\expr \\
%&\gbar \inl~\expr \gbar \inr~\expr \\
%&\gbar \ifthen {\expr}{\expr}{\expr}  \\
%&\gbar \caseof {\expr}{x_1.\expr}{x_2.\expr} 
%\end{aligned} $
%& 
%$\begin{aligned}
%\type &::=  \rmunit~|~\text{int}~|~\text{bool} \\
%&\gbar \type \times \type  \\
%&\gbar \type + \type 
%\\
%\context &::= \emptyC \\
%&\gbar \context, \var : \type
%\\ \\ \\ \\
%\end{aligned} $
%\end{tabular}
%\end{figure}


\begin{figure}
\centering
$\begin{aligned}
\world &::= \bbone \gbar \bbtwo \\
\type &::= \text{unit}~|~\text{int}~|~\text{bool} \\
&\gbar \type \times \type 
 \gbar \type + \type \\
&\gbar \type \to \type
 \gbar \fut \type \\
&\gbar \alpha \gbar \mu \alpha.\tau \\
\expr &::= ()\gbar\inte\gbar\bool\gbar \var  \\
&\gbar \lam{\var}{\type}{\expr} 
 \gbar \expr~\expr \\
&\gbar (\expr, \expr) 
 \gbar \pio~\expr 
 \gbar \pit~\expr \\
&\gbar \inl~\expr 
 \gbar \inr~\expr \\
&\gbar \caseof {\expr}{\var.\expr}{\var.\expr} \\
&\gbar \ifthen {\expr}{\expr}{\expr} \\
&\gbar \letin{\var}{\expr}{\expr} \\
&\gbar \next~\expr 
 \gbar \prev~\expr 
 \gbar \pause~\expr \\
\context &::=\emptyC \gbar \context, \colwor \var \type
\end{aligned} $
\caption{\lamStaged~Syntax}
\label{fig:lamStagedSyntax}
\end{figure}
%% %!TEX root = ../paper.tex

%\begin{figure}
%\caption{\lang~Valid Types}
%\label{fig:validTypes}
%\end{figure}

\begin{figure*}
\begin{abstrsyn}
\begin{mathpar}
\fbox {Valid Types} \and
\infertypeswor [\rmunit] 	{\Delta \vdash {\tt unit} \istypewor}						{\cdot} 																	\and
\infertypeswor [int]		{\Delta \vdash {\tt int} \istypewor}						{w \in \{\bbonep,\bbtwo\}}													\and
\infertypeswor [\times]		{\Delta \vdash A~\mathcal{O}~B \istypewor}					{\Delta \vdash A \istypewor 
																						& \Delta \vdash B \istypewor
																						& \mathcal O \in \{+,\times,\to \}} 										\and
\infertypeswor [\mu]		{\Delta \vdash \mu \alpha. A \istypewor}					{\Delta, \alpha \istypewor \vdash A \istypewor} 							\and
\infertypeswor [var]		{\Delta \vdash \alpha \istypewor}							{\alpha \istypewor \in \Delta} 												\and
\infertypeswor [\fut]		{\Delta \vdash \fut A \istypemix}							{\Delta \vdash A \istypetwo} 												\and
\infertypeswor [\fut]		{\Delta \vdash \curr A \istypemix}							{\Delta \vdash A \istypepure} 															
\end{mathpar}
\hrule
\begin{mathpar}
\fbox {Standard Typing} \and
\infertypeswor [\rmunit] 	{\ty {\tup{}}\rmunit}								{\cdot} 																	\and
\infertypeswor [int]		{\ty {i} \rmint}									{w \in \{\bbonep,\bbtwo\}}													\and
\infertypeswor [hyp]		{\ty x A}											{\col x A \in \Gamma} 														\and
\infertypeswor [\to I]		{\ty {\lam {x}{e}} {A \to B}}						{A \istypewor & \ty [,\col x A] e B} 										\and
\infertypeswor [\to E]		{\ty {\app {e_1}{e_2}} {B}}							{\ty {e_1} {A \to B} & \ty {e_2} A} 										\and
\infertypeswor [\times I]	{\ty {\tup{e_1,e_2}}{A\times B}}					{\ty {e_1} A & \ty {e_2} B} 												\and
\infertypeswor [\times E_1]	{\ty {\pio e} A}									{\ty e {A\times B}} 														\and
\infertypeswor [\times E_2]	{\ty {\pit e} B}									{\ty e {A\times B}} 														\and
\infertypeswor [+ I_1]		{\ty {\inl e} {A + B}}								{\ty e A} 																	\and
\infertypeswor [+ I_2]		{\ty {\inr e} {A + B}}								{\ty e B} 																	\and
\infertypeswor [\mu I]		{\ty {\roll e} {\mu \alpha.\tau}}					{\ty e {[\mualphatau / \alpha]\tau}} 										\and
\infertypeswor [\mu E]		{\ty {\unroll e} {[\mualphatau / \alpha]\tau}}		{\ty e \mualphatau} 														\and
\infertypeswor [+ E]		{\ty {\caseof{e_1}{x_2.e_2}{x_3.e_3}} C}			{\ty {e_1}{A+B} & \ty[,\col {x_2} A]{e_2} C & \ty[,\col {x_3} B]{e_3} C}
\end{mathpar}
\hrule
\begin{mathpar}
\fbox {Staging Features} \and
\infertypeswor [hold]		{\typesone {\pause e} {\fut \rmint}}				{\typesone e {\curr \rmint}}		 										\and
\infertypeswor [\fut I]		{\typesone {\next e}{\fut A}}						{\typestwo e A} 															\and
\infertypeswor [\fut E]		{\typestwo {\prev e} A}								{\typesone e {\fut A}} 														\and
\infertypeswor [\curr I]	{\typesone {\pure e} {\curr A}}						{\typespure e A} 															\and
\infertypeswor [\curr E]	{\typesone {\letp x {e_1} {e_2}} B}					{\typesone {e_1} {\curr A} & \typesone [\Gamma,\colpure x A] {e_2} B} 		\and
%\infertypeswor [lift] 		{\typesone {\lifttag e} {\curr A + \curr B}}		{\typesone e {\curr \tup{A+B}}} 											\and
\infertypeswor [+ E]		{\typesone {\caseP{e_1}{x_2.e_2}{x_3.e_3}} C}		{\typesone {e_1}{\curr(A+B)} 
																				&\typesone[\Gamma,\colmix {x_2} {\curr A}]{e_2} C 
																				&\typesone[\Gamma,\colmix {x_3} {\curr B}]{e_3} C} 							
\end{mathpar}
% \hrule
% \begin{mathpar}
% \fbox {Derivable $\curr$ Rules} \and
% \infertypeswor [\to E]		{\ty {\app {e_1}{e_2}} {\curr B}}					{\ty {e_1} {\curr\tup{A \to B}} & \ty {e_2} {\curr A}} 							\and
% \infertypeswor [\times E_1]	{\ty {\pio e} {\curr A}}							{\ty e {\curr\tup{A\times B}}}													\and
% \infertypeswor [\times E_2]	{\ty {\pit e} {\curr B}}							{\ty e {\curr\tup{A\times B}}}													\and
% \infertypeswor []			{\typesone {x} {\curr A}}							{\colpure x A \in \Gamma}													\and
% \infertypeswor []			{\typespure x A}									{\colone x {\curr A} \in \Gamma}													
% \end{mathpar}
\end{abstrsyn}
\caption{\lang~Static Semantics}
\label{fig:statics}
\end{figure*}


%% \lang\ is a typed two-stage lambda calculus with products, sums, and
%% isorecursive types.  Its statics are adapted from the linear temporal
%% logic given by \cite{davies96}, restricted to two stages and extended
%% with general sums and recursion.%
%% \footnote{This restriction to two stages is made primarily for simplicity of
%% presentation. All of these techniques could feasibly be extended to more
%% complicated stage systems.} 
%% Although \cite{davies96} was mostly interested in describing partial evaluation,
%% the system is equally apt as an input to splitting.

%% \subsection{Statics}

%% In \lang, the $\next$ constructor includes a stage \bbtwo\ term in a
%% stage~\bbone\ term. To ensure that the staging features are only applied to
%% valid terms, we index our judgments not only by a type, but also a stage $w$.

%% The type validity judgment $\Delta \vdash A \istypewor$, defined in
%% \cref{fig:validTypes}, ensures that a type $A$ exists at stage $w$; for
%% example, any type $\fut A$ only exists at stage \bbone.  The typing judgment
%% $\typeswor e A$ says that $e$ has type $A$ at stage $w$, in the context
%% $\Gamma$, and is defined in \cref{fig:statics}.  As one would expect, the
%% typing judgment only produces valid types.
%% %(by induction on the derivation of the typing judgment).
%% % under the assumptions in $\Delta$. This context is only augmented by recursive types.

%% Except for the staging features $\next$, $\prev$, and $\pause$, the ordinary
%% language features (i.e., the introductory and eliminatory forms for product,
%% sum, and function types) preserve the stage of their subterms, and work at both
%% stages \bbone\ and \bbtwo. Thus, if we were to remove the staging features,
%% \lang\ would simply consist of two non-interacting copies of a standard lambda
%% calculus. Variables in the context are annotated with the stage at which they
%% were introduced and can only be used at the same stage.

%% Note that there is no way to eliminate a term of type $\fut A$ into a stage
%% \bbone\ term---the only elimination form is $\prev$, which results in a stage
%% \bbtwo\ term. This is why well-typed terms cannot have any information flow
%% from stage \bbtwo\ to stage \bbone, and what makes splitting possible.

%% \subsection{Dynamics}
%% \label{sec:stagedsemantics}


%% Specifically, the dynamic semantics consist of two evaluation
%% relations $\redonesym$ and $\redtwosym$, called {\em 1-evaluation} and
%% {\em 2-evaluation}, that respectively evaluates stage-1 and stage-2
%% expressions. The relation 1-evaluation maps a stage-1 term to a
%% partial value and a residual table. A {\em partial value}, can be
%% thought as a value except that it might include references to stage-2
%% code stored in the {\em residual table}, which maps variables to
%% residuals.  The relation 2-evaluation takes a stage-2 term and maps to
%% a residual.  To generate the residual table, the dynamics semantics
%% relies on a {\em reification} relation, $\reifysym$, that maps a table
%% to a residual, essentially generating a target expression representing
%% the computations in the table.  We can evaluate a \lang\ term as
%% follows: we first apply 1-evaluation to obtain a partial value and a
%% residual table, we then construct a 2-term from the partial value and
%% the residual table by using reification, and then reduce this term to
%% a residual via 2-evaluation.


%% In the quickselect example, stage~\bbone\ terms are contained within
%% stage~\bbtwo\ terms---for example, in the recursive call to {\tt qsStaged}. In
%% this case, if we use an ordinary evaluation strategy reducing outermost redexes
%% first, we would no longer have staged evaluation, as stage~\bbtwo\ code would
%% be evaluated before stage~\bbone\ code. 

%% Thus, the dynamic semantics for \lang\ evaluates all of a term's stage \bbone\
%% subexpressions before any of its stage~\bbtwo\ subexpressions. This results in
%% a stage \bbtwo\ term with no stage~\bbone\ content, 
%% which can be described as a term in a monostaged language called \langTwo. 
%% Then, at stage~\bbtwo, we complete
%% evaluating this term with a standard dynamic semantics, $\tworedsym$. (The
%% rules for this judgment are not shown, but they are standard.)

%% \subsubsection{Stage~\bbone\ Evaluation}

%% To gain intuition about the challenges of implementing this staged dynamic
%% semantics, consider the program:
%% \begin{lstlisting}
%% 1`#2 (next {`2`f 20`1`}, 2+3)`
%% \end{lstlisting}
%% This is a stage~\bbone\ expression of type $\rmint~@~\bbone$; the pair inside it is a
%% stage~\bbone\ expression of type $(\fut\rmint)\times\rmint~@~\bbone$; and ${\tt f} :
%% \rmint \to \rmint~@~\bbtwo$ is an expensive stage~\bbtwo\ function which we would like
%% to avoid evaluating. A conventional call-by-value semantics would fully
%% evaluate both components of the pair before projecting the second component.
%% The problem is that, while \verb|next {f 20}| is not a value (in the sense that
%% additional stage-\bbtwo\ computation steps are necessary to produce a numeral),
%% evaluating the contents of \verb|next| cannot occur as part of stage~\bbone\
%% evaluation.

%% Intuitively, the solution is to designate \verb|next {f 20}| as a value
%% \emph{in stage \bbone}, even though it requires additional evaluation in stage
%% \bbtwo. Therefore, the pair evaluates to
%% \begin{lstlisting}
%% 1`(next {`2`f 20`1`}, 5)`
%% \end{lstlisting}
%% and the projection, in turn, immediately evaluates to \verb|5|.

%% Now consider a more complex example where stage~\bbone\ evaluation must
%% substitute such an incompletely-evaluated expression. The following
%% stage-\bbtwo\ term has type $\rmint$:
%% \begin{lstlisting} 
%% 2`prev{`
%%   1`let x = (next {`2`f 20`1`}, 3+4) in
%%   next{` 2`prev{`1`#1 x`2`} * prev{`1`#1 x`2`} * hold{`1`#2 x`2`}` 1`}`
%% 2`}`
%% \end{lstlisting}
%% As with the simpler example, this term does not fully reduce at stage~\bbone, 
%% because it depends on the value of \verb|f 20|, which is not reduced until stage \bbtwo.

%% Again treating \verb|(next {f 20}, 7)| as a value during stage~\bbone, we
%% substitute it for the three occurrences of \verb|x| in the body of the
%% \verb|let| expression, yielding
%% \begin{lstlisting} 
%% 2`prev{`
%%   1`next{` 2`prev{`1`#1 (next {`2`f 20`1`}, 7)`2`} 
%%       * prev{`1`#1 (next {`2`f 20`1`}, 7)`2`} 
%%       * hold{`1`#2 (next {`2`f 20`1`}, 7)`2`} `1`}`2`
%% }`
%% \end{lstlisting}
%% The outermost $\next$ here is not yet a value, because there remains stage~\bbone\ work to do in its body.
%% So to proceed, the semantics must first search the body of the $\next$ for stage~\bbone\ subterms and reduce them in place.
%% This search process is called {\em specialization}.
%% In our example this means reducing all three projections to give,
%% \begin{lstlisting} 
%% 2`prev{`1`
%%   next{`2` 
%%     prev{`1`next {`2`f 20`1`}`2`} * prev{`1`next {`2`f 20`1`}`2`} * 7 
%%   `1`}`2`
%% }`
%% \end{lstlisting}
%% Since all stage~\bbone\ subterms are reduced, the $\prev$s and $\next$s cancel to yield the final residual:
%% \begin{lstlisting} 
%% 2`(f 20) * (f 20) * 7`
%% \end{lstlisting}
%% But now stage \bbtwo\ must compute the expensive function call \verb|f 20| twice! 
%% There are some systems, typically called {\em metaprogramming}, 
%% where this duplication would be considered intended behavior.
%% For our applications, we find the duplication undesirable and avoid it.

%% To get this behavior, we do not treat \verb|next {f 20}| as a value to be substituted, 
%% but instead bind \verb|f 20| to a fresh stage~\bbtwo\ variable $\mathtt{\hat y}$
%% and then substitute that variable.

%% More precisely, when evaluating the \verb|next|, we construct an explicit
%% substitution $\mathtt{[\hat y\mapsto f~20]}$ binding its old contents to the
%% fresh variable $\mathtt{\hat y}$. This substitution is placed at the top of the
%% containing $\prev$ block:
%% \begin{lstlisting} 
%% 2`prev {
%% [yhat|->f 20]
%%   `1`let x = (next{`2`yhat`1`}, 7) in
%%   next{`2`prev{`1`#1 x`2`} * prev{`1`#1 x`2`} * hold{`1`#2 x`2`}`1`}`2`
%% }`
%% \end{lstlisting}
%% %As a convention, we render the new variable with a %stylish and fashionable
%% %hat.  
%% We proceed with stage~\bbone\ evaluation, which now duplicates the variable
%% $\mathtt{\hat y}$ rather than the expression \verb|f 20|.
%% \begin{lstlisting} 
%% 2`prev {
%% [yhat|->f 20]
%%   `1`next{`2`
%%     prev{`1`#1 (next {`2`yhat`1`}, 7)`2`} * 
%%     prev{`1`#1 (next {`2`yhat`1`}, 7)`2`} *
%%     hold{`1`#2 (next {`2`yhat`1`}, 7)`2`}
%%   `1`}`2`
%% }`
%% \end{lstlisting}
%% We then specialize to reduce the projections and cancel the $\next$s and $\prev$s, giving:
%% \begin{lstlisting} 
%% 2`prev {
%% [yhat|->f 20]
%%     `1`next{`2`yhat * yhat * 7`1`}
%% `2`}`
%% \end{lstlisting}
%% Once again, we lift the results of specialization into a substitution:
%% \begin{lstlisting} 
%% 2`prev {
%% [yhat|->f 20,
%%  zhat|->yhat * yhat * 7]
%%     `1`next{`2`zhat`1`}`2`
%% }`
%% \end{lstlisting}
%% Finally, when canceling the last $\next$ and $\prev$, the semantics {\em reifies} the contained substitutions into let statements, yielding
%% \begin{lstlisting} 
%% 2`let yhat = f 20 in
%% let zhat = yhat * yhat * 7 in zhat`
%% \end{lstlisting}

%% This concludes stage \bbone\ evaluation of the program---we have reduced all
%% stage \bbone\ redexes, resulting in a monostaged term in \langTwo. Stage \bbtwo\
%% evaluation, $\tworedsym$, then reduces this to a numeral.

%% \subsubsection{Evaluation and Specialization}
%% \label{ssec:dynamics}

%% %!TEX root = ../paper.tex

\begin{figure*}
\begin{abstrsyn}
\begin{mathpar}
\fbox{1-Evaluation}	\and
\inferdiaone[\rmunit]
{\red {\tup{}}{\cdot;\tup{}}}
{\cdot}
\and
\inferdiaone [\to I]
{\red {\lam{x}{e}} {\cdot;\lam{x}{e}}}
{\cdot}
\and
\inferdiaone [\to E]
{\red {\app {e_1}{e_2}} {\gcomp 1 2, \xi';v'}}
{\red {e_1} {\xi_1;\lam{x}{e'}} & \sub [2] & \red  [\Gamma,\dom{\xi_1},\dom{\xi_2}] {[v_2/x]e'}{\xi';v'}}
\and
%
\inferdiaone [\times I]
{\red {\tup{e_1,e_2}}{\gcomp 1 2;\valprod{v_1}{v_2}}}
{\sub [1] & \sub [2]}
\and
\inferdiaone [\times E_1]
{\red {\pio{e}}{\xi;v_1}}
{\red{e}{\xi;\valprod{v_1}{v_2}}}
\and
\inferdiaone [\times E_2]
{\red {\pit{e}}{\xi;v_2}}
{\red{e}{\xi;\valprod{v_1}{v_2}}}
\and
\inferdiaone [+ I_1]
{\red {\inl{e}} {\xi;\inl{v}}}
{\sub}
\and
\inferdiaone [+ I_2]
{\red {\inr{e}} {\xi;\inr{v}}}
{\sub}
\and
%
\inferdiaone [+ E_1]
{\red {\caseof{e_1}{x_2.e_2}{x_3.e_3}}{\gcomp 1 2;v_2}}
{\red {e_1}{\xi_1;\inl{v}} & \red
  [\Gamma,\dom{\xi_1}]{[v/x_2]e_2}{\xi_2;v_2}}
\and
\inferdiaone [+ E_2]
{\red {\caseof{e_1}{x_2.e_2}{x_3.e_3}}{\gcomp 1 3;v_3}}
{\red {e_1}{\xi_1;\inr{v}} & \red
  [\Gamma,\dom{\xi_1}]{[v/x_3]e_3}{\xi_3;v_3}}
\and
\inferdiaone[\mu I]
{\red {\roll{e}}{\xi;\roll{v}}}
{\sub}
\and
\inferdiaone[\mu E]
{\red {\unroll{e}}{\xi; v}}
{\red {e}{\xi; \roll{v}}}
\end{mathpar}

\hrule
\begin{mathpar}
\fbox{Residualization}
\and
\inferdiaspc[\rmunit]
{\red {\tup{}}{\tup{}}}
{\cdot}
\and
%\inferdiaspc[int]
%{\red {i}{i}}
%{\cdot}
%\and
%% %\inferdiaspc [bool]{\red {b}{b}}
%% {\cdot}
%% \and
\inferdiaspc[hyp]
{\red {x}{x}}
{\cdot}
\and
\inferdiaspc[\to I]
{\red {\lam{x}{e}}{\lam{x}{q}}}
{\diatwo [\Gamma,x] e q} 
\and
\inferdiaspc [\to E]
{\red {\app {e_1} {e_2}}{q_1~q_2}}
{\sub [1] & \sub [2]}
\and
\inferdiaspc [\times I]
{\red {\tup{e_1,e_2}}{\tup{q_1,q_2}}}
{\sub [1] & \sub [2]} 
\and
\inferdiaspc [C]
{\red {\scriptCapp e}{\scriptCapp q}}
{\sub & \scriptC \in \{\mathtt{pi1},\mathtt{pi2},\mathtt{inl},\mathtt{inr},\mathtt{roll},\mathtt{unroll}\}}
%\inferdiaspc[+ I_2]
%{\red {\inr~e}{\inr~q}}
%{\sub}
\and
\inferdiaspc[+ E_1]
{\red {\caseof{e_1}{x_2.e_2}{x_3.e_3}}
{\caseof{q_1}{x_2.q_2}{x_3.q_3}}}
{\sub [1] & \diatwo [\Gamma,x_2] {e_2} {q_2} & \diatwo [\Gamma,x_3] {e_3} {q_3}} 
%\inferdiaspc [\mu E]   		{\red {\unroll~e}{\unroll~v}}
%	{\sub}										\and
%\inferdiaspc [let]			{\red {\letin{x}{e_1}{e_2}}{\letin{x}{q_1}{q_2}}}			{\sub [1] & \diatwo [\Gamma,x] {e_2} {q_2}} 					\and
%\inferdiaspc [if_T] 			{\red {\ifthen{e_1}{e_2}{e_3}}{\ifthen{q_1}{q_2}{q_3}}}	{\sub [1] & \sub [2] & \sub [3]} 		\and
%
\end{mathpar}
\hrule
\begin{mathpar}
\fbox{Staging Features} \and
\inferdiaone [\fut I]	{\red {\next{e}}{\hat y \mapsto q;\next{\hat y}}}			{\diatwo e q}														\and
\inferdiaspc [\fut E]	{\red {\prev{e}} q}											{\diaone e {\xi; \next{\hat y}} & \reify{\xi}{\hat y}q}				\and
%\inferdiaone [hold]		{\red {\pause e} {\xi, \hat y \mapsto i; \next {\hat y}}}	{\red e {\xi; \pure i}}												\and
\infer					{\reify {\cdot}{q}{q}}										{\cdot}																\and
\infer					{\reify {y \mapsto q_1, \xi}{q_2}{\letin{y}{q_1}{q'}}}		{\reify{\xi}{q_2}{q'}}												\and
\inferdiaone			{\red {\pure e} {\cdot; \pure v}}							{\reduce e v} 														\and
\inferdiaone			{\red {\letp x {e_1} {e_2}} {\xi_1, \xi_2; v_2}}			{\red {e_1} {\xi_1;\pure {v_1}} & \red {[v_1/x]e_2} {\xi_2;v_2}} 	\and
%\inferdiaone			{\red {\lifttag e} {\xi;\inl{\pure v}}}						{\red e {\xi; \pure{\inl v}}}										\and
%\inferdiaone			{\red {\lifttag e} {\xi;\inr{\pure v}}}						{\red e {\xi; \pure{\inr v}}}										
\inferdiaone [+ E_1]	{\red {\caseP{e_1}{x_2.e_2}{x_3.e_3}}{\gcomp 1 2;v_2}}		{\red {e_1}{\xi_1;\pure{\inl{v}}} 
																					&\red [\Gamma,\dom{\xi_1}]{[\pure v/x_2]e_2}{\xi_2;v_2}}			\and
\inferdiaone [+ E_2]	{\red {\caseP{e_1}{x_2.e_2}{x_3.e_3}}{\gcomp 1 3;v_3}}		{\red {e_1}{\xi_1;\pure{\inr{v}}}
																					&\red [\Gamma,\dom{\xi_1}]{[\pure v/x_3]e_3}{\xi_3;v_3}}		
\end{mathpar}

\end{abstrsyn}
\caption{\lang~Dynamic Semantics.}
\label{fig:diaSemantics}
\end{figure*}


%% The dynamics for \lang\ implements the stage \bbone\ evaluation algorithm
%% described above using three judgments: $\redonesym$ (evaluation), $\redtwosym$
%% (specialization), and the auxiliary $\reifysym$ (reification). 
%% Along with these are three judgments identifying forms of values:
%% $\pvalsym$ (partial values), $\ressym$ (residuals), 
%% and the auxiliary $\fcon$ (residual table).
%% These judgments are related as follows:
%% \begin{itemize}
%% \item Evaluation sends a stage~\bbone\ term to a partial value.
%% In order to avoid duplicating subterms, evaluation collects explicit
%% substitutions into a single {\em residual table} that is also the output of
%% evaluation.
%% \item Specialization sends a stage~\bbtwo\ term to a residual.
%% \item Reification sends a single residual and a residual table to another residual.  
%% \item Evaluation depends on specialization for $\next$ terms, which have
%% stage~\bbtwo\ subterms.
%% \item Specialization depends on evaluation for $\prev$ and $\pause$, which have
%% stage~\bbone\ subterms. The residual table produced by evaluation is also
%% reified here.
%% \end{itemize}

%% \emph{Partial values} are stage~\bbone\ terms that have been fully
%% evaluated, but which may contain stage~\bbtwo\ variables wrapped in $\next$
%% blocks.  In the example above, 
%% \begin{lstlisting} 
%% 1`(next {`2`yhat`1`},7)`
%% \end{lstlisting}
%% is a partial value.
%% \emph{Residuals} ($\ressym$es) are \langTwo\ terms---stage \bbtwo\ terms whose
%% stage \bbone\ subexpressions have all been fully evaluated. In the example
%% above,
%% \begin{lstlisting} 
%% 2`let zhat = yhat*yhat*7 in zhat`
%% \end{lstlisting}
%% is a residual.

%% Note that partial values and residuals are both terms for which stage \bbone\
%% computation has completed; however, partial values are stage~\bbone\ terms,
%% while residuals are stage~\bbtwo\ terms.

%% The $\redonesym$ judgment, defined in \ref{fig:diaSemantics}, takes an stage
%% \bbone\ term%
%% \footnote{ The input to $\redonesym$ may be open on stage \bbtwo\ variables, but
%% the type system ensures that those must occur under a $\next$. Consequently
%% $\redonesym$ will never directly encounter a variable.}
%% %
%% to a {\em residual table} $\xi$ and a partial value $v$. The residual
%% table implements the explicit substitutions in our example---it maps fresh
%% stage \bbtwo\ variables (which may appear inside $\next$ blocks in $v$) to
%% residuals. For non-staging features of \lang, $\redonesym$ is essentially
%% standard call-by-value evaluation, and gathers the subterms' residual tables
%% into a single one.

%% \crem{Can we cut \ref{fig:diaSemanticsSpec} from the paper? Perhaps just put a
%% few representative rules into \ref{fig:diaSemantics}?}

%% At $\next~e$, in contrast, $\redonesym$ specializes into $e$ for stage~\bbone\ 
%% subexpressions to evaluate.  This is implemented by the $\redtwosym$ judgment,
%% which takes a stage \bbtwo\ term to a residual. 
%% For the most part (\ref{fig:diaSemanticsSpec}), specialization simply recurs into
%% all subexpressions; at $\prev$, however, it resumes $\redonesym$ evaluation,
%% which produces a residual table $\xi$ and (by canonical forms) an expression
%% $\next\{\hat y\}$. $\xi$ is then reified using $\reifysym$ into a series of let
%% bindings enclosing $\mathtt{\hat y}$.
%% Once specialization reduces $e$ to a residual $q$, the output of $\redonesym$ is a
%% fresh variable wrapped in a $\next$ block ($\next~\hat y$), along with a residual
%% table which maps that variable to the residual ($\hat y \mapsto q$).

%% Within specialization lies a subtle---if perhaps unintuitive---feature.  
%% Observe that specialization will traverse into both branches of a stage-\bbtwo\ {\tt if} or {\tt case} 
%% statement in its search for stage~\bbone\ code. 
%% Thus the evaluation of that stage~\bbone\ code will occur {\em regardless of the eventual value of the predicate},
%% and so a term like 
%% \begin{lstlisting} 
%% 1`next{
%%   `2`if true 
%%   then hold{`1`3+4`2`} 
%%   else prev{`1`spin() (* loops forever *)`2`}`1`
%% }`
%% \end{lstlisting}
%% will fail to evaluate at stage \bbone.
%% Although it may seem undesirable, this behavior is a deliberate feature.
%% Indeed it is critical to the quickselect example, 
%% wherein all three branches of the $n$-$k$ comparison are evaluated at stage~\bbone.
%% As always, it is the programmer's responsibility to ensure that her staging annotations 
%% do not cause a program to loop forever.  

%% The context ($\Gamma$) keeps track of stage \bbtwo\ variables in the input term. 
%% These both appear in the original program at stage \bbtwo\ and are inserted by the semantics.

%% As an optimization, we can include the special-case rule,
%% \begin{mathpar}
%% \inferdiaone [hat] {\red {\next~\hat y}{\cdot,\next~\hat y}}{\cdot}
%% \end{mathpar}
%% to avoid one-for-one variable bindings in the residual.

%% \subsubsection {Top-Level Evaluation}
%% \label{sec:partialeval}

%% % KAYVONF: good statement, but hold out for now
%% %The hope of partial evaluation is that $f_x$ is cheaper to execute than $f$, meaning that we can save work if we must %evaluate it many times.

%% These dynamics allow us to define a partial evaluator for \lang\ by identifying
%% static with stage \bbone\ and dynamic with stage \bbtwo.  Specifically, we
%% encode $f$ as a \lang\ expression with a function type of the form
%% $A\to\fut(B\to C)$.%
%% %\cprotect\footnote{We can rewrite \texttt{fexp} in this form, or simply apply
%% %the following higher-order function which makes the adjustment:
%% %\begin{lstlisting} 
%% %let adjust (f : $int * int -> $int) =
%% %  fn (p : int) => 
%% %    next{
%% %      fn (b : int) => 
%% %        prev{f (next {b}, p)}
%% %    }
%% %\end{lstlisting}}
%% %
%% Here $A$ is the static input, $B$ is the dynamic input, and $C$ is the output.

%% Once a stage \bbone\ argument $a:A$ is provided, we can evaluate the partially-applied
%% function:
%% $\cdot\vdash f~a \mathop{\redonesym} [\xi,v]$.
%% The result is an environment $\xi$ and a partial value $v$ of type $\fut(B\to
%% C)$, which by canonical forms must have the form $v = \next~\hat y$. 
%% Next, we reify this environment into a sequence of \verb|let|-bindings
%% enclosing $\hat y$, via $\reify\xi{\hat y}{f_a}$. 
%% Because reification preserves types, the resulting residual $f_a$ has type $B\to C$ in \langTwo, so we can apply it to some $b:B$
%% and compute the final result of the function, $f_a~b \mathop{\tworedsym} c$.

%% That this sequence of evaluations is in fact staged follows from our
%% characterizations of partial values and residuals, that $\redonesym$
%% outputs a partial value, and that $\reifysym$ outputs an expression in \langTwo.

%% For example, to specialize quickselect to a particular list, 
%% \begin{lstlisting}
%% qsStaged [5,2,7,4,1]
%% \end{lstlisting}
%% \TODO finish example


%\begin{remark}
%For any $\colone{e}{A}$ containing no $\next$ subexpressions, $\redonesym$ will
%always compute an empty environment, and a partial value identical to the result
%of call-by-value evaluation of $e$.
%%derivationally equivalent to standard call-by-value evaluation.
%\end{remark}

%\subsection{Metatheory}
%
%Recall that residuals live in \langTwo; we will indicate typing judgments in
%\langTwo\ with $\vdash_\bbtwo$.
%
%%\begin{definition}
%%Context $\Gamma$ is well-formed ($\Gamma\wf$) if it
%%contains only stage-2 variables.
%%\end{definition}
%
%\begin{definition}
%An environment $\xi$ is well-formed ($\Gamma\vdash\xi\wf$) if either:
%\begin{enumerate}
%\item $\xi = \cdot$; or
%\item $\xi = \xi',x:B\mapsto e$ where
%$\Gamma\vdash\xi'\wf$ and
%$\typeslangTwo[\Gamma,\dom{\xi'}] e B$
%%$\Gamma,\dom{\xi'}\vdash \coltwo{e}{B}$ and
%%$\Gamma,\dom{\xi'}\vdash e \res$.
%\end{enumerate}
%\end{definition}
%
%\begin{theorem}
%If $\typeswor e A$ then $\Gamma\wf$ and $A\istypewor$.
%\end{theorem}
%
%\begin{theorem}
%If $\diaonesub$ and $\typesone e A$ then
%\begin{enumerate}
%\item $\Gamma\vdash\xi\wf$;
%\item $\Gamma,\dom\xi\vdash \colone{v}{A}$; and
%\item $\Gamma,\dom\xi\vdash v\pval$.
%\end{enumerate}
%\end{theorem}
%
%\begin{theorem}
%If $\diatwosub$ and $\typestwo e A$ then
%\begin{enumerate}
%\item $\typeslangTwo q A$; and
%\item $\Gamma\vdash_\bbtwo q\val$.
%\end{enumerate}
%\end{theorem}
%
%\begin{theorem}\label{thm:reify-type}
%If $\Gamma\vdash\xi\wf$ and
%$\Gamma,\dom\xi\vdash \colone{\next\ \hat y}{\fut A}$
%then 
%$\reify{\xi}{\hat y}{q}$ and
%$\typeslangTwo q A$.
%\end{theorem}

%\TODO
%Note somewhere how to run stage-one non-$\fut A$ terms. For example, a stage-one
%integer term is guaranteed not to depend on the table, although one might be
%produced. One may either discard the table, or evaluate everything in the table
%(and terminating with the partial value iff everything in the table terminates).



%!TEX root = paper.tex

\section{Splitting Algorithm}
\label{sec:splitting}

\begin{abstrsyn}

The primary strength of \lang\ is the expressive power afforded by
its ability to syntactically interleave work intended for stage \bbone\
and work intended for stage \bbtwo, as well as its ability to form
mixed-stage abstractions. 

In this section, we provide a {\em splitting algorithm} to translate the multistage terms of \lang\
into a simpler but equivalent form.
The output of splitting is simpler because it is {\em syntactically separated}; that is, 
stage \bbone\ and stage \bbtwo\ are represented by distinct subterms and are only connected at the top syntactic level.
The output of splitting is equivalent (to its input) because it can be evaluated to the same residual 
that would be produced by the semantics of \ref{sec:semantics}.

More precisely, for any $\coltwo e A$ which reduces to a residual $q$ (via $\diatwo e q$),
splitting translates $e$ into some syntactically separated program $\mathcal{P}$ 
(via the new relation $e \splittwosym \mathcal{P}$), which
can also be reduced to $q$ (via the new relation $\mathcal{P} \sepredtwosym q$).
We enforce the ``syntactically separated'' condition by saying that $\mathcal{P}$ 
can only have one form, namely $\pipeS p l r$, where  
$p$ (called the {\em precomputation}) is a monostage term encoding all the stage \bbone\ computation that was in $e$, 
and $l.r$ (called the {\em resumer}) is a monostage term encoding all the stage \bbtwo\ computation that was in $e$.  
Likewise, $\sepredtwosym$ is defined by a single rule, which evaluates the precomputation and plugs it in for $l$:
\[
\infer{\sepredtwo {\pipeS p l r} {[b/l]r}} {\reduce p b}
\]
Here we call $b$ the {\em boundary value}, as it represents the communication at the boundary between stages.
A summary of these operations and the relationship between them is provided on the right side of \ref{fig:splittingSummary}.

Since terms at world \bbtwo\ can depend on terms at \bbonem\ (via \texttt{prev}),
we also provide a way to split those as well.
This is given by the new relation $e \splitonesym \mathcal{P}$,
where $\mathcal{P}$ must have the form $\pipeM c l r$ where  
$c$ (called the {\em combined term}) is a monostage term encoding all the stage \bbone\ computation that was in $e$, 
and $l.r$ (again called the {\em resumer}) is a monostage term encoding all the stage \bbtwo\ computation that was in $e$.

Similar to above, we also define a $\sepredtwosym$ relation which can be used to evaluate \texttt{sepM} terms,
with one complication arising from the fact that \lang\ terms at world \bbonem\ produces multistage output.
For example, consider the term
\begin{lstlisting}
(next {1+2}, pure{3+4})
\end{lstlisting}
which reduces (via $\redonesym$) to the residual table and partial value
\begin{lstlisting}
[yhat |-> 1+2] (next {yhat}, pure{7})
\end{lstlisting}
Since this output has interleaved stages, so theres no way a syntactically seperated term could directly reduce to it.

We resolve this mismatch by introducing one more translation, called {\em masking} and written $[\xi;v] \vsplito \mathcal{V}$,
which converts residual tables and partial values like the those above into a syntactically seperated version $\mathcal{V}$.
As before, we enforce the ``syntactically seperated'' property structurally,
saying that $\mathcal{V}$ must have the form $\mval i q$ (called a {\em masked value}) where 
$i$ is a monostage value respresenting the stage \bbone\ part of $v$,
and $q$ is a monostage term representing all of the stage \bbtwo\ computation in $\xi$ and $v$.

With this machinery in place, we can define $\sepredtwosym$ as just creating masked values with the single rule:
\[
\infer{\sepredone {\pipeM c l r} {\mval i {[b/l]r}}} {\reduce c {(i,b)}}
\]
Here we see the primary difference between terms at world \bbtwo\ and world \bbone.
Since $\coltwo e A$ reduces to an entirely stage \bbtwo\ result, 
its stage \bbone\ subcomputations only exist to internally pass their value to stage \bbtwo;
but since $\colmix e A$ reduces to a multistage result, 
its stage \bbone\ subcomputations exist {\em both} to pass their value to stage \bbtwo\ {\em and} to be present in the result.
Correspondingly, in the $\sepredtwosym$ rule, the precomptuation $p$ reduces only to a boundary value which is passed to the resumer;
but in the $\sepredonesym$ rule, the combined term $c$ reduces to a tuple containing both the immediate result and the boundary value,
with only the latter being passed to the resumer.

A summary of these operations and the relationship between them is provided on the left side of \ref{fig:splittingSummary}.

This section proceeds by defining masking ($\vsplito$), 
then using masking to motivate the definition of \bbonem-translation ($\splitonesym$),
and finally coming back to define \bbtwo-translation ($\splittwosym$).

\subsection{Masking}

The point of the masking operation is simply to convert a partial value into a masked value $\mval i q$
by assigning all of the stage \bbone\ content to $i$ and all of the stage \bbtwo\ content to $q$.
The rules of the masking relation are given in \ref{fig:valMask}.

Masking operates by first inducting on the entries of the residual table.  
Being purely stage~\bbtwo\ content, these are reified into let statements at the top of the resumer.
Once the table is empty, masking inducts on value itself.

Masking assigns purely stage~\bbone\ values to the immediate value
and likewise assign references into the residual table to the resumer.
In both cases, the offside component is assigned to \texttt{()}, to represent trivial information.

Masking distributes into tuples, injections, and rolls, since their subvalues may have content at both stages.
However, the tags of injections and rolls are replicated only in the immediate value, 
since they represent stage~\bbone\ information.

Since lambdas may represent multi-stage computations, 
masking splits the body of lambdas as general stage~\bbone\ terms (as described in \ref{sec:split-one}), 
and packages the resulting terms as functions.

\subsection{Term Splitting at \bbonem}
\label{sec:split-one}

We now show how to translate terms $\colmix e A$ into the form $\pipeM c l r$,
pursuant to the correctness condition given in \ref{fig:splittingSummary}.
The algorithm is specified by the $\splitonesym$ relation (\cref{fig:termSplit}), 
which proceeds recursively on the structure of~$e$.

When $e$ is a terminal (a variable or unit)
splitting yields a combined term formed by tupling $e$ with a \texttt{()} precomputation, 
and the trivial resumer \texttt{()}. 
(Stage~\bbone\ terminals, by definition, contain no stage~\bbtwo\ subcomputations.)  
For example, the integer constant \texttt{3} splits into the combined term \texttt{(3,())} and resumer \texttt{\_=>()}.

For all non-terminals (except \texttt{next}),
splitting descends into $e$, recursively splitting its $n$ subterms
to produce their respective combined terms $c_1,\ldots,c_n$ and resumers $r_1, \ldots, r_n$.
The combined term of $e$ is formed by binding $c_1,\ldots,c_n$
to the patterns $(y_1,z_1),\ldots,(y_n,z_n)$
to isolate immediate results from boundary values. Then,
the immediate result of $e$ is formed by replacing $e$'s subterms with $y_1,\ldots,y_n$.
The resumer binds the boundary values $b_1,\ldots,b_n$ to an
argument $(l_1,\ldots,l_n)$ in a term that has the same structure
of~$e$ but where each subterm is replaced by its resumer ($r_i$'s).

Splitting {\tt case} yields a combined term that executes one of the branches' combined terms based on the immediate result $y_1$ of the predicate.
The boundary value $b_i$ for this branch is injected and bundled with that of the predicate ($b_1$).   
$b_i$ is cased in the resumer to determine which branch's resumer should be executed.
{\tt case} and \texttt{hold} are the only two rules where splitting adds non-trivial logic is added to the precomputation.

Function introduction has a \texttt{()} boundary value,
since functions are already fully reduced in our semantics.
However, since the body of a function may itself be multi-stage, splitting must continue into it.
The immediate result is a new function formed from the stage~\bbone\ part of the original body.
The resumer is a new function formed out of the stage~\bbtwo\ part of the original body.
It is the responsibility of the application site to save the precomputation of the function body
and pass it to the resumer version of the function.

Since the results of splitting \texttt{next} terms depend on the output of splitting its stage~\bbtwo\ subterm,
we defer description of \texttt{next} until after describing stage~\bbtwo\ term splitting.

\subsection{Term Splitting at \bbtwo}

Because stage~\bbtwo\ terms in \lang\ reduce to monostage residuals (as opposed to partial values),
term splitting at \bbtwo\ assumes a simpler form than the version at \bbonem\ does. 
The algorithm is specified by the $\splittwosym$ relation in \cref{fig:termSplit}.

In the terminal cases of
constants and variables, splitting generates trivial precomputations that are \texttt{()}, and resumers consisting of the original term.
For example, the integer constant \texttt{3} splits into the
precomputation \texttt{()} and resumer \texttt{\_=>3}.

More generally, for all (except \texttt{prev} and \texttt{hold}) 
$n$-ary terms $e = \mathcal{C}\ttlpar e_1 \ttsemi \ldots \ttsemi e_n \ttrpar$ 
the precomputation is the tupled precomputations of $e$'s $n$ subterms:
$p=(p_1,\ldots,p_n)$.  The resumer binds each boundary value to an
argument $(l_1,\ldots,l_n)$ in a term that has the same structure
of~$e$ but where each subterm is replaced by its corresponding resumer:
$r = \mathcal{C}\ttlpar r_1 \ttsemi \ldots \ttsemi r_n \ttrpar$ .
Notably, at \texttt{case}s and functions the
precomputation of subterms is lifted out from underneath stage \bbtwo\ binders.  
% TODO: should probably draw a parallel to the same behavior in dynamics

Splitting \texttt{prev} generates a precomputation that projects the immediate result of its stage~\bbone\ subterm.
Since the argument to \texttt{prev} is of $\fut$ type, its immediate result reduces to $\tup{}$, justifying why it can be thrown away.
\texttt{hold} treats the entire combined term of its stage~\bbone\ subexpression as a precomputation, 
and projects out the integer result in the resumer. 
\footnote{The resumer of an integer expression is usually trivial, 
but we have to include it here for termination purposes.} 
Finally, splitting \texttt{next} simply tuples up the precomputation of its stage~\bbtwo\ subterm with a trivial immediate result $\tup{}$.

\subsection {Three World System?}

Why do we need a three world system?

\end{abstrsyn}

%!TEX root = ../paper.tex

\begin{abstrsyn}
\begin{figure}[t]
\begin{mathpar}
\infer {x 					\vsplito \mval x x}														{\cdot}												\and
\infer {\tup{} 				\vsplito \mval {\tup{}} {\tup{}}}										{\cdot}												\and
\infer {\pure m 			\vsplito \mval m {\tup{}}}												{\cdot}												\and
\infer {\next y 			\vsplito \mval {\tup{}} y}												{\cdot}												\and
\infer {\tup{v_1,v_2} 		\vsplito \mval {\tup{i_1,i_2}} {\tup{q_1,q_2}}}							{v_1 \vsplito \mval {i_1} {q_1} 
																									&v_2 \vsplito \mval {i_2} {q_2}}					\and
\infer {\scont v 			\vsplito \mval {\scont i} q}											{v \vsplito \mval i q
																									& \scont \dash \in \{ 
																										\inl \dash , 
																										\inr \dash,
																										\roll \dash 
																									\}}								\and
\infer {\fix f x e			\vsplito \mval {\fix f x {\letin {\tup{x,y}} c {\tup{x,\roll y}}}} 
							{\fix f {\tup{x,\roll l}} r}}											{\splitone e A c {l.r}}								\and
\end{mathpar}
\caption{Value splitting rules.}
\label{fig:valueSplit}
\end{figure}

\begin{figure}
\begin{mathpar}
\infersplitone 					{\spl {\exv v} A {\exv{\tup{i,\tup{}}}} {\_.\exv q}} {v \vsplito \mval i q} \and 
\infersplitone [common2]		{\spl {\scont e}{A}{\letin{\tup{y,z}}{c}
									{\tup{\scont {\exv y},\exv z}}}{l. r}}															{\sub {} {A} 
																																	& \scont \dash \in \{ 
																																	\inl \dash , 
																																	\inr \dash , 
																																	\roll \dash,
																																	\unroll \dash
																																	\}} 		\and
\infersplitone [\fut I]			{\spl {\next e}{\fut A}{\tup{\tup{},p}}{l.r}}														{\splittwosub {} A} 			\and
\infersplitone [common1]		{\spl {\scont e}{A}{\letin{\tup{y,z}}{c}
									{\tup{\scont {\exv y},\exv z}}}{l.\scont r}}													{\sub {} {A} 
																																	& \scont \dash \in \{ 
																																	\pio \dash , 
																																	\pit \dash
																																	\}} 		\and
\infersplitone [\times I] 		{\splitonetall {\tup{e_1,e_2}}{A\times B}
									{\left(
										\talllet{\tup{y_1,z_1}}{c_1}{
										\talllet{\tup{y_2,z_2}}{c_2}{
										\exv{\tup{\tup{y_1, y_2},\tup{z_1, z_2}}}\ttrpar\ttrpar}}
									\right)}
									{\tup{l_1,l_2}.\tup{r_1,r_2}}}																	{\sub 1 A & \sub 2 B} \and
\infersplitone [\to E]			{\splitoneTall {\app {e_1}{e_2}}{B}{\left(
									\talllet{\tup{y_1,z_1}}{c_1}{
									\talllet{\tup{y_2,z_2}}{c_2}{
									\talllet{\tup{y_3,z_3}}{\app{y_1}{y_2}}{\exv{\tup{y_3,\tup{z_1,z_2,z_3}}}\ttrpar\ttrpar\ttrpar}}}
									\right)}
									{\tup{l_1,l_2,l_3}.\app{\exv {r_1}}{\exv{\tup{r_2,l_3}}}}}										{\sub 1 {A \to B} & \sub 2 A}							\and
\infersplitone				{\spl {\pure e} {\curr A} {\tup{e,\tup{}}}{\_.\tup{}}}								{\cdot}							\and
\infersplitone				{\splitoneTall {\letp x{e_1}{e_2}} {?} 
							{\letin {\tup{x,z_1}} {c_1} {
							 \letin {\tup{y_2,z_2}} {c_2} {\tup{y_2,\tup{z_1,z_2}}}}}
							 {\tup{l_1,l_2}.\letin{\_}{r_1}{r_2}}}												{\sub 1 ? & \sub 2 ?}			\and
\infersplitone [+ E]		{\splitoneTall {\caseP{e_1} {x_2.e_2} {x_3.e_3}}{C}
								{\left(
								\talllet{\tup{y_1,z_1}}{c_1}{
									\tallcase{y_1}
									{x_2.\letin{\tup{y_2,z_2}}{c_2}{\exv {\tup{y_2,\tup{z_1,\inl{z_2}}}}}}
									{x_3.\letin{\tup{y_3,z_3}}{c_3}{\exv {\tup{y_3,\tup{z_1,\inr{z_3}}}}}\ttrpar}
								}\right)}
								{\tup{l_1,l_b}.{\ttlpar r_1 \ttsemi \caseof{\exv{l_b}}{l_2.[\tup{}/x_2]{r_2}}{l_3.[\tup{}/x_3]{r_3}\ttrpar}
								}}}																									{\sub 1 {A+B} 
																																	& \sub [\Gamma,\col{x_2} A] 2 C 	
																																	& \sub [\Gamma,\col{x_3} B] 3 C} 		
\end{mathpar}
\hrule
\begin{mathpar}
\infersplittwo 					{\spl {\exv q} A {\exv {\tup{}}} \_ q} 														{\cdot} 						\and 
\infersplittwo [\fut E]			{\spl {\prev e}{A} {\pit c} l r }															{\splitonesub {} {\fut A}} 		\and 
\infersplittwo [\times E_1]		{\spl {\scont e}{A}{p}{l}{\scont r}}														{\sub {} {A\times B} 
																															& \scont \dash \in \{ 
																																\pio \dash , 
																																\pit \dash,  
																																\inl \dash,  
																																\inr \dash,  
																																\roll \dash, 
																																\unroll \dash, 
																																\fix fx \dash 
																															\}} 				\and  
\infersplittwo [\to E]			{\spl {\scont{e_1,e_2}}{B}{\tup{p_1,p_2}}{\tup{l_1,l_2}}{\scont{r_1,r_2}}}				{\sub 1 {A \to B} & \sub 2 A
																															& \scont {\dash, \dash} \in \{ 
																																\tup {\dash,\dash} , 
																																\app \dash \dash,
																																\letin x \dash \dash 
																															\}}									\and			
\infersplittwo [+ E]			{\splittwo {\caseof{e_1}{x_2.e_2}{x_3.e_3}}{C}
								{\tup{p_1,p_2,p_3}}{ \tup{l_1,l_2,l_3}}{\caseof{r_1}{x_2.r_2}{x_3.r_3}}}					{\sub 1 {A+B} & \sub [\Gamma,\col{x_2} A] 2 C 
																															& \sub [\Gamma,\col{x_3} B] 3 C} 								
\end{mathpar} 
\caption{Splitting rules for terms at \bbonem\ and \bbtwo.}
\label{fig:termSplit}
\end{figure}
\end{abstrsyn}


%!TEX root = ../paper.tex

\begin{figure*}
\begin{abstrsyn}
\begin{mathpar}
\infer {\rtab {\hat y \mapsto q,\xi} v	\vsplito \mval i {\letin {\hat y} q r}}				{\rtab \xi v \vsplito \mval i r}					\and
\infer {\rtab \cdot {\tup{}} 			\vsplito \mval {\tup{}} {\tup{}}}					{\cdot}												\and
\infer {\rtab \cdot {\pure m} 			\vsplito \mval m {\tup{}}}							{\cdot}												\and
\infer {\rtab \cdot {\next{\hat y}} 	\vsplito \mval {\tup{}} {\hat y}}					{\cdot}												\and
\infer {\rtab \cdot {\tup{v_1,v_2}} 	\vsplito \mval {\tup{i_1,i_2}} {\tup{r_1,r_2}}}		{\rtab \cdot {v_1} \vsplito \mval {i_1} {r_1} 
																							&\rtab \cdot {v_2} \vsplito \mval {i_2} {r_2}}		\and
\infer {\rtab \cdot {\scriptCapp v} 	\vsplito \mval {\scriptCapp i} r}					{\rtab \cdot v \vsplito \mval i r
																							&\scriptC \in 
																							\{\mathtt{inl},\mathtt{inr},\mathtt{roll}\}}		\and
\infer {\rtab \cdot {\lam x e}			\vsplito \mval {\lam x c} {\lam {\tup{x,l}} r}}		{\splitone e A c {l.r}}								\and
\end{mathpar}
\end{abstrsyn}
\caption{Masking separates a residual table and its associated partial value into its first- and second-stage components.}
\label{fig:valMask}
\end{figure*}



\special{papersize=8.5in,11in}
\setlength{\pdfpageheight}{\paperheight}
\setlength{\pdfpagewidth}{\paperwidth}

\conferenceinfo{CONF 'yy}{Month d--d, 20yy, City, ST, Country} 
\copyrightyear{20yy} 
\copyrightdata{978-1-nnnn-nnnn-n/yy/mm} 
\doi{nnnnnnn.nnnnnnn}

% Uncomment one of the following two, if you are not going for the 
% traditional copyright transfer agreement.

%\exclusivelicense                % ACM gets exclusive license to publish, 
                                  % you retain copyright

%\permissiontopublish             % ACM gets nonexclusive license to publish
                                  % (paid open-access papers, 
                                  % short abstracts)

\titlebanner{banner above paper title}        % These are ignored unless
\preprintfooter{short description of paper}   % 'preprint' option specified.

\title{Stage-Splitting a Modal Language}

\authorinfo{Name1}
           {Affiliation1}
           {Email1}
\authorinfo{Name2\and Name3}
           {Affiliation2/3}
           {Email2/3}

\maketitle

\begin{abstract}
This is the text of the abstract.
\end{abstract}

\category{CR-number}{subcategory}{third-level}

% general terms are not compulsory anymore, 
% you may leave them out
\terms
term1, term2

\keywords
keyword1, keyword2

\section{Introduction}

Staged computation has existed as a programming technique for over three decades.  Jorring et al. (\cite{jorring86}) identify three classes of staging techniques: meta-programming, partial evaluation, and stage-splitting.  The first two of these have received significantly more attention than the third.

We call these techniques {\em staged} in that part of their input comes in the first stage, and part comes in the second stage.  Likewise, some computation occurs at each stage as well.

Countless meta-programming systems exist (...twelve thousand citations...\cite{devito13}), and their background theory and type-systems are well understood (\cite{davies01}).  Partial evaluation, too, is well-understood.  Partial evaluation systems exist... . This paper explores both theory and applications for stage-splitting.  

\section{Stage-Splitting Definition and Comparison to Partial Evaluation}

First, we review the definition of partial evaluation.  Informally, a partial evaluator takes the code for a function $f$, as well as the first-stage input $x$ to that function, and produces the code for a version of that function {\em specialized} to the first input, often called $f_x$.  This $f_x$ function can then be evaluated with the second-stage input to produce the same final answer that $f$ would have.  The goal of the process is that $f_x$ should be cheaper to evaluate than $f$, although this can't be guaranteed for all inputs.  We now state this theorem more formally: a partial evaluator is some function $p$ such that,
\[
	\forall f,x. \exists f_x. [p(f,x) = f_x \text{ and } \forall y.\llbracket f \rrbracket(x,y)=\llbracket f_x \rrbracket(y)]
\]
where (borrowing notation from \cite{jones96}), $\llbracket f \rrbracket$ means the mathematical function corresponding to the code given by $f$.

Informally, we define stage-splitting to be the process of taking some function $f$ into two other functions, $f_1$ and $f_2$, where $f_1$ computes a partial result from the first-stage input, and $f_2$ uses that partial result and the second-stage input to compute a final result which is the same as if we had just run the original $f$ on both inputs.  Again, more formally, a stage-splitter is some $s$ such that,
\[
	\forall f. \exists f_1,f_2. [s(f) = (f_1,f_2) \text{ and } 
	\forall x,y.\llbracket f \rrbracket(x,y)=\llbracket f_2 \rrbracket(\llbracket f_1 \rrbracket(x),y)]
\]

We first discuss a few similarities between partial-evaluation and stage-splitting.  First off, both techniques have the same form of input, namely a bivariate function where the first input comes at stage one, and the second input comes at stage two.  

Again in both cases, the governing equations are too weak to fully determine the definitions of $p$ and $s$.  Indeed, both admit completely trivial definitions.  Consider the stage-splitter which always returns the identity for $f_1$ and $f$ for $f_2$, or analogously the partial evaluator which always returns an $f_x$ that just closes over the input $x$ and internally calls $f$ once $y$ is available. The ambiguity of these equations (modulo standard program equivalence of the outputs) can be resolved by adding annotations to $f$ to clearly specify the parts of the computation that are first stage and the parts that are second stage.  Later, we show that the same annotations suffice for both partial evaluation and stage-splitting.  

The differences between stage-splitting and partial evaluation are likewise evident from these governing equations.  For instance in partial evaluation, the existential $f_x$ depends on $x$, which means that the partial evaluator cannot be run until $x$ is known.  Moreover, if one wishes to specialize $f$ for multiple $x$'s, then the partial evaluator must be run several times.  Depending on the use case and cost of partial evaluation, this may be prohibitively expensive.  Alternatively, a stage-splitter need only be run once, and this can be done entirely before any $x$ is known.

\subsection{Partial Evaluator from Stage-Splitter}

We can recover a valid partial evaluator from a stage-splitter by stage-splitting the input function $f$ into $f_1$ and $f_2$, computing $\llbracket f_1 \rrbracket(x)$ to obtain $\bar x$, and then returning an $f_x$ such that $\llbracket f_x \rrbracket(y) = \llbracket f_2 \rrbracket(\bar x, y)$.  Note that this does not mean that stage-splitting is a strict generalization of partial evaluation.  In practice, partial evaluators easily perform optimizations (such as branch elimination, discussed later) which are beyond the scope of stage-splitting, and would require further technology than has been developed here.  It is best to think of stage-splitting as simply the first half of partial evaluation, where the back half is an optimizer. [Might be able to come up with a futamura projection-like statement here, which would be really really really cool.]

\subsection{Stage-Splitter from Partial Evaluator}

Likewise, we can easily recover a stage-splitter from a partial evaluator.  If $p$ is a valid partial evaluator, then we can define a stage splitter $s$ such that $s(f)=(f_1,f_2)$, where
\begin{align*}
[f_1](x) &= p (f,x) \\
[f_2](l,y) &= [l] (y)
\end{align*}
This implicitly requires that the languages in which $f_1$ and $f_2$ are expressed are strong enough to write a partial evaluator, but that is the case in this paper.  A stage-splitter defined this way leaves much to be desired.  Firstly, partial evaluation of $f$ may be too expensive for the context in which $f_1$ needs to run.  Additionally, the intermediate data structure created this way may be much larger than necessary, as it would contain all of the residual code.

\section{A Two-Stage Modal Language}

Introduce the next and prev concepts, along with typesystem.  Introduce binding time analysis here, and explain that we don't care about it.  Show some examples.  Introduce a hold operation.

\section{Diagonal Semantics}

We desire a semantics that meets the following goals:

\begin{enumerate}
\item In absence of $\next$ expressions, it should be equivalent to standard call-by-value evaluation.
\item $\next$ and $\prev$ should be inverses of the other.
\item All of the first stage work should be completed before second stage work.
\item The result for a stage-2 expression should be the same as if all of the $\next$s and $\prev$s were interpreted as the identity function.
\end{enumerate}

In this section, we introduce a dynamic semantics for our language.  Although our type system is virtually identical to that in \cite{davies96}, the semantics is different in terms of cost and termination behavior.  The semantics comprises the following judgments:

\begin{center}
\begin{tabular}{|c|p{2.5cm}|p{2.5cm}|} \hline
Judgment & Reads & Conditions \\ \hline 
$\isvalone e$ & ``$e$ is a $\bbone$-value under context $\Xi$'' & ... \\ \hline 
$\Xi \vdash \diaonesub$ & ``$e$ evaluates to future table $\gamma$ and value $v$'' 
& $\typesone e A$ 
	\newline $\isvalone v$ 
	\newline ... \\ \hline 
$\diatwosub$ & ``$e$ speculates to $q$'' & ... \\ \hline 
...&...&... \\ \hline
\end{tabular}
\end{center}

Evaluation and speculation are the main judgments here, the rest being largely administrative.  The evaluation judgment operates on stage-1 code whereas the speculation operates on stage-2 code.  Since $\next$ and $\prev$ are the crossover points between 1-code and 2-code, they are correspondingly the only places where the evaluation and speculation judgments depend on the other. 

We can think of our evaluation as proceeding in the standard way for stage-1 code. But when the evaluator encounters stage-2 code, it places that code off to the side in a table, and then keeps a reference to the table entry.  These references are then passed around as stage-1 values, and the tables are managed accordingly.

The evaluation judgment is very similar to standard call-by-value evaluation, as goal 1 would suggest.  The input to evaluation is a stage-1 expression (usually $e$), as well as two administrative contexts ($\Gamma$ and $\Xi$), covered later.  Evaluation has two outputs: the {\em future table} (usually $\xi$) and the {\em partial value} (usually $v$).  We cover the latter first.  The partial value is essentially the result of the first-stage portion of $e$, and must be a 1-value of the same type as $e$.  As usual, this means that $v$ must be composed only of base primitives, tuples, injections, and lambdas (corresponding to base types, products, sums, and functions).  Analogously, the value corresponding to $\fut$ is a construct called a {\em hatted variable} (written $\hat y$), which signifies a reference to some stage-2 computation.  Those stage-2 computations are held in

\subsection{Adherence to Goals}

We meet our first (discuss) and second goals (show examples).

That we meet our third goal is clear from just the structure of our judgments (...).

Under the strictest interpretation, we do not meet our third goal.  Consider the following examples (show speculating down an if).

But as it turns out, a semantics which had the erasure behavior would not meet goal 3! (explain using the same example)

\section{Splitting Algorithm}

[Present the splitting judgement.  Give statements of type and value correctness for splitting.  Give all of the splitting rules.  Talk through a few of them.]

The theorem that we want to be true is...

%\begin{center}
%\begin{tabular}{l}
%If $\splittwo{e}{A}{p,l.r}$ \\
%and $\reducetwo{e}{v}$ \\
%then $\reduce{p}{u}$ \\
%and $\reduce{[u/l]r}{w}$ \\
%and $v = w$ 
%\end{tabular}
%\end{center}

%\begin{center}
%\begin{tabular}{l}
%If $\splitone {e}{A}{c,l.r}$ \\
%and $\colone {\gamma_1}{\Gamma}$ \\
%and $\coltwo {\gamma_2}{\Gamma}$\\
%and $\rsone{\gamma_1(e)}{v}{q}$ \\
%and $\reducetwo{\gamma_2(q)}{w}$ \\
%then $\reduce{|\gamma_1|(c)}{\valprod{v}{u}}$ \\
%and $\reduce{[u/l]|\gamma_2|(r)}{w}$
%\end{tabular}
%~~~
%\begin{tabular}{l}
%If $\splittwo{e}{B}{p,l.r}$ \\
%and $\colone {\gamma_1}{\Gamma}$ \\
%and $\coltwo {\gamma_2}{\Gamma}$\\
%and $\rstwo{\gamma_1(e)}{q}$ \\
%and $\reducetwo{\gamma_2(q)}{w}$ \\
%then $\reduce{|\gamma_1|(p)}{u}$ \\
%and $\reduce{[u/l]|\gamma_2|(r)}{w}$ 
%\end{tabular}
%\end{center}

\section{Examples for Staged Pipelines}

Give the gist of one-to-one pipeline example (like client/server).
Then talk about a one-to-many pipeline.
Then talk about a many-to-one pipeline like spark.  It clear how to target something like this for known base types on the boundary, and for product types.  But sums on the boundary are hard!  We leave many-to-one as future-work.


\section{Examples of Algorithm Derivation}

Fast exponent example.  

\lstset{language=ML, columns=fullflexible, basicstyle=\ttfamily, tabsize=4, escapeinside={"*}{*"}}
\begin{lstlisting} 
let exp (b : $int, e : int) : $int =
	if e == 0 then
		next{1}
	else if (e mod 2) == 0 then
		next{let x = prev{exp(b,e/2)} in x*x}
	else
		next{prev{b} * prev{exp (b,e-1)}}		
\end{lstlisting}

splits into

\begin{lstlisting} 
let exp (b, e) =
	((), roll (
		if e == 0 then
			inL ()
		else 
			inR (
				if (e mod 2) == 0 then
					inL (#2 (exp (b,e/2)))
				else
					inR (#2 (exp (b,e-1)))
			)
	))
\end{lstlisting}

and

\begin{lstlisting} 
let exp ((b, e), p) =
	case unroll p of
	  () => 1
	| d  =>
		case d of
		  r => let x = exp ((b,()),r) in x*x
		| r => b * exp ((b,()),r)
\end{lstlisting}

Quickselect example.

\begin{lstlisting} 
let qs (l : "*$\mathtt{\mu \alpha. }$*"() + "*$\mathtt{int*\alpha}$*", i: $int) = 
	case unroll l of
	  () => next {0}
	| (h,t) => 
		let (left,right,n) = partition h t in
		next{
			let n = prev{hold n} in
			case compare prev{i} n of
			  () (*LT*) => prev {qs left i}
			| () (*EQ*) => hold h
			| () (*GT*) => 
				prev {qs right next{prev{i}-n-1}}
		}	
\end{lstlisting}

Things to try: an interpreter which, partially evaluated, does cps or something.

For each of these examples, talk about what partial evaluation would do and why that might be bad.

\section{Related Work}

Our stage-splitting algorithm was first suggested in \cite{jorring86} under the name {\em pass separation}.  They essentially proposed that a function $f$ could be split into two others, $f_1$ and $f_2$, such that $f(x,y)=f_2(f_1(x),y)$.  They did not distinguish binding time analysis from stage splitting, and so pass separation inherits the former's ambiguity.  The main goal of \cite{jorring86} was to motivate pass separation and other staging transformations as a powerful way to think about compilation and optimization.  Accordingly, their approach was entirely informal, with no implementation realized.  Moreover, they predicted that ``the [pass separation] approach will elude full automation for some time."  

Implementations of the stage-splitting algorithm have appeared in the literature exclusively (and coincidentally) in the context of graphics pipelines.  The first of these (\cite{knoblock96}) uses a binding time analysis to separate those parts of graphics shaders that are input-invariant from those which are not, and then uses a stage splitting algorithm to factor that into two shaders, thereby minimizing recomputation.  Their shaders are written in a C-like language with basic arithmetic and if statements.  Although their analysis does not give an explicit account of the type-level behavior of the splitting algorithm, it effectively can synthesize product and sum boundary type.  

Like the previous example, the Spark language (\cite{sparkThesis}) uses staging to minimize recomputation in real-time rendering applications.  But instead of using a binding-time analysis, Spark allows the programmer to manually target stages of the graphics pipeline.  Since the modern graphics pipeline is inherently a many-to-one system, this is difficult to reconcile with sum types on the boundary.  Fortunately, Spark has a set of syntactic restrictions that prevent sum boundary types.  Spark does not clearly identify this conflict, but the authors did note that first-stage if statements were difficult to provide meaning to [need quote].

[RTSL and SH]

[Discuss Yong's recent paper here.  It does some pretty sophisticated binding time analysis, with a somewhat straightforward splitting after that.  They have the same many-to-one use case as Spark, but syntactic restrictions prevent sum types on the boundary, sort of.  If we wanted to faithfully represent their system in ours, we would need some mechanisms for abstraction over stage, which we do not have.]

Davies (\cite{davies96}) explored the connection between linear temporal logic and its corresponding type system [circle](which we adapted into [circle sub 2]), and showed the equivalence between [circle] and existing systems for binding time analysis. That work provided $\beta$ and $\eta$ rules for the next and prev operators, but did not consider a full dynamic semantics for the whole language. Whereas [name of our type system] is appropriate for stage-splitting and partial evaluation, \cite{davies01} provides a similar system, [square], that is appropriate for meta-programming.  The main difference is that terms inside a [prev] operator do not see any stage-2 bindings declared outside of it.  They note that where [circle] corresponds to the non-branching temporal logic, [square] corresponds to the branching version.

[Meta-ML eases off on this restriction but does not (I think?) eliminate it.]

[What's going on with names and necessity?]

[Our work bears a lot of similarity to ML5, which also uses a modal type system.  The difference is that we target stages systems (each stage talks to the next), whereas they target distributed ones (all stages talk to all others). The type systems reflect this directly in the world accessibility relation.  There might be some analogue of stage-splitting in the ML5 work, but I have not yet isolated it (might be buried in CPS conversion).]

\appendix
\section{Appendix Title}

This is the text of the appendix, if you need one.

\acks

Acknowledgments, if needed.

% We recommend abbrvnat bibliography style.

\bibliography{paper}
\bibliographystyle{abbrvnat}


\end{document}
