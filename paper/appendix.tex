\section{Additional examples}

Here we include some additional examples of our splitting algorithm.

\subsection{Trie}

\crem{pending Nico}

\subsection {Mixed Map Combinator}

In \ref{sec:exampleQS}, we suggested that staged quickselect could be applied to one list and many queries.
In this example, we implement that behavior by passing \texttt{qs} to a higher order function.
Such a combinator, which operates over both stages at once, is possible because of the generality of \lang's abstractions.

Note that we have to define a datatype for integer lists at the second stage, which is given by \texttt{list2}.
\begin{lstlisting} 
1`atsignnext{`2`
  datatype list2 = Empty2 | Cons2 of int * list2
`1`}

type qsType = ^list*$`2`int`1`->$`2`int`1`

// map : qsType -> ^list * $`2`list2`1` -> $`2`list2`1`
fun map f (l, q) = 
next {`2`
  let 
  fun m Empty2 = Empty2
    | m (Cons2(h,t)) = Cons2(prev{`1`f (l,next{`2`h`1`})`2`}, m t)
  in m prev{`1`q`2`}
`1`}

val qsMany = map qs`
\end{lstlisting}
The \texttt{map} function splits into the following two functions:
\begin{lstlisting} 
2`datatype list2 = Empty2 | Cons2 of int * list2

`1`fun map1 f = (fn (l,()) => ((), #2 (f (l, ()))), ())
`2`fun map2 (f,()) ((l,q), p : tree) =
  let 
  fun m Empty2 = Empty2
    | m (Cons2(h,t)) = Cons2(f ((l,h), p), m t) 
  in m q`
\end{lstlisting}

As desired, \texttt{qs1} (which is passed to \texttt{map1} via variable \texttt{f}) is executed only once per invocation of \texttt{map},
whereas \texttt{qs2} (which is passed to \texttt{map2} via variable \texttt{f}) is executed once per query.

