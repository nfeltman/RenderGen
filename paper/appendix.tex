\section{Additional examples}

Here we include some additional examples of our splitting algorithm.

\subsection{Trie}

\crem{pending Nico}

\subsection {Mixed Map Combinator}

In \ref{sec:exampleQS}, we suggested that staged quickselect could be applied to one list and many queries.
In this example, we implement that behavior by passing \texttt{qs} to a higher order function.
Such a combinator, which operates over both stages at once, is possible because of the generality of \lang's abstractions.

Note that we have to define a datatype for integer lists at the second stage, which is given by \texttt{list2}.
\begin{lstlisting} 
1`atsignnext{`2`
  datatype list2 = Empty2 | Cons2 of int * list2
`1`}

type qsType = ^list*$`2`int`1`->$`2`int`1`

// map : qsType -> ^list * $`2`list2`1` -> $`2`list2`1`
fun map f (l, q) = 
next {`2`
  let 
  fun m Empty2 = Empty2
    | m (Cons2(h,t)) = Cons2(prev{`1`f (l,next{`2`h`1`})`2`}, m t)
  in m prev{`1`q`2`}
`1`}

val qsMany = map qs`
\end{lstlisting}
The \texttt{map} function splits into the following two functions:
\begin{lstlisting} 
2`datatype list2 = Empty2 | Cons2 of int * list2

`1`fun map1 f = (fn (l,()) => ((), #2 (f (l, ()))), ())
`2`fun map2 (f,()) ((l,q), p : tree) =
  let 
  fun m Empty2 = Empty2
    | m (Cons2(h,t)) = Cons2(f ((l,h), p), m t) 
  in m q`
\end{lstlisting}

As desired, \texttt{qs1} (which is passed to \texttt{map1} via variable \texttt{f}) is executed only once per invocation of \texttt{map},
whereas \texttt{qs2} (which is passed to \texttt{map2} via variable \texttt{f}) is executed once per query.

\subsection{Trie}

Our last example is one which generates a trie intermediate data structure.

The function {\tt exists} below determines whether a query string (s) occurs in a given set of strings (from the alphabet $\{A, B,C\}$).
It solves this problem by partitioning the provided set of strings into three subsets,
one which contains all words that start with $A$,
one which contains all words that start with $B$, and one for $C$.
When partitioning, we also throw away the first letter, since we know it's uniform for the whole subset.
Also while doing this partitioning, we also keep track of whether the empty word occurred at all.
We then look at the query string.  If it's empty, we return whether the empty word was in the set.
Otherwise, we look at the first letter, pick the appropriate subset, 
and recursively search for the rest of our query string in that.

We stage this algorithm by saying that the word set comes at stage \bbone\
and the query string comes at stage \bbtwo.

\begin{lstlisting} 
datatype letter = A | B | C
datatype string = EmptyS | ConsS of letter * string
datatype list = EmptyL | ConsL of string * list
/* helper function */
fun partition (l : list) : (bool * list * list * list) =
 case l of
   EmptyL => (false,EmptyL,EmptyL,EmptyL)
 | ConsL (s,ss) =>
     let (anyEmpty,a,b,c) = partition ss in
     case unroll s of
       EmptyS => (true,a,b,c)
     | ConsS (z,zs) =>
         case z of 
           A => (anyEmpty, ConsL(zs,a), b, c) 
         | B => (anyEmpty, a, ConsL(zs,b), c) 
         | C => (anyEmpty, a, b, ConsL(zs, c))
fun exists ((l,s) : list * $string) : $bool =
 case l of 
   EmptyL => next{false} 
 | ConsL _ => 
   let (anyEmpty, a, b, c) = partition l in
   next {
     case prev{s} of
       EmptyS => prev{
         if anyEmpty then next{true} else next{false}
         }
     | ConsS (z,zs) =>
         case z of
           A => prev{exists (a,next{zs})}
         | B => prev{exists (b,next{zs})} 
         | C => prev{exists (c,next{zs})}
   }
\end{lstlisting}

\noindent
Running the splitting algorithm on this code produces:

\begin{lstlisting} 
datatype _ = EmptyS | ConsS of _
datatype _ = EmptyL | ConsL of _
fun partition l =
 case l of
   EmptyL => (false,EmptyL,EmptyL,EmptyL)
 | ConsL (s,ss) =>
     let (anyEmpty,a,b,c) = partition ss in
     case unroll s of
       EmptyS => (true,a,b,c)
     | ConsS (z,zs) =>
         case z of 
           A => (anyEmpty, ConsL(zs,a), b, c) 
         | B => (anyEmpty, a, ConsL(zs,b), c) 
         | C => (anyEmpty, a, b, ConsL(zs, c))
fun exists1 (l,s) = roll ((), 
 case l of 
   EmptyL => inj1 ()
 | ConsL _ => 
   let (anyEmpty, a, b, c) = partition l in
   inj2 (
   	if anyEmpty then inj1 () else inj2 (), 
   	#2 (exists1 (a,())), 
   	#2 (exists1 (b,())), 
   	#2 (exists1 (c,()))
   )
 )

datatype _ = EmptyS | ConsS of _
fun exists2 ((l,s), p) =
 case unroll p of 
   inj1 () => false
 | inj2 (anyEmpty,p_a,p_b,p_c) => 
     case s of
       EmptyS => 
         case anyEmpty of 
           inj1 () => true 
         | inj2 () => false
     | ConsS (z,zs) =>
         case z of
           A => exists2 (((),zs),p_a)
         | B => exists2 (((),zs),p_b)
         | C => exists2 (((),zs),p_c)
\end{lstlisting}

As promised, this split version creates a trie in the first stage, and consumes it in the second stage.