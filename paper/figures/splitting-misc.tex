%!TEX root = ../paper.tex

\begin{figure}[t]
\begin{abstrsyn}
\begin{mathpar}
%
\infer {e \equiv e} {\cdot} \and
\infer {e' \equiv e} {e \equiv e'} \and
\infer {e \equiv e''} {e \equiv e' & e' \equiv e''} \and
\infer {e \equiv e'} { e \betastepsym \bbonep e' } \and
%
% \infer {\pio {\exv{\tup {v_1,v_2}}} \equiv \exv {v_1}} {\cdot} \and
% \infer {\pit {\exv{\tup {v_1,v_2}}} \equiv \exv {v_2}} {\cdot} \and
% \infer {\unroll {\exv{\roll v}} \equiv \exv v} {\cdot} \and
% \infer {\letin x {\exv v} e \equiv [v/x]e} {\cdot} \and
% \infer {\app {\exv {\fix f x e}} {\exv v} \equiv [\fix f x e, v/f,x]e} {\cdot} \and
% \infer {\caseof {\exv {\inl v}} {x_1.e_1} {x_2.e_2} \equiv [v/x_1]e_1} {\cdot} \and
% \infer {\caseof {\exv {\inr v}} {x_1.e_1} {x_2.e_2} \equiv [v/x_2]e_2} {\cdot} \and
%
\infer {\letin x q {\exv x} \equiv q} {\cdot} \and
\infer {\scont e \equiv \scont {e'}} {e \equiv e' &
	\scont \dash  \in \left \{
	\begin{array}{l}
	\pio \dash ,
	\pit \dash ,
	\inl \dash ,
	\inr \dash ,
	\roll \dash ,
	\unroll \dash ,
	\fix fx\dash , \\
	\tup {\dash,e} ,
	\tup {e,\dash},
	\app \dash e,
	\app e \dash,
	\letin x e \dash,
	\letin x \dash e, \\
	\caseof {\dash} {x_2.e_2} {x_3.e_3},
	\caseof {e_1} {x_2.\dash} {x_3.e_3},
	\caseof {e_1} {x_2.e_2} {x_3.\dash},
	\end{array}
	\right \} } \and
% \infer {\pio e \equiv \pio {e'}} {e \equiv e'} \and
% \infer {\pit e \equiv \pit {e'}} {e \equiv e'} \and
% \infer {\inl e \equiv \inl {e'}} {e \equiv e'} \and
% \infer {\inr e \equiv \inr {e'}} {e \equiv e'} \and
% \infer {\roll e \equiv \roll {e'}} {e \equiv e'} \and
% \infer {\unroll e \equiv \unroll {e'}} {e \equiv e'} \and
% \infer {\fix fxe \equiv \fix fx{e'}} {e \equiv e'} \and
%\infer {\tup{e_1,e_2} \equiv \tup {e_1',e_2'}} {e_1 \equiv e_1' & e_2 \equiv e_2'} \and
%\infer {\app{e_1}{e_2} \equiv \app {e_1'}{e_2'}} {e_1 \equiv e_1' & e_2 \equiv e_2'} \and
%\infer {\letin x {e_1} {e_2} \equiv \letin x {e_1'} {e_2'}} {e_1 \equiv e_1' & e_2 \equiv e_2'} \and
%
\infer {\scont {\letin x q e} \equiv \letin x q {\scont e}} {
	\scont \dash  \in \{
	\pio \dash ,
	\pit \dash ,
	\tup {\dash ,e},
	\tup {\exv v,\dash },
	\app \dash e,
	\app {\exv v} \dash ,
	\caseof \dash  {x_2.e_2} {x_3.e_3} \}
	}
% \infer {\pio {\letin x q e} \equiv \letin x q {\pio e}} {\cdot} \and
% \infer {\pit {\letin x q e} \equiv \letin x q {\pit e}} {\cdot} \and
% \infer {\tup {\letin x q {e_1}, e_2} \equiv \letin x q {\tup {e_1,e_2}}} {\cdot} \and
% \infer {\tup {\exv {v_1}, \letin x q {e_2}} \equiv \letin x q {\tup {\exv {v_1},e_2}}} {\cdot} \and
% \infer {\app {\letin x q {e_1}}{e_2} \equiv \letin x q {\app {e_1}{e_2}}} {\cdot} \and
% \infer {\app {\exv {v_1}}{\letin x q {e_2}} \equiv \letin x q {\app {\exv {v_1}}{e_2}}} {\cdot} \and
% \infer {\caseof {\letin x q e}{x_2.e_2}{x_3.e_3} \equiv \letin x q {\caseof e {x_2.e_2}{x_3.e_3}}} {\cdot}
\end{mathpar}
\end{abstrsyn}
\caption{Monostage equivalence relation, including reduction, congruence, and let-transposition rules. Since the \bbonep\ fragment of \lang\ is monostage, we simply use $e \betastepsym \bbonep e'$ to mean any standard reduction from $e$ to $e'$.}
\label{fig:equiv}
\end{figure}
