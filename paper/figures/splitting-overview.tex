%!TEX root = ../paper.tex

\begin{figure*}
\begin{abstrsyn}

\begin{theorem} \leavevmode
\label{thm:total} 
\begin{itemize} 
\item For term $e$, if $\typestwo e A$, then $\splittwo e A p l r$. 
\item For term $e$, if $\typesone e A$, then $\splitone e A c {l.r}$.
\item For partial value $v$, if $\typesone v A$, then $v \vsplito \mval i q$.
\end{itemize}
\end{theorem}

\begin{theorem} \leavevmode
\label{thm:unique}
\begin{itemize} 
\item If $\splittwo e A p l r$ and $\splittwo e A {p'} {l'} {r'}$, then $p = p'$, and $l.r = l'.r'$.
\item If $\splitone e A p {l.r}$ and $\splitone e A {c'} {l'.r'}$, then $c = c'$, and $l.r = l'.r'$.
\item If $v \vsplito \mval i q$ and $v \vsplito \mval {i'} {q'}$, then $i = i'$ and $q = q'$.
\end{itemize}
\end{theorem}

\begin{theorem}
\label{thm:allCorrect}
If $\typestwo [\cdot] e A$, $e \stepsym \bbtwo \cdots \stepsym \bbtwo \exv q$, and $\splittwo e A p l r$,
then $\reduce p b$ and $[b/l]r \equiv q$.
\end{theorem}

\begin{lemma}
\label{lem:twoCorrect}
If $e \stepsym \bbtwo e'$, $\splittwo e A p  l r$, $\splittwo {e'} A {p'} {l'} {r'}$, and $\reduce {p'} {b'}$,
then $\reduce p b$ and $[b/l]r \equiv [b'/l']r'$.
\end{lemma}

\begin{lemma}
\label{lem:oneCorrect}
If $e \stepsym \bbonem e'$, $\splitone e A p {l.r}$, $\splitone {e'} A {p'} {l'.r'}$, and $\reduce {c'} {\tup{i,b'}}$,
then $\reduce c {\tup {i,b}}$ and $[b/l]r \equiv [b'/l']r'$.
\end{lemma}

\begin{lemma}
\label{lem:oneCorrect}
If $e \stepsym \bbonep e'$ and $\reduce {e'} i$, then $\reduce e i$.
\end{lemma}

% \textbf{Splitting Terms at World \bbtwo\ ---
% $\splittwo {\coltwo e A} A p l r$}
% \begin{enumerate}
% \item[] Precomputation $p$ is a monostaged term containing the first-stage computations in $e$.
% \item[] Resumer $l.r$ is a monostaged term containing the second-stage computations in $e$.
% \end{enumerate}

% \textbf{Splitting Terms at World \bbonem\ ---
% $\splitone {\colmix e A} A c {l.r}$}
% \begin{enumerate}
% \item[] Combined result $c$ is a monostage term containing the first-stage computations in $e$.
% \item[] Resumer $l.r$ is a monostage term containing the second-stage computations in $e$.
% \end{enumerate}

% \textbf{Masking ---
% $\rtab \xi v \vsplito \mval i q$}
% \begin{enumerate}
% \item[] Input $\rtab \xi v$ is the residual table and partial value to be masked.
% \item[] Immediate value $i$ is a monostage value representing the first-stage components of $v$.
% \item[] Residual $q$ is a monostaged term representing the second-stage computations in $\xi$ and $v$.
% \end{enumerate}

% \hrule
% \vspace{1em}

% \centering
% \setlength{\unitlength}{2.8pt}
% \begin{tabular}{c|c}
% {\large \bf Splitting at World \bbonem} & 
% {\large \bf Splitting at World \bbtwo} \\

% \begin{picture} (80,54) (10,30)

% \thicklines
% \put(25,37){\oval(22,10)}
% \put(25,75){\oval(22,10)}
% \put(75,37){\oval(22,10)}
% \put(75,75){\oval(22,10)}
% \put(36,37){\vector(1,0){28}}
% \put(36,75){\vector(1,0){28}}
% \put(25,70){\vector(0,-1){28}}
% \put(75,70){\vector(0,-1){28}}

% \put(14,75){\raisebox{-0.5ex}{\makebox[22 \unitlength]{$\colmix e A$}}}
% \put(14,37){\raisebox{-0.5ex}{\makebox[22 \unitlength]{$\rtab \xi v$}}}
% \put(64,75){\raisebox{-0.5ex}{\makebox[22 \unitlength]{$\pipeM c l r$}}}
% \put(64,37){\raisebox{-0.5ex}{\makebox[22 \unitlength]{$\mval i q$}}}

% \put(25,78){\makebox[50 \unitlength]{Splitting $\left(\splitonesym\right)$}}
% \put(25,33){\makebox[50 \unitlength]{Masking $\left(\vsplito\right)$}}
% \put(76,55){\parbox[l]{20 \unitlength}{Separated \\ Evaluation: \\$\reduce c {\tup{i,b}}$\\$q\equiv [b/l]r$ }}
% \put(9,55){\parbox[r]{20 \unitlength}{First-Stage \\ \lang\ Eval.: \\$\diaone e \xi v$}}

% \put(28,66){\parbox[t]{44 \unitlength}{ 
% 	\textbf{Correctness Condition:} \\ 
% 	Splitting followed by evaluation (top and right arrows)
% 	produces the same result as 
% 	\lang\ evaluation followed by masking (left and bottom arrows),
% 	up to a term equivalence.
% }}
% \end{picture}

% &
% \begin{picture} (80,54) (10,30)

% \thicklines
% \put(25,75){\oval(22,10)}
% \put(75,37){\oval(22,10)}
% \put(75,75){\oval(22,10)}
% \put(25,37){\vector(1,0){39}}
% \put(36,75){\vector(1,0){28}}
% \put(25,70){\line(0,-1){33}}
% \put(75,70){\vector(0,-1){28}}

% \put(14,75){\raisebox{-0.5ex}{\makebox[22 \unitlength]{$\coltwo e A$}}}
% \put(64,75){\raisebox{-0.5ex}{\makebox[22 \unitlength]{$\pipeS p l r$}}}
% \put(64,37){\raisebox{-0.5ex}{\makebox[22 \unitlength]{$q$}}}

% \put(25,78){\makebox[50 \unitlength]{Splitting $\left(\splittwosym\right)$}}
% \put(76,55){\parbox[l]{20 \unitlength}{Separated \\ Evaluation: \\ $\reduce p b$\\$q\equiv [b/l]r$}}
% \put(9,55){\parbox[r]{20 \unitlength}{First-Stage \\ \lang\ Eval.: \\ $\diatwo e q$}}

% \put(28,66){\parbox[t]{44 \unitlength}{ 
% 	\textbf{Correctness Condition:} \\ 
% 	Splitting followed by evaluation (top and right arrows)
% 	produces the same result as 
% 	\lang\ evaluation (left and bottom arrow),
% 	up to a term equivalence.
% }}
% \end{picture}

% \\ 

% \end{tabular}


\end{abstrsyn}
\caption{\TODO put a real caption here and format the figure nicely}
\label{fig:splittingSummary}
\end{figure*}
