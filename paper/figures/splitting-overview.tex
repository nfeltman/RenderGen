%!TEX root = ../paper.tex

\begin{figure*}
\begin{abstrsyn}

\textbf{Splitting Terms at World \bbtwo\ ---
$\splittwo {\coltwo e A} A p l r$}
\begin{enumerate}
\item[] Precomputation $p$ is a monostaged term containing the first-stage computations in $e$.
\item[] Resumer $l.r$ is a monostaged term containing the second-stage computations in $e$.
\end{enumerate}

\textbf{Splitting Terms at World \bbonem\ ---
$\splitone {\colmix e A} A c {l.r}$}
\begin{enumerate}
\item[] Combined result $c$ is a monostage term containing the first-stage computations in $e$.
\item[] Resumer $l.r$ is a monostage term containing the second-stage computations in $e$.
\end{enumerate}

\textbf{Masking ---
$\rtab \xi v \vsplito \mval i q$}
\begin{enumerate}
\item[] Input $\rtab \xi v$ is the residual table and partial value to be masked.
\item[] Immediate value $i$ is a monostage value representing the first-stage components of $v$.
\item[] Residual $q$ is a monostaged term representing the second-stage computations in $\xi$ and $v$.
\end{enumerate}

\hrule
\vspace{1em}

\centering
\setlength{\unitlength}{2.8pt}
\begin{tabular}{c|c}
{\large \bf Splitting at World \bbonem} & 
{\large \bf Splitting at World \bbtwo} \\

\begin{picture} (80,54) (10,30)

\thicklines
\put(25,37){\oval(22,10)}
\put(25,75){\oval(22,10)}
\put(75,37){\oval(22,10)}
\put(75,75){\oval(22,10)}
\put(36,37){\vector(1,0){28}}
\put(36,75){\vector(1,0){28}}
\put(25,70){\vector(0,-1){28}}
\put(75,70){\vector(0,-1){28}}

\put(14,75){\raisebox{-0.5ex}{\makebox[22 \unitlength]{$\colmix e A$}}}
\put(14,37){\raisebox{-0.5ex}{\makebox[22 \unitlength]{$\rtab \xi v$}}}
\put(64,75){\raisebox{-0.5ex}{\makebox[22 \unitlength]{$\pipeM c l r$}}}
\put(64,37){\raisebox{-0.5ex}{\makebox[22 \unitlength]{$\mval i q$}}}

\put(25,78){\makebox[50 \unitlength]{Splitting $\left(\splitonesym\right)$}}
\put(25,33){\makebox[50 \unitlength]{Masking $\left(\vsplito\right)$}}
\put(76,55){\parbox[l]{20 \unitlength}{Separated \\ Evaluation: \\$\reduce c {\tup{i,b}}$\\$q\equiv [b/l]r$ }}
\put(9,55){\parbox[r]{20 \unitlength}{First-Stage \\ \lang\ Eval.: \\$\diaone e \xi v$}}

\put(28,66){\parbox[t]{44 \unitlength}{ 
	\textbf{Correctness Condition:} \\ 
	Splitting followed by evaluation (top and right arrows)
	produce the same result as 
	evaluation followed by masking (left and bottom arrows),
	up to a term equivalence.
}}
\end{picture}

&
\begin{picture} (80,54) (10,30)

\thicklines
\put(25,75){\oval(22,10)}
\put(75,37){\oval(22,10)}
\put(75,75){\oval(22,10)}
\put(25,37){\vector(1,0){39}}
\put(36,75){\vector(1,0){28}}
\put(25,70){\line(0,-1){33}}
\put(75,70){\vector(0,-1){28}}

\put(14,75){\raisebox{-0.5ex}{\makebox[22 \unitlength]{$\coltwo e A$}}}
\put(64,75){\raisebox{-0.5ex}{\makebox[22 \unitlength]{$\pipeS p l r$}}}
\put(64,37){\raisebox{-0.5ex}{\makebox[22 \unitlength]{$q$}}}

\put(25,78){\makebox[50 \unitlength]{Splitting $\left(\splittwosym\right)$}}
\put(76,55){\parbox[l]{20 \unitlength}{Separated \\ Evaluation: \\ $\reduce p b$\\$q\equiv [b/l]r$}}
\put(9,55){\parbox[r]{20 \unitlength}{First-Stage \\ \lang\ Eval.: \\ $\diatwo e q$}}

\put(28,66){\parbox[t]{44 \unitlength}{ 
	\textbf{Correctness Condition:} \\ 
	Splitting followed by evaluation (top and right arrows)
	produce the same result as 
	direct evaluation (left and bottom arrow),
	up to a term equivalence.
}}
\end{picture}

\\ 

\end{tabular}


\end{abstrsyn}
\caption{Overview of splitting.}
\label{fig:splittingSummary}
\end{figure*}
