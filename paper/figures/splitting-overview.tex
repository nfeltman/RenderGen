\begin{figure*}

\textbf{\lang\ Evaluation:}

\hspace{2em}\bbone-Evaluation: $e\mathbin{\redonesym}[\xi;v]$, where $\xi$ is a \emph{residual table} and $v$ is a \emph{partial value}. 

\hspace{2em}\bbtwo-Evaluation: $e\mathbin{\redtwosym}q$, where $q$ is a \emph{residual} in the unstaged language \langTwo

% $red_2$: 2-eval,  which we think of as being identical to specialization. 

\vspace{.75em}
\textbf{\lang\ Splitting:}

\hspace{2em}\bbone-Splitting Structure: $e \splitonesym [c,l.r]$, where:

\hspace{4em}$c$ is the \emph{combined term} (representing all stage~\bbone\ subcomputations in $e$)

\hspace{4em}$l.r$ is the \emph{resumer} (representating all stage~\bbtwo\ subcomputations in $e$)

\hspace{4em}$c~\redsym~(y,b)$, where $y$ is the \emph{\bbone-result} of $e$, and $b$ is the \emph{boundary value} of $e$

 
\hspace{2em}\bbone-Splitting Correctness: If $e\mathbin{\redonesym}[\xi;v]$, then:

\hspace{4em}$y$ is identical to $\masko{\xi;v}$

\hspace{4em}$\maskt{\xi;v}$ and \texttt{(let l=b in r)} reduce (via $\redsym$) to identical values.  

\hspace{2em}\bbtwo-Splitting Structure: $e \splittwosym [p,l.r]$, where:

\hspace{4em}$p$ is the \emph{precomputation} (representing all stage~\bbone\ subcomputations in $e$)

\hspace{4em}$l.r$ is the \emph{resumer} (representating all stage~\bbtwo\ subcomputations in $e$)

\hspace{2em}\bbtwo-Splitting Correctness: If $e~\redtwosym~q$, and $q~\redsym~v$, then \texttt{(let l=b in r)}~$\redsym~v$


\caption{Summary of \lang\ evaluation and splitting.}
\label{fig:termSplitTwo}
\end{figure*}

