%!TEX root = ../paper.tex

\begin{figure*}
\begin{subfigure}{0.5\textwidth}
\begin{lstlisting} 
datatype list = Empty | Cons of int * list

fun part (p, Empty) = (0,Empty, Empty) 
  | part (p, Cons (h,t)) = 
    let val (s,left,right) = part (p,t) in 
    if h<p 
    then (s+1,Cons(h,left),right) 
    else (s,left,Cons(h,right))

fun qSelect (Empty, k) = 0
  | qSelect (Cons ht, k) =
    let val (n,left,right) = part ht in
    case compare k n of
      LT => qSelect (left, k)
    | EQ => #1 ht
    | GT => qSelect (right, k-1-n)
\end{lstlisting}
\caption{Unstaged implementation of quickselect.}
\label{fig:qs-unstaged}
\end{subfigure}%
\begin{subfigure}{0.5\textwidth}
\begin{lstlisting} 
1`atsignpure{
  datatype list = Empty | Cons of int * list
  fun part (...) = ...
} 

`//1`qsStaged : ^list * $`2`int`1` -> $`2`int`1`
fun qsStaged (Empty,_) = next {`2`0`1`}
  | qsStaged (Cons ht,next{`2`k`1`}) = 
    let 
      val (n,le,ri) = part ht
      val next{`2`n`1`} = hold n 
    in next{`2`
      case compare k n of
        LT => prev {`1`qsStaged (le, next{`2`k`1`})`2`}
      | EQ => prev {`1`hold (#1 ht)`2`}
      | GT => prev {`1`qsStaged (ri, next{`2`k-1-n`1`})`2`}`1`
    }`
\end{lstlisting}
\caption{Staged implementation of quickselect in \lang.}
\vspace{1.3em}
\label{fig:qs-staged}
\end{subfigure}
\caption{Quickselect: traditional and staged.}
\end{figure*}


