\begin{figure*}
\begin{subfigure}{0.5\textwidth}
\begin{lstlisting} 
datatype list = Empty | Cons of int * list
fun partition (p : int, l : list) 
  : (int*list*list) =
  case unroll l of 
    Empty => (0,Empty, Empty) 
  | Cons (h,t) =>
      let (s,left,right) = partition (p,t) in
      if h<p 
      then (s+1,Cons(h,left),right)
      else (s,left,Cons(h,right))

fun qSelect (l : list, k : int) : int = 
  case l of
    Empty => 0
  | Cons (h,t) => 
      let (left,right,n) = partition h t in
        case compare k n of
          LT => qSelect (left, k)
        | EQ => h
        | GT => qSelect (right, k-n-1)
\end{lstlisting}
\caption{Unstaged implementation of quickselect.}
\label{fig:quickselect}
\label{fig:qs-unstaged}
\end{subfigure}%
\begin{subfigure}{0.5\textwidth}
\begin{lstlisting} 
1`datatype list = Empty | Cons of int * list
fun partition (p : int, l : list) = ...

fun qsStaged (l : list, k : $`2`int`1`) : $`2`int`1` = 
  case l of
    Empty => next {`2`0`1`}
  | Cons (h,t) => 
      let (left,right,n) = partition h t in
      next{
        `2`let n = hold{`1`n`2`} in
          case compare prev{`1`k`2`} n of
            LT => prev {`1`qsStaged (left k)`2`}
          | EQ => hold {`1`h`2`}
          | GT => prev {`1`qsStaged (right, 
                             next{prev{k}-n-1)}`2`}`1`
      }`
\end{lstlisting}
\caption{Staged implementation of quickselect in \lang.}
\label{fig:qsstaged}
\label{fig:qs-staged}
\end{subfigure}
\caption{Quickselect: traditional and staged.}
\end{figure*}


