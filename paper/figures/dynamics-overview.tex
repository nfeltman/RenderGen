%!TEX root = ../paper.tex

\begin{figure}
\begin{abstrsyn}

% \textbf{First-stage evaluation at \bbonep\ ---
% $\reduce e u$}
% \begin{enumerate}
% \item[] Given $\colpure e A$;
% $u$ is a value.
% \end{enumerate}

% \textbf{First-stage evaluation at \bbonem\ ---
% $\diaone e \xi v$}
% \begin{enumerate}
% \item[] Given $\colmix e A$;
% $\xi$ is a residual table, 
% $v$ is a partial value.
% \end{enumerate}

% \textbf{First-stage evaluation at \bbtwo\ ---
% $\diatwo e q$}
% \begin{enumerate}
% \item[] Given $\coltwo e A$;
% $q$ is a residual.
% \item[] $\redtwosym$ uses reification --- $\reify \xi q q'$ \\
% Given residual table $\xi$ and residual $q$; $q'$ is a residual.
% \end{enumerate}

% \textbf{Second-stage evaluation of residuals ---
% $\reduce q u$}
% \begin{enumerate}
% \item[] Given residual $q$;
% $u$ is a value.
% \end{enumerate}

\begin{theorem} [Progress] \leavevmode
\label{thm:progress} 
\begin{itemize} 
\item If $\typesone e A$, then either $e$ has the form $\exv v$, or $\stepmix e {e'}$, or $\lift e {e'}$.
\item If $\typespure e A$, then either $e$ has the form $\exv u$, or $\steppure e {e'}$.
\item If $\typestwo e A$, then either $e$ has the form $\exv q$, or $\steptwo e {e'}$.
\end{itemize}
\end{theorem}
\begin{theorem} [Preservation] \leavevmode
\label{thm:preservation} 
\begin{itemize} 
\item If $\typesone e A$ and $\lift e {e'}$, then $\typestwo q B$ and $\typesone [\Gamma,\coltwo y B] {e'} A$.
\item If $\typeswor e A$ and $\stepwor e {e'}$, then $\typeswor {e'} A$.
\end{itemize}
\end{theorem}

\end{abstrsyn}
\caption{Type Safety Theorems.}
\label{fig:typeSafety}
\end{figure}

