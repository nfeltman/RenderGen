\section{Introduction}
\label{sec:intro}

Consider a function $F$ which takes an $x$ and $y_1,\dots,y_m$, and computes
$f(x,y_i)$ for each $y_i$:
\begin{lstlisting}
fun @$F(x, y_1, y_2 \ldots, y_m)$@ = 
  @$f(x, y_1);  f(x, y_2); \ldots; f(x, y_m)$@
\end{lstlisting}
%
In this common scenario, a multivariate function is applied many times with one
argument fixed, so it is advantageous to find the computations in $f$ that
depend on $x$ and perform them only once. This is the intent of compiler
optimization techniques like loop hoisting and common subexpression elimination,
which identify the computations common to all invocations of $f$, and perform
them once at the beginning of the computation.

Programs featuring this structure are particularly important in real-time
computer graphics; in fact, modern graphics architectures \emph{require} that
graphics computations be organized into distinct {\em passes} which perform
increasingly fine-grained computations (for example, per object, per region of
the screen, and per pixel, where many computations in later passes use the
results from a single computation in an earlier pass)\,\cite{OpenGL4Spec}.
While performant, this paradigm of programming with explicit passes results in
complex code where invariants must hold across different passes, and local
changes for one pass may require changes to other passes to ``plumb'' inputs
from all other frequencies\,\cite{Foley:2011}.
%
In other words, requiring manual separation of passes breaks key programming
abstractions such as composition and modularity.
%
Graphics researchers therefore have suggested using explicitly staged programs
\,\cite{Proudfoot:2001,Foley:2011,He:2014}, deferring to the compiler the
mechanical task of separating such programs into passes.

Programming languages researchers have studied similar problems. In the 1980s,
J{\o}rring and Scherlis identified {\em frequency reduction} and {\em
precomputation}~\cite{JS86-staging} as mechanisms for efficiently evaluating
staged computations. To perform frequency reduction, one identifies and hoists
computations that are performed multiple times, in order to compute them only
once. To perform precomputation, one identifies computations that can be
performed in advance, and does so---for example, at compile time if the relevant
inputs are statically known.

There has been much work on language-based techniques for performing
precomputation. Partial evaluation~\cite{futamura71,jones96} achieves
precomputation by specializing functions to specific arguments. Going back to
our example, if $f$ specializes to a particular $v$, written $f_v$, such that
$f(v,y) = f_v(y)$, then $F$ can be specialized to $v$ as
\begin{lstlisting}
fun @$F_v(y_1, y_2, \ldots, y_m)$@ = 
  @$f_v(y_1);  f_v(y_2); \ldots; f_v(y_m)$@.
\end{lstlisting}
eliminating the need to compute $m$ times those parts of $f(v,-)$ which do not
depend on the second argument.

A closely related technique, {\em metaprogramming}, allows writing
programs that generate specialized
code~\cite{davies96,Taha97,DP01-modal,NP05-nn}. Metaprogramming
enables fine-grained control over specialization by requiring \emph{staging
annotations} that explicitly mark the stage of each computation. (In
partial evaluation, we typically assume that a \emph{binding-time
  analysis} has automatically determined which parts of $f$ can be
specialized.)

While partial evaluation and metaprogramming have been studied extensively,
frequency reduction techniques have not seen the same attention. In
\cite{JS86-staging}, J{\o}rring and Scherlis proposed {\em pass separation} as a
manual method for {\em splitting} a program into multiple passes.
%
In our example, pass separation would split the function $f$ into two passes
$f_1$ and $f_2$ such that $f(x,y_i) = f_2(f_1(x),y_i)$. Then we evaluate $F$ by
evaluating the first pass on $x$, and using the result $z$ to evaluate the
second pass on each $y_i$.
%
\begin{lstlisting}
fun @$F_\textrm{multipass}(x, y_1, y_2, \ldots, y_m)$@ = 
  let @$z = f_1(x)$@
  in @$f_2(z, y_1);  f_2(z, y_2); \ldots; f_2(z, y_m)$@
\end{lstlisting}
%
The key difference between pass separation and partial evaluation (or
metaprogramming) is that the former can be performed without access to
the first argument $x$; $F_\textrm{multipass}$ works for any $x$,
while $F_v$ is defined only for $x=v$.  Note that unlike
precomputation, frequency reduction does not require knowledge of
arguments; however, if a particular $v$ is known, we can use partial evaluation
to specialize $F_\textrm{multipass}$ to that $v$.

Prior work on partial evaluation and metaprogramming has shown that
precomputation can largely be automated for functional languages. In contrast,
automatic pass separation has been limited to simple languages that do not
support recursion or first-class
functions~\cite{knoblock96,Proudfoot:2001,Foley:2011,He:2014}. In this paper, we
extend this work by presenting a pass separation algorithm for a typed lambda
calculus.

As our starting point, we consider an explicitly-two-staged, typed lambda
calculus \lang\ in the style of Davies~\cite{davies96}, with a $\fut$ modality
denoting computation in a later stage (\ref{sec:semantics}). We then present a
splitting algorithm to pass-separate a \lang\ program into two passes
corresponding to the stages of computation in that program
(\ref{sec:splitting}). These passes are monostaged programs, expressed in a
conventional functional language.

The most interesting case involves splitting a mixed-stage function (like $f$ in
the example) which is moreover recursive. This yields a first pass function
$f_1$ which computes from its first-stage argument $x$ a \emph{boundary} data
structure encoding the interstage communication in $f(x,-)$. When $f$ recurs on
$x$, this boundary structure itself has a recursive type.
%
The second pass is a function $f_2$ which takes the boundary data structure
$f_1(x)$ and a second-stage argument $y_i$, and completes the computation
$f(x,y_i)$, traversing the boundary structure in light of the particular
argument $y_i$.

We have implemented this splitting algorithm (\ref{sec:implementation}) and
applied it to a number of staged programs (\ref{sec:examples}) ranging from
straight-line arithmetic operations to recursive and higher-order functions. In
the case of the quickselect algorithm, which we discuss next, the split code is
in fact asymptotically faster over many $y_i$.

%The crux of the splitting algorithm is splitting recursive mixed stage
%functions that involve both a first stage and second stage computations.  When
%split, such a function turns into a first-pass function that yields a {\em
%boundary value} that encodes the ``first-stage'' part of the computation.  If
%the function recurs on a first-stage argument, then the boundary value is
%typically a recursive data structure.  In other words, the splitting algorithm
%maps computational recursion into a recursive data type. If the boundary
%structure is generated as a result of a conditional, it is typically encoded as
%a (recursive) sum type, such as a tree, where the kind of the node indicates
%the recursion status of the first-stage evaluation.  The second-pass part of
%the computation, which corresponds to the ``second stage'' takes the boundary
%value as an argument as well as the second-stage argument and uses them to
%complete the computation, often by traversing the now precomputed boundary in
%light of the now available second-stage argument.

