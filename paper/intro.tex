\section{Introduction}

We often find ourselves in situations where the input to a computation arrives in two parts, at different points in time.  

The natural response is to try and do some part of the computation early, and the rest of the computation once the second input arrives.  

Since the computation runs in two parts, the total cost can be split as well, into $m+n$.

Additionally, we can run the second stage multiple times.  Cost is now $m+bn$.

Jorring et al. (\cite{jorring86}) identify three classes of staging techniques: meta-programming, partial evaluation, and stage-splitting.  The first two of these have received significantly more attention than the third.  Countless meta-programming systems exist (...twelve thousand citations...\cite{devito13}), and their background theory and type-systems are well understood (\cite{davies01}).  Partial evaluation, too, is well-understood.  Partial evaluation systems exist... . This paper explores both theory and applications for stage-splitting.  
