\section{Introduction}

Consider a function that takes as arguments an $x$ and $y_1 \ldots
y_{m-1}$ and invokes a function $f(\cdot, \cdot)$ on consecutive
values of $y$'s.  Using ``;'' for  sequential composition, we can
write such a function as
\begin{lstlisting}
fun @$F(x, y_0, y_1 \ldots y_{m-1}$@ = 
  @$f(x, y_0);  f(x, y_1)  \ldots f(x, y_{m-1})$@.
\end{lstlisting}
%
%% Let's assume for concreteness that $f(x,y)$ takes time to the size of
%% the first argument.  The function $F(x_0, y_0 \ldots y_{m-1}$ then
%% takes time $O(n \cdot m)$ where $n$ is the size of the input $x_0$.

Such computations, where we apply a multivariate function to arguments
where one of the argument remains the same are common.  For example,
many algorithm-design techniques revolve around the idea of finding
the computation in $f$ that depends only on $x$, and performing them
once and for all to use.  Similarly, crucial compiler optimization
techniques such as loop hoisting and common subexpression elimination
identify common computations across all invocations of $f$ and perform
them once and for all at the beginning of the computation.  A
particular application area where such algorithm-design and code
optimization techniques are widely applied include real-time computer
graphics.  There, to achieve high-performance, modern graphics
architectures requires writing such programs.  Programmers therefore
manually separate code into distinct {\em passes} based on their
required frequency of execution.  For example, graphics computations
are expressed as a sequence of passes where each pass may perform a
computation at increasingly finer grain, e.g., per object, per region
of the screen, and per pixel, where later computations use the results
from the prior computations.

%
While performant, programming with manually separated passes results
in complex code where invariants must hold across different passes,
and local changes for one pass may require changes to other passes to
``plumb'' inputs from all other frequencies\,\cite{Foley:2011}.
%
In other words, manual pass separation breaks key programming
abstractions such as composition and modularity.
%
Graphics researchers therefore have suggested using explicitly staged
programming languages\,\cite{Proudfoot:2001,Foley:2011,He:2014},
deferring to the compiler the tedious task of separating the code into
passes. However, all prior efforts have been limited to simple
languages that do not support recursion or first-class functions.




Programming-languages researchers have studied similar problems. In
the 1980s, J{\o}rring and Scherlis identified {\em frequency
  reduction} or {\em precomputation}~\cite{JS86-staging} as the common
mechanisms for evaluating efficiently staged computations.  The idea
behind frequency reduction is to identify computations that are
performed multiple times and hoist them so that they can be performed
once and used later as needed.  The idea behind precomputation is to
identify computations that can be performed earlier---for example, at
compile time if their inputs are known statically---and perform them
at that earlier time.

There has been much work on language-based techniques for performing
precomputation. Partial evaluation~\cite{} achieves precomputation by
specializing functions for specific values.  Going back to our
example, we can specialize $f$ at $[1,2, \ldots, n]$, written
$f_{[1,2, \ldots, n]}$, such that $f(x,y) = f_{[1,2, \ldots, n]}(y)$
and specialize $F$ for the same input as
\begin{lstlisting}
fun @$F_{[1,2, \ldots, n]}(y_0, y_1 \ldots y_{m-1})$@ = 
  @$(f_{[1,2, \ldots, n]}(y_0);  f_{0}(y_1)  \ldots f_{0}(y_{m-1}))$@.
\end{lstlisting}
%
Such a partial evaluation typically proceed by performing a
binding-time analysis to determine the static and dynamic parts of $f$
and $F$ and specializes them based on the results of the binding-time
analysis. A closely related technique, {\em metaprogramming}, enables
the programmer to write programs that generate specialized code.
Metaprogramming enables fine-grained control over specialization by
requiring staging annotations that make explicit the stage of each
computation.


While previous work shows that precomputation can be performed
essentially automatically (via partial evaluation or meta
programming), frequency reduction has been less explored.  In their
original paper, J{\o}rring and Scherlis proposed {\em pass separation}
as a manual method for {\em splitting} a program into multiple passes.
%
Applied to our example, pass separation would split the function $f$
into $f_1$ and $f_2$ such that $f(x,y) = f_2(f_1(x),y)$.  We can then
rewrite our example as a multi-pass program:
%
\begin{lstlisting}
fun @$F(x, y_0, y_1 \ldots y_{m-1})$@ = 
  let @$fx = f_1(x)$@
  in @$(f_2(fx, y_0);  f_2(fx, y_1)  \ldots f_2(fx, y_{m-1}))$@.
\end{lstlisting}
%
The key difference between partial evaluation (meta programming) and
pass separation is that the latter does not require the first
argument~$x$: it works with all arguments, whereas the partially
evaluated program is specialized for a particular value of $x$ ($x =
[1,2, \ldots, n]$ in the example). Specifically, if desired, partial
evaluation can be applied subsequently in our example, by partially
evaluating function~$f_2$ on $f_1(x)$.

However, unlike for precomputation, there is no available technique
for pass separation except for very simple languages
\cite{knoblock96,Proudfoot:2001,Foley:2011,He:2014}.  Notably omitted
features include recursion and first-class functions.  We are thus
interested in the problem of splitting automatically a functional
language.
%


As our starting point, we take the modal language \lang\ with explicit
staging in the style of Davies~\cite{davies96}, where the ``circle''
modality denotes computation in a later stage, and present a staged
operational semantics similar for \lang\ (\ref{sec:stagedsemantics}).
We then present a splitting algorithm that stratifies a \lang\ program
into a two mono-staged programs, one for the first stage and for the
second stage (\ref{sec:splitting}).  Since each program is monostaged,
they are expressed in conventional functional languages.

The crux of the splitting algorithm is splitting recursive mixed stage
functions that involve both a first stage and second stage
computations.  When split, such a function turns into a stratum-1
function that yields a {\em boundary value} that encodes the
``first-stage'' part of the computation.  If the function recurs on a
first-stage argument, then the boundary value is typically a recursive
data structure.  In other words, the splitting algorithm maps
computational recursion into a recursive data type. If the boundary
structure is generated as a result of a conditional, it is typically
encoded as a (recursive) sum type, such as a tree, where the kind of
the node indicates the recursion status of the first-stage evaluation.
The stratum-2 part of the computation, which corresponds to the
``second stage'' takes the boundary value as an argument as well as
the second-stage argument and uses them to complete the computation,
often by traversing the now precomputed boundary in light of the now
available second-stage argument.

Since it deals with unrestricted recursion and a rich language
consisting of first-class functions, recursion, and sum and recursive
types, the splitting algorithm is naturally complex. With some care,
the algorithm nevertheless can be specified reasonably succinctly.  In
fact, we have implemented the algorithm (\ref{sec:implementation}) and
applied to several examples (\ref{sec:examples}).  As an interesting
example, we show that a function 

\begin{lstlisting}
fun @$F(x, y_0, y_1 \ldots y_{m-1}$@ = 
  @$f(x, y_0);  f(x, y_1)  \ldots f(x, y_{m-1})$@,
\end{lstlisting}
where $f(x,y)$ selects the element of the list $x$ with rank $y$ by
using the ``quickselect'' algorithm.  When applied to this function,
our split algorithm yields an asymptotically more efficient function
that runs in expected $O(n\log{n} + m\log{n})$ time instead of
expected $O(n \cdot m)$ time---a near linear time improvement.
Interestingly the code output by our splitting algorithm
auto-generates the ``quicksort'' algorithm, which it uses to build a
binary search tree as a boundary value in the first stage.  For the
second stage, the splitting algorithm generates a ``binary search''
that takes the tree and performs a binary search on the tree as guided
by the given rank.  To the best of our knowledge, no prior approaches
can perform such complex transformations on higher-order code, nor can
they yield such asymptotic improvements in run time.

\paragraph{Old intro below.}

Multi-argument functions can frequently perform useful work before receiving all
of their inputs, or are often called numerous times with one argument fixed. An
important program optimization is therefore to \emph{specialize} such a function
$f$ to its fixed argument $a$, by executing those computations in $f$ which
depend only on $a$. This ensures that calls to the specialized function require
only computations which depend on its varying argument.

\emph{Program specialization}, or partial evaluation \cite{futamura71,jones96},
is a well-known specialization technique which, given $a$, transforms $f$ into a
new function $f_a(-)$ which computes $f(a,-)$. This transformation essentially
substitutes $a$ for the first argument of $f$, then evaluates in place any
subexpressions of $f$ depending only on that argument.

\emph{Data specialization} \cite{knoblock96,JS86-staging} 
is a technique for specializing $f$ \emph{without} the fixed argument $a$,
instead splitting $f$ into a pair of functions $f_1$ and $f_2$. $f_1(a)$
produces a data structure containing the results of the computations which
depend only on $a$; $f_2$ then completes the computation, given this data
structure and the varying argument; that is, $f_2(f_1(a),-)$ computes $f(a,-)$.
Crucially, none of the generated code ($f_1$ or $f_2$) depends on $a$!

Previous work on data specialization has been limited to simple, imperative
languages. In this paper, we extend data specialization to a typed lambda
calculus, allowing us to specialize a broader class of programs.

When splitting certain recursive functions, like \texttt{quickselect}, our
algorithm synthesizes recursive data structures and traversal algorithms which
yield asymptotic speedups over the original function. Splitting higher-order
combinators, like \texttt{map}, provides compositional reasoning at the source
level while cross-cutting fixed runtime stages, as in graphical shading
languages like Spark~\cite{Foley:2011}.

We start in \ref{sec:example} with an extended example of splitting
\texttt{quickselect}.
In \ref{sec:semantics}, we describe our staged language \lang, including its
type system and operational semantics.
In \ref{sec:splitting,sec:implementation}, we describe our stage-splitting
algorithm for performing data specialization.
Finally, in \ref{sec:examples}, we show how our stage-splitting algorithm
transforms a variety of other programs.
