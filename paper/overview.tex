%!TEX root = paper.tex

\section{Overview}
\label{sec:overview}

We present an overview of our approach by considering an example.  To
highlight the ideas, we abstract over some technical details, which
are fully formalized in the rest of the paper.  The example itself is
also implemented in our implementation.

\paragraph{Computing order statistics.}

\begin{abstrsyn}

%!TEX root = ../paper.tex

\begin{figure*}
\begin{subfigure}[t]{0.45\textwidth}
\begin{lstlisting} 

datatype list = Empty | Cons of int * list

fun partition (p, Empty) = (0,Empty, Empty) 
  | partition (p, Cons (h,t)) = 
    let val (s,left,right) = partition (p,t) 
    in if h < p then (s+1,Cons(h,left),right) 
             else (s,left,Cons(h,right))

qs: list * int -> int
fun qs (Empty, k) = 0
  | qs (Cons ht, k) =
    let val (i,left,right) = partition ht

    in case compare k m of
         LT => qs (left, k)
       | EQ => #1 ht
       | GT => qs (right, k-i-1)
\end{lstlisting}
\caption{Unstaged implementation of quickselect.}
\label{fig:qs-unstaged}
\end{subfigure}
\hfill
\begin{subfigure}[t]{0.55\textwidth}
\begin{lstlisting} 
3`atsigng{`  
1`datatype list = Empty | Cons of int * list

fun partition (p, Empty) = ...  (* Same as in unstaged *)



3`}`.

1`qss : ^list * $`2`int`1` -> $`2`int`1`
fun qss (`3`g{`1`Empty`3`}`1`,_) = next {`2`0`1`}
  | qss (`3`g{`1`Cons ht`3`}`1`,next{`2`k`1`}) = 
    let val `3`g{`1`(i0,left,right)`3`}`1` = `3`g{`1`partition ht`3`}`1`
        val next{`2`i`1`} = hold `3`g{`1`i0`3`}`1`
    in next{`2` case compare k i of
               LT => prev {`1`qss (`3`g{`1`left`3`}`1`, next{`2`k`1`})`2`}
             | EQ => prev {`1`hold `3`g{`1`#1 ht`3`}`2`}
             | GT => prev {`1`qss (`3`g{`1`right`3`}`1`, next{`2`k-i-1`1`})`2`}`1`}`
\end{lstlisting}
\caption{Staged implementation of quickselect in \lang.}

%\vspace{1.3em}
\label{fig:qs-staged}
\end{subfigure}
\caption{Quickselect: traditional and staged.}
\end{figure*}



Suppose that we wish to compute a series of order statistics queries
on a list \texttt{l}. To this end, we can use
quickselect~\cite{Hoare:1961}, which given a list \texttt{l} and a rank
\texttt{k}, returns the element of \texttt{l} with rank \texttt{k}.
As implemented in an ML-like language in \ref{fig:qs-unstaged},
function \texttt{qs} partitions the list by using the first element as
a pivot and then recurs on one of the two resulting sides to find the
desired element based on the relationship of \texttt{k} to the size of
the first half \texttt{i}.  Assuming that the input is uniformly
randomly ordered (which can be achieved by pre-permutation of the
input), function \texttt{qs} runs in expected $O(n)$ time.
%
Using function \texttt{qs}, we can perform many order statistics
queries, for example with $m$ different ranks
$\mathtt{k_1},\dots,\mathtt{k_m}$ as follows
%
\begin{lstlisting}
qs l @$\mathtt{k_1}$@; qs l @$\mathtt{k_2}$@; @$\ldots$@; qs l @$\mathtt{k_m}$@.
\end{lstlisting}
%
Unfortunately, this approach requires $O(n \cdot m)$ time, making it
expensive to perform order statistics on more than a small number of
times. This is unfortunate: we wish to be able to map \texttt{qs} to
over another list and do so efficiently.

One way to regain efficiency is to use a multi-pass algorithm that
performs frequently performed computations first so that they can be
reused.  It is nontrivial what such computations are in this example
but after some reflection, we can realize that they are the
comparisons performed by the partition function.  To minimize them, we
can pre-sort the input list \texttt{l} into a sorted list \texttt{s}
in the first pass, and then perform lookups in \texttt{s} in the
second pass.  Unfortunately, since \texttt{s} is a list, a lookup
would require linear time (also on average), leading to no improvement
in efficiency.  But this is not too hard: after sorting \texttt{s} we
can copy in into an array and use binary search to find the element
with the desired rank.  As another option, we can construct a binary
search tree from the input and perform size-based searches on the
binary search tree to find the elements with the desired rank.  Such
an approach would fit nicely into the functional paradigm but requires
one more additional algorithmic ingenuity: to ensure efficiency we
would need to store at each note the size of its subtree (these are
sometimes called ``augmented trees'').  In summary, we are able to
improve efficiency by replacing our simple solution with a multi-pass
algorithm that first preprocesses the input to generate a lookup data
structure and then in the second pass performs fast lookups.  As
outlined above, in addition, this transformation is highly non-trivial
as it involves reasoning about intricate algorithmic concerns, and
implementing more complex algorithms and data structures.  Both of
these approaches however improve the run-time to (expected)
$\Theta(n\log{n} + m\log{n})$, a near linear-time improvement in $m$.

Such transformations, called pass-separation by J{\o}rring and
Scherlis are commonly employed.  For example, as briefly mentioned in
\ref{sec:intro}, modern graphics software is written exactly in this
way.  Unfortunately, as the example illustrates, multipass programs
are naturally complex.  We thus ask: is it possible to
perform such transformations automatically.  We answer this question
affirmatively.  Our approach is to start with a staged language and
then use a splitting algorithm to generate multi-pass programs from
staged programs.


\subsection{Staging}

The idea behind staged programming is to use staging annotations or
types to indicate the stage of each subterm.  To apply this
technique to our example, suppose that we have a language with two
worlds representing the two stages and a third, {\em ground} world
that contains terms with no staging.

We can write a staged version of \texttt{qs}, called \texttt{qss},
whose code is shown in \ref{fig:qs-staged}, where the first-stage code
are typeset in red, and the second-stage code are typeset in blue.  To
obtain \texttt{qss} from \texttt{qs}, we first make the input list 
first-stage and the rank second-stage.
%
More precisely the input list has type $\curr\mathrm{list}$ (an
integer list ``now''), the rank has type $\fut\rmint$ (an integer in
the ``future''), and the return has type $\fut\rmint$.
%
We can then obtain \texttt{qss} by annotating the body of \texttt{qs}
with the staging annotations \texttt{prev} and \texttt{next}, which
transition between first-stage and second-stage code, and~$\texttt{g}$, 
which marks ground terms.  
%
We also use a third function
\texttt{hold}, which can be implemented with \texttt{prev} and
\texttt{next}, for promoting a first-stage integer to a second-stage integer;
the signature of \texttt{hold} is
% 
\lstinline{1`hold : ^int -> $`2`int`}.

Our type system ensures that the staging annotations are consistent,
in the sense that computations marked as first-stage cannot depend on
ones marked as second-stage.
%
The process of adding staging annotations to unstaged code has been
the subject of extensive research (\ref{sec:related}). We assume that
these annotations have been provided; we do not consider the problem
of generating them. In this example, there are other ways to annotate
\texttt{qs}, but we chose annotations that maximize the work performed
in the first stage.


\subsection{Splitting Staged Programs}

\begin{figure}
%\begin{subfigure}{0.5\textwidth}
\begin{lstlisting}
1`datatype list = Empty | Cons of int * list
fun partition (p : int, l : list) = ...`
	
datatype tree = Branch of int * int * tree * tree
                | Leaf

1`fun qSelect1 (l : list) : tree =
  case l of
    Empty => Leaf
  | Cons (h,t) => 
      let (left,right,n) = partition h t in
      Branch (n, h, qSelect1 left, qSelect1 right)`

2`fun qSelect2 (p : tree, k : int) : int = 
  case p of
    Leaf => 0
  | Branch (n,h,p1,p2) => 
      case compare k n of
        LT => qSelect2 (p1,k)
      | EQ => h
      | GT => qSelect2 (p2,k-n-1)`
\end{lstlisting}
\caption{Split (two-phase) implementation of quickselect.}
\label{fig:qs-split}
%\end{subfigure}
%\caption{Caption place holder}
\end{figure}



The staged quickselect code shown in \ref{fig:qs-staged} makes
explicit the staging of all terms, making it natural to ask, whether
we can transform this code into manually implemented multi-pass code
described above and match the efficiency improvements of the manual
implementation. Considering the algorithmic knowledge and the
considerations needed, this may seem like a tall order.
Interestingly, our splitting algorithm achieves exactly this.
In the rest of this section, we present a brief, high-level overview
of the main ideas behind this algorithm using the quickselect example.
%

When applied to the staged code in \ref{fig:qs-staged}, our splitting
algorithm yields a two-pass program that uses the binary-search-tree
based implementation.  Specifically, in its first pass, the two-pass
program takes the input list and constructs a probabilistically
balanced binary search tree, which is isomorphic to a treap data
structure~\cite{SeidelAr96}.  In the second pass, the program performs,
for each rank, a binary search tree lookup, by walking the tree to
find the element with the desired rank.
%
\ref{fig:qs-staged} illustrates the code for the first and second
passes \texttt{qs1} and \texttt{qs2} of the staged function
\texttt{qss}.


To create the multi-pass algorithm, the splitting algorithm operates
by composing local transformations on the subterms of the input
program.  In particular, the algorithm has no special knowledge of the
quickselect algorithm, binary search trees, or how to perform lookups
on binary search trees, but it is able to derive them from the input
program.
%

The splitting algorithm scans the program code for first-stage
computations (which depend only on first-stage values) and separates
them into the function, \texttt{qs1} of the first pass. This function
performs the first-stage computations and places the results into a
boundary data structure that both records the control flow and the
results from the first stage at relevant control-flow points.  More
specifically, the function \texttt{qs1} performs all the recursive
calls and evaluates all instances of the partition function, which
depend only on the input list.  The function produces a boundary data
structure that collects the results from all recursive calls along
with a tag that indicates the control branches taken.  Since
\texttt{qss} has a binary control structure (casing on the list), the
boundary is a binary tree.  To ensure that the trace contains the
necessary results to complete the execution in the second pass, where
second-stage values may be used, the splitting algorithm includes in
the nodes of the tree information such as the "pivot" used for
splitting and the size of the subtree at that note.  The resulting
tree is thus not just a tree but a binary search tree, keyed by the
pivot and augmented with "size" information.

As the splitting algorithm scans \texttt{qss} for computations that
can be performed in the first stage, it also collects computations
that must be left to the second stage in a separate function
\texttt{qs2}.  This function, which is executed in the second pass,
takes as argument the boundary and the second-stage argument (the
rank) and performs a lookup in the boundary data structure.  But how
does \texttt{qs2} knows to traverse the tree? It does not. Since
the splitting algorithm recorded the control flow of the first-stage
execution in the boundary data structure, \texttt{qs2} simply follows
this control structure and performs at each point the parts of the
computation from \texttt{qss} that can now be performed in the
presence of the second-stage argument (the rank).  In the context of
our example, this performs a lookup on the boundary data structure by
using the supplied rank.

\begin{theorem}
  Consider an execution of \texttt{qs1} with a randomly permuted input
  of $n$ keys and performing $m$ executions of \texttt{qs2} with the
  result of \texttt{qs1}.  The total run-time for this computation is
  $\Theta(n\log{n} + m\log{n})$ in expectation.
\end{theorem}
\begin{proof}
  We use a few standard facts.  We note first that \texttt{qs1} is
  isomorphic to a quick-sort algorithm but builds a binary search tree
  instead of combining the returned results in a list.  The run-time
  for \texttt{qs1} is thus (expected) $\Theta(n\log{n})$.
%
  The depth of the tree generated by \texttt{qs1} is isomorphic to the
    depth of a run of the quickselect algorithm or equivalently to the
    span of a parallel implementation of the quicksort algorithm, both
    of which are known to be (expected) $\Theta(\log{n})$.
%
    By inspection of \texttt{qs2}, we can see that the algorithm takes
    time proportional to the depth of the binary search tree, which is
    $\Theta(\log{n})$. Consider now a random variable for each of the
    $m$ invocations of \texttt{qs2}, each of which takes expected
    $\Theta(\log{n})$ time.  Since the total time is the sum of these
    random variables and since the expectation of the sum is sum of
    the expectations of the random variables, the bound follows.
\end{proof}

\end{abstrsyn}
