\section {Implementation and Optimization}
\label{sec:implementation}

We have implemented the semantics and splitting algorithm of \lang\ in Standard ML.
In addition to the features shown here, our implementation includes more base types, 
$n$-ary products and sums, if and let statements, and a basic elaboration step.

Applying the rules of \cref{sec:splitting} naively can yield inefficient split programs.
In many cases, the results can be improved by considering special cases in individual rules.
For instance, trivial subterm precomputations needn't be saved at tuples:
\[
\infer {\splittwo {(e_1,e_2)} A {p,l.(r_1,r_2)}}{\splittwo {e_1} A {(),\_.r_1} & \splittwo {e_2} A {p,l.r_2}}
\]
This sort of optimization does not affect the contract of splitting at all,
because the contract only requires that there be a precomputation that the residual can accept,
with no restrictions on the form of that precomputation.
Further gains can be realized by changing the contract of splitting itself.

Let a type be {\em present} if it has no $\to$s or $\fut$s in its type.
Observe that for any $v$ with a present type, $\maskt{v}$ will always be a nested bunch of \texttt{()}s,
which is to say that it carries no information.

\nr{We probably need an example of this.}

Of course, \texttt{()} is a more compact representation that carries no information.
Thus, we can change the definition of masking so that $\maskt{v}$ is always \texttt{()},
and change the splitting rules accordingly.

Back to the boundary type problem.
The problem becomes more difficult when trying to optimize across functions, including recursive calls.
For example, consider the stage \bbtwo\ term,
\begin{lstlisting}
2`prev{
  1`letfun fact (n : int) : int = 
    if n <= 0 then 1 else fact(n-1)*n
  in next{`2`hold {`1`fact 5`2`}-100`1`}`2`
}`
\end{lstlisting}
Noting that the precomputation of the body of \texttt{fact} could be given the type,
\begin{lstlisting}
datatype prec = L | R of prec
\end{lstlisting}
the whole expression splits into:
\begin{lstlisting}
1`letfun fact (n : int) : int * prec = 
  if n <= 0 then (1,L) 
  else let (y,z) = fact(n-1) in (y*n,R z)
in fact 5`

2`l => 
letfun fact (n : unit, l0 : prec) : unit = 
  case l0 of L => () | R l1 => fact ((),l0)
in (fact (pi2 l); pi1 l)-100`
\end{lstlisting}

Observe that it's wasteful to run the stage \bbtwo\ version of \texttt{fact}, since it
always returns \texttt{()}; worse yet, it has linear runtime!
But recognizing and optimizing this away is in general a global operation,
so we instead solve this issue by adding a new staging annotation to \lang, called
\texttt{mono}. The term $\monoSt~e$ is stage \bbone, and requires that $e$ contains
no stage \bbtwo\ subexpressions. This adds a new stage \bbmono\ and two new
rules to \lang:
\begin{mathpar}
\infer{\typesone{\monoSt~e}A}{\typesmono e A & A~\mathrm{safe}} \and
\infer{A\to B~\mathrm{safe}}{A~\mathrm{present} & B~\mathrm{safe}} \and
\infer{A~\mathrm{safe}}{A~\mathrm{present}} \and
\infer{\diaone{\monoSt~e}{\cdot;v}}{\diaone{e}{\cdot;v}} \and
\infer{\splitone{\monoSt~e}A {(e',()), \_.r} }{e \overset{C}{\rightarrow} e' & A \overset{G}{\rightarrow} r} \and
\infer{A \overset{G}{\rightarrow} ()}{A ~\mathrm{present}} \and
\infer{(A \to B)\overset{G}{\rightarrow} r}{B \overset{G}{\rightarrow} \lambda \_.r }
\end{mathpar}
Here, $\overset{C}{\rightarrow}$ translates functions to the signature expected for stage \bbone\ splitting outputs
by performing the two rewrites in place:
\begin{align*}
\lam x A e &\overset{C}{\rightarrow} \lambda x.(e,())\\
e_1~e_2 &\overset{C}{\rightarrow} \pio~(e_1~e_2)
\end{align*}
And then $\overset{G}{\rightarrow}$ generates a resumer that corresponds to the type $A$.
It operates inductively on the safety judgement of $A$.
At the safe $A$ base case, we make use of the prior optimization that says all safe $A$s have a resumer that evaluates to \texttt{()}.
