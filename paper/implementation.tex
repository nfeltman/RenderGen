%!TEX root = paper.tex

\section {Implementation}
\label{sec:implementation}

\begin{abstrsyn}
We have implemented the semantics and splitting algorithm of \lang\ in Standard ML.
Specifically, our implemention includes parsers, printers, and interpretters for both \lang\ 
and the output language, as well as a splitting algorithm which mediates between the two.
In addition to the features shown here, our implementation includes base types, 
$n$-ary products and sums, patterns, if and let statements, and a basic elaboration step.
We cover some of the details here.

\paragraph {Syntax}
Our concrete syntax is used in examples throughout the paper.
Its main notable property is that world boundaries are always noted by braces.

Functions and datatypes declarations are elaborated in the standard way.
We also implement declaration-level staging features.  For example,
\begin{lstlisting}
1`atsigngrnd{
  datatype list = Empty | Cons of int * list
  fun part (...) = ...
}
fun qsStaged (...) = ...`
\end{lstlisting}
indicates that the \texttt{list} datatype and \texttt{part} function
are declared and defined at world \bbonep, though this is accomplished by 
translating the above into,
\begin{lstlisting}
1`val grnd{Empty} = grnd{roll (inj ...)}
val grnd{Cons}  = grnd{fn (...) => roll (inj ...)}
val grnd{part}  = grnd{fn (...) => ...}
val qsStaged    = fn (...) => ...`
\end{lstlisting}

\paragraph {Patterns}
Our implemention includes patterns in the input language to eliminate products, $\fut$, and $\curr$.
As implied by our splitting rules, we also include product patterns in the output language.
For the sake of readability, we split input patterns directly (in a way similar to the standard elim forms), 
rather than translate them away before splitting.

\paragraph {Specialized Splitting} 
Our implemented splitting algorithm also includes many special cases to create more readable outputs.
Most important among these are cases for eliminating trivial precomputations.
These special cases all fit within the presented framework of our correctness condition.

\end{abstrsyn}