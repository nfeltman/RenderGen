\section {Implementation and Optimization}
\label{sec:implementation}

We have implemented the semantics and splitting algorithm of \lang\ in Standard ML.
In addition to the features shown here, our implementation includes more base types, 
$n$-ary products and sums, if and let statements, and a basic elaboration step.

Applying the rules of \cref{sec:splitting} naively can yield inefficient split programs.
In many cases, the results can be improved by considering special cases in individual rules.
For instance, trivial subterm precomputations needn't be saved at tuples:
\[
\infer {\splittwo {(e_1,e_2)} A {p,l.(r_1,r_2)}}{\splittwo {e_1} A {(),\_.r_1} & \splittwo {e_2} A {p,l.r_2}}
\]

This sort of optimization does not affect the contract of splitting at all,
because the contract only requires that there be a precomputation that the residual can accept,
with no restrictions on the form of that precomputation.

The problem becomes more difficult when trying to optimize across function boundaries.
For example, consider the stage \bbtwo\ term,
\begin{lstlisting}
2`prev{
  1`letfun double (n : int) = n + n
  in next{`2`hold {`1`double 5`2`}-7`1`}`2`
}`
\end{lstlisting}
Using the rules above, this term splits into:
\begin{lstlisting}
1`letfun double (n : int) = (n+n, ())
in double 5`

2`l => 
letfun double (n : unit, l0 : unit) = ()
in (double (pi2 l); pi1 l)-100`
\end{lstlisting}

Observe that it's wasteful to run the stage \bbtwo\ version of \texttt{double}, since it always returns \texttt{()}. 
In this case, the wasted work is small, but if \texttt{double} were a recursive function,
then it's runtime may be unbounded.

There are two ways plausible ways to solve this problem:
either we could define an optimization step that comes 
after splitting and show that it tends to fix the inefficiencies that splitting creates; 
or we could give more structure to the input language so that splitting can generate more efficient code directly.

We choose to investigate the latter approach.
Specifically, we add new staging features to \lang\ which allow us to indicate that a piece of code is {\em strictly} 
stage \bbone---{\em i.e.} that it contains no $\next$ blocks.
At the type level, this manifests as a constructor called $\monoType$.
So for instance, $\rmint \to \rmint$ is a stage \bbone\ function from one integer to another,
which might fail to terminate at stage \bbtwo.  
Alternatively $\monoType (\rmint \to \rmint)$ is a stage \bbone\ function from one integer to another
which {\em cannot} have any content at stage \bbtwo.
Naturally, the well-formedness of $\monoType A$ requires that $A$ contain no $\fut$s.

At the term level, we add the $\monoTerm$, $\lifttag$, \texttt{let mono} constructs to move information
in and out of $\monoType$.


\begin{figure*}
\begin{mathpar}
\infer	{\typesone {\monoTerm~e} {\monoType A}}					{\typesmono e A} 																\and
\infer	{\typesone {\letmono x {e_1} {e_2}} B}					{\typesone {e_1} {\monoType A} & \typesone [\Gamma,\colmono x A] {e_2} B} 		\and
\infer	{\typesone {\lifttag~e} {\monoType A + \monoType B}}	{\typesone e {\monoType (A+B)}} 												\and
\inferdiaone	{\red {\monoTerm~e} {\cdot; \monoTerm~v}}				{\reduce e v} 															\and
\inferdiaone	{\red {\letmono x {e_1} {e_2}} {\xi_1, \xi_2; v_2}}		{\red {e_1} {\xi_1;\monoTerm~v_1} & \red {[v_1/x]e_2} {\xi_2;v_2}} 		\and
\inferdiaone	{\red {\lifttag~e} {\xi;\inl(\monoTerm~v)}}				{\red e {\xi; \monoTerm(\inl~v)}} 										\and
\inferdiaone	{\red {\lifttag~e} {\xi;\inr(\monoTerm~v)}}				{\red e {\xi; \monoTerm(\inr~v)}} 										\and
\end{mathpar}
\caption{Mono Statics and Dynamics}
\label{fig:monoRules}
\end{figure*}

