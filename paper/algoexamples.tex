

\section{Examples of Algorithm Derivation}
\label{sec:examples}

\TODO put the \verb|fix|es in this code explicitly?

Fast exponent example.  

\begin{lstlisting} 
let fexp (b : $int, e : int) : $int =
	if e == 0 then
		next{1}
	else if (e mod 2) == 0 then
		next{let x = prev{fexp(b,e/2)} in x*x}
	else
		next{prev{b} * prev{fexp (b,e-1)}}		
\end{lstlisting}

splits into

\begin{lstlisting} 
let fexp (b, e) =
	((), roll (
		if e == 0 then
			inL ()
		else 
			inR (
				if (e mod 2) == 0 then
					inL (#2 (fexp (b,e/2)))
				else
					inR (#2 (fexp (b,e-1)))
			)
	))
\end{lstlisting}

and

\begin{lstlisting} 
let fexp ((b, e), p) =
	case unroll p of
	  () => 1
	| d  =>
		case d of
		  r => let x = fexp ((b,()),r) in x*x
		| r => b * fexp ((b,()),r)
\end{lstlisting}

Quickselect example.

\begin{lstlisting} 
let qs (l : "*$\mathtt{\mu \alpha. }$*"() + "*$\mathtt{int*\alpha}$*", i: $int) = 
	case unroll l of
	  () => next {0}
	| (h,t) => 
		let (left,right,n) = partition h t in
		next{
			let n = hold{n} in
			case compare prev{i} n of
			  () (*LT*) => prev {qs left i}
			| () (*EQ*) => hold {h}
			| () (*GT*) => 
				prev {qs right next{prev{i}-n-1}}
		}	
\end{lstlisting}

Things to try: an interpreter which, partially evaluated, does cps or something.

For each of these examples, talk about what partial evaluation would do and why that might be bad.


[Meta-ML eases off on this restriction but does not (I think?) eliminate it.]

[What's going on with names and necessity?]

[Our work bears a lot of similarity to ML5, which also uses a modal type system.  The difference is that we target stages systems (each stage talks to the next), whereas they target distributed ones (all stages talk to all others). The type systems reflect this directly in the world accessibility relation.  There might be some analogue of stage-splitting in the ML5 work, but I have not yet isolated it (might be buried in CPS conversion).]

