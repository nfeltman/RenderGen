\section{The \lang\ Language}

%
%
%\begin{figure}
%\caption{\ellStaged~Syntax}
%\label{fig:ellStagedSyntax}
%\centering
%\begin{tabular}{ll} 
%$\begin{aligned}
%\typeo &::= \text{unit}~|~\text{int}~|~\text{bool} \\
%&\gbar \typeo \times \typeo \\
%&\gbar \fut \typet \\
%\expro &::= ()~|~\inte~|~\bool  \\
%&\gbar \letin{\var}{\expro}{\expro} \\
%&\gbar \var \\
%&\gbar (\expro, \expro) \\
%&\gbar \pi_1~\expro \gbar \pi_2~\expro \\
%&\gbar \ifthen {\expro}{\expro}{\expro} \\
%&\gbar \next~\exprt \\
%\contextot &::=\emptyC \\
%&\gbar \contextot, \var : \typeo ^\bbone \\
%&\gbar \contextot, \var : \typet ^\bbtwo
%\end{aligned} $ 
%& 
%$\begin{aligned}
%\typet &::=  \text{unit}~|~\text{int}~|~\text{bool} \\
%&\gbar \typet \times \typet \\
%\\
%\exprt &::= ()~|~\inte~|~\bool \\
%&\gbar \letin{\var}{\exprt}{\exprt} \\
%&\gbar \var \\
%&\gbar (\exprt, \exprt) \\
%&\gbar \pi_1~\exprt \gbar \pi_2~\exprt \\
%&\gbar \ifthen {\exprt}{\exprt}{\exprt} \\
%&\gbar \prev~\expro \\
%\\
%\\
%\\
%\end{aligned} $
%\end{tabular}
%\end{figure}

%
%\begin{figure}
%\caption{\ellTarget~Syntax}
%\label{fig:ellTargetSyntax}
%\centering
%\begin{tabular}{ll} 
%$\begin{aligned}
%\expr &::= ()~|~\inte~|~\bool \\
%&\gbar \letin{\var}{\expr}{\expr} \\
%&\gbar \var \\
%&\gbar (\expr, \expr) \\
%&\gbar \pi_1~\expr \gbar \pi_2~\expr \\
%&\gbar \inl~\expr \gbar \inr~\expr \\
%&\gbar \ifthen {\expr}{\expr}{\expr}  \\
%&\gbar \caseof {\expr}{x_1.\expr}{x_2.\expr} 
%\end{aligned} $
%& 
%$\begin{aligned}
%\type &::=  \rmunit~|~\text{int}~|~\text{bool} \\
%&\gbar \type \times \type  \\
%&\gbar \type + \type 
%\\
%\context &::= \emptyC \\
%&\gbar \context, \var : \type
%\\ \\ \\ \\
%\end{aligned} $
%\end{tabular}
%\end{figure}


\begin{figure}
\centering
$\begin{aligned}
\world &::= \bbone \gbar \bbtwo \\
\type &::= \text{unit}~|~\text{int}~|~\text{bool} \\
&\gbar \type \times \type 
 \gbar \type + \type \\
&\gbar \type \to \type
 \gbar \fut \type \\
&\gbar \alpha \gbar \mu \alpha.\tau \\
\expr &::= ()\gbar\inte\gbar\bool\gbar \var  \\
&\gbar \lam{\var}{\type}{\expr} 
 \gbar \expr~\expr \\
&\gbar (\expr, \expr) 
 \gbar \pio~\expr 
 \gbar \pit~\expr \\
&\gbar \inl~\expr 
 \gbar \inr~\expr \\
&\gbar \caseof {\expr}{\var.\expr}{\var.\expr} \\
&\gbar \ifthen {\expr}{\expr}{\expr} \\
&\gbar \letin{\var}{\expr}{\expr} \\
&\gbar \next~\expr 
 \gbar \prev~\expr 
 \gbar \pause~\expr \\
\context &::=\emptyC \gbar \context, \colwor \var \type
\end{aligned} $
\caption{\lamStaged~Syntax}
\label{fig:lamStagedSyntax}
\end{figure}
%!TEX root = ../paper.tex

%\begin{figure}
%\caption{\lang~Valid Types}
%\label{fig:validTypes}
%\end{figure}

\begin{figure*}
\begin{abstrsyn}
\begin{mathpar}
\fbox {Valid Types} \and
\infertypeswor [\rmunit] 	{\Delta \vdash {\tt unit} \istypewor}						{\cdot} 																	\and
\infertypeswor [int]		{\Delta \vdash {\tt int} \istypewor}						{w \in \{\bbonep,\bbtwo\}}													\and
\infertypeswor [\times]		{\Delta \vdash A~\mathcal{O}~B \istypewor}					{\Delta \vdash A \istypewor 
																						& \Delta \vdash B \istypewor
																						& \mathcal O \in \{+,\times,\to \}} 										\and
\infertypeswor [\mu]		{\Delta \vdash \mu \alpha. A \istypewor}					{\Delta, \alpha \istypewor \vdash A \istypewor} 							\and
\infertypeswor [var]		{\Delta \vdash \alpha \istypewor}							{\alpha \istypewor \in \Delta} 												\and
\infertypeswor [\fut]		{\Delta \vdash \fut A \istypemix}							{\Delta \vdash A \istypetwo} 												\and
\infertypeswor [\fut]		{\Delta \vdash \curr A \istypemix}							{\Delta \vdash A \istypepure} 															
\end{mathpar}
\hrule
\begin{mathpar}
\fbox {Standard Typing} \and
\infertypeswor [\rmunit] 	{\ty {\tup{}}\rmunit}								{\cdot} 																	\and
\infertypeswor [int]		{\ty {i} \rmint}									{w \in \{\bbonep,\bbtwo\}}													\and
\infertypeswor [hyp]		{\ty x A}											{\col x A \in \Gamma} 														\and
\infertypeswor [\to I]		{\ty {\lam {x}{e}} {A \to B}}						{A \istypewor & \ty [,\col x A] e B} 										\and
\infertypeswor [\to E]		{\ty {\app {e_1}{e_2}} {B}}							{\ty {e_1} {A \to B} & \ty {e_2} A} 										\and
\infertypeswor [\times I]	{\ty {\tup{e_1,e_2}}{A\times B}}					{\ty {e_1} A & \ty {e_2} B} 												\and
\infertypeswor [\times E_1]	{\ty {\pio e} A}									{\ty e {A\times B}} 														\and
\infertypeswor [\times E_2]	{\ty {\pit e} B}									{\ty e {A\times B}} 														\and
\infertypeswor [+ I_1]		{\ty {\inl e} {A + B}}								{\ty e A} 																	\and
\infertypeswor [+ I_2]		{\ty {\inr e} {A + B}}								{\ty e B} 																	\and
\infertypeswor [\mu I]		{\ty {\roll e} {\mu \alpha.\tau}}					{\ty e {[\mualphatau / \alpha]\tau}} 										\and
\infertypeswor [\mu E]		{\ty {\unroll e} {[\mualphatau / \alpha]\tau}}		{\ty e \mualphatau} 														\and
\infertypeswor [+ E]		{\ty {\caseof{e_1}{x_2.e_2}{x_3.e_3}} C}			{\ty {e_1}{A+B} & \ty[,\col {x_2} A]{e_2} C & \ty[,\col {x_3} B]{e_3} C}
\end{mathpar}
\hrule
\begin{mathpar}
\fbox {Staging Features} \and
\infertypeswor [hold]		{\typesone {\pause e} {\fut \rmint}}				{\typesone e {\curr \rmint}}		 										\and
\infertypeswor [\fut I]		{\typesone {\next e}{\fut A}}						{\typestwo e A} 															\and
\infertypeswor [\fut E]		{\typestwo {\prev e} A}								{\typesone e {\fut A}} 														\and
\infertypeswor [\curr I]	{\typesone {\pure e} {\curr A}}						{\typespure e A} 															\and
\infertypeswor [\curr E]	{\typesone {\letp x {e_1} {e_2}} B}					{\typesone {e_1} {\curr A} & \typesone [\Gamma,\colpure x A] {e_2} B} 		\and
%\infertypeswor [lift] 		{\typesone {\lifttag e} {\curr A + \curr B}}		{\typesone e {\curr \tup{A+B}}} 											\and
\infertypeswor [+ E]		{\typesone {\caseP{e_1}{x_2.e_2}{x_3.e_3}} C}		{\typesone {e_1}{\curr(A+B)} 
																				&\typesone[\Gamma,\colmix {x_2} {\curr A}]{e_2} C 
																				&\typesone[\Gamma,\colmix {x_3} {\curr B}]{e_3} C} 							
\end{mathpar}
% \hrule
% \begin{mathpar}
% \fbox {Derivable $\curr$ Rules} \and
% \infertypeswor [\to E]		{\ty {\app {e_1}{e_2}} {\curr B}}					{\ty {e_1} {\curr\tup{A \to B}} & \ty {e_2} {\curr A}} 							\and
% \infertypeswor [\times E_1]	{\ty {\pio e} {\curr A}}							{\ty e {\curr\tup{A\times B}}}													\and
% \infertypeswor [\times E_2]	{\ty {\pit e} {\curr B}}							{\ty e {\curr\tup{A\times B}}}													\and
% \infertypeswor []			{\typesone {x} {\curr A}}							{\colpure x A \in \Gamma}													\and
% \infertypeswor []			{\typespure x A}									{\colone x {\curr A} \in \Gamma}													
% \end{mathpar}
\end{abstrsyn}
\caption{\lang~Static Semantics}
\label{fig:statics}
\end{figure*}


In this section, we describe \lang, a simple two-stage language
for which we define and analyze stage splitting.

%The grammar and type system for \lang\ is given in
%\ref{fig:grammar,fig:statics}.
The language \lang\ is essentially two copies of a typed lambda calculus with
products, sums, and isorecursive types.%
\footnote{It is possible for different stages to have different sets of
features, but for simplicity we do not consider this.}
Every valid expression has both a type and a stage, either \bbone~or \bbtwo,
expressing in which copy of the language the term resides. Intuitively, the
stage of a term expresses \emph{when} to evaluate it---all stage-\bbone\
subexpressions are evaluated before stage-\bbtwo\ ones.

In addition, there is a stage-\bbone\ type $\fut A$, containing encapsulated
stage-\bbtwo\ expressions of type $A$. Terms of type $\fut A$ are treated
opaquely by stage-\bbone\ code, as they cannot be evaluated until stage \bbtwo.
While $\fut$ allows us to embed stage-\bbtwo\ types within stage-\bbtwo\ types,
there is no way to embed a stage-\bbone\ type within a stage-\bbtwo\ type.
The $\next$ constructor embeds stage-\bbtwo\ expressions into stage \bbone,
while $\prev$ embeds stage-\bbone\ expressions into stage \bbtwo.  These are the
only ways in \lang\ to alter the stage of a term; we surround their arguments
with braces in \lang\ syntax to clearly indicate stage boundaries within a
program.

The grammar and type system for \lang\ is given in
\ref{fig:grammar,fig:statics}. (Typing judgments and context variables are annotated with stages.)
Only $\fut$ and its introductory and eliminatory forms $\next$ and $\prev$ affect the stage
of a term or type.
%We formulated our typing judgments in the style of \cite{davies96}, where the
%whole judgment is annotated with a stage.  
%The grammar and type system for \lang\ is given in
%\ref{fig:grammar,fig:statics}.
% We annotate typing judgments and context variables with stages;
%This is made manifest as rules which are entirely abstract over stage.
%In addition to determining the stage, $\next$ and $\prev$ are the introduction and elimination forms for $\fut$ types.
Specifically, given an argument with type $A$ at stage \bbtwo, $\next$ forms a $\fut A$ at stage \bbone.  
%That is, it forms the promise of a future $A$ out of a construction for an $A$ at the next timestep.
Stage \bbtwo\ expressions can obtain the original stage \bbtwo\ argument via the $\prev$ construct.  
Since $\prev$ operates at stage \bbtwo, this ensures no violation of causality\,\cite{cave14}.
The $\pause$ construct serves to wrap stage \bbone\ integers for use in stage \bbtwo.  
Although it is possible to implement $\pause$ from other \lang\ features, 
we provide it as a core primitive to simplify our examples. 

\subsection{Staged Evaluation}

As a simple example of expressing staged programs in \lang, consider the following fast exponentiation algorithm, which 
calculates $b^p$ in $\log p$ time:
\[
	\mathit{fexp}(b,p) = \left \{ \begin{array}{ll} 
		1 &  p = 0 \\ 
		\mathit{fexp}(b,p/2)^2 & p \text{ even} \\ 
		b \cdot \mathit{fexp}(b,p-1) & p \text{ odd} \end{array}
	\right .
\]

% Ignoring for the moment how to define recursive functions in \lang,
\noindent
The stage-\bbone\ term below defines the function {\tt fexp} that implements the fast exponentiation algorithm:%
%\footnote{We render $\fut$ as {\tt \$} in code examples.}

\begin{lstlisting} 
let fexp (b : $int, e : int) : $int =
	if e == 0 then
		next{1}
	else if (e mod 2) == 0 then
		next{let x = prev{fexp(b,p/2)} in x*x}
	else
		next{prev{b} * prev{fexp(b,p-1)}}		
\end{lstlisting}

Although the code for {\tt fexp} looks very much like the unstaged mathematical
definition above, it is in fact a staged program: the {\tt if} predicates and exponent
decomposition are all stage-\bbone\ terms, since they occur within $\prev$ blocks.
%\needsfix{Essentially, the type system is sufficient to prove the observation that all of
%the decomposition of the exponent can be done at the stage \bbone\ without
%requiring disruption of the elegant functional structure of the code.}
We note that our work does not consider the problem of {\em binding time analysis}---transforming an unstaged program into a staged \lang\ program by inserting appropriate $\next$, $\prev$, and $\pause$ constructs. We assume that binding time analysis is performed manually by a programmer (as done above) or via automatic analysis of unstaged code to arrive at a valid \lang\ program used as input for subsequent splitting.

A key attribute of the {\tt fexp} code example is the nesting of stage-\bbone\ and stage-\bbtwo\ expressions. Ordinary term evaluation eliminates outermost redexes first, 
however in the case that stage \bbone\ expressions are contained inside
stage \bbtwo\ ones (such as in the example above), this strategy conflicts with the precept of staged execution: that all stage \bbone\ code be evaluated before the evaluation of stage \bbtwo\ code. 
Accordingly, a suitably staged dynamic semantics for \lang\ must evaluate
\emph{all} the stage \bbone\ subexpressions of a term before its stage
\bbtwo\ subexpressions.

\subsection{Non-Duplicating First-Stage Evaluation}

\TODO clean up the next few paragraphs; unclear how much of it should go in the
paper

To understand the difficulty of staged evaluation, consider the following
example:
\begin{lstlisting}
#2 (next {1+2}, 3+4)
\end{lstlisting}
This is a stage-\bbone\ expression of type $\rmint$; the pair is a stage-\bbone\
expression of type $(\fut\rmint)\times\rmint$. A conventional call-by-value
evaluation strategy demands that we evaluate both components of the pair to
values before we project from the pair. However, at stage \bbone\ we are not
allowed to evaluate under the \verb|next|; on the other hand, \verb|next {1+2}|
is not a value, in the sense that it requires additional computation steps.

Intuitively, the solution is to say that \verb|next {1+2}| is a value \emph{at
stage \bbone}, even though once we proceed to stage-\bbtwo\ evaluation, it will
no longer be a value. Therefore, we evaluate the pair to
\begin{lstlisting}
#2 (next {1+2}, 7)
\end{lstlisting}
and get \verb|7| as the answer.

To complicate matters, consider a situation where we must substitute a stage
\bbone\ value into some other expression. In the stage-\bbtwo\ term
\begin{lstlisting} 
prev{
  let x = (next {1+2}, 3+4) in
  next{ prev{#1 x} * prev{#1 x} * hold{#2 x} }
}
\end{lstlisting}
our stage-\bbone\ evaluation will clearly yield a term which is not fully
reduced---since \verb|1+2| is needed to compute the end result, but we cannot
reduce it to \verb|3| until stage \bbtwo.

On the other hand, if we simply treat \verb|(next {1+2}, 7)| as a value, then we
will duplicate the \verb|1+2| computation in the term, and in fact we will
evaluate it twice, in each computation of \verb|#1 (next {1+2}, 7)|.

Our solution is to hoist these residuals, etc.

\vspace{2em}
\TODO old stuff below here. merge with above text:

Our goal is to evaluate away all of the stage-\bbone\ portions of this code, but there is more than one reasonable way to do this.
One option, taken by \cite{davies96}, is to treat \verb|next{1+2}| as a value for the purposes of stage-\bbone~evaluation, 
consequently duplicating it during substitution for {\tt x}.  The two $\prev$ constructs eliminate the $\next$ wrappers,
leaving the contained stage-\bbtwo\ code in place.
This produces:
\begin{lstlisting}
(1+2) * (1+2) * 7
\end{lstlisting}

We take a different approach.  
Instead of duplicating the contents of the first $\next$ expression, we bind them to some variable (here, $\mathtt{y}$) and
duplicate only a reference to that variable.  This produces:
\begin{lstlisting} 
let y = 1+2 in y * y * 7
\end{lstlisting}

Achieving this behavior mechanically requires us to resolve a contradiction:
we must substitute for \texttt{x} at stage \bbone, but we cannot evaluate inside the $\next$ block bound to it. 
Our solution is to replace the contents of the $\next$ with a new variable and create an explicit substitution (shown with a $\mapsto$) binding that variable to the $\next$'s old contents.  
This substitution then floats up to the top of the containing $\prev$ block:
\begin{lstlisting} 
prev {
[yhat|->1+2]
  let x = (next{yhat}, 7) in
  next{prev{#1 x} * prev{#1 x} *  hold{#2 x}}
}
\end{lstlisting}
As a convention, we render the new variable with a %stylish and fashionable
hat.  We're now free to perform the stage-\bbone~substitution for {\tt x} without duplicating stage-\bbtwo\ work.
\begin{lstlisting} 
prev {
[y|->1+2]
  next{
    prev{#1 (next {yhat}, 7)} * 
    prev{#1 (next {yhat}, 7)} *
    hold{#2 (next {yhat}, 7)}
  }
}
\end{lstlisting}
To evaluate the remaining $\next$, we must first partially evaluate the body by finding all of the contained stage-\bbone~terms and reducing them. 
As a rule, these will reduce to $\next$ expressions, which the $\prev$ eliminates, leaving the variable in place:
\begin{lstlisting} 
prev {
[yhat|->1+2]
    next{ yhat * yhat * 7 }
}
\end{lstlisting}
Once again, we lift the contents of the $\next$ into a substitution:
\begin{lstlisting} 
prev {
[yhat|->1+2]
[zhat|->yhat*yhat*7]
    next{ zhat }
}
\end{lstlisting}
Finally, when evaluating the outer $\prev$, we must {\em reify} the contained substitutions into let statements, yielding
\begin{lstlisting} 
let yhat = 1+2 in
let zhat = yhat * yhat * 7 in z
\end{lstlisting}

Thus we have evaluated all of the stage \bbone\ expressions of this program without duplicating the contents of $\next$ blocks.

\subsection{Dynamics}

%!TEX root = ../paper.tex

\begin{figure*}
\begin{abstrsyn}
\begin{mathpar}
\fbox{1-Evaluation}	\and
\inferdiaone[\rmunit]
{\red {\tup{}}{\cdot;\tup{}}}
{\cdot}
\and
\inferdiaone [\to I]
{\red {\lam{x}{e}} {\cdot;\lam{x}{e}}}
{\cdot}
\and
\inferdiaone [\to E]
{\red {\app {e_1}{e_2}} {\gcomp 1 2, \xi';v'}}
{\red {e_1} {\xi_1;\lam{x}{e'}} & \sub [2] & \red  [\Gamma,\dom{\xi_1},\dom{\xi_2}] {[v_2/x]e'}{\xi';v'}}
\and
%
\inferdiaone [\times I]
{\red {\tup{e_1,e_2}}{\gcomp 1 2;\valprod{v_1}{v_2}}}
{\sub [1] & \sub [2]}
\and
\inferdiaone [\times E_1]
{\red {\pio{e}}{\xi;v_1}}
{\red{e}{\xi;\valprod{v_1}{v_2}}}
\and
\inferdiaone [\times E_2]
{\red {\pit{e}}{\xi;v_2}}
{\red{e}{\xi;\valprod{v_1}{v_2}}}
\and
\inferdiaone [+ I_1]
{\red {\inl{e}} {\xi;\inl{v}}}
{\sub}
\and
\inferdiaone [+ I_2]
{\red {\inr{e}} {\xi;\inr{v}}}
{\sub}
\and
%
\inferdiaone [+ E_1]
{\red {\caseof{e_1}{x_2.e_2}{x_3.e_3}}{\gcomp 1 2;v_2}}
{\red {e_1}{\xi_1;\inl{v}} & \red
  [\Gamma,\dom{\xi_1}]{[v/x_2]e_2}{\xi_2;v_2}}
\and
\inferdiaone [+ E_2]
{\red {\caseof{e_1}{x_2.e_2}{x_3.e_3}}{\gcomp 1 3;v_3}}
{\red {e_1}{\xi_1;\inr{v}} & \red
  [\Gamma,\dom{\xi_1}]{[v/x_3]e_3}{\xi_3;v_3}}
\and
\inferdiaone[\mu I]
{\red {\roll{e}}{\xi;\roll{v}}}
{\sub}
\and
\inferdiaone[\mu E]
{\red {\unroll{e}}{\xi; v}}
{\red {e}{\xi; \roll{v}}}
\end{mathpar}

\hrule
\begin{mathpar}
\fbox{Residualization}
\and
\inferdiaspc[\rmunit]
{\red {\tup{}}{\tup{}}}
{\cdot}
\and
%\inferdiaspc[int]
%{\red {i}{i}}
%{\cdot}
%\and
%% %\inferdiaspc [bool]{\red {b}{b}}
%% {\cdot}
%% \and
\inferdiaspc[hyp]
{\red {x}{x}}
{\cdot}
\and
\inferdiaspc[\to I]
{\red {\lam{x}{e}}{\lam{x}{q}}}
{\diatwo [\Gamma,x] e q} 
\and
\inferdiaspc [\to E]
{\red {\app {e_1} {e_2}}{q_1~q_2}}
{\sub [1] & \sub [2]}
\and
\inferdiaspc [\times I]
{\red {\tup{e_1,e_2}}{\tup{q_1,q_2}}}
{\sub [1] & \sub [2]} 
\and
\inferdiaspc [C]
{\red {\scriptCapp e}{\scriptCapp q}}
{\sub & \scriptC \in \{\mathtt{pi1},\mathtt{pi2},\mathtt{inl},\mathtt{inr},\mathtt{roll},\mathtt{unroll}\}}
%\inferdiaspc[+ I_2]
%{\red {\inr~e}{\inr~q}}
%{\sub}
\and
\inferdiaspc[+ E_1]
{\red {\caseof{e_1}{x_2.e_2}{x_3.e_3}}
{\caseof{q_1}{x_2.q_2}{x_3.q_3}}}
{\sub [1] & \diatwo [\Gamma,x_2] {e_2} {q_2} & \diatwo [\Gamma,x_3] {e_3} {q_3}} 
%\inferdiaspc [\mu E]   		{\red {\unroll~e}{\unroll~v}}
%	{\sub}										\and
%\inferdiaspc [let]			{\red {\letin{x}{e_1}{e_2}}{\letin{x}{q_1}{q_2}}}			{\sub [1] & \diatwo [\Gamma,x] {e_2} {q_2}} 					\and
%\inferdiaspc [if_T] 			{\red {\ifthen{e_1}{e_2}{e_3}}{\ifthen{q_1}{q_2}{q_3}}}	{\sub [1] & \sub [2] & \sub [3]} 		\and
%
\end{mathpar}
\hrule
\begin{mathpar}
\fbox{Staging Features} \and
\inferdiaone [\fut I]	{\red {\next{e}}{\hat y \mapsto q;\next{\hat y}}}			{\diatwo e q}														\and
\inferdiaspc [\fut E]	{\red {\prev{e}} q}											{\diaone e {\xi; \next{\hat y}} & \reify{\xi}{\hat y}q}				\and
%\inferdiaone [hold]		{\red {\pause e} {\xi, \hat y \mapsto i; \next {\hat y}}}	{\red e {\xi; \pure i}}												\and
\infer					{\reify {\cdot}{q}{q}}										{\cdot}																\and
\infer					{\reify {y \mapsto q_1, \xi}{q_2}{\letin{y}{q_1}{q'}}}		{\reify{\xi}{q_2}{q'}}												\and
\inferdiaone			{\red {\pure e} {\cdot; \pure v}}							{\reduce e v} 														\and
\inferdiaone			{\red {\letp x {e_1} {e_2}} {\xi_1, \xi_2; v_2}}			{\red {e_1} {\xi_1;\pure {v_1}} & \red {[v_1/x]e_2} {\xi_2;v_2}} 	\and
%\inferdiaone			{\red {\lifttag e} {\xi;\inl{\pure v}}}						{\red e {\xi; \pure{\inl v}}}										\and
%\inferdiaone			{\red {\lifttag e} {\xi;\inr{\pure v}}}						{\red e {\xi; \pure{\inr v}}}										
\inferdiaone [+ E_1]	{\red {\caseP{e_1}{x_2.e_2}{x_3.e_3}}{\gcomp 1 2;v_2}}		{\red {e_1}{\xi_1;\pure{\inl{v}}} 
																					&\red [\Gamma,\dom{\xi_1}]{[\pure v/x_2]e_2}{\xi_2;v_2}}			\and
\inferdiaone [+ E_2]	{\red {\caseP{e_1}{x_2.e_2}{x_3.e_3}}{\gcomp 1 3;v_3}}		{\red {e_1}{\xi_1;\pure{\inr{v}}}
																					&\red [\Gamma,\dom{\xi_1}]{[\pure v/x_3]e_3}{\xi_3;v_3}}		
\end{mathpar}

\end{abstrsyn}
\caption{\lang~Dynamic Semantics.}
\label{fig:diaSemantics}
\end{figure*}


%{{{ stuff nico commented out
%Previous work (\cite{davies96}) focuses on the correspondence between the type system and existing temporal logics, whereas we care more about operational behavior and cost.  In this section, we'll consider a few proposals for our language before settling on one we like.  All of the proposals are call-by-value, differing primarily in how they handle values of $\fut$ type.

%\subsection{The Erasure Semantics}
%
%We first consider the {\em erasure semantics}, so called because it corresponds to what one would get by interpreting \lamStaged as a single-stage language, ignoring all of the $\next$ and $\prev$ terms.  This gives us two judgments, $\erasone$ and $\erastwo$, corresponding to {\em multistage evaluation} at \bbone and at \bbtwo.  We call these judgments  ``multistage'' because they cause work to happen at both stages.
%
%Both judgments behave normally at non-staged features.  We cover their behavior at staged features below:
%
%\begin{mathpar}
%\infer {\next~e \erasone \next~v} {e \erastwo v} \and
%\infer {\prev~e \erastwo \next~v} {e \erasone v} \and
%\infer {\pause~e \erasone \next~i} {e \erasone i}
%\end{mathpar}
%
%Essentially, we immediately evaluate under the $\next$, yielding a value for $\fut$ types. The $\prev$ terms remove this wrapper.  As expected, $\pause$ also gives us a way to inject into the wrapper at integers.
%
%The erasure semantics has some undesirable properties.  By intention, it interleaves the execution of stage-\bbone~and stage-\bbtwo~code, so the evaluation can't really be said to be staged (i.e. stage-\bbone~work is done before stage-\bbtwo~work).  Moreover, the erasure semantics cannot be equivalent to any semantics which does have this property!  To see why, consider the following code, which types to int at \bbtwo:
%
%\begin{lstlisting} 
%if 5*4*3*2 > 111 then
%	hold{2+4}
%else
%	prev{ loopForever () (* does not terminate *) }
%\end{lstlisting}
%
%Under the erasure semantics (using $\erastwo$), this code takes the top branch and evaluates to 6.  But in order know that the {\tt loopForever} function need not be called, the predicate had to be evaluated prior.  But the predicate is stage-\bbtwo, whereas {\tt loopForever} is stage-\bbone.  To borrow terminology from \cite{cave14}, this violates causality.  In order to avoid this problem, a valid semantics must {\em speculate} down the branches of any stage-\bbtwo~if or case statement (or similarly into the body of a stage-\bbtwo function) to find and evaluate all of the stage-\bbone~code.  Both of the other semantics we will consider have this property.
%
%The benefit of the erasure semantics is that it's very natural, and has a familiar cost model.  It also {\em obviously} produces the ``correct'' answer, so we can use the erasure semantics as a reference to prove the reasonableness of any other semantics.
%
%\subsection{Meta Semantics}
%
%A different semantics was provided in \cite{davies96}.  We briefly review a two-stage version of that semantics here.
%
%In the erasure semantics, $\next v$ is a value only if $v$ is fully reduced.  But in the meta semantics, $\next e$ is a value only if $e$ has no $\prev$ terms; $v$ is allowed to have unreduced stage-2 computation. 
%
%...
%
%The Davies semantics is comprised of two mutually recursive judgments: $\daviesz$ and $\davieso$.  For some $\colone e A$, the $\daviesz$ judgment evaluates only the first-stage parts of $e$, leaving unevaluated second-stage code within $\next$s.  (... This essentially gives us partial evaluation, and we're left to just evaluate the residual normally to get multi-stage evaluation...) 
%
%The benefit of the meta semantics is that it gives us a very explicit notion of partial evaluation.  This also, by construction, means that the meta semantics does all of the first stage-work {\em before} the second stage work begins. 
%
%The $\next$-by-name behavior of the meta semantics of course means that it happily duplicates second-stage code, which could increase the cost.  This makes reasoning about second-stage cost rather difficult.
%
%\begin{lstlisting} 
%let x = next {4+5} in
%next{prev{x} * prev{x}}
%\end{lstlisting}
%
%\subsection{Our Semantics}
%
%We desire a semantics that meets the following goals:
%
%\begin{enumerate}
%\item Modulo termination, it should be equivalent to the erasure semantics.
%\item All of the first stage work should be completed before second stage work.  Ideally, it should just have a notion of partial evaluation, like the meta semantics.
%\item Should be $\next$-by-value, rather than $\next$-by-name, like the erasure semantics.  
%\end{enumerate}
%
%We meet all of these goals.
%}}}
%{{{ stuff carlo commented out
%Abstractly, we can think of our evaluation as proceeding in the standard way for
%stage-1 code. When the evaluator encounters a $\next \{e\}$ expression, it
%places $e$ off to the side in a context and replaces the whole expression with a
%reference to the context entry.  These references are then passed around as
%stage-1 values for $\fut$ types.  But what if $e$ contains $\prev$ expressions?
%To ensure that all stage-1 code is evaluated before any stage-2 code, we must
%evaluate all of the 1-code contained in $e$ before inserting it into the table.
%This entails searching $e$ for all contained $\prev$s and evaluating them in
%place.  

%The evaluation judgment operates on stage-1 code, whereas the
%speculation operates on stage-2 code.  Since $\next$ and $\prev$ are the
%crossover points between 1-code and 2-code, they are correspondingly the only
%places where the evaluation and speculation judgments depend on the other. 
%}}}

The algorithm described above creates three different kinds of expressions which
cannot be evaluated further at a particular stage:
\begin{itemize}
\item 
Partial values ($\pvalsym$s) are stage \bbone\ terms that have been fully evaluated, 
but which may contain stage-\bbtwo\ variables wrapped in $\next$ blocks. 
In the example above, 
\verb|(|$\next \{\mathtt{\hat y}\}$\verb|,7)|.
\item Residuals ($\ressym$es) are stage \bbtwo\ terms whose stage \bbone\
subexpressions have all been fully evaluated. In the example above,
\verb|(1+2)| and \verb|(let z = y*y*7 in z)|.
\item Values ($\valsym$s) are stage \bbtwo\ terms which are fully evaluated; these
are the results of a computation after both stages have been completed. In the
example above, the term evaluates to \verb|63|.
\end{itemize}
The $\redonesym$ judgment takes an open stage \bbone\ term to a {\em future environment} $\xi$ and a partial value $v$.
The future environment is a mapping from newly-created variables 
(which then may appear within $\next$ blocks in $v$)
to residuals.  It is the formal manifestation of the floating substitutions in our example.
For all of the normal features of \lang (\ref{fig:diaSemanticsCore}), first stage evaluation has the same behavior and effect on values as standard evaluation,
and the final future environment is gotten by merging the future environments of the subterms.

When this judgment encounters a $\next$ block (\ref{fig:diaSemanticsNP}), it searches into the block's stage \bbtwo\ content to find any contained stage
\bbone\ subexpressions and evaluate them in place.  
We call this search process \emph{speculation}. It is implemented by the $\redtwosym$ judgment, which takes a stage \bbtwo\ term to a residual.
Once the contents of the $\next$ block are speculated into a residual ($q$), 
we then return a fresh variable wrapped in a $\next$ block ($\next~\hat y$), 
along with a future environment which maps that variable to our residual ($\hat y \mapsto q$).

At all of the normal features (\ref{fig:diaSemanticsSpec}), speculation does nothing but recursively speculate into every subexpression.
Once speculation finds a $\prev$ block, we resume stage \bbone\ evaluation of the contents, which produces a future context and (by canonical forms) a $\next$-wrapped variable ($\next\{\hat y\}$).
The context is then reified (using the $\reifysym$ judgment) into a series of let bindings, with $\mathtt{\hat y}$---stripped of its $\next$---at the bottom.

The context ($\Gamma$) keeps track of stage \bbtwo\ variables in the input term. 
These both appear in the original program at stage \bbtwo\ and are inserted by the semantics.

As an optimization, we can include the special-case rule,
\begin{mathpar}
\inferdiaone [hat] {\red {\next~\hat y}{\cdot,\next~\hat y}}{\cdot}
\end{mathpar}
to avoid one-for-one variable bindings in the residual.
We used this implicitly in the example in the previous section.

If we change the $\next$ and $\prev$ rules to 
\begin{mathpar}
\infer {\diaone {\next~e}{\cdot,\next~q}} {\diatwo e q} \and
\infer {\diatwo{\prev~e}{v}} {\diaone e {\cdot,\next~v}} 
\end{mathpar}
and treat $\next~q$ as a partial value, where $q$ is a residual,
then we get precisely the semantics from \cite{davies96}. This essentially bypasses the
environment bookkeeping in $\redonesym$, by inlining residuals instead of
hoisting them in \verb|let|-bindings.
From this it's clear that the semantics of \cite{davies96} and ours always produce the same value when they both terminate.
But moreover, because a reified residual will always be evaluated whether or not its result is consumed, 
our semantics terminates strictly less often than that of \cite{davies96}.

\subsection {A Partial Evaluation System}

A function $f$ which accepts inputs at both stages \bbone{} and \bbtwo{} can be
given a type of the form $A\to\fut(B\to C)$.%
\cprotect\footnote{We can rewrite \texttt{fexp} in this form, or simply apply
the following higher-order function which makes the adjustment:
\begin{lstlisting} 
let adjust (f : $int * int -> $int) =
  fn (p : int) => 
    next{
      fn (b : int) => 
        prev{f (next {b}, p)}
    }
\end{lstlisting}}
%
Given its stage \bbone\ argument $a:A$, we can evaluate the partially-applied
function:
$\cdot\vdash f~a \mathop{\redonesym} [\xi,v]$.
The result is an environment $\xi$ and a partial value $v$ of type $\fut(B\to
C)$. Next, we reify this environment into a sequence of \verb|let|-bindings
enclosing $v$, via $\reify\xi{v}{f_a}$. A canonical forms theorem on $v$
ensures that the resulting $f_a$ has type $B\to C$ in the residual language.
Finally, given a stage \bbtwo{} argument $b:B$, we can stage-\bbtwo{} evaluate
the ultimate result of the function, $f_a~b \mathop{\tworedsym} c$.

\TODO make sure this is true (haven't written the canonical forms theorems yet)

That this sequence of evaluations is in fact staged follows from our
characterizations of partial values, residuals, and values, that $\redonesym$
outputs a partial value, and that $\reifysym$ outputs an expression in \langTwo.

\begin{remark}
For any $\colone{e}{A}$ containing no $\next$ subexpressions, $\redonesym$ will
always compute an empty environment, and a partial value identical to the result
of call-by-value evaluation of $e$.
%derivationally equivalent to standard call-by-value evaluation.
\end{remark}

\subsection{Metatheory}

\begin{definition}
Contexts $\Gamma$ is well-formed ($\Gamma\wf$) if it
contains only stage-2 variables.
\end{definition}

\begin{definition}
An environment $\xi$ is well-formed ($\Gamma\vdash\xi\wf$) if either:
\begin{enumerate}
\item $\xi = \cdot$; or
\item $\xi = \xi',x\mapsto e$ where
$\Gamma\vdash\xi'\wf$ and
$\Gamma,\dom{\xi'}\vdash \coltwo{e}{B}$ and
$\Gamma,\dom{\xi'}\vdash e \res$.
\end{enumerate}
\end{definition}

\begin{theorem}
If $\typesone e A$ then $\Gamma\wf$ and $A\istypeone$.
\end{theorem}

\begin{theorem}
If $\diaonesub$ and $\typesone e A$ then
\begin{enumerate}
\item $\Gamma\vdash\xi\wf$;
\item $\Gamma,\dom\xi\vdash \colone{v}{A}$; and
\item $\Gamma,\dom\xi\vdash v\pval$.
\end{enumerate}
\end{theorem}

\begin{theorem}
If $\diatwosub$ and $\typestwo e A$ then
\begin{enumerate}
\item $\Gamma\vdash \coltwo{q}{A}$; and
\item $\Gamma\vdash q\val$.
\end{enumerate}
\end{theorem}


