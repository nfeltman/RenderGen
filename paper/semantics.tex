%!TEX root = paper.tex

\section{\texorpdfstring{\lang}{λ12} Statics and Dynamics}
\label{sec:semantics}

\begin{figure}
\textbf{Languages:}
\begin{itemize}
\item \lang: two-staged lambda calculus with product, sum, and
  recursive types
\item \langmono: an unstaged lambda calculus with products.
\end{itemize}

\textbf{\lang\ Evaluation Relations:}
\begin{itemize}
\item 
\bbone-Evaluation: $e\mathbin{\redonesym}[\xi;v]$, where $\xi$ is a \emph{residual table} and $v$ is a \emph{partial value}. 

\item
\bbtwo-Evaluation: $e\mathbin{\redtwosym}q$, where $q$ is a \emph{residual} in the unstaged language \langTwo

\item 
Table reification: \ur{Fill in...}
\end{itemize}



% $red_2$: 2-eval,  which we think of as being identical to specialization. 

\vspace{.75em}
\textbf{\lang\ Splitting:}

\hspace{2em}\bbone-Splitting Structure: $e \splitonesym [c,l.r]$, where:

\hspace{4em}$c$ is the \emph{combined term} (representing all stage~\bbone\ subcomputations in $e$)

\hspace{4em}$l.r$ is the \emph{resumer} (representating all stage~\bbtwo\ subcomputations in $e$)

\hspace{4em}$c~\redsym~(y,b)$, where $y$ is the \emph{\bbone-result} of $e$, and $b$ is the \emph{boundary value} of $e$

 
\hspace{2em}\bbone-Splitting Correctness: If $e\mathbin{\redonesym}[\xi;v]$, then:

\hspace{4em}$y$ is identical to $\masko{\xi;v}$

\hspace{4em}$\maskt{\xi;v}$ and \texttt{(let l=b in r)} reduce (via $\redsym$) to identical values.  

\hspace{2em}\bbtwo-Splitting Structure: $e \splittwosym [p,l.r]$, where:

\hspace{4em}$p$ is the \emph{precomputation} (representing all stage~\bbone\ subcomputations in $e$)

\hspace{4em}$l.r$ is the \emph{resumer} (representating all stage~\bbtwo\ subcomputations in $e$)

\hspace{2em}\bbtwo-Splitting Correctness: If $e~\redtwosym~q$, and $q~\redsym~v$, then \texttt{(let l=b in r)}~$\redsym~v$

\caption{Summary of \lang\ evaluation and splitting.}
\label{fig:terminology}
\label{fig:termSplitSummary}
\end{figure}



%!TEX root = ../paper.tex

\begin{figure}
\begin{abstrsyn}
\[\begin{aligned}
\text{Values}\ \ 
u &::= \tup{}
 \gbar \lam {\var}{e} \\
&\gbar \tup{u, u}
 \gbar \inl u 
 \gbar \inr u \\
\text{Partial Values}\ \ 
v &::= \tup{}
 \gbar \lam {\var}{\expr} \\
&\gbar \tup{v, v}
 \gbar \inl v 
 \gbar \inr v \\
&\gbar \next {\hat y}
 \gbar \pure u \\
\text{Residuals}\ \ 
q &::= \tup{}\gbar\var%\bool
 \gbar \lam {\var}{q} 
 \gbar \app q q \\
&\gbar \tup{q, q} 
 \gbar \pio q 
 \gbar \pit q \\
&\gbar \inl q 
 \gbar \inr q
 \gbar \caseof {q}{\var.q}{\var.q} \\
&\gbar \roll e
 \gbar \unroll e\\
\text{Residual Table}\ \ 
\xi &::= \cdot
 \gbar \xi, {\hat y} \mapsto q \\
\end{aligned}\]
\end{abstrsyn}
\caption{Outputs of first-stage evaluation.}
\label{fig:values}
\end{figure}




%
%
%\begin{figure}
%\caption{\ellStaged~Syntax}
%\label{fig:ellStagedSyntax}
%\centering
%\begin{tabular}{ll} 
%$\begin{aligned}
%\typeo &::= \text{unit}~|~\text{int}~|~\text{bool} \\
%&\gbar \typeo \times \typeo \\
%&\gbar \fut \typet \\
%\expro &::= ()~|~\inte~|~\bool  \\
%&\gbar \letin{\var}{\expro}{\expro} \\
%&\gbar \var \\
%&\gbar (\expro, \expro) \\
%&\gbar \pi_1~\expro \gbar \pi_2~\expro \\
%&\gbar \ifthen {\expro}{\expro}{\expro} \\
%&\gbar \next~\exprt \\
%\contextot &::=\emptyC \\
%&\gbar \contextot, \var : \typeo ^\bbone \\
%&\gbar \contextot, \var : \typet ^\bbtwo
%\end{aligned} $ 
%& 
%$\begin{aligned}
%\typet &::=  \text{unit}~|~\text{int}~|~\text{bool} \\
%&\gbar \typet \times \typet \\
%\\
%\exprt &::= ()~|~\inte~|~\bool \\
%&\gbar \letin{\var}{\exprt}{\exprt} \\
%&\gbar \var \\
%&\gbar (\exprt, \exprt) \\
%&\gbar \pi_1~\exprt \gbar \pi_2~\exprt \\
%&\gbar \ifthen {\exprt}{\exprt}{\exprt} \\
%&\gbar \prev~\expro \\
%\\
%\\
%\\
%\end{aligned} $
%\end{tabular}
%\end{figure}

%
%\begin{figure}
%\caption{\ellTarget~Syntax}
%\label{fig:ellTargetSyntax}
%\centering
%\begin{tabular}{ll} 
%$\begin{aligned}
%\expr &::= ()~|~\inte~|~\bool \\
%&\gbar \letin{\var}{\expr}{\expr} \\
%&\gbar \var \\
%&\gbar (\expr, \expr) \\
%&\gbar \pi_1~\expr \gbar \pi_2~\expr \\
%&\gbar \inl~\expr \gbar \inr~\expr \\
%&\gbar \ifthen {\expr}{\expr}{\expr}  \\
%&\gbar \caseof {\expr}{x_1.\expr}{x_2.\expr} 
%\end{aligned} $
%& 
%$\begin{aligned}
%\type &::=  \rmunit~|~\text{int}~|~\text{bool} \\
%&\gbar \type \times \type  \\
%&\gbar \type + \type 
%\\
%\context &::= \emptyC \\
%&\gbar \context, \var : \type
%\\ \\ \\ \\
%\end{aligned} $
%\end{tabular}
%\end{figure}


\begin{figure}
\centering
$\begin{aligned}
\world &::= \bbone \gbar \bbtwo \\
\type &::= \text{unit}~|~\text{int}~|~\text{bool} \\
&\gbar \type \times \type 
 \gbar \type + \type \\
&\gbar \type \to \type
 \gbar \fut \type \\
&\gbar \alpha \gbar \mu \alpha.\tau \\
\expr &::= ()\gbar\inte\gbar\bool\gbar \var  \\
&\gbar \lam{\var}{\type}{\expr} 
 \gbar \expr~\expr \\
&\gbar (\expr, \expr) 
 \gbar \pio~\expr 
 \gbar \pit~\expr \\
&\gbar \inl~\expr 
 \gbar \inr~\expr \\
&\gbar \caseof {\expr}{\var.\expr}{\var.\expr} \\
&\gbar \ifthen {\expr}{\expr}{\expr} \\
&\gbar \letin{\var}{\expr}{\expr} \\
&\gbar \next~\expr 
 \gbar \prev~\expr 
 \gbar \pause~\expr \\
\context &::=\emptyC \gbar \context, \colwor \var \type
\end{aligned} $
\caption{\lamStaged~Syntax}
\label{fig:lamStagedSyntax}
\end{figure}

We express two-stage programs as terms in \lang, a typed, modal lambda calculus
with products, sums, and isorecursive types. Its type system has three distinct
\emph{worlds} \bbonep, \bbonem, and \bbtwo, which respectively classify
\emph{ground first-stage} computations (\bbonep); 
first-stage computations containing a \emph{mix} of code from each stage
(\bbonem); 
and second-stage computations (\bbtwo).

\subsection{Statics}

%!TEX root = ../paper.tex

%\begin{figure}
%\caption{\lang~Valid Types}
%\label{fig:validTypes}
%\end{figure}

\begin{figure*}
\begin{abstrsyn}
\begin{mathpar}
\fbox {Valid Types} \and
\infertypeswor [\rmunit] 	{\Delta \vdash {\tt unit} \istypewor}						{\cdot} 																	\and
\infertypeswor [int]		{\Delta \vdash {\tt int} \istypewor}						{w \in \{\bbonep,\bbtwo\}}													\and
\infertypeswor [\times]		{\Delta \vdash A~\mathcal{O}~B \istypewor}					{\Delta \vdash A \istypewor 
																						& \Delta \vdash B \istypewor
																						& \mathcal O \in \{+,\times,\to \}} 										\and
\infertypeswor [\mu]		{\Delta \vdash \mu \alpha. A \istypewor}					{\Delta, \alpha \istypewor \vdash A \istypewor} 							\and
\infertypeswor [var]		{\Delta \vdash \alpha \istypewor}							{\alpha \istypewor \in \Delta} 												\and
\infertypeswor [\fut]		{\Delta \vdash \fut A \istypemix}							{\Delta \vdash A \istypetwo} 												\and
\infertypeswor [\fut]		{\Delta \vdash \curr A \istypemix}							{\Delta \vdash A \istypepure} 															
\end{mathpar}
\hrule
\begin{mathpar}
\fbox {Standard Typing} \and
\infertypeswor [\rmunit] 	{\ty {\tup{}}\rmunit}								{\cdot} 																	\and
\infertypeswor [int]		{\ty {i} \rmint}									{w \in \{\bbonep,\bbtwo\}}													\and
\infertypeswor [hyp]		{\ty x A}											{\col x A \in \Gamma} 														\and
\infertypeswor [\to I]		{\ty {\lam {x}{e}} {A \to B}}						{A \istypewor & \ty [,\col x A] e B} 										\and
\infertypeswor [\to E]		{\ty {\app {e_1}{e_2}} {B}}							{\ty {e_1} {A \to B} & \ty {e_2} A} 										\and
\infertypeswor [\times I]	{\ty {\tup{e_1,e_2}}{A\times B}}					{\ty {e_1} A & \ty {e_2} B} 												\and
\infertypeswor [\times E_1]	{\ty {\pio e} A}									{\ty e {A\times B}} 														\and
\infertypeswor [\times E_2]	{\ty {\pit e} B}									{\ty e {A\times B}} 														\and
\infertypeswor [+ I_1]		{\ty {\inl e} {A + B}}								{\ty e A} 																	\and
\infertypeswor [+ I_2]		{\ty {\inr e} {A + B}}								{\ty e B} 																	\and
\infertypeswor [\mu I]		{\ty {\roll e} {\mu \alpha.\tau}}					{\ty e {[\mualphatau / \alpha]\tau}} 										\and
\infertypeswor [\mu E]		{\ty {\unroll e} {[\mualphatau / \alpha]\tau}}		{\ty e \mualphatau} 														\and
\infertypeswor [+ E]		{\ty {\caseof{e_1}{x_2.e_2}{x_3.e_3}} C}			{\ty {e_1}{A+B} & \ty[,\col {x_2} A]{e_2} C & \ty[,\col {x_3} B]{e_3} C}
\end{mathpar}
\hrule
\begin{mathpar}
\fbox {Staging Features} \and
\infertypeswor [hold]		{\typesone {\pause e} {\fut \rmint}}				{\typesone e {\curr \rmint}}		 										\and
\infertypeswor [\fut I]		{\typesone {\next e}{\fut A}}						{\typestwo e A} 															\and
\infertypeswor [\fut E]		{\typestwo {\prev e} A}								{\typesone e {\fut A}} 														\and
\infertypeswor [\curr I]	{\typesone {\pure e} {\curr A}}						{\typespure e A} 															\and
\infertypeswor [\curr E]	{\typesone {\letp x {e_1} {e_2}} B}					{\typesone {e_1} {\curr A} & \typesone [\Gamma,\colpure x A] {e_2} B} 		\and
%\infertypeswor [lift] 		{\typesone {\lifttag e} {\curr A + \curr B}}		{\typesone e {\curr \tup{A+B}}} 											\and
\infertypeswor [+ E]		{\typesone {\caseP{e_1}{x_2.e_2}{x_3.e_3}} C}		{\typesone {e_1}{\curr(A+B)} 
																				&\typesone[\Gamma,\colmix {x_2} {\curr A}]{e_2} C 
																				&\typesone[\Gamma,\colmix {x_3} {\curr B}]{e_3} C} 							
\end{mathpar}
% \hrule
% \begin{mathpar}
% \fbox {Derivable $\curr$ Rules} \and
% \infertypeswor [\to E]		{\ty {\app {e_1}{e_2}} {\curr B}}					{\ty {e_1} {\curr\tup{A \to B}} & \ty {e_2} {\curr A}} 							\and
% \infertypeswor [\times E_1]	{\ty {\pio e} {\curr A}}							{\ty e {\curr\tup{A\times B}}}													\and
% \infertypeswor [\times E_2]	{\ty {\pit e} {\curr B}}							{\ty e {\curr\tup{A\times B}}}													\and
% \infertypeswor []			{\typesone {x} {\curr A}}							{\colpure x A \in \Gamma}													\and
% \infertypeswor []			{\typespure x A}									{\colone x {\curr A} \in \Gamma}													
% \end{mathpar}
\end{abstrsyn}
\caption{\lang~Static Semantics}
\label{fig:statics}
\end{figure*}


\begin{abstrsyn}

The typing judgment $\typeswor e A$, defined in \ref{fig:statics}, means that
$e$ has type $A$ at world $w$, in the context $\Gamma$. (Variables in the
context are also annotated with the world at which they live.)

All three worlds contain unit, product, sum, function, and recursive types
defined in the usual fashion. For example, given two terms $\typesone{e_1}A$ and
$\typesone{e_2}B$ at world \bbonem, one can form the pair
$\typesone{\texttt{<}e_1,e_2\texttt{>}}{A\times B}$ also at world \bbonem. 

Differing worlds (and hence, differing stages of computation) interact by means
of the $\fut$ and $\curr$ type formers. $\fut A$ is a type in world \bbonem\
which classifies second-stage computations of type $A$. Given a term $e$ of type
$A$ at world \bbtwo, $\next{e}$ has type $\fut A$ at \bbonem. This essentially
encapsulates $e$ as a computation to be evaluated in the future---first-stage
computations can shuffle it around, but not use its result.  The only way to use
a $\fut A$ is to wrap it with $\prev$, yielding an $A$ at \bbtwo.

%This is why well-typed terms cannot have any information flow from stage
%\bbtwo\ to stage \bbone, and what makes splitting possible.

Although our products and functions are restricted to types at the same world,
$\fut$ essentially allows us to build ``mixed-stage'' products and functions.
For example, quickselect is a function at world \bbonem\ which takes a
$\curr{\rm list} \times \fut\rmint$ (a purely-first-stage list and a
second-stage computation of an integer) to a $\fut\rmint$ (a second-stage
computation of an integer).

$\curr A$ is a type in world \bbonem\ which classifies purely-first-stage
computations of type $A$. Given a world \bbonep\ term $e$ of type $A$,
$\pure{e}$ has type $\curr A$ at world \bbonem. ($e$ is guaranteed not to contain
second-stage computations because $\fut$ types are not available in world
\bbonep.)

An $e$ of type $\curr A$ at \bbonem\ can be unwrapped as an $A$ at \bbonep\
using the $\letp xe{e'}$ construct, which binds $\colpure xA$ in a \bbonem\ term
$e'$. This allows us to compute under $\curr$---for example, given a $\colmix
p{\curr(A\times B)}$, the term $\letp{x}{p}{\pure{\pio x}}$ computes its first
projection, of type $\curr A$.

Lastly, to case on a term $e$ of type $\curr(A+B)$ at world \bbonem, we need the
$\caseP{e}{x.e_1}{x.e_2}$ construct, whose branches are world \bbonem\ terms
open on $\curr{A}$ and $\curr{B}$ respectively.

The example code in this paper makes liberal use of $\rmint$s and various
functions on these, as well as a function $\pause$ which takes a $\curr\rmint$
to a $\fut\rmint$.

\end{abstrsyn}

\subsection{Dynamics}
\label{sec:stagedsemantics}

%!TEX root = ../paper.tex

\begin{figure*}
\begin{abstrsyn}
\begin{mathpar}
\fbox{1-Evaluation}	\and
\inferdiaone[\rmunit]
{\red {\tup{}}{\cdot;\tup{}}}
{\cdot}
\and
\inferdiaone [\to I]
{\red {\lam{x}{e}} {\cdot;\lam{x}{e}}}
{\cdot}
\and
\inferdiaone [\to E]
{\red {\app {e_1}{e_2}} {\gcomp 1 2, \xi';v'}}
{\red {e_1} {\xi_1;\lam{x}{e'}} & \sub [2] & \red  [\Gamma,\dom{\xi_1},\dom{\xi_2}] {[v_2/x]e'}{\xi';v'}}
\and
%
\inferdiaone [\times I]
{\red {\tup{e_1,e_2}}{\gcomp 1 2;\valprod{v_1}{v_2}}}
{\sub [1] & \sub [2]}
\and
\inferdiaone [\times E_1]
{\red {\pio{e}}{\xi;v_1}}
{\red{e}{\xi;\valprod{v_1}{v_2}}}
\and
\inferdiaone [\times E_2]
{\red {\pit{e}}{\xi;v_2}}
{\red{e}{\xi;\valprod{v_1}{v_2}}}
\and
\inferdiaone [+ I_1]
{\red {\inl{e}} {\xi;\inl{v}}}
{\sub}
\and
\inferdiaone [+ I_2]
{\red {\inr{e}} {\xi;\inr{v}}}
{\sub}
\and
%
\inferdiaone [+ E_1]
{\red {\caseof{e_1}{x_2.e_2}{x_3.e_3}}{\gcomp 1 2;v_2}}
{\red {e_1}{\xi_1;\inl{v}} & \red
  [\Gamma,\dom{\xi_1}]{[v/x_2]e_2}{\xi_2;v_2}}
\and
\inferdiaone [+ E_2]
{\red {\caseof{e_1}{x_2.e_2}{x_3.e_3}}{\gcomp 1 3;v_3}}
{\red {e_1}{\xi_1;\inr{v}} & \red
  [\Gamma,\dom{\xi_1}]{[v/x_3]e_3}{\xi_3;v_3}}
\and
\inferdiaone[\mu I]
{\red {\roll{e}}{\xi;\roll{v}}}
{\sub}
\and
\inferdiaone[\mu E]
{\red {\unroll{e}}{\xi; v}}
{\red {e}{\xi; \roll{v}}}
\end{mathpar}

\hrule
\begin{mathpar}
\fbox{Residualization}
\and
\inferdiaspc[\rmunit]
{\red {\tup{}}{\tup{}}}
{\cdot}
\and
%\inferdiaspc[int]
%{\red {i}{i}}
%{\cdot}
%\and
%% %\inferdiaspc [bool]{\red {b}{b}}
%% {\cdot}
%% \and
\inferdiaspc[hyp]
{\red {x}{x}}
{\cdot}
\and
\inferdiaspc[\to I]
{\red {\lam{x}{e}}{\lam{x}{q}}}
{\diatwo [\Gamma,x] e q} 
\and
\inferdiaspc [\to E]
{\red {\app {e_1} {e_2}}{q_1~q_2}}
{\sub [1] & \sub [2]}
\and
\inferdiaspc [\times I]
{\red {\tup{e_1,e_2}}{\tup{q_1,q_2}}}
{\sub [1] & \sub [2]} 
\and
\inferdiaspc [C]
{\red {\scriptCapp e}{\scriptCapp q}}
{\sub & \scriptC \in \{\mathtt{pi1},\mathtt{pi2},\mathtt{inl},\mathtt{inr},\mathtt{roll},\mathtt{unroll}\}}
%\inferdiaspc[+ I_2]
%{\red {\inr~e}{\inr~q}}
%{\sub}
\and
\inferdiaspc[+ E_1]
{\red {\caseof{e_1}{x_2.e_2}{x_3.e_3}}
{\caseof{q_1}{x_2.q_2}{x_3.q_3}}}
{\sub [1] & \diatwo [\Gamma,x_2] {e_2} {q_2} & \diatwo [\Gamma,x_3] {e_3} {q_3}} 
%\inferdiaspc [\mu E]   		{\red {\unroll~e}{\unroll~v}}
%	{\sub}										\and
%\inferdiaspc [let]			{\red {\letin{x}{e_1}{e_2}}{\letin{x}{q_1}{q_2}}}			{\sub [1] & \diatwo [\Gamma,x] {e_2} {q_2}} 					\and
%\inferdiaspc [if_T] 			{\red {\ifthen{e_1}{e_2}{e_3}}{\ifthen{q_1}{q_2}{q_3}}}	{\sub [1] & \sub [2] & \sub [3]} 		\and
%
\end{mathpar}
\hrule
\begin{mathpar}
\fbox{Staging Features} \and
\inferdiaone [\fut I]	{\red {\next{e}}{\hat y \mapsto q;\next{\hat y}}}			{\diatwo e q}														\and
\inferdiaspc [\fut E]	{\red {\prev{e}} q}											{\diaone e {\xi; \next{\hat y}} & \reify{\xi}{\hat y}q}				\and
%\inferdiaone [hold]		{\red {\pause e} {\xi, \hat y \mapsto i; \next {\hat y}}}	{\red e {\xi; \pure i}}												\and
\infer					{\reify {\cdot}{q}{q}}										{\cdot}																\and
\infer					{\reify {y \mapsto q_1, \xi}{q_2}{\letin{y}{q_1}{q'}}}		{\reify{\xi}{q_2}{q'}}												\and
\inferdiaone			{\red {\pure e} {\cdot; \pure v}}							{\reduce e v} 														\and
\inferdiaone			{\red {\letp x {e_1} {e_2}} {\xi_1, \xi_2; v_2}}			{\red {e_1} {\xi_1;\pure {v_1}} & \red {[v_1/x]e_2} {\xi_2;v_2}} 	\and
%\inferdiaone			{\red {\lifttag e} {\xi;\inl{\pure v}}}						{\red e {\xi; \pure{\inl v}}}										\and
%\inferdiaone			{\red {\lifttag e} {\xi;\inr{\pure v}}}						{\red e {\xi; \pure{\inr v}}}										
\inferdiaone [+ E_1]	{\red {\caseP{e_1}{x_2.e_2}{x_3.e_3}}{\gcomp 1 2;v_2}}		{\red {e_1}{\xi_1;\pure{\inl{v}}} 
																					&\red [\Gamma,\dom{\xi_1}]{[\pure v/x_2]e_2}{\xi_2;v_2}}			\and
\inferdiaone [+ E_2]	{\red {\caseP{e_1}{x_2.e_2}{x_3.e_3}}{\gcomp 1 3;v_3}}		{\red {e_1}{\xi_1;\pure{\inr{v}}}
																					&\red [\Gamma,\dom{\xi_1}]{[\pure v/x_3]e_3}{\xi_3;v_3}}		
\end{mathpar}

\end{abstrsyn}
\caption{\lang~Dynamic Semantics.}
\label{fig:diaSemantics}
\end{figure*}


\begin{abstrsyn}

We evaluate \lang\ terms in two stages. The first stage of evaluation:
\begin{enumerate}
\item fully reduces terms in \bbonep, as they are purely first-stage;
\item reduces terms in \bbonem, skipping over any world \bbtwo\ terms they may
contain (inside $\next$);
\item does not reduce terms in \bbtwo, but scans them for any \bbonem\ terms
they may contain (inside $\prev$) and reduces those.
\end{enumerate}
These tasks are accomplished by the $\redsym$, $\redonesym$, and $\redtwosym$
big-step judgments, respectively (all defined in \ref{fig:diaSemantics}).

This process ultimately produces a monostaged term, called a \emph{residual},
all of whose redexes are second-stage computations. The second stage of
evaluation is therefore just ordinary (unstaged) evaluation of this residual,
which we notate $\redsym$. (We assume there is a single notion of unstaged
evaluation applicable to both \bbonep\ terms and residuals. For space reasons,
we omit its definition.)

Notice that our type system makes it possible to evaluate terms in a staged
fashion, but doing so still requires some machinery, since the stages are
syntactically interleaved within an individual term. For example,
\texttt{qsStaged} contains a first-stage recursive call inside a second-stage
\texttt{case}; thus, an ordinary outermost-first evaluation strategy would not
be properly staged.

\paragraph{First-stage evaluation at \bbonem\ ($\redonesym$)}
The difficulty of evaluating terms at \bbonem\ is that they contain both terms
at \bbonep\ (via $\curr$) and at \bbtwo\ (via $\fut$); the other evaluation
rules are ordinary call-by-value rules. When we encounter $\pure{e}$, we
evaluate $e$ as a monostaged term; \texttt{letg} and \texttt{caseg} simply reach
inside $\pure$ and evaluate like \texttt{let} and \texttt{case}.

$\next{e}$ is trickier, because we should not evaluate $e$ (at \bbtwo) yet.
There are two difficulties---$e$ may contain other code at \bbonem\ (which we
handle with $\redtwosym$, discussed below); and $\next{e}$ may be part of a
first-stage redex, which we \emph{do} need to reduce. For example, in
\begin{lstlisting} 
1`(fn x : $int => e') (next{`2`e`1`})`
\end{lstlisting}
we must reduce this first-stage function application, but cannot reduce $e$ yet.
We could reduce this to $[\next{e}/x]e'$, but this may potentially duplicate an
expensive computation $e$. Instead, we choose to hoist $e$ outside, binding it
to a temporary variable $\hat{y}$, and substituting that variable instead:
\begin{lstlisting} 
2`let yhat = e in `1`[next{`2`yhat`1`}/x]e'`
\end{lstlisting}
then proceed by evaluating $[\next{\hat{y}}/x]e'$.

We achieve this by threading through a \emph{residual table} $\xi$, which
functions as a top-level sequence of \texttt{let}-bindings of residuals $e$ to
variables $\hat{y}$ in an evaluated \bbonem\ term. Thus, $\redonesym$ evaluation
yields a pair $[\xi;v]$ of a residual table and a \emph{partial value} $v$
containing variables bound in that table. The grammar of residual tables and
partial values (terms fully evaluated at the first but not second stage) is
given in \ref{fig:values}.

%The input to $\redonesym$ may be open on stage \bbtwo\ variables, but the type
%system ensures that those must occur under a $\next$. Consequently $\redonesym$
%will never directly encounter a variable.

\paragraph{First-stage evaluation at \bbtwo\ ($\redtwosym$)}
Since terms at world \bbtwo\ represent second-stage computations, we should not
evaluate them yet. However, we do need to evaluate any first-stage computations
they contain (inside $\prev$), even if those computations are underneath
binders.

For all constructs other than $\prev{e}$, $\redtwosym$ evaluation proceeds
recursively on all subterms, yielding a monostaged residual $q$. At $\prev{e}$,
we evaluate $e$ using $\redonesym$ to $[\xi;v]$. Because $\colmix e{\fut A}$,
$v$ must be $\next{\hat{y}}$ for some $\hat{y}$ bound in $\xi$. We \emph{reify}
$\xi$ into a sequence of second-stage \texttt{let} bindings around $\hat{y}$,
using the auxiliary judgment $\reify\xi q{q'}$ defined in
\ref{fig:diaSemantics}, then return the resulting residual $q'$.

\end{abstrsyn}

%Note that partial values and residuals are both terms for which stage \bbone\
%computation has completed; however, partial values are stage~\bbone\ terms,
%while residuals are stage~\bbtwo\ terms.

%As an optimization, we can include the special-case rule,
%\begin{mathpar}
%\inferdiaone [hat] {\red {\next~\hat y}{\cdot,\next~\hat y}}{\cdot}
%\end{mathpar}
%to avoid one-for-one variable bindings in the residual.

\subsection {Top-Level Evaluation}

\crem{TODO below here}

We can use the evaluation relations in the dynamic semantics to
evaluate a term to evaluate a \lang\ term as follows. We first apply
1-evaluation to obtain a partial value and a residual table. We then
construct a 2-term from the partial value and the residual table by
using reification, and then reduce this term to a residual via
residualization.  Finally, we evaluate the residual, which is a term in a
our monostaged target language \langmono, to a value by using  a
standard evaluation relation for \langmono.

We can evaluate a \lang\ term as
follows: we first apply 1-evaluation to obtain a partial value and a
residual table, we then construct a 2-term from the partial value and
the residual table by using reification, and then reduce this term to
a residual via residualization.

These dynamics allow us to define a partial evaluator for \lang\ by identifying
static with stage \bbone\ and dynamic with stage \bbtwo.  Specifically, we
encode $f$ as a \lang\ expression with a function type of the form
$A\to\fut(B\to C)$.%

Here $A$ is the static input, $B$ is the dynamic input, and $C$ is the output.

Once a stage \bbone\ argument $a:A$ is provided, we can evaluate the partially-applied
function:
$\cdot\vdash f~a \mathop{\redonesym} [\xi,v]$.
The result is an environment $\xi$ and a partial value $v$ of type $\fut(B\to
C)$, which by canonical forms must have the form $v = \next~\hat y$. 
Next, we reify this environment into a sequence of \verb|let|-bindings
enclosing $\hat y$, via $\reify\xi{\hat y}{f_a}$. 
Because reification preserves types, the resulting residual $f_a$ has type $B\to C$ in \langTwo, so we can apply it to some $b:B$
and compute the final result of the function, $f_a~b \mathop{\tworedsym} c$.

That this sequence of evaluations is in fact staged follows from our
characterizations of partial values and residuals, that $\redonesym$
outputs a partial value, and that $\reifysym$ outputs an expression in \langTwo.

For example, to specialize quickselect to a particular list, 
\begin{lstlisting}
qsStaged [5,2,7,4,1]
\end{lstlisting}
\TODO finish example

%We can rewrite \texttt{fexp} in this form, or simply apply
%the following higher-order function which makes the adjustment:
%\begin{lstlisting} 
%let adjust (f : $int * int -> $int) =
%  fn (p : int) => 
%    next{
%      fn (b : int) => 
%        prev{f (next {b}, p)}
%    }
%\end{lstlisting}
