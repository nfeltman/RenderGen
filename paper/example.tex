%!TEX root = paper.tex

\section {Example: Staged Quickselect in \texorpdfstring{\lang}{λ12}}
\label{sec:example}

\begin{abstrsyn}

%!TEX root = ../paper.tex

\begin{figure*}
\begin{subfigure}{0.5\textwidth}
\begin{lstlisting} 
datatype list = Empty | Cons of int * list

fun part (p, Empty) = (0,Empty, Empty) 
  | part (p, Cons (h,t)) = 
    let val (s,left,right) = part (p,t) in 
    if h<p 
    then (s+1,Cons(h,left),right) 
    else (s,left,Cons(h,right))

fun qSelect (Empty, k) = 0
  | qSelect (Cons ht, k) =
    let val (n,left,right) = part ht in
    case compare k n of
      LT => qSelect (left, k)
    | EQ => #1 ht
    | GT => qSelect (right, k-1-n)
\end{lstlisting}
\caption{Unstaged implementation of quickselect.}
\label{fig:qs-unstaged}
\end{subfigure}%
\begin{subfigure}{0.5\textwidth}
\begin{lstlisting} 
1`atsignpure{
  datatype list = Empty | Cons of int * list
  fun part (...) = ...
} 

`//1`qsStaged : ^list * $`2`int`1` -> $`2`int`1`
fun qsStaged (Empty,_) = next {`2`0`1`}
  | qsStaged (Cons ht,next{`2`k`1`}) = 
    let 
      val (n,le,ri) = part ht
      val next{`2`n`1`} = hold n 
    in next{`2`
      case compare k n of
        LT => prev {`1`qsStaged (le, next{`2`k`1`})`2`}
      | EQ => prev {`1`hold (#1 ht)`2`}
      | GT => prev {`1`qsStaged (ri, next{`2`k-1-n`1`})`2`}`1`
    }`
\end{lstlisting}
\caption{Staged implementation of quickselect in \lang.}
\vspace{1.3em}
\label{fig:qs-staged}
\end{subfigure}
\caption{Quickselect: traditional and staged.}
\end{figure*}




Suppose that we wish to issue order statistics queries on a collection of items
represented as a list \texttt{l}. We can use quickselect \cite{Hoare:1961}, an
expected linear time algorithm which takes a list \texttt{l} and a rank
\texttt{k} and returns the element with rank \texttt{k} in \texttt{l}.

Quickselect, which we define in \ref{fig:qs-unstaged}, first partitions the list
by using the first element as a pivot%
\footnote{We assume that the list is prepermuted to guarantee the expected
linear time behavior.}
and then recurs on one of the two resulting sides to find the desired element.
The side chosen is determined by the relationship of \texttt{k} to the size of
the first half \texttt{n}, which is returned by the \texttt{part}ition function
along with the two sides themselves.

Now, consider an application where we perform many order statistics queries on
the same collection \texttt{l}, but with $m$ different ranks
$\mathtt{k_1},\dots,\mathtt{k_m}$.
Certainly, it is possible to implement this with $m$ calls to
\texttt{quickselect}:
%
\begin{lstlisting}
quickselect l @$\mathtt{k_1}$@
quickselect l @$\mathtt{k_2}$@
 @$\vdots$@ 
quickselect l @$\mathtt{k_m}$@.
\end{lstlisting}
%
Can we do better?

\subsection{Staging}




An astute programmer might notice that much of the code in \texttt{quickselect}
does \emph{not} depend on the rank \texttt{k}; for example, the list is
partitioned before \texttt{k} is ever used. Moreover, while the recursive calls
to \texttt{qSelect} are guarded by a comparison to \texttt{k}, \texttt{k} does
not determine the arguments to those recursive calls---it only determines which
call is made. Thus, if we are willing to change the evaluation order of the
language (and evaluate under the \texttt{case} statement), it should be possible
to force \emph{all} the computations involving \texttt{l} to be performed before
\emph{any} of the computations involving \texttt{k}.

We can formalize this intuition by writing quickselect in a \emph{staged}
language. In this paper, we choose \lang\ (\ref{sec:semantics}), a staged, typed
lambda calculus. This allows us to directly express the idea that the argument
\texttt{l} is known at the \emph{first} stage of the computation, while the
argument \texttt{k} is only known at the \emph{second}, and all
first-stage computations occur before any second-stage computations.

We define a staged version of quickselect in \ref{fig:qs-staged}, writing
first-stage computations in red, and second-stage computations in blue. While
\texttt{qSelect} sends a \textrm{list} and $\rmint$ to an $\rmint$,
\texttt{qsStaged} has a more precise type---it is a first-stage function which
takes a $\curr\mathrm{list}$ (an integer list now) and a $\fut\rmint$
(an $\rmint$ \emph{in the future}), and returns a $\fut\rmint$.

Ignoring for the moment all \texttt{grnd} annotations, the first-stage code in
\texttt{qsStaged} looks like \texttt{qSelect}: in the first stage, terms of
non-circle type are available for immediate use. Indeed, we case on the list as
usual, and in the \texttt{Cons} branch, \texttt{part}ition it.

The \texttt{case} expression in \texttt{qSelect} depends on \texttt{k}, whose
type $\fut\rmint$ indicates that it is only available to second-stage
computations. Since \texttt{qsStaged} itself produces a $\fut\rmint$, the
remainder of the function is second-stage code.

The $\next$ wraps a second-stage expression of type $\rmint$ (the \texttt{case}
expression) as a first-stage expression of type $\fut\rmint$ (the result 
of \texttt{qsStaged}). Inside the \texttt{LT} and \texttt{GT} branches, the
$\prev$ unwraps first-stage $\fut\rmint$s (the results of the recursive calls
to \texttt{qsStaged}) as second-stage $\rmint$s. In the \texttt{EQ} branch,
$\pause$ promotes a first-stage $\rmint$ (the head of \texttt{l}) directly to a
second-stage $\rmint$.

The process of adding staging annotations ($\fut$ types, $\next$, and $\prev$)
to unstaged code has been the subject of extensive research under the
name of \emph{binding time analysis}. In this paper, we assume that these
annotations have been provided by the programmer (or perhaps a binding time
analysis tool), and do not consider the problem of generating such annotations.
There are many ways to annotate any program, including \texttt{qSelect}; we
chose annotations which maximize the work performed in the first stage.

\lang's type system, discussed in \ref{sec:semantics}, ensures that the staging
annotations are consistent, in the sense that computations marked as
first-stage cannot depend on ones marked as second-stage. The $\curr$ and
\texttt{grnd} (``ground'') annotations further distinguish those first-stage computations
which do not contain any embedded second-stage code. (Code \emph{inside}
\texttt{grnd}, like the \texttt{part}ition function, is guaranteed not to
contain second-stage code; first stage code outside a \texttt{grnd}, 
like \texttt{qsStaged}, may.) We will
discuss the importance of this additional distinction in \ref{sec:splitting}.



\subsection{Splitting Staged Programs}

\begin{figure}
%\begin{subfigure}{0.5\textwidth}
\begin{lstlisting}
1`datatype list = Empty | Cons of int * list
fun partition (p : int, l : list) = ...`
	
datatype tree = Branch of int * int * tree * tree
                | Leaf

1`fun qSelect1 (l : list) : tree =
  case l of
    Empty => Leaf
  | Cons ht => 
      let (left,right,n) = partition ht in
      Branch (n, #1 ht, qSelect1 left, qSelect1 right)`

2`fun qSelect2 (p : tree, k : int) : int = 
  case p of
    Leaf => 0
  | Branch (n,h,p1,p2) => 
      case compare k n of
        LT => qSelect2 (p1,k)
      | EQ => h
      | GT => qSelect2 (p2,k-n-1)`
\end{lstlisting}
\caption{Split (two-phase) implementation of quickselect.}
\label{fig:qs-split}
%\end{subfigure}
%\caption{Caption place holder}
\end{figure}



It is well-known that for any given input list with $n$ elements, the
quick-select algorithm requires $\Theta(n)$ expected time (for any
second argument) if we select keys for the partition function
uniformly randomly or identically if we permute the input. Under this
assumption, our example program requires expected $\Theta(nm)$ time.

Suppose now that $m$ is large (close to $n$ for example).  In this
case, our program is essentially a quadratic-time algorithm, which can
be very slow even for modest values of $m$ and $n$. In this case, to
improve the run-time it appears that we have no choice but rewrite our
program to use a different algorithm.  One way to do this would be to
pre-sort the input list $l$ and then lookup the element with the
desired rank (as specified by the second argument).  While this
approach might sound sensible, attaining the desired performance
requires answering one more important question: what should the data
structure for storing the sorted input be?  Unfortunately using a list
does not work, because each look up in a list requires linear time on
average. One option is to use arrays but in this case, we would need
to implement a binary search to perform the lookup.  Another option is
to implement a balanced binary search tree data structure for storing
the elements in the list and then use balanced binary-search-tree
lookups, which are logarithmic time.  Using either one of these
techniques, we can reduce the run-time to $\Theta(n\log{n} + m\log{n})$,
which is significantly better that $\Theta(n^2)$.


In summary, to achieve the desired improvement, we wrote a two-pass
program, where the first pass sorts the input and the second pass
performs a binary search either on an array or a balanced tree. As is
probably apparent to the reader, the resulting program is both
algorithmically and structurally more complex: it requires conversions
between different data structures (list to array or binary search
tree) and requires implementing additional algorithms such as sorting
and binary search or possibly binary search trees.  

%% As suggested by this relatively small example and experienced for
%% example by the graphics community, the design, implementation, and
%% maintenance of such multi-pass programs can be difficult due to the
%% increased complexity.


Instead of relying an writing multipass programs manually, in this
paper, we present techniques for automatically splitting staged
programs into multi-pass programs.  The key innovation behind our
approach is a {\em splitting algorithm} that splits a multistage
program into separate passes where each pass is a conventional program
expressed in a conventional functional language.  Our splitting
algorithm makes no restrictions on the input program: essentially any
functional program, including higher order programs, can be split.
Furthermore, the splitting algorithm is able to perform highly
non-trivial transformations on the program that can improve efficiency
significantly. 

When applied to the quick-select example as described, our splitting
algorithm yields a two-pass program that uses the binary-search-tree
based implementation outlined above.  Specifically, in the first pass,
the program takes the input list and constructs a probabilistically
balanced binary search tree, which is isomorphic to a treap data
structure~\cite{treaps}.  In the second pass, the program performs,
for each rank, a binary search tree lookup, by walking the tree to
find the element with the desired rank.  The resulting program, shown
in \ref{fig:????}  takes the $O(n\log{n} + m\log{n})$.

To create the multi-pass algorithm, the splitting algorithm operates
by composing local transformations on the subterms of the input
program.  In particular, the algorithm has no special knowledge of the
quick-select algorithm, binary search trees, or how to perform lookups
on binary search trees, but it is able to derive all of these from the
input program.  But how?
%
The details of the algorithm are relatively involved but in the hope
that it aids understanding we present a brief informal overview of how
the algorithm operates.  
%
%% This exposition is necessarily imprecise but the algorithm is made
%% precise in \ref{sec:splitting} and implemented
%% (\ref{sec:implementation}{sec:examples}).
%


At a high level, the splitting algorithm operates on our example by
using the program code and the knowledge that the list (the first
argument) is first-stage and the rank ( the second argument) is second
stage, the algorithm then derives two functions: \texttt{qsStage1} and
\texttt{qsStage2} as follows.


\paragraph{Pass-1 code.}
The splitting algorithm scans the program and imitates an execution of
\texttt{qs} with just the first-stage input by evaluating as much code
as possible with the available first-stage values.  Specifically the
algorithm generates a function \texttt{qsStage1} to perform all the
recursive calls of \texttt{qs} and to evaluate all instances of the
partition function, which depend only on the input list.  The function
\texttt{qsStage1} produces a trace of the execution of \texttt{qs} on
its first input by generating a tree that collects the results from
all recursive calls along with a tag that indicates the control
branches taken.  The trace is represented as a tree because
\texttt{qs} has a binary control structure (casing on the list), which
leads to a binary tree data structure.  To ensure that the trace
contains the necessary results to complete the execution in the second
pass, where second-stage values may be used, the function
\texttt{qsStage1} includes in the nodes of the tree information such
as the key used for splitting the list at that recursive call.  The
resulting tree is thus a binary search tree, keyed by the ``pivot''
used by the partition.  In general we refer to the result data
structure of \texttt{qsStace1} as the {\em boundary}, as it passes
information from the first pass to the second.

\paragraph{Pass-2 code}
As the splitting algorithm scans \texttt{qs} for computations that can
be performed in the first stage and collects them into
\texttt{qsStage1}, it also collects computations that must be left to
the second stage in a separate function \texttt{qsStage2}.  This
function, which is executed in the second pass, takes as argument the
boundary and the second-stage argument (the rank) and performs a
lookup in the boundary data structure.  But how does \texttt{qsStage2}
knows to traverse the tree? It does not. Since the splitting algorithm
recorded the control flow of the first stage execution in the boundary
data structure, \texttt{qsStage2} simply follows this control
structure and performs at each point the parts of the computation from
\texttt{qs} that can now be performed in the presence of the
second-stage argument (the rank).  In the context of our example, this
performs a lookup on the boundary data structure by using the supplied
rank.  

\paragraph{Summary.}
In summary, the splitting algorithm scans the program code for
first-stage computations (which depend only on first-stage values) and
separates them into function of the first pass. This function performs
the stage-1 computations and places the results into a boundary data
structure that both records the control flow and the results from the
first stage computation at each control-flow point.  As the splitting
function generates the code for the first pass, it also generates the
code for the second pass as a function that walks both the boundary
data (which essentially imitates execution) and at each control point
performs the part of the computation that depends on second-stage
values. We evaluate the resulting program by first evaluating the
function for the first pass and passing its return value (the
boundary) to the second pass, which then returns the desired result.



\begin{comment}

An astute programmer, having noticed that \texttt{quickselect} can be staged in
this fashion, might try to split it into a pair of functions, one which performs
all the work depending only on \texttt{l} (the first stage), and one which uses
that partial result and \texttt{k} to compute the element with
rank \texttt{k} in \texttt{l}. 

Intuitively, \texttt{l} determines the result of all calls to
\texttt{part}ition, and \texttt{k} only determines which calls are made. So we
can preprocess \texttt{l} by recursively dividing it into halves smaller and
greater than the pivot---that is, building a binary search tree. Then, once we
have \texttt{k}, we can recur on this tree, choosing whichever branch has the
\texttt{k}${}^\textit{th}$ leftmost element until we reach a leaf. And because
\texttt{part}ition contains no second-stage code, we can run it entirely in the
first stage.

We have implemented this splitting of quickselect in \ref{fig:qs-split}.
\texttt{qSelect1} builds a binary search tree from the list \texttt{l}, and
\texttt{qSelect2} takes such a tree and a rank \texttt{k} and computes the
answer. This allows us to efficiently perform many order statistics queries on
\texttt{l} by caching the tree and reusing it for many different ranks
$\mathtt{k_1},\dots,\mathtt{k_m}$:
%
\begin{lstlisting}
let b = qSelect1 l in
  qSelect2 b @$k_1$@
  qSelect2 b @$k_2$@
   @$\vdots$@ 
  qSelect2 b @$k_m$@.
\end{lstlisting}

Assuming \texttt{l} contains $n$ elements, this optimization changes the
asymptotic complexity from expected (randomized) $\Theta(n \cdot m)$ to
$\Theta(n\log{n} + m\log{n})$, which for any $m \approx n$ reduces the
complexity from $\Theta (n^2)$ to $\Theta(n\log{n})$---a significant improvement. 

In this paper, we develop a splitting algorithm
(\ref{sec:splitting,sec:implementation}) which, given a program $e$ in \lang,
produces an equivalent pair of programs which correspond precisely to the two
stages of computation in $e$.
(Splitting is always possible because the staging annotations in $e$ are
consistent, because $e$ is well-typed in \lang.) In the case of
\texttt{qsStaged}, our splitting algorithm produces the algorithm described
above.
\end{comment}
%\texttt{qSelect l k = qSelect2 (qSelect1 l) k}.

%Because the tree passes information across the stage boundary, we call it the
%\emph{boundary data structure}.

%Note that the desired optimized code shows above is intellectually more
%sophisticated than the code that we have started with: the optimized code is
%able to create a data structure, a balanced binary tree augmented with indexing
%information, and use a binary search technique over this tree to compute the
%result asymptotically more efficiently.

%In fact, based our teaching experience, we can imagine this kind of problem to
%be a moderately difficult exam question in an undergraduate algorithms class, as
%it not only requires understanding of data structures such as binary search
%trees but also requires modifying them to augment with indexing information to
%support rank-based search.

%(I also want to make it clear that recognizing \lang's appropriateness for this
%is itself is a contribution.)

\subsection{Quickselect in more detail}

Carlo writes this.



\end{abstrsyn}
