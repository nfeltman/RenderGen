%!TEX root = paper.tex

\section {Example: Staged Quickselect in \texorpdfstring{\lang}{λ12}}
\label{sec:example}

\begin{abstrsyn}

%!TEX root = ../paper.tex

\begin{figure*}
\begin{subfigure}[t]{0.45\textwidth}
\begin{lstlisting} 

datatype list = Empty | Cons of int * list

fun partition (p, Empty) = (0,Empty, Empty) 
  | partition (p, Cons (h,t)) = 
    let val (s,left,right) = partition (p,t) 
    in if h < p then (s+1,Cons(h,left),right) 
             else (s,left,Cons(h,right))

qs: list * int -> int
fun qs (Empty, k) = 0
  | qs (Cons ht, k) =
    let val (i,left,right) = partition ht

    in case compare k m of
         LT => qs (left, k)
       | EQ => #1 ht
       | GT => qs (right, k-i-1)
\end{lstlisting}
\caption{Unstaged implementation of quickselect.}
\label{fig:qs-unstaged}
\end{subfigure}
\hfill
\begin{subfigure}[t]{0.55\textwidth}
\begin{lstlisting} 
3`atsigng{`  
1`datatype list = Empty | Cons of int * list

fun partition (p, Empty) = ...  (* Same as in unstaged *)



3`}`.

1`qss : ^list * $`2`int`1` -> $`2`int`1`
fun qss (`3`g{`1`Empty`3`}`1`,_) = next {`2`0`1`}
  | qss (`3`g{`1`Cons ht`3`}`1`,next{`2`k`1`}) = 
    let val `3`g{`1`(i0,left,right)`3`}`1` = `3`g{`1`partition ht`3`}`1`
        val next{`2`i`1`} = hold `3`g{`1`i0`3`}`1`
    in next{`2` case compare k i of
               LT => prev {`1`qss (`3`g{`1`left`3`}`1`, next{`2`k`1`})`2`}
             | EQ => prev {`1`hold `3`g{`1`#1 ht`3`}`2`}
             | GT => prev {`1`qss (`3`g{`1`right`3`}`1`, next{`2`k-i-1`1`})`2`}`1`}`
\end{lstlisting}
\caption{Staged implementation of quickselect in \lang.}

%\vspace{1.3em}
\label{fig:qs-staged}
\end{subfigure}
\caption{Quickselect: traditional and staged.}
\end{figure*}



Suppose that we wish to issue order statistics queries on a collection of items
represented as a list \texttt{l}. We can use quickselect \cite{Hoare:1961}, an
expected linear time algorithm which takes a list \texttt{l} and a rank
\texttt{k} and returns the element with rank \texttt{k} in \texttt{l}.

Quickselect, which we define in \ref{fig:qs-unstaged}, first partitions the list
by using the first element as a pivot%
\footnote{We assume that the list is prepermuted to guarantee the expected
linear time behavior.}
and then recurs on one of the two resulting sides to find the desired element.
The side chosen is determined by the relationship of \texttt{k} to the size of
the first half \texttt{n}, which is returned by the \texttt{part}ition function
along with the two sides themselves.

Now, consider an application where we perform many order statistics queries on
the same collection \texttt{l}, but with $m$ different ranks
$\mathtt{k_1},\dots,\mathtt{k_m}$.
Certainly, it is possible to implement this with $m$ calls to
\texttt{quickselect}:
%
\begin{lstlisting}
quickselect l @$\mathtt{k_1}$@
quickselect l @$\mathtt{k_2}$@
 @$\vdots$@ 
quickselect l @$\mathtt{k_m}$@.
\end{lstlisting}
%
Can we do better?

\subsection{Staging}

Umut edits this.

An astute programmer might notice that much of the code in \texttt{quickselect}
does \emph{not} depend on the rank \texttt{k}; for example, the list is
partitioned before \texttt{k} is ever used. Moreover, while the recursive calls
to \texttt{qSelect} are guarded by a comparison to \texttt{k}, \texttt{k} does
not determine the arguments to those recursive calls---it only determines which
call is made. Thus, if we are willing to change the evaluation order of the
language (and evaluate under the \texttt{case} statement), it should be possible
to force \emph{all} the computations involving \texttt{l} to be performed before
\emph{any} of the computations involving \texttt{k}.

We can formalize this intuition by writing quickselect in a \emph{staged}
language. In this paper, we choose \lang\ (\ref{sec:semantics}), a staged, typed
lambda calculus. This allows us to directly express the idea that the argument
\texttt{l} is known at the \emph{first} stage of the computation, while the
argument \texttt{k} is only known at the \emph{second}, and all
first-stage computations occur before any second-stage computations.

We define a staged version of quickselect in \ref{fig:qs-staged}, writing
first-stage computations in red, and second-stage computations in blue. While
\texttt{qSelect} sends a \textrm{list} and $\rmint$ to an $\rmint$,
\texttt{qsStaged} has a more precise type---it is a first-stage function which
takes a $\curr\mathrm{list}$ (an integer list now) and a $\fut\rmint$
(an $\rmint$ \emph{in the future}), and returns a $\fut\rmint$.

Ignoring for the moment all \texttt{grnd} annotations, the first-stage code in
\texttt{qsStaged} looks like \texttt{qSelect}: in the first stage, terms of
non-circle type are available for immediate use. Indeed, we case on the list as
usual, and in the \texttt{Cons} branch, \texttt{part}ition it.

The \texttt{case} expression in \texttt{qSelect} depends on \texttt{k}, whose
type $\fut\rmint$ indicates that it is only available to second-stage
computations. Since \texttt{qsStaged} itself produces a $\fut\rmint$, the
remainder of the function is second-stage code.

The $\next$ wraps a second-stage expression of type $\rmint$ (the \texttt{case}
expression) as a first-stage expression of type $\fut\rmint$ (the result 
of \texttt{qsStaged}). Inside the \texttt{LT} and \texttt{GT} branches, the
$\prev$ unwraps first-stage $\fut\rmint$s (the results of the recursive calls
to \texttt{qsStaged}) as second-stage $\rmint$s. In the \texttt{EQ} branch,
$\pause$ promotes a first-stage $\rmint$ (the head of \texttt{l}) directly to a
second-stage $\rmint$.

The process of adding staging annotations ($\fut$ types, $\next$, and $\prev$)
to unstaged code has been the subject of extensive research under the
name of \emph{binding time analysis}. In this paper, we assume that these
annotations have been provided by the programmer (or perhaps a binding time
analysis tool), and do not consider the problem of generating such annotations.
There are many ways to annotate any program, including \texttt{qSelect}; we
chose annotations which maximize the work performed in the first stage.

\lang's type system, discussed in \ref{sec:semantics}, ensures that the staging
annotations are consistent, in the sense that computations marked as
first-stage cannot depend on ones marked as second-stage. The $\curr$ and
\texttt{grnd} (``ground'') annotations further distinguish those first-stage computations
which do not contain any embedded second-stage code. (Code \emph{inside}
\texttt{grnd}, like the \texttt{part}ition function, is guaranteed not to
contain second-stage code; first stage code outside a \texttt{grnd}, 
like \texttt{qsStaged}, may.) We will
discuss the importance of this additional distinction in \ref{sec:splitting}.

\subsection{Splitting Staged Programs}

\begin{figure}
%\begin{subfigure}{0.5\textwidth}
\begin{lstlisting}
1`datatype list = Empty | Cons of int * list
fun partition (p : int, l : list) = ...`
	
datatype tree = Branch of int * int * tree * tree
                | Leaf

1`fun qSelect1 (l : list) : tree =
  case l of
    Empty => Leaf
  | Cons (h,t) => 
      let (left,right,n) = partition h t in
      Branch (n, h, qSelect1 left, qSelect1 right)`

2`fun qSelect2 (p : tree, k : int) : int = 
  case p of
    Leaf => 0
  | Branch (n,h,p1,p2) => 
      case compare k n of
        LT => qSelect2 (p1,k)
      | EQ => h
      | GT => qSelect2 (p2,k-n-1)`
\end{lstlisting}
\caption{Split (two-phase) implementation of quickselect.}
\label{fig:qs-split}
%\end{subfigure}
%\caption{Caption place holder}
\end{figure}



An astute programmer, having noticed that \texttt{quickselect} can be staged in
this fashion, might try to split it into a pair of functions, one which performs
all the work depending only on \texttt{l} (the first stage), and one which uses
that partial result and \texttt{k} to compute the element with
rank \texttt{k} in \texttt{l}. 

Intuitively, \texttt{l} determines the result of all calls to
\texttt{part}ition, and \texttt{k} only determines which calls are made. So we
can preprocess \texttt{l} by recursively dividing it into halves smaller and
greater than the pivot---that is, building a binary search tree. Then, once we
have \texttt{k}, we can recur on this tree, choosing whichever branch has the
\texttt{k}${}^\textit{th}$ leftmost element until we reach a leaf. And because
\texttt{part}ition contains no second-stage code, we can run it entirely in the
first stage.

We have implemented this splitting of quickselect in \ref{fig:qs-split}.
\texttt{qSelect1} builds a binary search tree from the list \texttt{l}, and
\texttt{qSelect2} takes such a tree and a rank \texttt{k} and computes the
answer. This allows us to efficiently perform many order statistics queries on
\texttt{l} by caching the tree and reusing it for many different ranks
$\mathtt{k_1},\dots,\mathtt{k_m}$:
%
\begin{lstlisting}
let b = qSelect1 l in
  qSelect2 b @$k_1$@
  qSelect2 b @$k_2$@
   @$\vdots$@ 
  qSelect2 b @$k_m$@.
\end{lstlisting}

Assuming \texttt{l} contains $n$ elements, this optimization changes the
asymptotic complexity from expected (randomized) $\Theta(n \cdot m)$ to
$\Theta(n\log{n} + m\log{n})$, which for any $m \approx n$ reduces the
complexity from $\Theta (n^2)$ to $\Theta(n\log{n})$---a significant improvement. 

In this paper, we develop a splitting algorithm
(\ref{sec:splitting,sec:implementation}) which, given a program $e$ in \lang,
produces an equivalent pair of programs which correspond precisely to the two
stages of computation in $e$.
(Splitting is always possible because the staging annotations in $e$ are
consistent, because $e$ is well-typed in \lang.) In the case of
\texttt{qsStaged}, our splitting algorithm produces the algorithm described
above.

%\texttt{qSelect l k = qSelect2 (qSelect1 l) k}.

%Because the tree passes information across the stage boundary, we call it the
%\emph{boundary data structure}.

%Note that the desired optimized code shows above is intellectually more
%sophisticated than the code that we have started with: the optimized code is
%able to create a data structure, a balanced binary tree augmented with indexing
%information, and use a binary search technique over this tree to compute the
%result asymptotically more efficiently.

%In fact, based our teaching experience, we can imagine this kind of problem to
%be a moderately difficult exam question in an undergraduate algorithms class, as
%it not only requires understanding of data structures such as binary search
%trees but also requires modifying them to augment with indexing information to
%support rank-based search.

%(I also want to make it clear that recognizing \lang's appropriateness for this
%is itself is a contribution.)

\subsection{Quickselect in more detail}

Carlo writes this.



\end{abstrsyn}
