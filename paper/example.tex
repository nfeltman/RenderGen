\section {Example: Staged Quickselect in \texorpdfstring{\lang}{λ12}}
\label{sec:staging}

\ur{
\begin{itemize}
\item 
I find it hard to distinguish between blue and black on printed paper.
\item
$\pause$ should be called ``promote'' or something like that. I
  believe this is what Taha calls cross stage persistence.  THat work should
  be cited.
\end{itemize}
}

%!TEX root = ../paper.tex

\begin{figure*}
\begin{subfigure}{0.5\textwidth}
\begin{lstlisting} 
datatype list = Empty | Cons of int * list

fun part (p, Empty) = (0,Empty, Empty) 
  | part (p, Cons (h,t)) = 
    let val (s,left,right) = part (p,t) in 
    if h<p 
    then (s+1,Cons(h,left),right) 
    else (s,left,Cons(h,right))

fun qSelect (Empty, k) = 0
  | qSelect (Cons ht, k) =
    let val (n,left,right) = part ht in
    case compare k n of
      LT => qSelect (left, k)
    | EQ => #1 ht
    | GT => qSelect (right, k-1-n)
\end{lstlisting}
\caption{Unstaged implementation of quickselect.}
\label{fig:qs-unstaged}
\end{subfigure}%
\begin{subfigure}{0.5\textwidth}
\begin{lstlisting} 
1`atsignpure{
  datatype list = Empty | Cons of int * list
  fun part (...) = ...
} 

`//1`qsStaged : ^list * $`2`int`1` -> $`2`int`1`
fun qsStaged (Empty,_) = next {`2`0`1`}
  | qsStaged (Cons ht,next{`2`k`1`}) = 
    let 
      val (n,le,ri) = part ht
      val next{`2`n`1`} = hold n 
    in next{`2`
      case compare k n of
        LT => prev {`1`qsStaged (le, next{`2`k`1`})`2`}
      | EQ => prev {`1`hold (#1 ht)`2`}
      | GT => prev {`1`qsStaged (ri, next{`2`k-1-n`1`})`2`}`1`
    }`
\end{lstlisting}
\caption{Staged implementation of quickselect in \lang.}
\vspace{1.3em}
\label{fig:qs-staged}
\end{subfigure}
\caption{Quickselect: traditional and staged.}
\end{figure*}




Suppose that we wish to issue order statistics queries on a collection of items
represented as a list \texttt{l}. We can use quickselect \cite{quickselect}, an
expected linear time algorithm which takes a list \texttt{l} and a rank
\texttt{k} and returns the element with rank \texttt{k} in \texttt{l}.

Quickselect, which we define in \ref{fig:qs-unstaged}, first partitions the list
by using the first element as a pivot%
\footnote{We assume that the list is prepermuted to guarantee the expected
linear time behavior.}
and then recurs on one of the two resulting sides to find the desired element.
The side chosen is determined by the relationship of \texttt{k} to the size of
the first half \texttt{n}, which is returned by the \texttt{partition} function
along with the two sides themselves.

Now, consider an application where we perform many order statistics queries on
the same collection \texttt{l}, but with $m$ different ranks \texttt{k}.
Certainly, it is possible to implement this with $m$ calls to
\texttt{quickselect}:
%
\begin{lstlisting}
quickselect l @$k_1$@
quickselect l @$k_2$@
 @$\vdots$@ 
quickselect l @$k_m$@.
\end{lstlisting}
%
Can we do better?

\subsection{Staging}

An astute programmer might notice that much of the code in \texttt{quickselect}
does \emph{not} depend on the rank \texttt{k}---for example, the list is
partitioned before \texttt{k} is ever used. Moreover, while the recursive calls
to \texttt{qSelect} are guarded by a comparison to \texttt{k}, the lists passed
in to those recursive calls also don't depend on \texttt{k}. Thus, if we are
willing to change the evaluation order of the language (and evaluate under the
\texttt{case} statement), it should be possible to set things up so that
\emph{all} the computation involving \texttt{l} is performed before \emph{any}
of the computation involving \texttt{k}.

We can formalize this intuition by writing quickselect in \lang\
(\ref{sec:semantics}), a \emph{staged}, typed lambda calculus. In \lang, we can
directly express the idea that the argument \texttt{l} is known at the
\emph{first} stage of the computation, while the argument \texttt{k} is only
known at the \emph{second}, and all purely-stage-\bbone\ computation occurs
before any stage \bbtwo\ computation.

A staged version of quickselect is written in \ref{fig:qs-staged}, where stage
\bbone\ code is red, and stage \bbtwo\ code is blue. Whereas before
\texttt{qSelect} took a \textrm{list} and \rmint\ and returned an \rmint, now we
must be more careful---at stage \bbone, \texttt{qsStaged} takes a \textrm{list}
and a $\fut\rmint$ (an $\rmint$ \emph{in the future}), and returns a
$\fut\rmint$. 

At stage \bbone, all terms of non-circle type are available for
immediate use. Indeed, we are able to case on \texttt{l} as usual, and in the
\texttt{Cons} branch, pass it to the stage \bbone\ function \texttt{partition}.

\crem{Very rough text for the remainder of the section. Be a bit more explicit.}

However, \texttt{k} has type $\fut\rmint$, so we cannot use it as part of a
stage \bbone\ computation. Luckily, we are trying to produce a $\fut\rmint$. To
create one of these, we can wrap $\next$ around a stage \bbtwo\ computation of
an integer; in that computation, we can wrap $\prev$ around the $\fut\rmint$
\texttt{k} in order to use it as a stage \bbtwo\ integer.

(There's a subtle difference here between an \emph{int later} at the first
stage, and an int at the second stage.)

Essentially, $\next$ allows us to pass around opaque stage \bbtwo\ computations
at stage \bbone, and the $\prev$ construct lets us unwrap those computations for
use at stage \bbtwo.

For convenience, the code also uses $\pause$ blocks. This construct has same
stage signature as $\prev$, but instead of ``unwrapping'' $\fut$ types it simply
promotes integers from stage \bbone\ to stage \bbtwo.%
\footnote{$\pause$ can be implemented from our other language features, with
some effort. We provide it as a primitive to shorten examples. Furthermore, it
would be wise to extend $\pause$ to other base types, if we had them, and to
products and sums thereof.  This is related to the notion of {\em mobility} in
\cite{murphy05} and {\em stability} in \cite{krishnaswami13}.}

The process of adding staging annotations to unstaged code has been the subject
of extensive research under the name of \emph{binding time analysis}. In this
paper, we assume that these annotations have been provided by the programmer (or
perhaps a binding time analysis tool), and do not consider the problem of
generating such annotations. There are many ways to annotate any program,
including \texttt{qselect}; the version presented here matches the intuition for
the maximal amount of work that can be performed at stage \bbone.

The purpose of writing quickselect in \lang\ is that, because \texttt{qsStaged}
typechecks, it is a valid way of dividing the program into two stages, where the
stage \bbone\ computations can all be performed prior to the stage \bbtwo\
computations. In \ref{sec:semantics} we provide an evaluation strategy for
\lang\ which validates that claim.

%The construct $\next$ occurs in a stage \bbone\ context and indicates that the
%contents of its block are stage \bbtwo, whereas $\prev$ occurs in a stage
%\bbtwo\ context and indicates that the contents of its block is stage \bbone.

%(We adopt the convention that the top-level context is stage \bbone.) 

\subsection{Splitting Staged Programs}

\begin{figure}
%\begin{subfigure}{0.5\textwidth}
\begin{lstlisting}
1`datatype list = Empty | Cons of int * list
fun partition (p : int, l : list) = ...`
	
datatype tree = Branch of int * int * tree * tree
                | Leaf

1`fun qSelect1 (l : list) : tree =
  case l of
    Empty => Leaf
  | Cons ht => 
      let (left,right,n) = partition ht in
      Branch (n, #1 ht, qSelect1 left, qSelect1 right)`

2`fun qSelect2 (p : tree, k : int) : int = 
  case p of
    Leaf => 0
  | Branch (n,h,p1,p2) => 
      case compare k n of
        LT => qSelect2 (p1,k)
      | EQ => h
      | GT => qSelect2 (p2,k-n-1)`
\end{lstlisting}
\caption{Split (two-phase) implementation of quickselect.}
\label{fig:qs-split}
%\end{subfigure}
%\caption{Caption place holder}
\end{figure}



Because we have found a valid staging \texttt{qsStaged}, one might ask what
happens if we split this function into \emph{two} functions, one of which
performs all the stage \bbone\ computations given \texttt{l}, and the second
which performs all the stage \bbtwo\ computations given \texttt{k} and the
results of the previous.

Intuitively, what should this be? All calls to \texttt{partition} are in stage
\bbone, so this means we build a binary search tree from \texttt{l}. All
decisions on which side to recur down happen in stage \bbtwo, so this means we
traverse the tree once we have the rank \texttt{k}.

(There are two kinds of things we must pass to the second function---which stage
\bbtwo\ computations to use, and also anything that was held.)

\crem{introduce boundary terminology}

Formally, we wish to derive two functions, \texttt{qSelect1} and
\texttt{qSelect2}, which respectively contain all the stage \bbone\ and stage
\bbtwo\ content of \texttt{qSelect}, and are moreover semantically equivalent to
\texttt{qSelect} in the sense that:
\\
\texttt{qSelect l k = qSelect2 (qSelect1 l) k}.
\\

Since \texttt{quickselect1} requires only on a first stage
argument \texttt{l}, we can run it once to produce a boundary data
structure that can be used for all subsequent queries on \texttt{l},
e.g., we can perform $m$ queries by simply running:
%
\begin{lstlisting}
let b = qSelect1 l in
  qSelect2 b @$k_1$@
  qSelect2 b @$k_2$@
   @$\vdots$@ 
  qSelect2 b @$k_m$@.
\end{lstlisting}

Indeed, in \ref{sec:splitting,sec:implementation} we describe an algorithm for
splitting staged terms, which on \texttt{qsStaged} actually emits the algorithm
in \ref{fig:qssplit}.

Assuming that the list~\texttt{l} contains $n$ elements, this optimization
changes the asymptotic complexity from expected (randomized) $\Theta(n \cdot m)$
to $\Theta(n\log{n} + m\log{n})$, which for any $m \approx n$ reduces the
complexity from $\Theta (n^2)$ to $\Theta(n\log{n})$---a near linear time
improvement. 

%Note that the desired optimized code shows above is intellectually more
%sophisticated than the code that we have started with: the optimized code is
%able to create a data structure, a balanced binary tree augmented with indexing
%information, and use a binary search technique over this tree to compute the
%result asymptotically more efficiently.
%In fact, based our teaching experience, we can imagine this kind of problem to
%be a moderately difficult exam question in an undergraduate algorithms class, as
%it not only requires understanding of data structures such as binary search
%trees but also requires modifying them to augment with indexing information to
%support rank-based search.

%(I also want to make it clear that recognizing \lang's appropriateness for this
%is itself is a contribution.)
